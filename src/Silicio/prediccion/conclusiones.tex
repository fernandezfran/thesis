% Copyright (c) 2024, Francisco Fernandez
% License: CC BY-SA 4.0
%   https://github.com/fernandezfran/thesis/blob/main/LICENSE
\section{Conclusiones del capítulo}

Los resultados obtenidos en este capítulo pueden resumirse destacando que 
se utilizó el modelo de DFTB desarrollado en el capítulo previo para generar 
estructuras amorfas de Li-Si para distintas concentraciones de Li, cubriendo
el intervalo experimental usualmente estudiado mediante técnicas electroquímicas. La función distribución radial de a pares 
obtenida de sus configuraciones atómicas resultó en una concordancia excelente
con los datos experimentales. Cuando las mismas estructuras fueron analizadas 
con los modelos de vecinos más cercanos propuestos para predecir el corrimiento
químico de RMN de $^7$Li y la separación entre los picos en espectroscopia de 
Mössbauer también se encontró una buena concordancia con los experimentos.
