\section{Resultados y discuciones}

Los resultados se presentan a continuación en el orden en el que se emplean 
los pasos en la librería de Python \path{galpynostatic}: una primera subsección
para el preprocesamiento de los datos experimentales, luego otra para el ajuste
de estos datos con el modelo heurístco, y, por último, la utilización de este
para predecir las condiciones del tamaño de partícula pára lograr una carga
rápida en 15 minutos. Sumado a esto, también se comparan el comportamiento que
tendrían los distintos materiales, dados sus parámetros fundamentales, a 
distintos tamaños.

\subsection{Preprocesamiento de los datos experimentales}

Un procedimiento experimental usual para evaluar los materiales de las baterías
consiste en medir los perfiles galvanostáticos a distintos valores de C-rate.
En la Figura \ref{fig:preproc} se muestran como ejemplo las mediciones realizadas
por Wang \textit{et al.} \cite{wang2019high} para LiCoO$_2$ (LCO) recubierto con 
TiO$_2$. Además, se agrega una línea punteada horizontal que se corresponde
con el potencial de equilibrio reportado en el trabajo citado, 3.9 V, y otra
0.15 V por debajo, que es el valor que corresponde al potencial de corte
elegido en este capítulo. Esta es la región de interés en el gráfico, ya que 
los valores en los que el SOC se intersecta con esta última curva (SOC$_{\max}$)
son los que se utilizan para ajustar el modelo en función de C-rate.
\begin{figure}[h!]
    \centering
    \includegraphics[width=0.7\textwidth]{FastCharging/un/resultados/preprocesamiento/preprocesamiento.png}
    \caption{Perfiles galvanostáticos para distintos valores de C-rate para el
    sistema LCO recubierto de TiO$_2$. Las líneas horizontales indican el 
    potencial de equilibrio y el de corte utilizado para determinar la 
    capacidad máxima normalizada alcanzada (SOC$_{\max}$) a cada C-rate. 
    Reproducido del trabajo de Wang \textit{et al.} \cite{wang2019high}.}
    \label{fig:preproc}
\end{figure}

Es importante destacar que, en el trabajo citado, los perfiles galvanostáticos
se presentan en función del SOC normalizado, que no siempre es el caso. La 
forma usual en la que estos resultados son reportados es en función de la 
capacidad de descarga. En estos casos, es necesario normalizarla con respecto
a la capacidad máxima ($Q_{\max}$) alcanzada por el material, para así obtener
el SOC normalizado. El criterio utilizado en este capítulo para encontrar 
$Q_{\max}$ fue considerar el valor máximo de la capacidad alcanzado por la 
medición a la C-rate más baja. Gráficos similares al presentado en la Figura 
\ref{fig:preproc} son obtenidos en el resto de los trabajos experimentales que
se utilizan en los ajustes que siguen.


\subsection{Ajuste del modelo}

\begin{figure}[h!]
    \centering
    \includegraphics[width=0.7\textwidth]{FastCharging/un/resultados/ajuste/ajustes.png}
    \caption{Ajuste del modelo a los datos SOC$_{\max}$ \textit{versus} C-rate
    para los distintos materiales de electrodos considerados: (a) Grafito 
    amorfo \cite{mancini2022}, (b) LTO \cite{he2012}, (c) LFP \cite{lei2015}, 
    (d) LCO \cite{wang2019high}, (e) LMO \cite{bak2011}, (f) LNMO
    \cite{nishikawa2017}.}
    \label{fig:ajustes}
\end{figure}

En la Figura \ref{fig:ajustes} se muestran los datos experimentales y los 
ajustes del modelo para el SOC$_{\max}$ alcanzado al potencial de cortes 
\textit{versus} la C-rate para los resultados de la Figura \ref{fig:preproc}
y otros materiales de uso común en lso electrodos de las baterías de ion-litio.
Puede observarse una buena concordancia en general entre el modelo y los 
experimentos. Esto también puede observarse en la Figura \ref{fig:pred_vs_exp},
donde se muestran los valores predichos para SOC$_{\max}$ en función de los
experimentales, junto con el coeficiente de determinación de cada ajuste.

\begin{figure}[h!]
    \centering
    \includegraphics[width=0.7\textwidth]{FastCharging/un/resultados/ajuste/pred_vs_exp.png}
    \caption{Predicciones del SOC$_{\max}$ \textit{versus} valores 
    experimentales, junto al coeficiente de determinación de cada sistema.}
    \label{fig:pred_vs_exp}
\end{figure}

El trabajo de Mancini \textit{et al} \cite{mancini2022} aportó nuevos 
conocimientos sobre el efecto de esferoidización en las características de las
partículas de grafito y su impacto en el comportamiento electroquímico. A
continuación se hace referencia a estos datos como grafito amorfo, como puede
verse en la Figure \ref{fig:ajustes}a. En este caso, el rango de C-rates
reportadas cubre una amplia región del SOC$_{\max}$, desde un material 
totalmente cargado hasta uno casi totalmente descargado. Esto no es lo habitual,
ya que en la mayoría de los experimentos sólo se reportan curvas con un buen 
rendimiento (alta capacidad), lo que limita los ajustes realizados a una región
concreta del diagrama. El coeficiente de difusión y la constante cinética 
obtenidos en el ajuste para este sistema son $1.23\times10^{-10}$ cm$^2$/s y 
$2.31\times10^{-7}$ cm/s, respectivamente. Para esto se consideró un tamaño
de partícula de $7.5 \mu$m y una geometría esférica. Este valor se corresponde
con la media de la distribución de tamaños, que es reportada junto a los 
cuartiles en el trabajo citado. Para los casos que siguen, en los que no se
tiene información precisa de la distribución de tamaños, se considera el punto
medio del rango reportado para el tamaño de las partículas y utiliza para 
definir el parámetro $d$ en el modelo.

Los ánodos de Li$_4$Ti$_5$O$_{12}$ (LTO) presentan características excelentes
de seguridad y una capacidad teórica de 175 mAhg$^{-1}$. He \textit{et al} 
\cite{he2012} sintetizaron nanopartículas cristalinas y esféricas de LTO a 
diferentes proporciones atómicas de Li/Ti, con un tamaño entre los 0.5 $\mu$m 
y los 3 $\mu$m. Se asume entonces un valor de $d=1.75 \mu$m y se utilizan los
datos de la proporción usual del LTO para ajustar el modelo, teniendose como 
resultado un valor de $D$ de $6.58\times10^{-12}$ cm$^2$/s. El valor experimental 
de $D$ reportado por He \textit{et al} para esta proporción atómica fue de
$5.12\times10^{-12}$ cm$^2$/s. Con respecto al valor de $k^0$, se obtuvo 
$8.11\times10^{-8}$ cm/s. Comparar este valor con el experimental no es tan 
directo, ya que lo que reportan es la densidad de corriente de intercambio, 
$i^0 = 2.7\times10^{-4}$ mA/cm$^2$. Utilizando la ecuación \ref{eq:bv} de 
Butler-Volmer se tiene una relación entre $k^0$ e $i^0$ dada por
\begin{equation}\label{eq:i0k0}
    i^0 = F \frac{\rho}{M_r} k^0 \left(x_s\right)^{\alpha} \left(1 - x_s\right)^{1-\alpha},
\end{equation}
donde las definiciones de los parámetros están dadas en la Tabla \ref{t:params}.
Asumiendo un valor de 0.5 para el coeficiente de transferencia $\alpha$, un 
SOC de 0.5 y utilizando los valores del precusor LTO para $M_r = 459.1$ g/mol y
$\rho = 3.48$ g/cm$^3$ \cite{osti_1284125} se obtiene un valor para $k^0$ de
$7.38\times10^{-10}$ cm/s, que presenta una discrepancia de dos ordenes de 
magnitud con respecto al ajustado en el modelo. Sin embargo, en la literatura
se encuentran valores de $i^0$ con una gran dispersión entre 
$i^0 = 1.1\times10^{-3}$ mA/cm$^2$ \cite{medina2015} y $i^0 = 0.5$ mA/cm$^2$ 
\cite{umirov2019} que darían valores de $k^0$ entre $3.00\times10^{-9}$ cm/s y 
$1.37\times10^{-6}$ cm/s, respectivamente. Por lo cual puede afirmarse que el 
valor estimado por el modelo es razonable, dada la simplicidad del mismo.

Otro sistema en el cual las C-rates a las que se midieron los perfiles 
galvanostáticos cubren un rango amplio de valores de SOC$_{\max}$, de 
completamente cargado a completamente descargado, es el de LiFePO$_4$ (LFP) de 
Lei \textit{et al} \cite{lei2015}, como puede verse en la Figura 
\ref{fig:ajustes}. En este trabajo consideraron sistemas LFP/nanotubos de 
carbono/grafeno (LFP-CNT-G) como materiales catódicos con una capacidad 
superior a velocidades de carga alta y un desmpeño favorable en sucesivos 
ciclados a densidades de corriente relativamente altas, comparados con 
sistemas LFP-CNT y LFP-G. Para este caso seleccionado, obtuvieron coeficientes 
de difusión, a partir de la pendiente de un ajuste lineal a mediciones de 
espectroscopia de impedancia electroquímica (EIS), de $1.04\times10^{-12}$ 
cm$^2$/s, $1.738\times10^{-13}$ cm$^2$/s y $8.225\times10^{-13}$ cm$^2$/s, 
respectivamente para cada uno de los sistemas mencionados. Mientras que al
ajustar los datos experimentales del primer sistema mencionado, con un tamaño
de partícula de $0.35 \mu$m, se obtuvo un valor de $2.85\times10^{-13}$ cm$^2$/s
para este parámetro. Como puede observarse, se aprecia una discrepancia de un 
orden de magnitud pero dentro de los valores obtenidos en las otras sintesís.
El valor obtenido para $k^0$ utilizando la ecuación \ref{eq:i0k0} y el dato 
$i^0=5.127\times10^{-4}$ mA/cm$^2$ es $1.23\times10^{-9}$ cm/s, con un valor
de $M_r$ de $157.75$ g/mol y $\rho$ de $1.36$ g/cm$^3$ \cite{jin2018}.
En este caso se encuentra una correspondencia excelente con el valor ajustado
de $1.00\times10^{-9}$ cm/s.



