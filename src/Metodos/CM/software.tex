\subsection{Software}

Las simulación de perfiles galvanostáticos y la construcción de diagramas de 
diagnóstico con el modelo de una sola partícula se realizaron con un código 
escrito en \path{C++}. Luego, para seguir el enfoque de ajustar datos 
experimentales a esta superficie construida (ver capítulo \ref{ch:un}) se 
escribió una librería en Python (apéndice \ref{software:galpynostatic}), que 
también se utilizó para proponer una métrica universal en el capítulo 
\ref{ch:umbem}.
