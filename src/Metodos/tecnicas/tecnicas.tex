\section{Técnicas de simulación a distintas escalas}

Dentro de la física, la química y las ciencias de los materiales existen diversas
técnicas computacionales para desarrollar modelos capaces de predecir y entender 
las propiedades de algún sistema en particular. Estos modelos no son más que 
abstracciones matemáticas o lógicas de la realidad que permiten obtener dicha 
información de interés a costas de renunciar a otra \cite{franco2013}. 
En la Figura \ref{fig:escalas} se muestran aplicaciones específicas de 
simulaciones numéricas que cubren distintos intervalos de escalas espaciales y 
temporales. En esta tesis se utilizan las técnicas resaltadas con colores en
dicha figura para estudiar distintos aspectos de materiales de interés en el 
área de las baterías de litio.

\begin{figure}[h!]
    \centering
    \includegraphics[width=\textwidth]{Metodos/tecnicas/escalas.png}
    \caption{Esquema de escalas espaciales y temporales relativas de distintas 
    técnicas de simulaciones computacionales. Se destacan con colores aquellas que 
    se utilizaron en esta tesis mientras que se mencionan otras.}
    \label{fig:escalas}
\end{figure}

Los métodos tradicionales de prueba y error requieren demasiado tiempo como 
para seguir el ritmo rápido de crecimiento en la demanda de sistemas de 
almacenamiento de energía, que actualmente se encuentran limitados por los
materiales disponibles. Esto ha atraído la atención de investigadores e ingenieros 
hacia el desarrollo de modelos computacionales, desde la escala atómica hasta la 
escala del continuo, y la integración de las mismas, para tener una herramienta 
predictiva para resolver los problemas relacionados a los electrones, los átomos, 
los clusters, las partículas, los electrodos, las celdas e incluso el pack de 
baterías \cite{shi2016}. En particular, en la escala atómica usualmente se tienen
predicciones bajo equilibrio termodinámico de propiedades como la estructura, 
transformaciones de fase, energías de activación, entre otras, como se lo realiza
en la Parte \ref{p:silicio} de esta tesis. Además, se busca encontrar relaciones 
que mapeen dichas propiedades con descriptores más simples de obtener 
\cite{juan2021}, como se propone en la Parte \ref{p:fast-charging} de esta tesis.
