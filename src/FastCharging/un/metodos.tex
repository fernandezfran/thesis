\section{Métodos computacionales}

\subsection{Fundamentos matemáticos}\label{s:metodologia}

\begin{table}[h!]
    \centering
    \caption{Parámetros utilizados en las simulaciones y la construcción de los
    diagramas}
    \setlength\extrarowheight{2pt}\stackon{%
    \begin{tabular}{l l}
        \toprule
        \thead{\normalsize\bfseries Parámetro} & 
        \thead{\normalsize\bfseries Definición} \\
        \midrule
        $\Xi = k^0 \sqrt{t_h / (C_r D)}$ & Parámetro galvanostático cinético \\
        $\ell = d (V/A) (C_r / (D t_h))$ & Parámetro galvanostático de difusión finita \\
        $D$ & Coeficiente de difusión del Li$^+$ \\
        $t, t_h$ & Tiempo y tiempo para una hora \\
        $C_r$ & C-rate \\
        $x, x_s$ & Grado de intercalación en el electrodo y en la interfase \\ 
                 & electrodo/electrolito \\
        $Q, Q_{\max}$ & Capacidad y capacidad máxima \\ 
        SOC & Estado de la Carga, $\text{SOC} = Q / Q_{\max}$ \\ 
        $z$ & Parámetro para establecer la geometría de la partícula: \\
            & $z=1$ plana, $z=2$ cilíndrica, $z=3$ esférica \\
        $d$ & Longitud de difusión -- radio de la partícula -- espesor \\ 
            & de la lámina \\
        $r$ & Distancia definida entre 0 y $d$ ($0 \leq r \leq d$) \\
        $V$ & Volumen del material activo \\
        $A$ & Área superficial del electrodo/electrolito \\
        $V / A$ & Proporción volumen/superficie, $d$ plana, $d/2$ cilíndrica, \\ 
                & $d/3$ esférica \\
        $I_c$, $i_c$ & Corriente y densidad de corriente (constantes) \\
        $k^0$ & Constante cinética \\
        $E$, $E^0$, $E_{\text{off}}$ & Potencial del electrodo de trabajo vs Li/Li$^+$,\\
                                     & potencial de equilibrio y de corte \\
        $\alpha$ & Coeficiente de transferencia de carga \\
        $\rho$ & Densidad del material \\
        $M_r$ & Masa molecular del material \\
        $F$ & Constante de Faraday \\
        $R$ & Constante universal de los gases \\
        $T$ & Temperatura absoluta \\
        \bottomrule
    \end{tabular}
    }{}
    \label{t:params}
\end{table}

Para el fenómeno de la intercalación de iones de litio se consideró un modelo de 
una sola partícula. Los parámetros de todas las ecuaciones que siguen se
presentan en la Tabla \ref{t:params}. En el modelo, el grado de intercalación $x$
en el punto $r$ y a tiempo $t$ se obtiene de resolver la ecuación diferencial 1D
de Fick:
\begin{equation}\label{eq:fick}
    \frac{\partial x}{\partial t} = D \left[ \frac{\partial^2 x}{\partial r^2} + \frac{(z - 1)}{r} \left(\frac{\partial x}{\partial r}\right) \right],
\end{equation}
donde $D$ es el coeficiente de difusión y $z$ depende del tipo de geometría de la 
partícula \cite{vassiliev2016}. Para resolver la ecuación \ref{eq:fick} se 
utilizó el método de Crank-Nicolson mediante diferencias finitas 
\cite{crank-nicolson} con las siguientes condiciones de contorno en la superficie
de la partícula ($r = d$),
\begin{equation}
    \left(\frac{\partial x}{\partial r}\right)_d = - \frac{I_c}{F A \frac{\rho}{M_r}D},
\end{equation}
y al centro de la partícula ($r = 0$),
\begin{equation}
    \left(\frac{\partial x}{\partial r}\right)_0 = 0.
\end{equation}

La condición de corriente constante se fijó con la ecuación de Butler-Volmer
\begin{equation}\label{eq:bv}
    I_c = F A \frac{\rho}{M_r} k^0 \left\{x_s \exp\left[ \frac{(1-\alpha)F(E-E^0)}{RT} \right] - (1 - x_s) \exp\left[ -\frac{\alpha F (E-E^0)}{RT} \right] \right\}
\end{equation}

Las simulaciones fueron realizadas con un código escrito en \path{C++}.


\subsection{Derivación de los parámetros adimensionales} 

En esta sección se derivan los parámetros adimensionales de la ecuación de
Butler-Volmer en condiciones de corriente constante. Para ello definimos
$\xi = (nF/RT)(E - E^0)$ y $u = D t / d^2$, la ecuación \ref{eq:bv} se puede 
expresar de la siguiente forma integral \cite{aoki1984}
\begin{equation}\label{eq:aoki}
    \frac{i}{n F c^0} = k^0 \left(e^{\alpha_a \xi} + e^{-\alpha_c \xi}\right) \left\{ \frac{1}{1+e^{-\xi}} - \frac{d}{D} \int_0^u \frac{i}{n F c^0} \Theta_3(0|u - z) dz \right\},
\end{equation}
donde $c^0$ es la concentración máxima del ion en el material. $\Theta_3(u)$ es 
la función tita \cite{bieniasz2015}. Si consideramos una corriente constantes
$i_c$, la ecuación \ref{eq:aoki} nos da
\begin{equation}\label{eq:aoki2}
    \frac{i_c}{n F c^0} = k^0 \left(e^{\alpha_a \xi} + e^{-\alpha_c \xi}\right) \left\{ \frac{1}{1+e^{-\xi}} - \frac{d}{D} \frac{i_c}{n F c^0} \int_0^u \Theta_3(0|u - z) dz \right\}.
\end{equation}
Esta corriente constante se la puede expresar en términos de la C-rate como
\begin{equation}
    I_c = \frac{C_r Q}{t_h}
\end{equation}
y usando que la capacidad del electrodo es
\begin{equation}
    Q = V \left( \frac{\rho}{M} \right) F n
\end{equation}
y dividiendo por el área superficial y $n F c^0$ se obtiene que
\begin{equation}\label{eq:i_c}
    \frac{i_c}{n F c^0} = \frac{V C_r}{A t_hi}.
\end{equation}
Reemplazando esta ecuación \ref{eq:i_c} en el lado derecho de la ecuación 
\ref{eq:aoki2}, se tiene
\begin{equation}\label{eq:aoki3}
    \frac{i_c}{n F c^0} = k^0 \left(e^{\alpha_a \xi} + e^{-\alpha_c \xi}\right) \left\{ \frac{1}{1+e^{-\xi}} - \frac{d}{D} \frac{V C_r}{A t_h} \int_0^u \Theta_3(0|u - z) dz \right\},
\end{equation}
de donde el factor $d (V/A) (C_r / (D t_h))$ se utiliza para definir el 
parámetro de difusión finita presentado en la ecuación \ref{eq:ele}. Reemplazando 
ahora la ecuación \ref{eq:ele} en \ref{eq:aoki3} y normalizando por la densidad 
de corriente y multiplicando a ambos miembros de la igualdad por el término 
$\sqrt{t_h / (C_r D)}$ se llega a que
\begin{equation}\label{eq:aoki4}
    \frac{i_c}{n F c^0} \sqrt{\frac{t_h}{C_r D}} = k^0 \sqrt{\frac{t_h}{C_r D}}\left(e^{\alpha_a \xi} + e^{-\alpha_c \xi}\right) \left\{ \frac{1}{1+e^{-\xi}} - \ell \int_0^u \Theta_3(0|u - z) dz \right\},
\end{equation}
de donde emerge el parámetro $\Xi$ ya definido en la ecuación \ref{eq:xi} como 
$\Xi = k^0 \sqrt{t_h / (C_r D)}$.


\subsection{Construcción de los diagramas}

Como la respuesta de los perfiles galvanostáticos sólo depende de ambos parámetros
adimensionales $\Xi$ y $\ell$, se pueden construir diagramas universales. Para 
este fin se definió una grilla de $N$ puntos en el espacio 
$\log(\Xi)--\log(\ell)$ y se fijó un potencial de corte. Para cada punto en la 
grilla se realizó la simulación y se guardó el valor de intercalación de litio
alcanzado, denotado por $\text{SOC}_{\max}$. Luego, para la construcción de los
diagramas se utilizaron unos miles de puntos.


\subsection{Un modelo heurístico para ajustar sistemas complejos}

Muchos de los electrodos de intercalación de litio que se investigan en los 
laboratorios o son utilizados en las baterías comerciales, pueden ser simulados
con modelos matemáticos complejos que consideran las particularidades de cada
caso \cite{doyle1995}. Durante el proceso de manufactura, en el cual se prepara
el precursor con una distribución de tamaños, se agrega un material conductor y
un \textit{binder}, lo cual genera una estructura porosa, cuyos poros definen
caminos de difusión tortuosos para los iones de litio. Los modelos que poseen
este nivel de detalles \cite{doyle1995} aportan información sobre las propiedades
específicas de un electrodo determinado, a cambio de dificultar una descripción
más general del problema. El modelo introducido en las secciones anteriores 
se presenta como una alternativa para tener una descripción universal del 
comportamiento de distintos materiales, ya que permite obtener la capacidad en 
condiciones de cargado galvanostático en electrodos de una sola partícula en 
función de 4 parámetros: la C-rate, el tamaño de partícula, el coeficiente de 
difusión y la constante cinética.

Este modelo tiene algunas suposiciones:
\begin{itemize}
    \item El transporte de carga dentro de la partícula está limitado por el
        movimiento de los iones insertados, i.e. que el transporte electrónico
        es rápido.
    \item La difusión de los iones obedece la segunda Ley de Fick 1D para 
        geometría plana, cilíndrica o esférica.
    \item Se ignora la transferencia de masa con el electrolito dada una alta 
        concentraciones de iones en el mismo.
    \item El coeficiente de difusión, $D$, permanece constante en todo el 
        proceso de intercalación.
    \item El flujo de iones es cero en el centro de la partícula.
    \item La transferencia de carga en la interfase electrodo/electrolito se 
        describe mediante la ecuación de Butler-Volmer.
    \item La constante cinética, $k^0$, es la misma en todo el proceso de 
        intercalación.
    \item La resistencia de la celda está dada por un valor constante, 
        $R_{\Omega}$.
    \item Se ignora la interacción entre los iones intercalados.
    \item No se consideran los cambios volumétricos de la partícula durante su 
        carga/descarga.
    \item También se ignoran migraciones de los iones por la presencia de un 
        campo eléctrico y reacciones secundarias.
\end{itemize}
La influencia de la mayoría de estas limitaciones a la hora de obtener la 
capacidad de la partícula se discuten en la referencia \cite{gavilan2023}. Algunas
de estas suposiciones pueden no cumplirse en muchos experimentos; sin embargo, el 
modelo puede ser utilizado como una guía heurística para entender el 
comportamiento global de los distintos materiales y para diseñar los materiales
activos, en lo que respecta a su tamaño de partícula medio. La obtención de dicho
comportamiento universal solo puede obtenerse al resignar a las particularidades
de cada sistema. 

El modelo heurístico basado en simulaciones galvanostáticas, que se presenta a 
continuación, tiene asociado un software que fue escrito en Python, 
\path{galpynostatic}, es open-source y fácil de usar, el mismo se describe en el
Apéndice \ref{software:galpynostatic}. Este modelo pretende obtener información
cinética de una manera simple y rápida, además, también pretende obtener el 
tamaño óptimo de la partícula para obtener una carga del 80\% en 15 minutos.
El mismo consiste en la utilización de los diagramas universales, como el que 
se presenta en la Figura \ref{fig:diagnostico}, para determinar los valores de
$\Xi$ y $\ell$ que mejor se ajustan a datos experimentales. Esto se logra al 
realizar una búsqueda de grilla con distintas combinaciones posibles del 
coeficiente de difusión $D$ y de la constante cinética $k^0$, manteniendo 
constantes los otros parámetros involucrados en $\Xi$ y $\ell$, como el tamaño 
de la partícula $d$ y el factor geométrico $z$. Luego, para cada valor de C-rate
considerado se tiene el valor experimental del SOC y se lo compara con el 
predicho.

Para cada elección de parámetros ($D$, $k^0$) se calcula en valor de 
$\text{SOC}_{\max}$ con el diagrama y se obtiene el error cuadrático medio (MSE,
\textit{mean square error}) referido a los valores experimentales de la siguiente
manera
\begin{equation}
    \text{MSE} = \frac{1}{n} \sum_{i=1}^n \left(y_i^{\text{exp}} - y_i^{\text{pred}}\right)^2,
\end{equation}
donde $n$ es el número de mediciones, $y_i^{\exp}$ son los valores 
experimentales del SOC$_{\max}$ e $y_i^{\text{pred}}$ los predichos por el 
modelo. Luego de una exploración exhaustiva, se obtiene el conjunto de 
parámetros que minimizan el MSE. Esto nos provee los valores de $D$ y $k^0$ que
caracterizan a un material de intercalación en particular. Con esta información
se puede predecir el tamaño óptimo de la partícula para el cual el sistema 
alcanza una capacidad considerable a una dada C-rate.
