\chapter{Abstract}

The electric vehicle industry is growing due to the need to use renewable 
energies to reduce greenhouse gas emissions. In this context, energy storage and 
transportation systems become essential. One of the biggest challenges facing 
electric vehicles is to reach the range and recharge time of internal combustion
vehicles. To achieve this, electrodes with high capacity and fast charging times 
are required. This Ph.D. thesis addresses both of these aspects.

In general terms, it can be affirmed that the two great advances achieved in this 
Doctoral Thesis refer to the establishment of an unprecedented universal metric 
to predict and evaluate the fast charging of materials, and to the prediction of 
the properties of amorphous alloys of Li-Si, one of the most promising materials
to be used as anodes for next-generation Li-ion batteries.

In the first part of this thesis, a heuristic model is developed to predict the 
optimal particle size of active material in the electrodes to achieve a fast 
charge of 80\% State of Charge (SOC) within 15 minutes. This model is based on 
simulations of galvanostatic techniques taking into account ion diffusion within 
the material and interfacial charge transfer. With this model it is possible to 
fit experimental data of the maximum SOC as a function of galvanostatic charging rate (C-rate) for different 
materials and to obtain diffusion coefficients and kinetic constants in a fast 
and simple way. The estimations made for the analyzed systems were found to be 
within the expected range of experimental values. Then, a universal metric is 
proposed to standardize the performance comparisons of different materials 
considered for fast charging applications. This metric presents an improvement 
over a previous literature suggestion that assumes ultrafast interfacial charge 
transfer.

In the second part of this thesis, amorphous Li-Si alloys formed in the 
lithiation of silicon anodes are studied. In the first chapter, molecular 
dynamics simulations are performed using a reactive force field and an accelerated 
exploration of local minima approach is proposed. In the second chapter, a DFTB 
(\textit{density functional tight-binding}) model is parameterized with a fitting 
algorithm that weights the different structures in the training set. It
outperforms the state-of-the-art reactive force field for this system in its 
accuracy in predicting formation energies in both the training crystalline 
structures and the evaluation amorphous structures. The radial distribution 
function of an amorphous silicon structure obtained by simulated annealing with 
this model results in excellent agreement with experimental data. The last 
chapter describes a lithiation protocol to obtain amorphous structures at 
different lithium concentrations. The atomic configurations obtained are analyzed 
with nearest neighbor models proposed here to predict nuclear magnetic resonance (NMR), x-ray and Mössbauer 
measurements. These predictions also show good agreement with experiments.
