% Copyright (c) 2024, Francisco Fernandez
% License: CC BY-SA 4.0
%   https://github.com/fernandezfran/thesis/blob/main/LICENSE
\documentclass[12pt, spanish, a4paper, twoside]{book}
\usepackage[spanish, es-nodecimaldot]{babel}

\usepackage{xcolor}
\definecolor{mycolor}{RGB}{31, 119, 180}

\newcommand{\todo}[1]{\textcolor{red}{#1}}
\newcommand{\done}[1]{\textcolor{blue}{#1}}
\newcommand{\change}[1]{\textcolor{black}{#1}}

\usepackage[breaklinks=true, hidelinks]{hyperref} 

\usepackage{geometry}
\geometry{
	a4paper,
	total={170mm,257mm},
	left=20mm,
	top=20mm,
}

\setlength\parindent{12pt}

\usepackage{setspace}
\onehalfspacing

\usepackage[T1]{fontenc}
\usepackage{mathpazo}

\usepackage{changepage}

\usepackage{fancyhdr}
\pagestyle{fancy}
\fancyhf{}
\let\oldheadrule\headrule
\renewcommand{\headrule}{\color{mycolor}\oldheadrule}
\renewcommand{\headrulewidth}{0.8pt}
\renewcommand{\chaptermark}[1]{\markboth{#1}{}}
\renewcommand{\sectionmark}[1]{\markright{#1}}
\fancyhead[CE]{\footnotesize\color{mycolor}\normalfont\bfseries\itshape\nouppercase{
    \textsc{\leftmark}}
}
\fancyhead[CO]{\footnotesize\color{mycolor}\normalfont\bfseries\itshape\nouppercase{
    \textsc{\rightmark}}
}
\fancyfoot[CE,CO]{\color{mycolor}\bfseries\thepage}

\renewenvironment{itemize}{
\begin{list}{}{
\setlength{\leftmargin}{1.5em}
}
}{
\end{list}
}

\usepackage{adjustbox}
\usepackage{makecell,booktabs}
\usepackage{stackengine}
\setstackEOL{\cr}

\usepackage{imakeidx}
\makeindex[columns=3, title=Alphabetical Index, intoc]
\setcounter{tocdepth}{2}
\setcounter{secnumdepth}{4}

\usepackage{sectsty}
\usepackage{titlesec}
\titleformat{\chapter}
    [display]
    {\centering\Huge\bfseries\color{mycolor}}
    {\chaptername\ \thechapter}
    {0pt}
    {\huge}

\partfont{\color{mycolor}}
\sectionfont{\color{mycolor}}
\subsectionfont{\color{mycolor}}
\subsubsectionfont{\color{mycolor}}
\let\oldtextbf\textbf
\renewcommand{\textbf}[1]{\textcolor{mycolor}{\oldtextbf{#1}}}

\titleformat{\section}
  {\normalfont\Large\bfseries\color{mycolor}}
  {\thesection}{1em}{}[{\titlerule[0.8pt]}]

\usepackage{dirtytalk}

\usepackage{graphicx}
\graphicspath{{./}}

\usepackage{float}

\usepackage[fleqn]{amsmath} 
\setlength{\mathindent}{24pt}
\usepackage{mathabx}
\usepackage{mathtools}
\usepackage{array}

\usepackage[tableposition=top, labelfont={color=mycolor,bf}]{caption}
\addto\captionsspanish{
    \def\listtablename{\'Indice de tablas}%
    \def\tablename{Tabla}
}
\usepackage{colortbl}
\arrayrulecolor{mycolor}

\usepackage{listings}
\definecolor{code-green}{RGB}{44, 160, 44}
\definecolor{code-magenta}{RGB}{227, 119, 194}
\lstdefinestyle{mystyle}{
    xleftmargin=2cm,
    xrightmargin=2cm,
    backgroundcolor=\color{white},
    commentstyle=\color{code-magenta},
    keywordstyle=\color{mycolor},
    stringstyle=\color{code-green},
    basicstyle=\ttfamily\footnotesize,
    breakatwhitespace=false,
    breaklines=true,
    captionpos=b,
    keepspaces=true,
    numbersep=5pt,
    showspaces=false,
    showstringspaces=false,
    showtabs=false,
    extendedchars=true,
    tabsize=4
}
\lstset{
    style=mystyle,
    literate=
    {á}{{\'a}}1
    {é}{{\'e}}1
    {í}{{\'i}}1
    {ó}{{\'o}}1
    {ú}{{\'u}}1
}

\usepackage[square]{natbib} % [square,numbers,sort,sectionbib] natbib or chapterbib
\usepackage{tocbibind}
\renewcommand{\cite}{\citep}

% \newcommand\blankpage{%
%     \null
%     \thispagestyle{empty}%
%     \addtocounter{page}{-1}%
%     \newpage
% }


\begin{document}

\frontmatter

% \blankpage
% \blankpage
% \blankpage
% \blankpage

% % Copyright (c) 2024, Francisco Fernandez
% License: CC BY-SA 4.0
%   https://github.com/fernandezfran/thesis/blob/main/LICENSE
\thispagestyle{empty}
\begin{center}
{\large
    % logo

    \vspace{1cm}

    {\Huge Modelado computacional para el desarrollo de electrodos de baterías de ion-litio de próxima generación}
    
    \vspace{0.5cm}
    por
    \vspace{0.5cm}
    
    {\Large Francisco Fernandez}

    \vspace{0.5cm}

    Presentado ante la Facultad de Matemáticas, Astronomía, Física y Computación 
    como parte de los requerimientos para la obtención del grado de
    
    \vspace{0.5cm}

    {\Large Doctor en Física}

    \vspace{0.5cm}
    de la

    UNIVERSIDAD NACIONAL DE CÓRDOBA

    \vspace{0.5cm}
    % logo 
    
    mes, 202?

    \textcopyright FaMAF - UNC 202?

    \vspace{1.5cm}

    Director: Daniel Eugenio Barraco Díaz

    Codirector: Ezequiel Pedro Marcos Leiva

    % licencia
}
\end{center}


% \chapter{Resumen}

La industria de los vehículos eléctricos está en crecimiento, debido a la 
necesidad de utilizar energías renovables para reducir las emisiones de gases de 
efecto invernadero. En este contexto, los sistemas de almacenamiento y transporte 
de energía se vuelven cruciales. Uno de los mayores desafíos que enfrentan es que 
los automóviles eléctricos alcancen la autonomía y el tiempo de recarga de los 
vehículos de combustión interna. Para lograrlo, se requieren electrodos con una 
gran capacidad y que logren una carga rápida. Esta tesis contempla ambas 
cuestiones.

En la primera parte de la tesis se desarrolla un modelo heurístico para predecir 
el tamaño óptimo de partículas de material activo en los electrodos para que 
logren una carga rápida del 80 \% del Estado de la Carga (SOC) en 15 minutos. El 
mismo se basa en simulaciones de técnicas galvanostáticas teniendo en cuenta la 
difusión de los iones dentro del material y la transferencia de carga interfacial. 
Con este modelo se pueden ajustar datos experimentales del SOC máximo en función 
de C-rate para distintos materiales y obtener coeficientes de difusión y 
constantes cinéticas de una forma rápida y simple. Las estimaciones realizadas 
para los sistemas analizados resultaron estar dentro del intervalo de valores 
experimentales esperado. Luego, se propone una métrica universal para 
estandarizar las comparaciones del desempeño de distintos materiales considerados 
para aplicaciones de carga rápida. Esta métrica presenta una mejora con respecto 
a una propuesta de literatura previa que supone una transferencia de carga 
interfacial ultra-rápida.

En la segunda parte de la tesis se estudian las aleaciones amorfas de Li-Si que 
se forman en la litiación de los ánodos de silicio. En el primer capítulo se 
realizan simulaciones de dinámica molecular utilizando un potencial reactivo y 
proponiendo un método acelerado de exploración de mínimos locales. En el segundo 
capítulo se parametriza un modelo DFTB (\textit{density functional tight-binding})
con un algoritmo de ajuste que pondera las distintas estructuras en el conjunto 
de entrenamiento. El mismo supera en su exactitud al potencial reactivo del 
estado del arte para este sistema a la hora de predecir energías de formación 
tanto en las estructuras cristalinas de entrenamiento como en las estructuras 
amorfas de evaluación. La función distribución radial de una estructura de 
silicio amorfa obtenida mediante un templado simulado con este modelo resulta en 
una concordancia excelente con datos experimentales. En el último capítulo se 
describe un protocolo de litiación para obtener estructuras amorfas a distintas 
concentraciones de litio. Las configuraciones atómicas obtenidas se analizan con 
modelos de vecinos más cercanos propuestos aquí para predecir mediciones de RMN, 
rayos x y Mössbauer. Estas predicciones también presentan una buena concordancia 
con los experimentos.


% \chapter{Abstract}

The electric vehicle industry is growing due to the need to use renewable 
energies to reduce greenhouse gas emissions. In this context, energy storage and 
transportation systems become essential. One of the biggest challenges facing 
electric vehicles is to reach the range and recharge time of internal combustion
vehicles. To achieve this, electrodes with high capacity and fast charging times 
are required. This Ph.D. thesis addresses both of these aspects.

In the first part of this thesis, a heuristic model is developed to predict the 
optimal particle size of active material in the electrodes to achieve a fast 
charge of 80 \% State of Charge (SOC) within 15 minutes. This model is based on 
simulations of galvanostatic techniques taking into account ion diffusion within 
the material and interfacial charge transfer. With this model it is possible to 
fit experimental data of the maximum SOC as a function of C-rate for different 
materials and to obtain diffusion coefficients and kinetic constants in a fast 
and simple way. The estimations made for the analyzed systems were found to be 
within the expected range of experimental values. Then, a universal metric is 
proposed to standardize the performance comparisons of different materials 
considered for fast charging applications. This metric presents an improvement 
over a previous literature suggestion that assumes ultrafast interfacial charge 
transfer.

In the second part of this thesis, amorphous Li-Si alloys formed in the 
lithiation of silicon anodes are studied. In the first chapter, molecular 
dynamics simulations are performed using a reactive force field and an accelerated 
exploration of local minima approach is proposed. In the second chapter, a DFTB 
(\textit{density functional tight-binding}) model is parameterized with a fitting 
algorithm that weights the different structures in the training set. It
outperforms the state-of-the-art reactive force field for this system in its 
accuracy in predicting formation energies in both the training crystalline 
structures and the evaluation amorphous structures. The radial distribution 
function of an amorphous silicon structure obtained by simulated annealing with 
this model results in excellent agreement with experimental data. The last 
chapter describes a lithiation protocol to obtain amorphous structures at 
different lithium concentrations. The atomic configurations obtained are analyzed 
with nearest neighbor models proposed here to predict NMR, x-ray and Mössbauer 
measurements. These predictions also show good agreement with experiments.


% \tableofcontents

\mainmatter


% \part{Motivación y fundamentos}

% \chapter{Introducción}\label{ch:introduccion}
\thispagestyle{empty}

\vspace{50pt}

\begin{adjustwidth}{50pt}{50pt}
    TODO
\end{adjustwidth}

\clearpage
\newpage
\thispagestyle{empty}
\mbox{}
\newpage

% Copyright (c) 2024, Francisco Fernandez
% License: CC BY-SA 4.0
%   https://github.com/fernandezfran/thesis/blob/main/LICENSE
\section{Contextualización}

El calentamiento global aparece como el mayor problema ambiental de este siglo.
El mismo se refiere al aumento de la temperatura media de la atmósfera y por 
ende a sus consecuencias en  el clima. Esto es debido al efecto que producen 
las actividades humanas, 
como por ejemplo la quema de combustibles fósiles, que emite 
a la atmósfera grandes cantidades de CO$_2$, entre otros gases de efecto 
invernadero, o la deforestación. Estos gases absorben la radiación infrarroja emitida por la tierra y la reemiten, 
provocando un incremento de la temperatura de la misma que lleva asociado un 
aumento en la frecuencia y la intensidad de eventos climáticos extremos. %\cite{houghton2005}. 
De acuerdo a el Panel Intergubernamental del Cambio Climático 
\cite{IPCC}, desde la época preindustrial, las actividades humanas han provocado 
aproximadamente 1.0$^{\circ}$C de calentamiento global y al ritmo actual se van 
a sobrepasar los 1.5$^{\circ}$C antes del 2050, un cambio en la temperatura
media que las medidas previas al aumento de las actividades mencionadas no habían llegado a alcanzar. Limitar el 
calentamiento a esta temperatura requiere que se realicen rápidamente cambios 
sin precedentes en la tecnología y en el comportamiento humano. Uno de los 
cambios más importante es el de la matriz energética, en la cual las energías 
renovables deberán suministrar alrededor del 80\% de la energía para 2050, donde 
los vectores energéticos, como las baterías de litio, juegan un rol fundamental 
debido a la intermitencia de estas formas de generación de energía.

El litio es el metal más liviano de la tabla periódica y uno de los elementos más
importantes dentro de los minerales necesarios en la producción de baterías de
litio. En particular, para la Argentina tiene un interés económico, social, 
industrial y tecnológico ya que es uno de los países que integran, junto a 
Bolivia y Chile, el Triangulo de Litio, el cual acumula el 70\% de las reservas 
mundiales de fácil extracción de este mineral. Esto último debido a que esta cantidad de reservas
se encuentran en salares de los que, a grandes rasgos, es más barato
extraer litio de ellos en comparación a las rocas de las cuales se puede extraer 
litio en una míneria usual, como las pegmatitas. A pesar de esto se tienen que
llevar a cabo distintas consideraciones ambientales, sociales y legales del 
proceso de extracción e incentivar el desarrollo de valor agregado a dicha 
extracción \cite{gutierrez2022, petavratzi2022, obaya2021, romero2021, 
heredia2020, fornillo2019}.

En esta tesis se presentan estudios computacionales sobre materiales para el 
desarrollo de electrodos de baterías de ion-litio de próxima generación. Se 
abordan dos perspectivas, una con el objetivo de tener baterías que frente a una 
carga rápida retengan un porcentaje considerable de la capacidad y otra 
utilizando electrodos que permitan almacenar mayor cantidad de energía que los 
actuales.


\section{Energía, transporte y litio}

En la actualidad se utilizan distintas formas para generar energía y pueden 
dividirse en dos grandes tipos, las renovables y las no-renovables. Estas últimas
dominan la producción de energía mundial y están compuestas principalmente por 
combustibles fósiles y centrales nucleares, mientras que las energías renovables
abarcan más variantes como la biomasa, la hidráulica, la eólica y la solar, pero 
aún no son lo suficientemente utilizadas. Una de las particularidades de estas 
fuentes de energías renovables es su producción intermitente mientras que el 
consumo de la misma, independientemente de cómo se genere, es a demanda. Esto 
hace que sea necesario el involucramiento de vectores energéticos que permitan 
almacenar y transportar el excedente de energía que se genera en sus períodos de 
mayor producción para luego ser utilizada en los momentos de mayor demanda.

El sector del transporte terrestre, marítimo y aéreo es responsable de más de un 
tercio de las emisiones de CO$_2$ debido a su dependencia en los combustibles 
fósiles \todo{\cite{IEA}}. Dicho esto, está claro que se debe fomentar opciones de desplazamiento menos intensivas
en carbono y con tecnologías más eficientes, como los vehículos eléctricos (EVs),
\todo{cuyos motores poseen una eficiencia para convertir la energía eléctrica en energía para las ruedas que ronda el 80\%, compárese este valor con las
eficiencias entre el 12\% y el 30\% de los motores a combustión interna para la misma tarea \cite{DOE}}.

En los últimos años se ha producido un crecimiento exponencial en las ventas 
anuales de los EVs, como puede observarse en la Figura \ref{fig:evs}a \cite{EVV}. En la
última década, dichas ventas aumentaron aproximadamente un 500\% y se estima que
para la próxima década las ventas se multipliquen por 10. Estas ventas están 
concentradas en China y en algunos países y estados de Europa y Estados Unidos, 
respectivamente, debido a que en los países en desarrollo y emergentes influye 
negativamente su costo alto de adquisición y una falta de infraestructura para la 
recarga de sus baterías. En particular, durante el 2022 en Noruega el 79.3\% de 
los automóviles patentados fueron eléctricos. En el país que le sigue en la lista,
Suecia, se patentó un 32.1\% de EVs en dicho año \cite{PWC}.
\begin{figure}[h!]
    \centering
    \includegraphics[width=\textwidth]{Introduccion/energia/evs.png}
    \caption{(a) Ventas anuales de vehículos eléctricos en la última década. Se 
    proyecta que para el 2030 las ventas asciendan a las 40 millones de unidades 
    frente a las 3 millones del año 2020 \cite{EVV}. (b) Proyección del costo en 
    dólares de vehículos eléctricos y de combustión interna en países 
    desarrollados \cite{BLOOMBERG}.}
    \label{fig:evs}
\end{figure}

En la Figura \ref{fig:evs}b se muestra la proyección en el costo de los vehículos 
eléctricos y de combustión interna realizada por la empresa financiera Bloomberg 
para los países desarrollados \cite{BLOOMBERG}. Se espera que para el año 2026 
los costos se igualen y que para el 2030 los EVs sean aproximadamente un 15\% más
baratos que los vehículos de combustión interna. Este cambio se debe a la 
disminución en el precio de la producción de baterías, que actualmente representa
aproximadamente el 40\% del costo del EV.

El sector energético en Argentina depende altamente de la utilización de 
combustibles fósiles, donde la generación de energía está dominada por el gas 
natural (65\%) y le siguen las centrales hidroeléctricas (18\%), plantas nucleares
(8\%), parques eólicos (7\%) y solares (1\%) \cite{IEA}. En cuanto al potencial de 
producción de fuentes renovables, Argentina tiene una gran capacidad en sus 
fuentes eólicas y solares por desarrollar. Además, es el cuarto productor mundial más 
grande de litio, que es un mineral crítico para la manufactura de sistemas de 
almacenamiento y transporte de energía, claves para la transición energética. 
El mismo representa el 7\% de la demanda para vehículos eléctricos mientras que 
para almacenamiento en la red el porcentaje es del 10\%. Otros metales y 
minerales críticos se encuentran en la región de América Latina; por ejemplo, 
Paraguay posee la reserva más grande del mundo de titanio, Chile es el mayor 
productor de cobre, Brasil tiene las segundas reservas más grande de níquel y
hierro, las terceras de grafito y manganeso, la cuarta de aluminio y la quinta de 
fósforo, por último, Cuba se encuentra en el tercer puesto de reservas de cobalto.

En la Figura \ref{fig:iea-Li} se muestra la proyección en la demanda total de 
litio por año y por aplicación, donde la mayor contribución se encuentra para la 
utilización del mismo en vehículos eléctricos mientras que una menor contribución 
se espera en aplicaciones de sistemas de almacenamiento estacionarios y otras 
aplicaciones que incluyen dispositivos electrónicos, medicamentos, lubricantes, 
entre otras \cite{IEA}. Cabe destacar que para el almacenamiento estacionario 
las baterías de ion-litio es muy probable que compitan con baterías de sodio o magnesio, entre 
otras. En el histograma de la Figura \ref{fig:iea-Li} pueden diferenciarse dos 
regiones, la primera de ellas entre el año 2022 y el 2035, donde los aumentos
porcentuales de la demanda de litio con respecto a 5 años atrás son del 74\%, 
99\% y 76\%. Luego, del año 2035 al 2040, el cambio se encuentra en el 32\% y
dicho aumento porcentual continúa disminuyendo al 10\% y al 3\% en los períodos 
subsiguientes.
\begin{figure}[h!]
    \centering
    \includegraphics[width=.8\textwidth]{Introduccion/energia/iea-Li.png}
    \caption{Proyección de la demanda total de litio en kilotoneladas para el 
    período 2025-2050 para sus distintas aplicaciones: vehículos eléctricos (en 
    azul), sistemas de almacenamiento de energía estacionarios (en naranja) y
    otras aplicaciones (en verde). Fuente: \cite{IEA}.}
    \label{fig:iea-Li}
\end{figure}


\section{Baterías de ion-litio}

A finales del año 2019, año en el que se comenzó esta tesis, la Real Academia 
de Ciencias de Suecia le otorgó el Premio Nobel en Química a J. B. Goodenough, 
M. S. Whittingham y A. Yoshino por sus contribuciones al desarrollo de la batería 
de ion-litio. Esta batería recargable permitió los avances que se vieron en los 
teléfonos móviles y en las computadoras portátiles, entre otras aplicaciones.
Además, permite un mundo libre de combustibles fósiles ya que se utiliza en 
vehículos eléctricos y en almacenamientos estacionarios de energía para fuentes
renovables. Este galardón restaltó la importancia de muchos aspectos de la ciencia
moderna, como la investigación básica, la investigación la aplicada, la 
interdisciplina (JBG fue físico, MSW es un químico y AY un ingeniero) los 
desarrollos tecnológicos y los problemas concretos de las sociedades.
En la década del 1970, MSW desarrolló la primera batería utilizando un ánodo de
litio metálico y un cátodo de disulfuro de titanio. En 1980, JBG duplicó el 
voltaje original de dicha batería al introducir un cátodo de óxido de cobalto.
La desventaja de ambas se encontraba en el ánodo de litio metálico, que en los 
ciclos de carga y descarga se deposita preferentemente en sitios donde ya se 
ha depositado, dando lugar a estructuras ramificadas, llamadas dendritas, que 
pueden cortocircuitar la celda y llevar a la explosión de la misma. En 1985,
AY remplazó este material por uno carbonoso que incorpora los iones de litio
durante la carga y la descarga, disminuyendo los riesgos mencionados. Basandose
en este desarrolló, Sony comenzó a comercializar baterías de ion-litio en 1991.
La densidad de energía de estas baterías rondaba los 80 Wh/kg, en la actualidad
WeLion comercializa para los EVs de Nio una batería de ion-litio con una 
densidad de energía de 360 Wh/kg. En la Figura \ref{fig:whkg} se muestra la 
evolución de la densidad de energía en baterías de ion-litio comercializadas 
en los últimos 30 años. La importancia de esta característica para los EVs 
radica en la relación autonomía/peso.
\begin{figure}[h!]
    \centering
    \includegraphics[width=.8\textwidth]{Introduccion/baterias/whkg.png}
    \caption{Aumento en la densidad de energía en baterías de ion-litio comercializadas
    en los últimos 30 años. Figura adaptada de \cite{li2023700}.}
    \label{fig:whkg}
\end{figure}

Las baterías de ion-litio admiten una gran cantidad de recargas y las mismas están 
compuestas por celdas electroquímicas conectadas entre sí, las mismas son unidades 
fundamentales que permiten transformar la energía química almacenada en energía
eléctrica mediante una reacción redox (reducción-oxidación), en la cual uno de los 
componentes pierde electrones (se oxida) y el otro gana electrones (se reduce).
En la Figura \ref{fig:esquema_bateria} se muestra un esquema general con el 
funcionamiento que presenta una celda electroquímica de ion-litio y se destacan 
las componentes más relevantes: los electrodos positivo (cátodo) y negativo (ánodo) 
donde ocurren las reacciones redox en la carga/descarga de la celda, el electrolito 
por el cual difunden los iones de litio y el separador que suele ser un material 
poroso permeable al electrolito que se encarga de que los electrones circulen por 
el circuito externo. Durante la descarga de la reacción redox es espontánea y 
provoca la difusión de iones de litio por el electrolito desde el ánodo hacia el 
cátodo, junto con una corriente eléctrica en un circuito externo (flechas rojas). 
Durante la carga se debe aplicar una corriente eléctrica externa para tener la 
reacción inversa (flechas verdes).
\begin{figure}[h!]
    \centering
    \includegraphics[width=.8\textwidth]{Introduccion/baterias/esquema_bateria.png}
    \caption{Esquema de las componentes y el funcionamiento de una batería de 
    ion-litio.}
    \label{fig:esquema-bateria}
\end{figure}

En la Figura \ref{fig:scopus} se muestra el incremento en las últimas dos décadas
de los artículos científicos publicados en el área de las baterías de litio y, en 
particular, de las dos ramas estudiadas en esta tesis: la Carga rápida y los 
Ánodos de Si. En dicha figura se presentan datos extraídos de la base de datos 
Scopus \cite{SCOPUS} del número de publicaciones anuales normalizado con respecto 
al número de publicaciones en el año 2003, año en el que hubo 710 publicaciones 
en baterías de litio, 32 sobre ánodos de Si y 0 sobre carga rápida, por lo que 
se normalizó en este caso a la única publicación del 2004 en el tema.
\begin{figure}[h!]
    \centering
    \includegraphics[width=.8\textwidth]{Introduccion/baterias/scopus.png}
    \caption{Número de publicaciones anuales normalizado con respecto al año 2003. 
    Las consultas realizadas en Scopus \cite{SCOPUS} incluyen: 
    \texttt{lithium AND battery} (LIBs, en azul), \texttt{lithium AND battery AND 
    fast-charging} (Carga rápida, en naranja) y \texttt{lithium AND battery AND 
    silicon anodes} (Ánodos de Si, en verde).}
    \label{fig:scopus}
\end{figure}
La normalización y la escala logarítmica en la Figura \ref{fig:scopus} permiten
observar cualitativamente que la pendiente de crecimiento de publicaciones 
realcionadas a la carga rápida de baterías de litio es considerablemente mayor a 
de las otras dos. Además, los ánodos de Si se encuentran dentro de lo que sería
el creciemiento promedio del área de las baterías de litio. Un análisis de datos
cuantitativo permite determinar que en la última década el aumento de porcentaje
anual de publicaciones promedio fue del 15 \% y 16 \% para las baterías de litio 
y para los ánodos de silicio, respectivamente, mientras que para la carga rápida 
este porcentaje promedio asciende al 52 \%. Este análisis demuestra la relevancia
que la comunidad científica le da a los temas estudiados en esta tesis.


\section{Objetivos y estructura de la tesis}

Esta tesis tiene como objetivo estudiar materiales que se utilicen para el 
desarrollo de electrodos de baterías de ion-litio de próxima generación mediante 
distintos modelados computacionales. 
La misma se encuentra dividida en tres partes, la primera de ellas sobre la 
Motivación y fundamentos consistente de dos capítulos, el capítulo 
\ref{ch:introduccion} con esta introducción y el capítulo \ref{ch:metodos} con la
descripción de los distintos métodos computacionales utilizados. 
La Parte \ref{p:fast-charging} se divide en dos capítulos, ambos relacionados con 
la carga rápida de baterías de ion-litio. En el capítulo \ref{ch:un} se 
desarrolla un modelo para ajustar datos experimentales en condiciones 
galvanostáticas y predecir el tamaño óptimo de partículas que permite retener un 
80 \% de su capacidad frente a una carga realizada en 15 minutos 
\cite{fernandez2023towards}. El capítulo \ref{ch:umbem} busca una métrica 
universal que permita estandarizar las comparaciones del desempeño entre 
distintos materiales considerados en aplicaciones de carga rápida.
La Parte \ref{p:silicio} se centra en el estudio de las aleaciones presentes en 
los ánodos de silicio y se divide en tres capítulos. El capítulo 
\ref{ch:caracterizacion} caracteriza las estructuras de Li-Si encontradas con 
un potencial reactivo y con un método de exploración acelerada de mínimos locales
propuesto \cite{fernandez2021characterization}. En el capítulo \ref{ch:modelo} se
parametriza un modelo DFTB (\textit{denstity functional tight-binding}) para la 
interacción Li-Si mediante un algoritmo que asigna pesos a las distintas 
estructuras consideradas para el ajuste \cite{oviedo2023}. En el capítulo 
\ref{ch:prediccion} se proponen modelos de vecinos más cercanos para predecir 
mediciones de rayos x, RMN y Mössbauer a partir de las configuraciones atómicas
\cite{fernandez2023nmr}.
Cada uno de los capítulos mencionados en estas dos últimas partes se componen
de una introducción y detalles de los métodos computacionales utilizados, los 
resultados junto a las discusiones de los mismos y conclusiones parciales.
Por último, se cierra la tesis con el capítulo \ref{ch:comentarios} con los 
comentarios finales de la misma.



\section{Dinámica molecular}

La dinámica molecular (MD, de sus siglas en inglés, \textit{molecular dynamics})
es una técnica de simulación computacional que considera un sistema de $N$
partículas atómicas, que interactúan a través de un campo de fuerzas newtoniano,
de las cuales se obtiene su evolución temporal. La misma permite obtener
propiedades termodinámicas macroscópicas (temperatura, presión) de un sistema en 
equilibro a partir de cantidades microscopicas (posiciones, velocidades, fuerzas)
~\cite{frenkel2001, allen2017}.

Para entender mejor como trabaja esta técnica de simulación es conveniente ver
como funciona su código fuente, el mismo sigue, en la mayoria de los casos, la
siguiente forma:
\begin{enumerate}
    \item \underline{Inicialización del sistema}: se especifican las posiciones y
        velocidades iniciales de los átomos. También se elije un paso temporal, 
        un radio de corte para las interacciones y las condiciones de contorno que
        se van a respetar a lo largo de la simulación. 
    \item \underline{Cálculo de fuerzas}: con las posiciones específicadas se
        calcula la fuerza sobre cada uno de los átomos a través del campo de 
        fuerzas elegido.
    \item \underline{Integración de las ecuaciones de movimiento}: se integran las
        ecuaciones de Newton mediante algún integrador que obtiene las posiciones
        y velocidades del paso temporal siguiente a partir del actual.
    \item \underline{Computo de propiedades termodinámicas}: se realizan los
        cálculos de distintas cantidades de interés, como las energías potencial
        y cinéctica, la presión y la temperatura.
    \item De ser necesario, se aplica algún \underline{termostato o barostato}
        para realizar simulaciones en el ensamble termodinámico deseado.
    \item \underline{Evolución temporal}: se incrementa el tiempo adhiriendo un
        paso temporal y se vuelve al cálculo de las fuerzas con las nuevas 
        configuraciones.
\end{enumerate}

Estos pasos pueden verse en la figura \textcolor{red}{HACER UN DIAGRAMA DEL PSEUDO 
CÓDIGO PROPIO}. Veamos a continuación cada una de las partes en detalle.

\subsection{Configuraciones iniciales}

Mencionar algo sobre materials project.
Velocidades aleatorias.

\subsection{Condiciones de contorno}

Periódicas - Fijas

\subsection{Potenciales interatómicos}

\subsubsection{ReaxFF}

\subsubsection{DFTB}

\subsection{Integradores}

\subsection{Termostatos y barostatos}

\subsection{Métodos de exploración de la superficie energía-potencial}

\subsubsection{Minimizaciones locales}

\subsubsection{Templado simulado}

\subsubsection{Dinámica acelerada}

\section{Experimentos computacionales}

\subsection{Distribución radial de a pares}

\subsection{Número de coordinación}

\subsection{Difusión}



% \part{Carga rápida de baterías de ion-litio}\label{p:fast-charging}

% \chapter{Un modelo heurístico basado en simulaciones galvanóstaticas}\label{ch:un}
\thispagestyle{empty}

\vspace{50pt}

\begin{adjustwidth}{50pt}{50pt}
    Resumen...
\end{adjustwidth}

\clearpage
\newpage
\thispagestyle{empty}
\mbox{}
\newpage


% \chapter{UMBEM: Una métrica universal para comparar el desempeño de materiales de electrodos de carga rápida}\label{ch:umbem}
\thispagestyle{empty}

\vspace{50pt}

\begin{adjustwidth}{50pt}{50pt}
    Resumen...
\end{adjustwidth}

\clearpage
\newpage
\thispagestyle{empty}
\mbox{}
\newpage



% \part{Silicio como ánodo de baterías de ion-litio de próxima generación: Estudio de sus aleaciones}\label{p:silicio}

% La información experimental que puede obtenerse de la estructuras de las 
distintas fases amorfas que se forman durante el ciclado de los electrodos de 
silicio es bastante limitada. Las estructuras de la red o de la interfase son 
inestables y amorfas, lo cual dificulta su caracterización mediante técnicas
experimentales tradicionales. Por ejemplo, la difracción de rayos x permitió
caracterizar la fase cristalina Li$_{15}$Si$_4$ que está presente en el electrodo
cuando este se encuentra completamente cargado ~\cite{obrovac2004}, pero esta 
técnica tiene ciertas limitaciones a la hora de estudiar estructuras amorfas 
que se encuentran en los procesos de carga y descarga. Por otro lado, el análisis
de la función distribución de a pares de Si \textit{ex-situ} de datos de rayos x
hizo posible investigar el orden a corto alcance de las estructuras amorfas de
Li$_x$Si ~\cite{key2011} y proponer una explicación al mecanismo de litiación.
Sin embargo, como los elementos livianos como el Li tienen baja sensibilidad a 
los rayos x, las conclusiones están simplificadas a la formación de pequeños 
clusters de Si o de átomos de Si aislados durante la litación, lo cual limita la 
descripción de la estructura a una escala mayor.

Dentro de este contexto, las simulaciones computacionales se posicionan como una
herramienta poderosa para acceder al comportamiento microscópico de las 
estructuras de Li$_x$Si y los cambios que sufren durante la litiación. Actualmente
no existe un único modelo computacional robusto que permita estudiar todos los
diferentes procesos presentes en los electrodos de silicio, por lo que se han 
llevado a cabo distintos esfuerzos en los últimos años para estudiar este sistema.
Donde el mayor obstáculo está relacionado con la naturaleza intrínseca de 
multi-escala del silicio. A pesar de su gran precisión, los estudios de DFT se
encuentran drásticamente limitados en el número de átomos que se pueden utilizar
para modelar las estructuras complejas de las fases litiadas. Una solución a este
problema es utilizar potenciales interatómicos semi-empíricos, para los cuales 
se necesita una parametrización que se robusta y transferible. Como esta 
parametrización esta fuera de los objetivos que plantea esta tesis, en este 
capítulo se utiliza una realizada para un potencial reactivo, introducido en 
la sección \ref{s:reaxff}, para sistemas de Li-Si previamente realizada por Fan 
\textit{et al.} ~\cite{fan2013}, en la cual optimizaron el campo de fuerzas usando 
cálculos de DFT, considerando datos de las energías, distintas geometrías y cargas
de las fases cristalinas de Li, Si y aleaciones de Li-Si. En su trabajo lo 
utilizaron en simulaciones de MD para caracterizar las propiedades mecánicas de
las estructuras amorfas de Li$_x$Si, incluyendo litiación de capa fina, compresión
biaxial, tensión y compresión uniaxial y la tensión que puede soportar el sistema 
antes de deformarse.


\section{Campo de fuerzas}

El campo de fuerzas de Fan \textit{et al.} ha sido ampliamente utilizado en 
simulaciones de MD para estudiar el proceso de litiación de distintas estructuras
de silicio, desde estructuras periódicas a nanoestructuras. Previo a la 
realización del trabajo de este capítulo, se consultó la bibliografía para 
verificar esto y asegurarnos de la transferibilidad del potencial. 

Además de los resultados reportados por Fan \textit{et al.} ~\cite{fan2013}, 
la estructura, el estrés y la difusividad fue estudiada durante la litiación de 
silicio amorfo (a-Si) y silicio cristalino (c-Si) en diferentes orientaciones 
cristalográficas ~\cite{chen2020, kim2015}. Ding \textit{et al.} ~\cite{ding2017} 
reportó la variación de la velocidad de migración en la frontera de fases y la 
difusividad de Li en función del estrés externo aplicado, exponiendo que la 
tensión acelera la razón de litiación, mientras que la compresión la retarda. Kim 
\textit{et al.} ~\cite{kim2014} realizó simulaciones de MD para caracterizar la 
evolución estructural de la frontera de fases entre c-Si, con diferentes planos 
de orientación, con una fase amorfa de litiación máxima. Posteriormente, Fan 
\textit{et al.} ~\cite{fan2018} estudió nanoestructuras, computando la respuesta
mecánica de nanopilares de c-Si en la orientación (111) durante la litiación.
Un trabajo similar, pero para la orientación (100), fue realizado por Cao 
\textit{et al.} ~\cite{cao2019}. Tang \textit{et al.} ~\cite{tang2019} investigó
la evolución y la permanencia de la porosidad de nanocapas de Si durante los 
procesos de litiación y delitiación. Ostadhossein \textit{et al.} 
~\cite{ostadhossein2015} caracterizó la litiación de nanohilos de c-Si y mostró
que este potencial ReaxFF reproduce precisamente las barreras de energía de la
migración de Li obtenidas por DFT, tanto en c-Si como en a-Si.

Este potencial no estuvo sólo limitado al uso en MD, sino que fue aplicado a otros
métodos de simulación, por ejemplo, simulaciones de Monte Carlo en el ensamble
gran canónico fueron realizadas para estudiar un ciclo de litiación y delitiación
de un electrodo de a-Si ~\cite{basu2019}. Trochet y Mousseau ~\cite{trochet2017}
caracterizaron el paisaje energético a concentraciones relativamente bajas de 
impurezas de Li en c-Si, usando una técnica de activación-relajación cinética. 
Kim \textit{et al.} ~\cite{kim2017} desarrolló un algoritmo para investigar la 
respuesta a la delitiación de una capa delgada de silicio recubierta de óxido de 
aluminio. El ReaxFF también fue combinado con otros campos de fuerza, como los
potenciales de Tersoff y Lennard-Jones, para simular la litiación de 
nanopartículas de Si recubiertas con carbono, que permitieron observar una 
correlación entre el crecimiento del estrés y la densidad de carga 
~\cite{zheng2019,zheng2020}. Propiedades mecánicas de interfase Si/SiO$_2$ litiada 
fueron reportadas por Verners y Simone ~\cite{verners2019}. 

No es posible llevar a cabo un estudio sobre las propiedades electrónicas con el 
uso del ReaxFF ya que es una de sus limitaciones.  Sin embargo, de la discusión 
previa, puede observarse que ha sido capaz de predecir un número importante de
propiedades del sistema Li-Si.


\section{Configuraciones iniciales}

\subsection{Estructuras cristalinas}

En este capítulo se estudian las propiedades de las estructuras amorfas Li$_x$Si
para distintos valores de $x$ en un rango que va de 0.21 a 4.2. Para algunas de
estas concentraciones se encuentran estructuras cristalinas de LiSi, las cuales 
fueron extraídas del Materials Project ~\cite{materials_project} 
(mp-1314, mp-672287, mp-569849, mp-29720) y sus posiciones utilizadas para definir
los estados iniciales. Las celdas primitivas de las estructuras cristalinas se
muestran en la figura \ref{fig:cristalinas}, donde están en orden creciente de 
concentración de Li, las mismas son Si, LiSi, Li$_{12}$Si$_7$, Li$_7$Si$_3$, 
Li$_{13}$Si$_4$, Li$_{15}$Si$_4$, Li$_{21}$Si$_5$ y Li. En la estructura de LiSi
se tiene una remanencia del diamante de Si en los enlaces, en la Li$_{12}$Si$_7$ 
hay dos tipos de clusters de átomos de Si, pentagonos y estrellas, en la
Li$_7$Si$_3$ los átomos de Si se encuentran en mancuernas, en la Li$_{13}$Si$_4$ 
se tienen las mismas mancuernas junto a algunos átomos aislados y, por último, 
todos los átomos de Si se encuentran aislados tanto en la Li$_{15}$Si$_{4}$ como
en la Li$_{21}$Si$_5$. Estas estructuras cristalinas fueron observadas a 
temperaturas altas ~\cite{wen1981}, pero no se encuentran en el funcionamiento de
una batería ~\cite{obrovac2004}. Sin embargo, sus posiciones pueden tomarse como 
iniciales para simular estructuras amorfas a las concentraciones correspondientes.
\begin{figure}[t]
    \centering
    \includegraphics[width=\textwidth]{caracterizacion/cristalinas.png}
    \caption{Estructuras cristalinas de LiSi. Las estructuras no están a escala 
    entre sí. Los átomos de Si se muestran en azul y los de Li en verde, mientras
    que la celda periódica en gris. Los enlaces de Si-Si están graficados si la 
    distancia entre dos de estos átomos es menor a 2.5\AA.}
    \label{fig:cristalinas}
\end{figure}

\subsection{Protocolo de delitiación}

Para obtener configuraciones iniciales para valores de $x$ distintos a los de las 
cristalinas se siguió un protocolo de delitiación en el cual se selecciona la 
estructura cristalina más cercana con un valor de $x$ superior al deseado,
se le extrae un átomo de Li de manera aleatoria y se realiza una dinámica en el 
ensamble NPT durante 2 ps para relajar el volumen. Para estas simulaciones se 
utilizó el termostato de Nosé-Hoover ~\cite{nose1984a, nose1984b, hoover1985} a
300.0 K, un barostato a 0.0 atm y un paso temporal de 1 fs utilizando el
software \path{LAMMPS} ~\cite{lammps1, lammps2}. La extracción del átomo de Li y
la simulación en el ensamble NPT fueron repetidas hasta alcanzar una concentración
deseada. Por último, para algunas concentraciones en particular, se seleccionó la
estructura con la menor presión absoluta como estado inicial para la exploración 
acelerada de mínimos locales que se introduce en la siguiente sección.


\section{Exploración acelerada de mínimos locales}

Las simulaciones de MD tienen un gran poder predictivo para el estudio de 
procesos presentes en las baterías de litio, sin embargo, las escalas de tiempo
están limitadas de unos pocos ns o $\mu$s. El número de operaciones que se 
necesita para alcanzar las escalas de tiempo de la operación de una batería 
experimental son prohibitivos, incluso considerando el uso de potenciales 
semi-empíricos como el ReaxFF en supercomputadoras. Como consecuencia de esto,
la MD usual no es suficiente para una exploración del espacio de las fases y las
estructuras de Li-Si observadas van a estar cercanas a las configuraciones 
iniciales mientras que en el sistema real probablemente pueden aparecer otras
configuraciones. Un método simple y poderoso para acelerar la exploración de 
mínimos locales en sistemas moleculares es el templado simulado 
~\cite{kirkpatrick1983}, en el cual básicamente se busca mejorar la exploración
del espacio de las fases en simulaciones de MD utilizando temperaturas altas y
luego reduciéndola progresivamente hasta encontrar un mínimo de energía a 
temperatura ambiente. El templado simulado múltiple (MSA, de sus siglas en inglés 
\textit{Multiple simluated annealing}) utiliza esta idea en distintas copias del 
sistema y fue utilizado para explorar y encontrar distintas estructuras mínimas 
relevantes cercanas al equilibrio ~\cite{hao2015}.

La presente técnica de simulación, exploración acelerada de mínimos locales (AELM,
de sus siglas en inglés \textit{accelerated exploration of local minima}), es 
similar a la MSA pero en vez de calentar y enfriar lentamente el sistema, se 
utiliza un sesgo en la función de energía potencial para sobrepasar las barrearas
de energía y luego se realiza una minimización local, con algún minimizador local 
como gradientes conjugados o LBFGS, para encontrar el mínimo. Este método permite 
obtener muchas estructuras con energías mínimas relevantes, que son de interés a 
la hora de estudiar electrodos de Li-Si muy ciclados.

Las aleaciones de Li-Si presentan interacciones fuertes entre los átomos que las
conforman, especialmente en el enlace Si-Si donde la energía de enlace es del
orden de $\approx$2 eV ~\cite{wypych2018handbook}. Las barreras de 
energía potencial se espera que sean de ese orden de magnitud, por lo cual un 
muestreo de una MD a temperatura ambiente parece no tener solución. Para explorar 
ampliamente las distintas configuraciones del sistema, $\mathbf{r}$, se 
transforma la superficie de energía potencial (PES, de sus siglas en inglés 
\textit{potencial energy surface}), $V(\mathbf{r})$, usando un potencial sesgado
\begin{equation}\label{eq:bias}
    V_b(\mathbf{r}) = V(\mathbf{r}) + (\alpha - 1) V(\mathbf{r}) = \alpha V(\mathbf{r}),
\end{equation}
donde $\alpha$ es el factor de compresión. El principal efecto de la ecuación 
\ref{eq:bias} es reducir las barreras de la PES, por lo cual el tiempo de
residencia en estados meta-estables es menor que en el sistema sin sesgar y la 
exploración de configuraciones de sistemas diferentes es más eficiente y 
alcanzada en un tiempo de simulación razonable. El término 
$(\alpha - 1) V(\mathbf{r})$ es usualmente referido como la 
\say{función de sesgo}.

La adición de esta función de sesgo a la PES está en la base del método de 
hiper-dinámica (HD), desarrollado por Voter ~\cite{voter1997HD,voter1997method} 
para acelerar la exploración de un sistema sin perder su dinámica. En una 
simulación típica de HD, para recuperar el promedio de alguna propiedad, la 
configuración muestreada son repesadas por un factor $w$ que involucra una función
exponencial y depende del sesgo aplicado. Debido a que este sistema involucra 
cambios grandes en las energías de interacción, comparado con la energía térmica
$k_BT$, lo que implica que la función exponencial en $w$ toma valores muy grandes,
esto provoca que el procedimiento numérico sea inestable y la recuperación de 
la propiedad de interés, como la energía potencial, no sea posible. Ya que este
capítulo se centra en un estudio estructural de los sistemas, podemos descartar
el cálculo del tiempo real evolucionado en la simulación. Además, como el 
funcionamiento de las baterías luego se da a temperatura ambiente, es de esperar
que una vez que se alcanza un mínimo local el sistema explore configuraciones
cercanas a este. Por lo cual se aplica el método de gradientes conjugados (CG)
a cada una de las configuraciones de la HD y de esa forma se muestrea la 
multiplicidad de estructuras.

Este método de exploración introducido en esta tesis se asemeja al templado
simulado, aunque el objetivo es explorar muchas estructuras diferentes en vez de 
encontrar el mínimo global. El templado simulado fue utilizado anteriormente
con este mismo objetivo, Hao \textit{et al.} utilizó la técnica MSA para obtener
distintas estructuras de mínima energía de péptidos ~\cite{hao2015}. En este 
método AELM se usa HD en vez de temperaturas altas para favorecer la exploración,
y se realizan múltiples minimizaciones por CG en vez de simular un enfriado. 


\section{Resultados}

Las simulaciones aceleradas fueron llevadas a cabo en el ensamble NVT a 300.0 K
con un termostato de Langevin ~\cite{schneider1978} utilizando una versión 
modificada de \path{GEMS} ~\cite{gems}. A cada estructura se le realizó una 
minimización de gradientes conjugados con el software \path{LAMMPS} 
~\cite{lammps1, lammps2}. El tamaño de los sistemas y la cantidad de estructuras 
utilizadas para obtener los siguiente resultados se presentan en la tabla 
\ref{t:siminfo}.
\begin{table}[h]
    \centering
    \caption{Información del conjunto de datos.}
    \begin{tabular}{cccccccc}
    \hline
    $x$ en Li$_{x}$Si & N$_{Li}$ & N$_{Si}$ & N$_{estructuras}$ & E$_{mean}$ / N$_T$ [eV] & E$_{std}$ / N$_T$ [eV] & $\sqrt{kT}$ / N$_T$ [eV] \\
    \hline
    0.21 & 140 & 667 & 774 & -4.399 & 0.003 & 0.0002 \\
    0.62 & 416 & 670 & 1665 & -4.002 & 0.005 & 0.0001 \\
    1.25 & 839 & 671 & 1224 & -3.521 & 0.004 & 0.0001 \\
    1.71 & 1152 & 672 & 2132 & -3.286 & 0.002 & 0.0001 \\
    2.17 & 693 & 319 & 1699 & -3.126 & 0.002 & 0.0002 \\
    2.71 & 865 & 319 & 1504 & -2.964 & 0.002 & 0.0001 \\
    3.25 & 1040 & 320 & 1464 & -2.856 & 0.003 & 0.0001 \\
    3.75 & 1080 & 288 & 2660 & -2.777 & 0.002 & 0.0001 \\
    4.20 & 1344 & 320 & 1600 & -2.717 & 0.001 & 0.0001 \\
    \hline
    \end{tabular}
    \label{t:siminfo}
\end{table}

\begin{figure}[th]
    \centering
    \includegraphics[width=\textwidth]{caracterizacion/energias.png}
    \caption{Histogramas correspondientes a la energía potencial de las 
    estructuras obtenidas con el método AELM, con distintos valores de $\alpha$
    en la ecuación \ref{eq:bias}, para cada composición de Li$_x$Si estudiada.}
    \label{fig:energias}
\end{figure}
La figura \ref{fig:energias} muestra los histogramas para las energías mínimas
de las estructuras de Li$_x$Si obtenidas con el método AELM para los valores de
$x$ estudiados y distintos valores de compresión $\alpha$. En cada fila hay un 
histograma de energías representativo para las estructuras de concentración 
cercana, por lo cual en el análisis sólo son nombradas algunas de estas 
concentraciones. Para el primer caso, donde $x = 0.21$, se puede apreciar como el 
uso de valores más pequeños de $\alpha$ permite que estructuras con menor energía 
sean encontradas. El principal efecto de este factor $\alpha$ sobre la PES es la 
disminución de sus barreras de energía, mejorando la exploración del espacio de 
las fases. Este efecto se vuelve más drástico a medida que el valor de $\alpha$ 
tiende a cero. Para estas concentraciones representativas también se realizaron 
simulaciones de MD usuales, es decir, con un valor de $\alpha = 1$, estas no 
pueden sobrepasar las barreras de energías durante el tiempo simulado. El sistema 
permanece cercano al mínimo local asociado a la configuración inicial. Por otro 
lado, el uso de $\alpha = 0.2$ en el método de AELM resulta en un acceso rápido
a estructuras de energías menores. Un comportamiento similar se observa para 
$x = 2.17$, sin embargo, en este caso el valor más pequeño de $\alpha$ tiende a 
encontrar energías más altas que los otros casos, dando lugar a una distribución
similar a la de MD ordinaria pero con mayores fluctuaciones. Esto probablemente 
se deba a una exploración demasiado extensa, donde el sistema difunde a través
de una gran región del espacio de las fases y las minimizaciones múltiples de 
CG no son capaces de encontrar mínimos de menor energía. Por último, para la 
concentración más alta, correspondiente a un valor de $x = 4.2$, las simulaciones 
de AELM con un valor de $\alpha = 0.2$ son capaces de encontrar estructuras que 
reducen fuertemente la energía potencial del sistema. 

\begin{figure}[t]
    \centering
    \includegraphics[width=\textwidth]{caracterizacion/amorfas.png}
    \caption{Configuración amorfa representativa de cada valor de $x$. Los átomos
    de Si se muestran en azul mientras que los de Li en verde.}
    \label{fig:amorfas}
\end{figure}
De los histogramas de la figura \ref{fig:energias} se seleccionan los factores 
de aceleración más óptimos, es decir que producen energías menores, para obtener
las propiedades estructurales que se discuten a continuación. Para estos valores
se muestra una estructura representativa a cada composición en la figura 
\ref{fig:amorfas}. Para $x = 0.21$ puede verse que la red amorfa de silicio
permanece con su estructura tetraédrica desordenada. Algunos enlaces Si-Si 
comienzan a romperse a medida que la concentración de litio aumenta, como puede
verse para las estructuras cercanas a $x = 2.17$. Por último, para las 
concentraciones más altas de litio, se alcanzan estructuras que involucran 
cadenas unidimensionales periódicas de silicio. Una estructura similar ha sido 
reportada por Ostadhossein \textit{et al.} ~\cite{ostadhossein2015}. En las 
próximas secciones se caracterizan dichas estructuras.

\subsection{Comportamiento electroquímico}

\subsubsection{Cambio de volumen fraccionario}

El cambio de volumen fraccionario puede definirse utilizando una normalización 
relativa al número de átomos de Si en la estructura de acuerdo a
\begin{equation}\label{eq:fvc}
    fvc = \frac{N_{Si}}{V_{Si}} \left( \frac{V_{Si,x}}{N_{Si,x}} - \frac{V_{Si}}{N_{Si}} \right),
\end{equation}
donde $V_{Si}$ y $N_{Si}$ son el volumen y el número de átomos de Si en la celda
unidad de c-Si, $V_{Si,x}$ y $N_{Si,x}$ son el volumen y el número de átomos de Si
en la celda de simulación para el valor correspondiente de $x$. En la figura
\ref{fig:fvc} se muestran los valores calculados a partir de la ecuación 
\ref{eq:fvc} para las distintas estructuras de Li$_x$Si estudiadas. En la misma 
se comparan los valores obtenidos con datos experimentales de AFM (\textit{atomic 
force microscopy}, sus siglas en inglés) medidos por Beaulieu \textit{et al.} 
~\cite{beaulieu2003} y con predicciones de DFT con un cambio volumétrico fijo 
utilizado por Chevrier y Dahn ~\cite{chevrier2009}. Los mismos muestran que el
potencial ReaxFF proporciona una tendencia correcta tanto cualitativa como 
cuantitativamente.
\begin{figure}[th]
    \centering
    \includegraphics[width=0.8\textwidth]{caracterizacion/fvc.png}
    \caption{Cambio de volumen fraccionario en función de la composición de la 
    aleación. Los valores experimentales de AFM se muestran con cuadrados azules, 
    la línea recta se corresponde con cálculos de DFT y los círculos rojos son 
    resultados de este trabajo.}
    \label{fig:fvc}
\end{figure}

\subsubsection{Voltaje}

\begin{table}[h]
    \centering
    \caption{Energías de formación obtenidas a través de la ecuación \ref{eq:fe}}
    \begin{tabular}{ccc}
    \hline
    x en Li$_x$Si & Energía de formación [eV] & Desviación estándar [eV]\\
    \hline
    0.21  &  0.5027  &  0.0037 \\
    0.62  &  0.1206  &  0.0074 \\
    1.25  & -0.1160  &  0.0096 \\
    1.71  & -0.2358  &  0.0065 \\
    2.17  & -0.3551  &  0.0075 \\
    2.71  & -0.4098  &  0.0072 \\
    3.25  & -0.5187  &  0.0126 \\
    3.75  & -0.6202  &  0.0097 \\
    4.20  & -0.6995  &  0.0075 \\
    \hline
    \end{tabular}
    \label{t:fe}
\end{table}
Las energías obtenidas pueden ser utilizadas para testear el funcionamiento del 
modelo para predecir propiedades electroquímicas, como fue sugerido por Chevrier
y Dahn ~\cite{chevrier2009}. Primero, se define la energía de formación de las 
distintas estructuras amorfas como
\begin{equation}\label{eq:fe}
    E_f(x) = E_{Li_xSi} - (x E_{Li} + E_{Si}),
\end{equation}
donde $E_{Li_xSi}$ es la energía de la aleación Li$_x$Si por átomo de Si, E$_{Li}$
y E$_{Si}$ son las energías cohesivas de Li y Si en sus fases cristalinas. Usando
la ecuación \ref{eq:fe} como aproximación a la energía de formación de Gibbs, el 
potencial \textit{versus} Li metálico de Li$_x$Si puede obtenerse a partir de
\begin{equation}\label{eq:voltaje}
    V(x) = - \frac{dE_f(x)}{dx},
\end{equation}
donde $V$ es el potencial. Los datos obtenidos así pueden compararse con valores
experimentales y computacionales previos. Las energías de formación calculadas
a partir de la ecuación \ref{eq:fe} se muestran en la tabla \ref{t:fe}. 
Realizándoles un \textit{spline} a estos valores, mostrados en el recuadro de la
figura \ref{fig:voltaje}, los valores de $V(x)$ son obtenidos de la ecuación
\ref{eq:voltaje} y son graficados en función de la composición en la figura 
\ref{fig:voltaje} con una línea roja. Para comparar, se incluye en la misma figura
las curvas experimentales medidas para la litiación y la delitiación de silicio
amorfo ~\cite{hatchard2004} y la curva teórica de cálculos de primeros principios 
~\cite{chevrier2009}. Se puede afirmar que los resultados obtenidos con el ReaxFF 
son bastante satisfactorios.
\begin{figure}[th]
    \centering
    \includegraphics[width=0.8\textwidth]{caracterizacion/voltaje.png}
    \caption{Curvas potencial-concentración de la litiación de ánodos de Si.
    La línea negra se corresponde con cálculos de DFT, las líneas azules con 
    curvas medidas experimentalmente en la litiación de Si amorfo y la línea 
    roja es la derivada del \textit{spline} ajustado a los datos de la energía 
    de formación obtenidos con el ReaxFF, presentados en el recuadro.}
    \label{fig:voltaje}
\end{figure}

\subsection{Distribución radial de a pares}

La distribución radial de a pares, introducida en la sección \ref{s:observables},
puede ser utilizada para describir la estructura de materiales amorfos. Para el 
caso de sistemas que están conformados por más de un elemento se pueden analizar 
las distribuciones radiales de a pares parciales ~\cite{lamparter1995}, donde la 
$g_{ij}(r)$ representa la RDF de los átomos $j$ a una distancia $r$ alrededor de 
los átomos centrales $i$, y que es lo mismo que considerar $g_{ji}(r)$. Las 
figuras \ref{fig:rdf-LiLi}, \ref{fig:rdf-SiSi}, \ref{fig:rdf-SiLi} muestran los
resultados obtenidos para las RDF de Li-Li, Si-Si y Si-Li, respectivamente. En
cada una de ellas se analizan los cambios en la estructura que se dan para los
distintos valores de $x$ en Li$_x$Si estudiados, las curvas se calculan sobre los
\textit{frames} minimizados de la HD.

\begin{figure}[h!]
    \centering
    \includegraphics[width=0.8\textwidth]{caracterizacion/rdf-LiLi.png}
    \caption{Distribución radial de a pares para Li-Li de las estructuras 
    minimizadas. Cada curva se corresponde con un valor de concentración 
    distinto.}
    \label{fig:rdf-LiLi}
\end{figure}
Lo más relevante a destacar de la RDF$_{Li-Li}$ es que su primer pico comienza 
centrado en 2.45 \AA\ para las concentraciones de iones de Li más bajas y que 
luego dicho pico aumenta su posición a distancias más grandes a medida que aumenta 
$x$ hasta permanecer en 2.95 \AA\ para $x$ mayores a 1.71. La altura de este pico
aumenta en un 50\% luego de la litiación completa, relativa a la menor 
concentración.

\begin{figure}[h!]
    \centering
    \includegraphics[width=0.8\textwidth]{caracterizacion/rdf-SiSi.png}
    \caption{Distribución radial de a pares para Si-Si de las estructuras 
    minimizadas. Cada curva se corresponde con un valor de concentración 
    distinto.}
    \label{fig:rdf-SiSi}
\end{figure}
Este mismo efecto se ve en el primer pico de la RDF$_{Si-Si}$, el centro del mismo
se encuentra en 2.4 \AA\ para $x = 0.21$ y luego se desplaza a distancias
mayores, después de $x = 1.25$ el centro se encuentra entre 2.52 y 2.56 \AA.
Mientras que la altura del pico aumenta se ve un decrecimiento en en el ancho 
del pico, el valor del FWHM va de 0.14 \AA\ a 0.05 \AA\ para $x = 0.21$ y 
$x = 3.75$, respectivamente. Por otro lado, el segundo pico de la RDF$_{Si-Si}$
también se desplaza hacia distancias mayores, se divide en dos picos para valores 
de $x$ entre 0.62 y 1.71 y vuelve a comportarse como un sólo pico para para 
concentraciones mayores. Entre el primer y el segundo pico se observa un hombro,
como señaló previamente Fan \textit{et al.} ~\cite{fan2013}. Los resultados 
obtenidos para la RDF$_{Si-Si}$ están en concordancia con las mediciones 
experimentales reportadas por Key \textit{et al.} ~\cite{key2011}.

\begin{figure}[h!]
    \centering
    \includegraphics[width=0.8\textwidth]{caracterizacion/rdf-SiLi.png}
    \caption{Distribución radial de a pares para Si-Li de las estructuras 
    minimizadas. Cada curva se corresponde con un valor de concentración 
    distinto.}
    \label{fig:rdf-SiLi}
\end{figure}
Para el primer pico de la RDF$_{Si-Li}$ se ve el comportamiento contrario, el 
centro del mismo se desplaza a distancias menores a medida que la concentración
de litio aumenta. Esto es acompañado con un aumento de la altura del pico y una
disminución de su ancho. Para el segundo pico también se observa un desplazamiento
del mismo hacia distancias menores, pero por encima de $x = 1.71$ el pico se
divide en dos picos con distintas alturas dependiendo de la concentración. Esto
es analizado con mayor detalle en la sección \ref{s:intercionexion}.

\subsection{Número de coordinación}

De la misma manera que se definió la distribuciones radiales de a pares parciales,
se pueden obtener los números de coordinación a un dado tipo de átomos. CN$_{ij}$
se corresponde con la cantidad de átomos vecinos de tipo $j$ para un átomo central 
de tipo $i$ hasta una cierta distancia de corte. Para la elección de dicho valor 
se considera hasta el primer pico de la $g_{ij}(r)$ correspondiente. Debido a que 
en los materiales amorfos la primera y la segunda esfera de coordinación pueden 
llegar a estar superpuestas, el límite superior de integración no está definido 
unívocamente para todas las concentraciones consideradas ~\cite{lamparter1995}.
Para el número de coordinación promedio para átomos de Si vecinos de otros átomos 
de Si se calculó contando el número de dichos vecinos utilizando un radio de 
corte de 3 \AA. Lo mismo se realizó para Li-Li definiendo un radio de corte de 
4 \AA. Para el caso de Si-Li se utilizó el criterio de considerar como radio de 
corte el valor $r$ para el cual la $g(r)$ presenta un mínimo entre los dos picos
a primeros y segundo vecinos. Los resultados se muestran en la figura 
\ref{fig:cn1}.
\begin{figure}[th]
    \centering
    \includegraphics[width=0.8\textwidth]{caracterizacion/cn.png}
    \caption{Promedio del primer número de coordinación en función de la 
    concentración de litio para Li-Li, Si-Si y Si-Li. Como radio de corte se 
    utilizó la distancia posterior al primer pico de la RDF correspondiente. En 
    los casos que no se aprecia la barra de error, es porque la misma es menor 
    que el tamaño de los puntos.}
    \label{fig:cn1}
\end{figure}

Para el caso del CN$_{Si-Si}$, se tiene que esta cantidad decrece de 4 a 2, a 
medida que la concentración de Li aumenta. Esto indica que a valores pequeños de 
$x$ la estructura de Si mantiene sus conexiones tetraédricas, mientras que para
valores grandes de $x$ el Si tiende a formar cadenas periódicas unidimensionales.
En la red de silicio amorfa, analizada con más detalle en la sección 
\ref{s:clusters}, una estructura 3-periódica se presenta para valores bajos de 
$x$, donde el CN se encuentra alrededor de 4. Luego, una estructura 1-periódica se
alcanza para valores grandes de $x$, donde los enlaces Si-Si tienden a formar 
cadenas, que pueden verse para $x = 3.75$ donde se tiene $CN = 2.05$, por ejemplo.
El CN de Si-Li y Li-Li presenta valores pequeños para concentraciones 
bajas y aumenta monótonamente hasta alcanzar valores de 10 y 12, respectivamente, 
que se asemejan al valor de una estructura de empaquetamiento compacto.

\begin{figure}[th]
    \centering
    \includegraphics[width=0.8\textwidth]{caracterizacion/cn2.png}
    \caption{Promedio del segundo número de coordinación en función de la 
    concentración de litio para Li-Li, Si-Si y Si-Li. Para la elección de los 
    radios de corte que definen el cascarón en el cual se cuentan los vecinos,
    se consideró el segundo pico de la RDF correspondiente. En los casos que no 
    se aprecia la barra de error, es porque la misma es menor que el tamaño de 
    los puntos.}
    \label{fig:cn2}
\end{figure}
Los resultados para el segundo número de coordinación se presentan en la figura 
\ref{fig:cn2}. Estos resultados se obtuvieron considerando un cascarón con un 
radio de corte interno y otro externo, elegidos de manera tal que incluyan el 
segundo pico de la RDF. La elección de dichos valores varió dependiendo del tipo
de átomos que se consideraron. En todos ellos se tomó como radio de corte interno 
el radio de corte del primero número de coordinación. Luego, para el radio de 
corte externo se utilizaron valores de 5.0 \AA\ para Si-Si y 6.0 \AA\ para Li-Li
y Si-Li.

Para los valores de CN$_{Si-Si}$ se observa un aumento para concentraciones bajas
de Li si se lo compara con el CN de primeros vecinos. Para valores mayores de $x$,
se puede ver como el valor de CN también tiende a 2, lo cual es coherente con la
formación de cadenas que se notó previamente. La tendencia cualitativa del segundo
CN para Li-Li y Si-Li es la misma a la observada en el primer CN, sólo que ahora
empieza en un valor cercano a 5 y tiende a 35 y 40, respectivamente. Este valor 
es mucho mayor que el que se tiene para los segundos vecinos en una estructura 
de empaquetamiento compacto, que es 6 para la estructura cristalina FCC. Incluso 
es mayor a la suma del segundo (6) y del tercer vecino (24) esperado para la red 
FCC.

\subsection{Formación de clusters}\label{s:clusters}

Analizando la formación de clusters por medio del algoritmo DBSCAN 
~\cite{ester1996}, en el cual puede definirse un radio de corte para el cual se 
deja de considerar que los átomos están enlazados entre sí (es decir, formando 
clusters), se encuentra que las estructuras amorfas de silicio no pueden ser 
clasificadas en diferentes tipos de clusters, las mismas reflejan más bien 
una red amorfa. Esto viene de interpretar los gráficos que se presentan en la 
figura \ref{fig:clusters}. 

En particular, en la figura \ref{fig:clusters-isolated} se define el porcentaje 
de átomos de Si que están a una distancia mayor que $r_{cut}$ de otros átomos de 
Si, cuando el radio de corte es mayor que la distancia a la cual termina el 
primer pico de la RDF$_{Si-Si}$, no se tienen átomos de Si que cumplan esta 
propiedad, es decir, no hay átomos de Si que se encuentren aislados en el sistema, 
incluso a concentraciones altas de Li. Esto refleja que el a-Si se comporta como 
una red en la cual todos los átomos de silicio están interconectados entre sí, 
que también se puede deducir de la figura \ref{fig:clusters-nclusters} en la cual 
se tiene que cuando el radio de corte es menor al primer pico de la RDF$_{Si-Si}$ 
el número de clusters es igual al número de átomos de Si, pero que cuando este 
radio es más grande que la distancia a la cual termina el primer pico, hay un 
solo cluster.

\begin{figure}[h]
    \centering
    \begin{subfigure}{.475\textwidth}
        \centering
        \includegraphics[width=\textwidth]{caracterizacion/cluster-isolated.png}
        \caption{Porcentaje de átomos de Si aislados en función de la elección del
        radio de corte.}
        \label{fig:clusters-isolated}
    \end{subfigure}
    \hfill
    \begin{subfigure}{.475\textwidth}
        \centering
        \includegraphics[width=\textwidth]{caracterizacion/cluster-nclusters.png}
        \caption{Número de clusters de Si sobre el número total de átomos de Si.}
        \label{fig:clusters-nclusters}
    \end{subfigure}
    \caption{Formación de clusters indicando una red amorfa de silicio.}
    \label{fig:clusters}
\end{figure}

\subsection{Interconexión de clusters}\label{s:intercionexion}

Para determinar que es lo que causa la división en el segundo pico de la 
RDF$_{Si-Li}$ se realizó un análisis similar al reportado por Ding \textit{et al.}
~\cite{ding2015}. Estos autores analizaron la correlación en la distancia de a
pares de los segundos vecinos más cercanos en términos de las conexiones entre
clusters, definiendo un poliedro de coordinación alrededor del átomo central 
considerado para la RDF y sus segundos vecinos. El número de átomos compartidos
entre estos dos poliedros de coordinación enlazados fueron utilizados para 
establecer categorías y analizar sus contribuciones a la RDF. Estas categorías
dependen del hecho de que los poliedros comparten un vértice (1 átomo), una 
arista (2 átomos), una cara de los poliedros (3 átomos) o cuadriláteros 
distorsionados o tetraedros aplastados (4 átomos). De una forma similar a este 
trabajo, se deconvolucionó el segundo pico de la RDF calculando la RDF parcial 
de distintas categorías, donde cada categoría se define por el número de átomos de
Li que interconectan un átomo de Si con su segundo vecino de Li. El comportamiento
detallado se presenta en la figura \ref{fig:interconexiones}. Puede afirmarse a 
grandes rasgos que para concentraciones bajas de Li en las aleaciones, hay una 
predominancia de segundos vecinos de Li que tienen una o ninguna interconexión 
con los vecinos de Li de la primera esfera de coordinación Si-Li. Para $x > 1.0$
la contribución del segundo vecino de Li interconectado con dos o más átomos de 
Li de la primera esfera de coordinación Si-Li comienza a ser predominante y la
contribución de los átomos de Li sin conectarse empieza a decaer. Para $x > 3.0$,
la contribución del primer pico del segundo vecino de Li interconectado dos o
tres veces se vuelve relevante mientras que aparecen contribuciones de cuatro o
más interconexiones.
\begin{figure}[th]
    \centering
    \includegraphics[width=\textwidth]{caracterizacion/interconexiones.png}
    \caption{Interconexiones de los segundos vecinos más cercanos de Li con un 
    átomo central de Si para cada valor de $x$ en Li$_x$Si considerado. El número 
    de primeros vecinos más cercanos que conectan a los segundos vecinos más 
    cercanos con el átomo central de Si se indica en el recuadro de las figuras. 
    Además de la RDF$_{Si-Li}$ total, se grafica cada una de las contribuciones 
    de los diferentes tipos de interconexiones posibles.}
    \label{fig:interconexiones}
\end{figure}

Mientras que el comportamiento presentado en la figura \ref{fig:interconexiones}
es más bien complejo, pueden establecerse tendencias generales que ayudan a 
entender mejor que es lo que sucede. Si se divide la RDF$_{Si-Li}$ en dos 
contribuciones de segundos vecinos, la primera de ellas, que se encuentra a una
distancia entre 4.0 \AA\ y 5.0 \AA, se puede atribuir a los átomos que tienen dos 
o más interconexiones de Li, mientras que la segunda de ellas, entre 5.0 \AA\ y
5.6 \AA, se corresponde con los átomos que tiene una o ninguna interconexión de 
Li. Utilizando esta clasificación, se muestra en la figura 
\ref{fig:interconexiones-areas} la fracción del área que representa cada una de
estas categorías en función de la concentración de litio.
\begin{figure}[th]
    \centering
    \includegraphics[width=0.8\textwidth]{caracterizacion/interconexiones-areas.png}
    \caption{Evolución con la concentración de la fracción que representa cada
    categoría de interconexiones de Li al área total del segundo pico de la 
    RDF$_{Si-Li}$.}
    \label{fig:interconexiones-areas}
\end{figure}

\subsection{Orden de corto alcance}

El término orden de corto alcance (SRO, de sus siglas en inglés 
\textit{short-range order}) se utiliza para denotar el ordenamiento de los átomos
que rodean a uno específico en una cáscara. Del mismo modo, el término 
\textit{clustering} se ha se ha definido como la tendencia de los átomos 
similares a estar cerca unos de otros. Ambos conceptos se refieren a un orden 
estructural entre átomos vecinos, pero no son necesariamente persistentes a 
distancias más largas. Warren ~\cite{warren69} y Cowley ~\cite{cowley1950} 
definieron un parámetro (WCP) para caracterizar estos tipos de ordenamientos de 
la siguiente manera:
\begin{equation}
    WCP = 1 - \frac{p_{A-B}}{m_B} = 1 - \frac{p_{B-A}}{m_A},
\end{equation}
donde $p_{A-B}(p_{B-A})$ es la probabilidad de tener un átomo de tipo B(A) como
vecino de un átomo de tipo A(B) y $m_B(m_A)$ es la concentración global de átomos
B(A), expresadas en fracciones molares. La igualdad, en ambas definiciones 
posibles del WCP, viene del hecho de que la probabilidad de encontrar a un átomo 
de tipo A como vecino de un átomo de tipo B es igual a la de tener un átomo de 
tipo B como vecino de un átomo de tipo A, esto es $m_A p_{A-B} = m_B p_{B-A}$.

Los valores que se obtienen de utilizar el parámetro WCP en sistemas del tipo
A$_x$B indica una aleatoriedad completa si es igual a cero, preferencia por 
átomos de distinto tipo si $WCP < 0$ y preferencia por átomos del mismo tipo si 
$WCP > 0$. Aunque este parámetro permite un análisis cuantitativo notable, sólo
se define para sistemas cristalinos en los que cada átomo tiene el mismo número
de vecinos. ~\cite{warren69}

A continuación se extiende esta idea para definir un nuevo parámetro, 
$\theta_{A-B}$, que es adecuado para caracterizar estructuras amorfas, de la 
siguiente manera:
\begin{equation}
    \theta_{A-B} = \ln \left( \frac{C_{A-B}}{C_{Bulk}} \right),
\end{equation}
donde A indica la naturaleza del átomo que se considera como central y B el tipo
de átomo que se considera como vecino, equivalente a la definición de WCP. En
este caso, la relación entre la concentración local y la concentración global se 
calcula a partir de la integración de la distribución radial de a pares parcial,
$g_{A-B}(r)$, en una esfera al rededor del átomo central,
\begin{equation}
    \frac{C_{A-B}}{C_{Bulk}} = \frac{1}{V(r_{cut})} \int_0^{r_{cut}} g_{A-B}(r) dV,
\end{equation}
donde $r_{cut}$ y $V(r_{cut})$ son el radio de corte y el volumen de la esfera 
considerada. Ya que en $g_{A-B}(r)$ no hay dependencia angular, $dV$ puede 
escribirse como $4 \pi r^2 dr$. Esta cantidad puede pensarse como la 
concentración promedio dentro de la esfera relativa a la del material 
\textit{bulk}. Así, de forma análoga al parámetro de WCP, $\theta_{A-B}$ indica 
la tendencia SRO o el \textit{clustering} para cualquier tipo de átomo dado.
Si $\theta$ es positivo, indica una acumulación de átomos relativa al 
\textit{bulk}, mientras que si es negativo indica una disminución. Si es igual a 
cero se tiene una aleatoriedad completa. Este nuevo parámetro también satisface 
la relación $\theta_{A-B} = \theta_{B-A}$ de la misma manera que se discutió para
el parámetro de WCP, ya que por definición $g_{A-B}(r) = g_{B-A}(r)$. Por lo cual 
se tiene que el parámetro $\theta_{A-B}$ da información similar a la que provee 
el WCP, pero además es aplicable a sistemas amorfos.

La figura \ref{fig:sro} muestra la variación del parámetro $\theta$ en función 
de la concentración de Li. Hay tres posibilidades para el análisis de $\theta$ en
sistemas de Li$_x$Si ($\theta_{Li-Li}$, $\theta_{Si-Si}$ y $\theta_{Si-Li}$), ya
que $\theta_{Si-Li} = \theta_{Li-Si}$. Para todos los casos se consideraron los
mismos radios de corte que en los cálculos del número de coordinación, luego del
primer pico de la RDF correspondiente.
\begin{figure}[th]
    \centering
    \includegraphics[width=0.8\textwidth]{caracterizacion/sro.png}
    \caption{Parámetros $\theta_{Li-Li}$, $\theta_{Si-Li}$ y $\theta_{Si-Si}$ 
    en función de la concentración de Li. El primer subíndice indica el tipo de
    átomo que se considera como central mientras que el segundo es el vecino. El
    radio de corte se eligió luego del primer pico de la RDF correspondiente.}
    \label{fig:sro}
\end{figure}

Como tendencia general, puede notarse que todos los valores de $\theta$ aumentan
cuando crece la cantidad de litio en el sistema, $x$, y que se estabiliza para 
valores grandes de $x$. Este comportamiento monótono y la disminución en la 
pendiente para concentraciones altas está correlacionado con el comportamiento
presentado en el análisis de los números de coordinación.

En el caso de $\theta_{Si-Si}$, alcanza un valor positivo aproximadamente 
constante para $x > 2.5$, mostrando una correlación fuerte con la presencia de 
cadenas lineales de Si, previamente discutidas y observadas en la figura 
\ref{fig:amorfas}. Aunque la presencia de estas cadenas se puede inferir a partir
de los valores de los CN en $x$ altos, $\theta$ es más sensible al SRO, ya que
está normalizado por la concentración global. Esta propiedad de $\theta$ permite
un análisis más claro incluso si las cadenas están interactuando entre sí, como
es el caso para concentraciones bajas de litio.

$\theta_{Si-Li}$ presenta variaciones pequeñas y un valor positivo para todo 
$x > 0.5$, mostrando una acumulación constante de Li al rededor del Si, o, 
análogamente, una acumulación de Si alrededor del Li. Este comportamiento se le 
puede atribuir a la interacción atractiva fuerte entre Si-Li. En el caso de 
$\theta_{Li-Li}$, este parámetro es siempre negativo, lo que indica una 
interacción débil Li-Li y la correspondiente disminución de vecinos Li-Li. Por
último, el parámetro $\theta_{Si-Si}$ tiene un valor negativo para $x < 1.0$, 
sugiriendo que la presencia de concentraciones bajas de litio tiende a separar 
los átomos de silicio entre sí. Sin embargo, $\theta_{Si-Si}$ se vuelve positivo
para $x > 1.0$, implicando una acumulación de vecinos de Si sobre átomos de Si, 
relativo a la concentración global. Esto se debe a la formación de estructuras 
Si-Si. Para $x > 2.5$ puede observarse un valor constante de 
$\theta_{Si-Si} \approx 0.3$, revelando la formación de estructuras estables de 
Si-Si dadas por las cadenas previamente mencionadas.


\section{Conclusiones del capítulo}

Con el fin de emular las estructuras amorfas encontradas en muchos experimentos 
electroquímicos, en este capítulo se generaron estructuras desordenadas de 
aleaciones de Li$_x$Si para varios valores de $x$ utilizando un algoritmo de 
dinámica acelerada y un campo de fuerzas reactivo. La exploración acelerada de 
mínimos locales (AELM) permitió la caracterización de una amplia gama de 
estructuras amorfas. El cambio de volumen de las estructuras litiadas en relación 
con el Si está en concordancia con los resultados experimentales de AFM. Las
energías de las estructuras obtenidas representan bien el comportamiento 
electroquímico de la curva de potencial en función de la concentración de Li. Se 
analizó la función de distribución radial de a pares para los diferentes tipos de 
pares atómicos y se dilucidó la estructura compleja del segundo pico del RDF 
Si-Li mediante un análisis de interconexión de clusters. Además, haciendo un 
análisis de la formación de clústeres en función del radio de corte, se demostró 
que las estructuras amorfas no presentan diferentes enlaces de Si ni átomos de Si 
aislados. En su lugar, se encontró que el sistema se comporta como una red amorfa. 
Estudiando los números de coordinación de primeros y segundos vecinos para las 
diferentes concentraciones, se mostró que esta red amorfa mantiene las conexiones 
tetraédricas para bajas concentraciones de Li y que tiende a formar cadenas para 
altas concentraciones de Li. Por último, la definición de un nuevo parámetro 
permitió determinar el orden de corto alcance de las estructuras amorfas, definido 
por interacciones débiles Li-Li y fuertes Li-Si y Si-Si. El método propuesto AELM 
resulta ser un método rápido y eficaz para obtener mínimos energéticamente 
relevantes. Se hizo una analogía con el templado simulado múltiple. Un análisis 
detallado de la eficiencia de AELM en comparación con otros métodos eficientes 
como el templado simulado múltiple o los métodos de Monte Carlo es una motivación 
para trabajos futuros.

% \chapter{Modelo funcional de densidad de enlace estrecho}\label{ch:modelo}
\thispagestyle{empty}

\vspace{50pt}

\begin{adjustwidth}{50pt}{50pt}
    La capacidad de predicción de las simulaciones de dinámica molecular está 
    limitada principalmente por la escala temporal y por la fiabilidad del campo
    de fuerzas utilizado. La complejidad de las aleaciones amorfas de Li$_x$Si 
    requiere que ambas cuestiones se aborden simultáneamente. El método del 
    Funcional de Densidad de Enlace Estrecho (DFTB, \textit{Density functional 
    tight-binding}) aparece como una solución de complejidad intermedia a este
    problema, ya que es capaz de describir la naturaleza electrónica de diferentes 
    entornos, dándole una transferibilidad mayor que los campos de fuerzas clásicos, 
    a un costo computacional más bajo comparado a DFT (\textit{Density Functional
    Theory}). En este capítulo se presenta un conjunto de parámetros DFTB adecuados
    para modelar las aleaciones amorfas de Li$_x$Si. Dichos parámetros se 
    construyen haciendo especial hincapié en su transferibilidad para todo el 
    intervalo de carga. Esto se logra al introducir un algoritmo de optimización 
    que pesa las distintas estequiometrías para mejorar la predicción de algún 
    observable, en este caso sus energías de formación. El modelo resultante se 
    muestra robusto para la predicción de estructuras cristalinas y amorfas, 
    presentando una excelente concordancia con cálculos DFT y superando el 
    potencial ReaxFF del estado-del-arte para este sistema.
\end{adjustwidth}

\clearpage
\newpage
\thispagestyle{empty}
\mbox{}
\newpage

\chapter{Introducción}\label{ch:introduccion}
\thispagestyle{empty}

\vspace{50pt}

\begin{adjustwidth}{50pt}{50pt}
    TODO
\end{adjustwidth}

\clearpage
\newpage
\thispagestyle{empty}
\mbox{}
\newpage

% Copyright (c) 2024, Francisco Fernandez
% License: CC BY-SA 4.0
%   https://github.com/fernandezfran/thesis/blob/main/LICENSE
\section{Contextualización}

El calentamiento global aparece como el mayor problema ambiental de este siglo.
El mismo se refiere al aumento de la temperatura media de la atmósfera y por 
ende a sus consecuencias en  el clima. Esto es debido al efecto que producen 
las actividades humanas, 
como por ejemplo la quema de combustibles fósiles, que emite 
a la atmósfera grandes cantidades de CO$_2$, entre otros gases de efecto 
invernadero, o la deforestación. Estos gases absorben la radiación infrarroja emitida por la tierra y la reemiten, 
provocando un incremento de la temperatura de la misma que lleva asociado un 
aumento en la frecuencia y la intensidad de eventos climáticos extremos. %\cite{houghton2005}. 
De acuerdo a el Panel Intergubernamental del Cambio Climático 
\cite{IPCC}, desde la época preindustrial, las actividades humanas han provocado 
aproximadamente 1.0$^{\circ}$C de calentamiento global y al ritmo actual se van 
a sobrepasar los 1.5$^{\circ}$C antes del 2050, un cambio en la temperatura
media que las medidas previas al aumento de las actividades mencionadas no habían llegado a alcanzar. Limitar el 
calentamiento a esta temperatura requiere que se realicen rápidamente cambios 
sin precedentes en la tecnología y en el comportamiento humano. Uno de los 
cambios más importante es el de la matriz energética, en la cual las energías 
renovables deberán suministrar alrededor del 80\% de la energía para 2050, donde 
los vectores energéticos, como las baterías de litio, juegan un rol fundamental 
debido a la intermitencia de estas formas de generación de energía.

El litio es el metal más liviano de la tabla periódica y uno de los elementos más
importantes dentro de los minerales necesarios en la producción de baterías de
litio. En particular, para la Argentina tiene un interés económico, social, 
industrial y tecnológico ya que es uno de los países que integran, junto a 
Bolivia y Chile, el Triangulo de Litio, el cual acumula el 70\% de las reservas 
mundiales de fácil extracción de este mineral. Esto último debido a que esta cantidad de reservas
se encuentran en salares de los que, a grandes rasgos, es más barato
extraer litio de ellos en comparación a las rocas de las cuales se puede extraer 
litio en una míneria usual, como las pegmatitas. A pesar de esto se tienen que
llevar a cabo distintas consideraciones ambientales, sociales y legales del 
proceso de extracción e incentivar el desarrollo de valor agregado a dicha 
extracción \cite{gutierrez2022, petavratzi2022, obaya2021, romero2021, 
heredia2020, fornillo2019}.

En esta tesis se presentan estudios computacionales sobre materiales para el 
desarrollo de electrodos de baterías de ion-litio de próxima generación. Se 
abordan dos perspectivas, una con el objetivo de tener baterías que frente a una 
carga rápida retengan un porcentaje considerable de la capacidad y otra 
utilizando electrodos que permitan almacenar mayor cantidad de energía que los 
actuales.


\section{Energía, transporte y litio}

En la actualidad se utilizan distintas formas para generar energía y pueden 
dividirse en dos grandes tipos, las renovables y las no-renovables. Estas últimas
dominan la producción de energía mundial y están compuestas principalmente por 
combustibles fósiles y centrales nucleares, mientras que las energías renovables
abarcan más variantes como la biomasa, la hidráulica, la eólica y la solar, pero 
aún no son lo suficientemente utilizadas. Una de las particularidades de estas 
fuentes de energías renovables es su producción intermitente mientras que el 
consumo de la misma, independientemente de cómo se genere, es a demanda. Esto 
hace que sea necesario el involucramiento de vectores energéticos que permitan 
almacenar y transportar el excedente de energía que se genera en sus períodos de 
mayor producción para luego ser utilizada en los momentos de mayor demanda.

El sector del transporte terrestre, marítimo y aéreo es responsable de más de un 
tercio de las emisiones de CO$_2$ debido a su dependencia en los combustibles 
fósiles \todo{\cite{IEA}}. Dicho esto, está claro que se debe fomentar opciones de desplazamiento menos intensivas
en carbono y con tecnologías más eficientes, como los vehículos eléctricos (EVs),
\todo{cuyos motores poseen una eficiencia para convertir la energía eléctrica en energía para las ruedas que ronda el 80\%, compárese este valor con las
eficiencias entre el 12\% y el 30\% de los motores a combustión interna para la misma tarea \cite{DOE}}.

En los últimos años se ha producido un crecimiento exponencial en las ventas 
anuales de los EVs, como puede observarse en la Figura \ref{fig:evs}a \cite{EVV}. En la
última década, dichas ventas aumentaron aproximadamente un 500\% y se estima que
para la próxima década las ventas se multipliquen por 10. Estas ventas están 
concentradas en China y en algunos países y estados de Europa y Estados Unidos, 
respectivamente, debido a que en los países en desarrollo y emergentes influye 
negativamente su costo alto de adquisición y una falta de infraestructura para la 
recarga de sus baterías. En particular, durante el 2022 en Noruega el 79.3\% de 
los automóviles patentados fueron eléctricos. En el país que le sigue en la lista,
Suecia, se patentó un 32.1\% de EVs en dicho año \cite{PWC}.
\begin{figure}[h!]
    \centering
    \includegraphics[width=\textwidth]{Introduccion/energia/evs.png}
    \caption{(a) Ventas anuales de vehículos eléctricos en la última década. Se 
    proyecta que para el 2030 las ventas asciendan a las 40 millones de unidades 
    frente a las 3 millones del año 2020 \cite{EVV}. (b) Proyección del costo en 
    dólares de vehículos eléctricos y de combustión interna en países 
    desarrollados \cite{BLOOMBERG}.}
    \label{fig:evs}
\end{figure}

En la Figura \ref{fig:evs}b se muestra la proyección en el costo de los vehículos 
eléctricos y de combustión interna realizada por la empresa financiera Bloomberg 
para los países desarrollados \cite{BLOOMBERG}. Se espera que para el año 2026 
los costos se igualen y que para el 2030 los EVs sean aproximadamente un 15\% más
baratos que los vehículos de combustión interna. Este cambio se debe a la 
disminución en el precio de la producción de baterías, que actualmente representa
aproximadamente el 40\% del costo del EV.

El sector energético en Argentina depende altamente de la utilización de 
combustibles fósiles, donde la generación de energía está dominada por el gas 
natural (65\%) y le siguen las centrales hidroeléctricas (18\%), plantas nucleares
(8\%), parques eólicos (7\%) y solares (1\%) \cite{IEA}. En cuanto al potencial de 
producción de fuentes renovables, Argentina tiene una gran capacidad en sus 
fuentes eólicas y solares por desarrollar. Además, es el cuarto productor mundial más 
grande de litio, que es un mineral crítico para la manufactura de sistemas de 
almacenamiento y transporte de energía, claves para la transición energética. 
El mismo representa el 7\% de la demanda para vehículos eléctricos mientras que 
para almacenamiento en la red el porcentaje es del 10\%. Otros metales y 
minerales críticos se encuentran en la región de América Latina; por ejemplo, 
Paraguay posee la reserva más grande del mundo de titanio, Chile es el mayor 
productor de cobre, Brasil tiene las segundas reservas más grande de níquel y
hierro, las terceras de grafito y manganeso, la cuarta de aluminio y la quinta de 
fósforo, por último, Cuba se encuentra en el tercer puesto de reservas de cobalto.

En la Figura \ref{fig:iea-Li} se muestra la proyección en la demanda total de 
litio por año y por aplicación, donde la mayor contribución se encuentra para la 
utilización del mismo en vehículos eléctricos mientras que una menor contribución 
se espera en aplicaciones de sistemas de almacenamiento estacionarios y otras 
aplicaciones que incluyen dispositivos electrónicos, medicamentos, lubricantes, 
entre otras \cite{IEA}. Cabe destacar que para el almacenamiento estacionario 
las baterías de ion-litio es muy probable que compitan con baterías de sodio o magnesio, entre 
otras. En el histograma de la Figura \ref{fig:iea-Li} pueden diferenciarse dos 
regiones, la primera de ellas entre el año 2022 y el 2035, donde los aumentos
porcentuales de la demanda de litio con respecto a 5 años atrás son del 74\%, 
99\% y 76\%. Luego, del año 2035 al 2040, el cambio se encuentra en el 32\% y
dicho aumento porcentual continúa disminuyendo al 10\% y al 3\% en los períodos 
subsiguientes.
\begin{figure}[h!]
    \centering
    \includegraphics[width=.8\textwidth]{Introduccion/energia/iea-Li.png}
    \caption{Proyección de la demanda total de litio en kilotoneladas para el 
    período 2025-2050 para sus distintas aplicaciones: vehículos eléctricos (en 
    azul), sistemas de almacenamiento de energía estacionarios (en naranja) y
    otras aplicaciones (en verde). Fuente: \cite{IEA}.}
    \label{fig:iea-Li}
\end{figure}


\section{Baterías de ion-litio}

A finales del año 2019, año en el que se comenzó esta tesis, la Real Academia 
de Ciencias de Suecia le otorgó el Premio Nobel en Química a J. B. Goodenough, 
M. S. Whittingham y A. Yoshino por sus contribuciones al desarrollo de la batería 
de ion-litio. Esta batería recargable permitió los avances que se vieron en los 
teléfonos móviles y en las computadoras portátiles, entre otras aplicaciones.
Además, permite un mundo libre de combustibles fósiles ya que se utiliza en 
vehículos eléctricos y en almacenamientos estacionarios de energía para fuentes
renovables. Este galardón restaltó la importancia de muchos aspectos de la ciencia
moderna, como la investigación básica, la investigación la aplicada, la 
interdisciplina (JBG fue físico, MSW es un químico y AY un ingeniero) los 
desarrollos tecnológicos y los problemas concretos de las sociedades.
En la década del 1970, MSW desarrolló la primera batería utilizando un ánodo de
litio metálico y un cátodo de disulfuro de titanio. En 1980, JBG duplicó el 
voltaje original de dicha batería al introducir un cátodo de óxido de cobalto.
La desventaja de ambas se encontraba en el ánodo de litio metálico, que en los 
ciclos de carga y descarga se deposita preferentemente en sitios donde ya se 
ha depositado, dando lugar a estructuras ramificadas, llamadas dendritas, que 
pueden cortocircuitar la celda y llevar a la explosión de la misma. En 1985,
AY remplazó este material por uno carbonoso que incorpora los iones de litio
durante la carga y la descarga, disminuyendo los riesgos mencionados. Basandose
en este desarrolló, Sony comenzó a comercializar baterías de ion-litio en 1991.
La densidad de energía de estas baterías rondaba los 80 Wh/kg, en la actualidad
WeLion comercializa para los EVs de Nio una batería de ion-litio con una 
densidad de energía de 360 Wh/kg. En la Figura \ref{fig:whkg} se muestra la 
evolución de la densidad de energía en baterías de ion-litio comercializadas 
en los últimos 30 años. La importancia de esta característica para los EVs 
radica en la relación autonomía/peso.
\begin{figure}[h!]
    \centering
    \includegraphics[width=.8\textwidth]{Introduccion/baterias/whkg.png}
    \caption{Aumento en la densidad de energía en baterías de ion-litio comercializadas
    en los últimos 30 años. Figura adaptada de \cite{li2023700}.}
    \label{fig:whkg}
\end{figure}

Las baterías de ion-litio admiten una gran cantidad de recargas y las mismas están 
compuestas por celdas electroquímicas conectadas entre sí, las mismas son unidades 
fundamentales que permiten transformar la energía química almacenada en energía
eléctrica mediante una reacción redox (reducción-oxidación), en la cual uno de los 
componentes pierde electrones (se oxida) y el otro gana electrones (se reduce).
En la Figura \ref{fig:esquema_bateria} se muestra un esquema general con el 
funcionamiento que presenta una celda electroquímica de ion-litio y se destacan 
las componentes más relevantes: los electrodos positivo (cátodo) y negativo (ánodo) 
donde ocurren las reacciones redox en la carga/descarga de la celda, el electrolito 
por el cual difunden los iones de litio y el separador que suele ser un material 
poroso permeable al electrolito que se encarga de que los electrones circulen por 
el circuito externo. Durante la descarga de la reacción redox es espontánea y 
provoca la difusión de iones de litio por el electrolito desde el ánodo hacia el 
cátodo, junto con una corriente eléctrica en un circuito externo (flechas rojas). 
Durante la carga se debe aplicar una corriente eléctrica externa para tener la 
reacción inversa (flechas verdes).
\begin{figure}[h!]
    \centering
    \includegraphics[width=.8\textwidth]{Introduccion/baterias/esquema_bateria.png}
    \caption{Esquema de las componentes y el funcionamiento de una batería de 
    ion-litio.}
    \label{fig:esquema-bateria}
\end{figure}

En la Figura \ref{fig:scopus} se muestra el incremento en las últimas dos décadas
de los artículos científicos publicados en el área de las baterías de litio y, en 
particular, de las dos ramas estudiadas en esta tesis: la Carga rápida y los 
Ánodos de Si. En dicha figura se presentan datos extraídos de la base de datos 
Scopus \cite{SCOPUS} del número de publicaciones anuales normalizado con respecto 
al número de publicaciones en el año 2003, año en el que hubo 710 publicaciones 
en baterías de litio, 32 sobre ánodos de Si y 0 sobre carga rápida, por lo que 
se normalizó en este caso a la única publicación del 2004 en el tema.
\begin{figure}[h!]
    \centering
    \includegraphics[width=.8\textwidth]{Introduccion/baterias/scopus.png}
    \caption{Número de publicaciones anuales normalizado con respecto al año 2003. 
    Las consultas realizadas en Scopus \cite{SCOPUS} incluyen: 
    \texttt{lithium AND battery} (LIBs, en azul), \texttt{lithium AND battery AND 
    fast-charging} (Carga rápida, en naranja) y \texttt{lithium AND battery AND 
    silicon anodes} (Ánodos de Si, en verde).}
    \label{fig:scopus}
\end{figure}
La normalización y la escala logarítmica en la Figura \ref{fig:scopus} permiten
observar cualitativamente que la pendiente de crecimiento de publicaciones 
realcionadas a la carga rápida de baterías de litio es considerablemente mayor a 
de las otras dos. Además, los ánodos de Si se encuentran dentro de lo que sería
el creciemiento promedio del área de las baterías de litio. Un análisis de datos
cuantitativo permite determinar que en la última década el aumento de porcentaje
anual de publicaciones promedio fue del 15 \% y 16 \% para las baterías de litio 
y para los ánodos de silicio, respectivamente, mientras que para la carga rápida 
este porcentaje promedio asciende al 52 \%. Este análisis demuestra la relevancia
que la comunidad científica le da a los temas estudiados en esta tesis.


\section{Objetivos y estructura de la tesis}

Esta tesis tiene como objetivo estudiar materiales que se utilicen para el 
desarrollo de electrodos de baterías de ion-litio de próxima generación mediante 
distintos modelados computacionales. 
La misma se encuentra dividida en tres partes, la primera de ellas sobre la 
Motivación y fundamentos consistente de dos capítulos, el capítulo 
\ref{ch:introduccion} con esta introducción y el capítulo \ref{ch:metodos} con la
descripción de los distintos métodos computacionales utilizados. 
La Parte \ref{p:fast-charging} se divide en dos capítulos, ambos relacionados con 
la carga rápida de baterías de ion-litio. En el capítulo \ref{ch:un} se 
desarrolla un modelo para ajustar datos experimentales en condiciones 
galvanostáticas y predecir el tamaño óptimo de partículas que permite retener un 
80 \% de su capacidad frente a una carga realizada en 15 minutos 
\cite{fernandez2023towards}. El capítulo \ref{ch:umbem} busca una métrica 
universal que permita estandarizar las comparaciones del desempeño entre 
distintos materiales considerados en aplicaciones de carga rápida.
La Parte \ref{p:silicio} se centra en el estudio de las aleaciones presentes en 
los ánodos de silicio y se divide en tres capítulos. El capítulo 
\ref{ch:caracterizacion} caracteriza las estructuras de Li-Si encontradas con 
un potencial reactivo y con un método de exploración acelerada de mínimos locales
propuesto \cite{fernandez2021characterization}. En el capítulo \ref{ch:modelo} se
parametriza un modelo DFTB (\textit{denstity functional tight-binding}) para la 
interacción Li-Si mediante un algoritmo que asigna pesos a las distintas 
estructuras consideradas para el ajuste \cite{oviedo2023}. En el capítulo 
\ref{ch:prediccion} se proponen modelos de vecinos más cercanos para predecir 
mediciones de rayos x, RMN y Mössbauer a partir de las configuraciones atómicas
\cite{fernandez2023nmr}.
Cada uno de los capítulos mencionados en estas dos últimas partes se componen
de una introducción y detalles de los métodos computacionales utilizados, los 
resultados junto a las discusiones de los mismos y conclusiones parciales.
Por último, se cierra la tesis con el capítulo \ref{ch:comentarios} con los 
comentarios finales de la misma.



\section{Dinámica molecular}

La dinámica molecular (MD, de sus siglas en inglés, \textit{molecular dynamics})
es una técnica de simulación computacional que considera un sistema de $N$
partículas atómicas, que interactúan a través de un campo de fuerzas newtoniano,
de las cuales se obtiene su evolución temporal. La misma permite obtener
propiedades termodinámicas macroscópicas (temperatura, presión) de un sistema en 
equilibro a partir de cantidades microscopicas (posiciones, velocidades, fuerzas)
~\cite{frenkel2001, allen2017}.

Para entender mejor como trabaja esta técnica de simulación es conveniente ver
como funciona su código fuente, el mismo sigue, en la mayoria de los casos, la
siguiente forma:
\begin{enumerate}
    \item \underline{Inicialización del sistema}: se especifican las posiciones y
        velocidades iniciales de los átomos. También se elije un paso temporal, 
        un radio de corte para las interacciones y las condiciones de contorno que
        se van a respetar a lo largo de la simulación. 
    \item \underline{Cálculo de fuerzas}: con las posiciones específicadas se
        calcula la fuerza sobre cada uno de los átomos a través del campo de 
        fuerzas elegido.
    \item \underline{Integración de las ecuaciones de movimiento}: se integran las
        ecuaciones de Newton mediante algún integrador que obtiene las posiciones
        y velocidades del paso temporal siguiente a partir del actual.
    \item \underline{Computo de propiedades termodinámicas}: se realizan los
        cálculos de distintas cantidades de interés, como las energías potencial
        y cinéctica, la presión y la temperatura.
    \item De ser necesario, se aplica algún \underline{termostato o barostato}
        para realizar simulaciones en el ensamble termodinámico deseado.
    \item \underline{Evolución temporal}: se incrementa el tiempo adhiriendo un
        paso temporal y se vuelve al cálculo de las fuerzas con las nuevas 
        configuraciones.
\end{enumerate}

Estos pasos pueden verse en la figura \textcolor{red}{HACER UN DIAGRAMA DEL PSEUDO 
CÓDIGO PROPIO}. Veamos a continuación cada una de las partes en detalle.

\subsection{Configuraciones iniciales}

Mencionar algo sobre materials project.
Velocidades aleatorias.

\subsection{Condiciones de contorno}

Periódicas - Fijas

\subsection{Potenciales interatómicos}

\subsubsection{ReaxFF}

\subsubsection{DFTB}

\subsection{Integradores}

\subsection{Termostatos y barostatos}

\subsection{Métodos de exploración de la superficie energía-potencial}

\subsubsection{Minimizaciones locales}

\subsubsection{Templado simulado}

\subsubsection{Dinámica acelerada}

\section{Experimentos computacionales}

\subsection{Distribución radial de a pares}

\subsection{Número de coordinación}

\subsection{Difusión}


% Copyright (c) 2024, Francisco Fernandez
% License: CC BY-SA 4.0
%   https://github.com/fernandezfran/thesis/blob/main/LICENSE
\section{Resultados y discusiones}

Los resultados se presentan a continuación en el orden en el que se emplean 
los pasos en la librería \path{galpynostatic} de Python: una primera sección
para el preprocesamiento de los datos experimentales, luego otra para el ajuste
de estos datos con el modelo heurístico, y, por último, la utilización de éste
para predecir las condiciones del tamaño de partícula para lograr una carga
rápida de 15 y 5 minutos. Sumado a esto, también se compara el comportamiento que
tendrían los distintos materiales, dados sus parámetros fundamentales, a 
distintos tamaños.

\subsection{Preprocesamiento de los datos experimentales}

Un procedimiento experimental usual para evaluar los materiales de las baterías
consiste en medir los perfiles galvanostáticos a distintos valores de C-rate.
En la Figura \ref{fig:preproc} se muestran como ejemplo las mediciones realizadas
por Wang \textit{et al.} \cite{wang2019high} para LiCoO$_2$ (LCO) recubierto con 
TiO$_2$. Además, se agrega una línea punteada horizontal que se corresponde
con el potencial de equilibrio reportado en el trabajo citado, 3.9 V, y otra
0.15 V por debajo, que es el valor que corresponde al potencial de corte
elegido en este capítulo. Esta es la región de interés en el gráfico, ya que 
los valores en los que el SOC se intersecta con esta última curva (SOC$_{\max}$)
son los que se utilizan para ajustar el modelo en función de C-rate.
\begin{figure}[h!]
    \centering
    \includegraphics[width=0.7\textwidth]{FastCharging/un/resultados/preprocesamiento/preprocesamiento.png}
    \caption{Perfiles galvanostáticos para distintos valores de C-rate para el
    sistema LCO recubierto de TiO$_2$. Las líneas horizontales indican el 
    potencial de equilibrio y el de corte utilizado para determinar la 
    capacidad máxima normalizada alcanzada (SOC$_{\max}$) a cada C-rate. 
    Reproducido del trabajo de Wang \textit{et al.} \cite{wang2019high}.}
    \label{fig:preproc}
\end{figure}

Es importante destacar que, en el trabajo citado, los perfiles galvanostáticos
se presentan en función del SOC normalizado, que no siempre es el caso. La 
forma usual en la que estos resultados son reportados es en función de la 
capacidad de descarga. En estos casos, es necesario normalizarla con respecto
a la capacidad máxima ($Q_{\max}$) alcanzada por el material, para así obtener
el SOC normalizado. El criterio utilizado en este capítulo para encontrar 
$Q_{\max}$ fue considerar el valor máximo de la capacidad alcanzado por la 
medición a la C-rate más baja. Gráficos similares al presentado en la Figura 
\ref{fig:preproc} son obtenidos en el resto de los trabajos experimentales que
se utilizan en los ajustes que siguen.


\subsection{Ajuste del modelo}

\begin{figure}[h!]
    \centering
    \includegraphics[width=0.7\textwidth]{FastCharging/un/resultados/ajuste/ajustes.png}
    \caption{Ajuste del modelo a los datos SOC$_{\max}$ \textit{versus} C-rate
    para los distintos materiales de electrodos considerados: (a) Grafito 
    amorfo \cite{mancini2022}, (b) LTO \cite{he2012}, (c) LFP \cite{lei2015}, 
    (d) LCO \cite{wang2019high}, (e) LMO \cite{bak2011}, (f) LNMO
    \cite{nishikawa2017}.}
    \label{fig:ajustes}
\end{figure}

En la Figura \ref{fig:ajustes} se muestran los datos experimentales y los 
ajustes del modelo para el SOC$_{\max}$ alcanzado al potencial de cortes 
\textit{versus} la C-rate para los resultados de la Figura \ref{fig:preproc}
y otros materiales de uso común en lso electrodos de las baterías de ion-litio.
Puede observarse una buena concordancia en general entre el modelo y los 
experimentos. Esto también puede observarse en la Figura \ref{fig:pred_vs_exp},
donde se muestran los valores predichos para SOC$_{\max}$ en función de los
experimentales, junto con el coeficiente de determinación de cada ajuste.

\begin{figure}[h!]
    \centering
    \includegraphics[width=0.7\textwidth]{FastCharging/un/resultados/ajuste/pred_vs_exp.png}
    \caption{Predicciones del SOC$_{\max}$ \textit{versus} valores 
    experimentales, junto al coeficiente de determinación de cada sistema.}
    \label{fig:pred_vs_exp}
\end{figure}

El trabajo de Mancini \textit{et al} \cite{mancini2022} aportó nuevos 
conocimientos sobre el efecto de esferoidización en las características de las
partículas de grafito y su impacto en el comportamiento electroquímico. A
continuación se hace referencia a estos datos como grafito amorfo, como puede
verse en la Figure \ref{fig:ajustes}a. En este caso, el rango de C-rates
reportadas cubre una amplia región del SOC$_{\max}$, desde un material 
totalmente cargado hasta uno casi totalmente descargado. Esto no es lo habitual,
ya que en la mayoría de los experimentos sólo se reportan curvas con un buen 
rendimiento (alta capacidad), lo que limita los ajustes realizados a una región
concreta del diagrama. El coeficiente de difusión y la constante cinética 
obtenidos en el ajuste para este sistema son $1.23\times10^{-10}$ cm$^2$/s y 
$2.31\times10^{-7}$ cm/s, respectivamente. Para esto se consideró un tamaño
de partícula de $7.5 \mu$m y una geometría esférica. Este valor se corresponde
con la media de la distribución de tamaños, que es reportada junto a los 
cuartiles en el trabajo citado. Para los casos que siguen, en los que no se
tiene información precisa de la distribución de tamaños, se considera el punto
medio del rango reportado para el tamaño de las partículas y utiliza para 
definir el parámetro $d$ en el modelo.

Los ánodos de Li$_4$Ti$_5$O$_{12}$ (LTO) presentan características excelentes
de seguridad y una capacidad teórica de 175 mAhg$^{-1}$. He \textit{et al} 
\cite{he2012} sintetizaron nanopartículas cristalinas y esféricas de LTO a 
diferentes proporciones atómicas de Li/Ti, con un tamaño entre los 0.5 $\mu$m 
y los 3 $\mu$m. Se asume entonces un valor de $d=1.75 \mu$m y se utilizan los
datos de la proporción usual del LTO para ajustar el modelo, teniendose como 
resultado un valor de $D$ de $6.58\times10^{-12}$ cm$^2$/s. El valor experimental 
de $D$ reportado por He \textit{et al} para esta proporción atómica fue de
$5.12\times10^{-12}$ cm$^2$/s. Con respecto al valor de $k^0$, se obtuvo 
$8.11\times10^{-8}$ cm/s. Comparar este valor con el experimental no es tan 
directo, ya que lo que reportan es la densidad de corriente de intercambio, 
$i^0 = 2.7\times10^{-4}$ mA/cm$^2$. Utilizando la ecuación \ref{eq:bv} de 
Butler-Volmer se tiene una relación entre $k^0$ e $i^0$ dada por
\begin{equation}\label{eq:i0k0}
    i^0 = F \frac{\rho}{M_r} k^0 \left(x_s\right)^{\alpha} \left(1 - x_s\right)^{1-\alpha},
\end{equation}
donde las definiciones de los parámetros están dadas en la Tabla \ref{t:params}.
Asumiendo un valor de 0.5 para el coeficiente de transferencia $\alpha$, un 
SOC de 0.5 y utilizando los valores del precusor LTO para $M_r = 459.1$ g/mol y
$\rho = 3.48$ g/cm$^3$ \cite{osti_1284125} se obtiene un valor para $k^0$ de
$7.38\times10^{-10}$ cm/s, que presenta una discrepancia de dos ordenes de 
magnitud con respecto al ajustado en el modelo. Sin embargo, en la literatura
se encuentran valores de $i^0$ con una gran dispersión entre 
$i^0 = 1.1\times10^{-3}$ mA/cm$^2$ \cite{medina2015} y $i^0 = 0.5$ mA/cm$^2$ 
\cite{umirov2019} que darían valores de $k^0$ entre $3.00\times10^{-9}$ cm/s y 
$1.37\times10^{-6}$ cm/s, respectivamente. Por lo cual puede afirmarse que el 
valor estimado por el modelo es razonable, dada la simplicidad del mismo.

Otro sistema en el cual las C-rates a las que se midieron los perfiles 
galvanostáticos cubren un rango amplio de valores de SOC$_{\max}$, de 
completamente cargado a completamente descargado, es el de LiFePO$_4$ (LFP) de 
Lei \textit{et al} \cite{lei2015}, como puede verse en la Figura 
\ref{fig:ajustes}. En este trabajo consideraron sistemas LFP/nanotubos de 
carbono/grafeno (LFP-CNT-G) como materiales catódicos con una capacidad 
superior a velocidades de carga alta y un desmpeño favorable en sucesivos 
ciclados a densidades de corriente relativamente altas, comparados con 
sistemas LFP-CNT y LFP-G. Para este caso seleccionado, obtuvieron coeficientes 
de difusión, a partir de la pendiente de un ajuste lineal a mediciones de 
espectroscopia de impedancia electroquímica (EIS), de $1.04\times10^{-12}$ 
cm$^2$/s, $1.738\times10^{-13}$ cm$^2$/s y $8.225\times10^{-13}$ cm$^2$/s, 
respectivamente para cada uno de los sistemas mencionados. Mientras que al
ajustar los datos experimentales del primer sistema mencionado, con un tamaño
de partícula de $0.35 \mu$m, se obtuvo un valor de $2.85\times10^{-13}$ cm$^2$/s
para este parámetro. Como puede observarse, se aprecia una discrepancia de un 
orden de magnitud pero dentro de los valores obtenidos en las otras sintesís.
El valor obtenido para $k^0$ utilizando la ecuación \ref{eq:i0k0} y el dato 
$i^0=5.127\times10^{-4}$ mA/cm$^2$ es $1.23\times10^{-9}$ cm/s, con un valor
de $M_r$ de $157.75$ g/mol y $\rho$ de $1.36$ g/cm$^3$ \cite{jin2018}.
En este caso se encuentra una correspondencia excelente con el valor ajustado
de $1.00\times10^{-9}$ cm/s.




\subsection{Predicción del tamaño óptimo de partícula}

Como ya ha sido mencionado a lo largo de esta tesis, el criterio de carga rápida
está definido por la obtención del 80\% de la capacidad del electrodo en 15 
minutos, lo cual se traduce en un SOC$_{\max}$ de 0.8 y una C-rate de 4 C. La
Figura \ref{fig:prediccion} muestra donde se encuentra cada sistema analizado en
el diagrama $\log(\Xi)$--$\log(\ell)$ para dicha C-rate. También se presenta una
curva de nivel con una línea roja correspondiente a SOC$_{\max} = 0.8$. Puede
observarse que tres de los materiales ya se encuentran en la región de 
SOC$_{\max}$ mayor a 0.8 (LCO, LMNO y LNMO), mientras que los otros se encuentran
por debajo de este valor (LTO, Grafito amorfo y LFP).
\begin{figure}[h!]
    \centering
    \includegraphics[width=0.7\textwidth]{FastCharging/un/resultados/prediccion/prediccion.png}
    \caption{Diagrama de SOC$_{\max}$ mostrando la ubicación de los materiales 
    usuales de LIBs a 4 C para las referencias consideradas \cite{mancini2022,
    he2012, lei2015, wang2019high, bak2011, nishikawa2017}. En los casos en los 
    que la curva de cargado a 4 C no estaba disponible, el valor del punto fue 
    predicho con el modelo. La línea roja muestra la curva de nivel 
    correspondiente al valor 0.8 de SOC$_{\max}$. Las flechas muestran el cambio
    en el tamaño de la partícula que debería efectuarse para obtener dicho valor
    a la C-rate dada. Las curces sobre la línea muestran la posición de estos
    tamaños de partícula nuevos.}
    \label{fig:prediccion}
\end{figure}
Haciendo uso del diagrama se puede predecir una forma simple y rápida el tamaño 
de partícula requerido para satisfacer el criterio de carga rápida. Dado que los
valores de $D$ y $k^0$ ya fueron ajustados, el valor de $d$ seleccionado por el
experimento y el de C-rate por el criterio, el valor de $\Xi$ es constante. Luego,
para alcanzar el valor de 0.8 de SOC$_{\max}$ hay que variar $\ell$ y esto se
logra disminuyendo o aumentando el tamaño de la partícula, según sea necesario. 
Este desplazamiento necesario está representado por las flechas en la Figura 
\ref{fig:prediccion} para cada caso. Ya se ha apreciado que tres sistemas se 
encuentran en la región ya optimizada (LCO, LMO y LNMO), por lo que en estos casos
los tamaños predichos para alcanzar SOC$_{\max} = 0.8$ a 4 C serán mayores que 
los experimentales. Por el contrario, el resto de los materiales (LTO, Grafito
amorfo y LFP) tienen que ser mejorados con una reducción del tamaño de partícula
para cumplir la condición. En la tabla \ref{t:prediccion} se muestran los tamaños
de partícula predichos para todos los materiales en la tercera columna para este
criterio. Las incertidumbres se determinaron por propagación de errores con 
derivadas parciales. Ya que el tamaño de la partícula sólo aparece en el parámetro
$\ell$, al definir $\ell_{\text{opt}}$ como el valor al cual el SOC$_{\max}$ 
alcanza el valor deseado de 0.8 y usar que $V/A = d/z$ se puede despejar de la 
ecuación \ref{eq:ele} que
\begin{equation}
    d = \sqrt{\frac{t_h z D 10^{\ell_{\text{opt}}}}{C_r}}.
\end{equation}
Si además se supone que toda la incertidumbre está asociada al coeficiente de 
difusión $D$, al cual ya se le calculó su incerteza, se puede obtener que
\begin{equation}
    \Delta d = \frac{1}{2} \sqrt{\frac{t_h z 10^{\ell_{\text{opt}}}}{C_r D}} \Delta D.
\end{equation}

\begin{table}[h!]
    \centering
    \caption{Tamaño experimental y valores predichos para cargar el 80\% del
    electrodo en 15 y 5 minutos.} 
    \setlength\extrarowheight{2pt}\stackon{%
    \begin{tabular}{l c c c}
        \toprule
        \textbf{Material del} &
        \textbf{Tamaño} &  
        \textbf{Tamaño predicho} & 
        \textbf{Tamaño predicho} \\
        \textbf{electrodo} & 
        \textbf{experimental [$\mu$m]} &  
        \textbf{para 15 minutos [$\mu$m]} & 
        \textbf{para 5 minutos [$\mu$m]} \\
        \midrule
        Grafito amorfo & 7.5 & 4.027 $\pm$ 0.002 & 2.167 $\pm$ 0.001 \\
        LTO & 1.75 & 0.962 $\pm$ 0.004 & 0.530 $\pm$ 0.002 \\
        LFP & 0.35 & 0.084 $\pm$ 0.002 & 0.0309 $\pm$ 0.0006 \\
        LCO & 20 & 28.8 $\pm$ 0.6 & 16.4 $\pm$ 0.4 \\
        LMO & 0.025 & 0.0734 $\pm$ 0.0003 & 0.0418 $\pm$ 0.0002 \\
        LNMO & 7.999 & 13 $\pm$ 2 & 7.3 $\pm$ 0.8 \\
        \bottomrule
    \end{tabular}
    }{}
    \label{t:prediccion}
\end{table}

Al observarse un buen desempeño para la carga de 15 minutos, se puede exigir un 
poco más que este criterio y predecir el tamaño de partícula requerido para una
C-rate más alta, digamos 80\% de la carga en 5 minutos (12 C). Si bien esta figura
puede parecer sobredemandante a primera vista, reportes recientes consideran 
protocolos de carga de 10 minutos \cite{mattis2021, attia2020}. Los resultados
se muestran en la última columna de la Tabla \ref{t:prediccion}. Como puede 
observarse, el comportamiento depende del sistema y del experimento en particular
considerado. El único caso donde se cumple este último criterio de carga rápida 
es el LMO, ya que el tamaño experimental sobrecumple el criterio. Aunque el LCO 
y el LNMO no cumplen con este último criterio, los cambios en sus tamaños serían
menores, por lo que estos materiales requieren mejoras menores. En el resto de 
los casos, para el LFP se necesitaría una disminución de un orden de magnitud 
en su tamaño, mientras que para el grafito amorfo o el LTO se requeriría una
disminución de su tamaño en un factor de 3.


\subsection{Comparación de sistemas con un mismo tamaño}

La curva de los distintos experimentos ajustados sigue el mismo comportamiento 
para todos los sistemas, como se mostró en la Figura \ref{fig:ajustes-mapa},
esto es un efecto esperado debido a las definiciones de $\Xi$ y $\ell$, todas las
curvas exhiben una pendiente de $-1/2$ dada por la siguiente ecuación
\begin{equation}
    \log(\Xi) = \log(B) - \frac{1}{2}\log(\ell),
\end{equation}
donde el valor de $B$ puede obtenerse al eliminar C-rate de las ecuaciones 
\ref{eq:xi} y \ref{eq:ele} para obtener
\begin{equation}
    \log(\Xi) = \log\left(\frac{k^0 d}{D \sqrt{z}}\right) - \frac{1}{2}\log(\ell),
\end{equation}
donde se ve que la ordenada al origen contiene una composición de los parámetros
fundamentales considerados. 

\begin{figure}[t]
    \centering
    \includegraphics[width=\textwidth]{FastCharging/un/resultados/comparacion/comparacion.png}
    \caption{Comparación de los sistemas considerados con distintos tamaños de 
    partícula entre 0.1 $\mu$m y 10.0 $\mu$m en el rango experimental usual para 
    valores de C-rates: (a) SOC$_{\max}$ \textit{versus} C-rate. (b) Diagramas.}
    \label{fig:comparacion}
\end{figure}

Si se quieren comparar los méritos de los distintos materiales, en términos de 
sus propiedades intrínsecas de transferencia de carga en la interfase 
electrodo/electrolito ($k^0$) y difusión de iones dentro de ellos ($D$), se 
debería comparar el comportamiento de las partículas para un mismo tamaño a 
distintas C-rate, lo cual se presenta en la Figura \ref{fig:comparacion}. En 
particular, en la Figura \ref{fig:comparacion}a se muestra el SOC$_{\max}$ 
en función de la C-rate, considerando un conjunto de tamaños de partículas y 
C-rates en un rango físicamente razonable. Para obtener estos resultados se 
utilizaron los valores de $D$ y $k^0$ ajustados en la sección \ref{s:ajustes}.

En el gráfico (i) de la Figura \ref{fig:comparacion}a para 0.1 $\mu$m está
claro que los únicos materiales que muestran una pérdida de la capacidad para
C-rates altas son LFP y LMO, el resto retiene más del 80\% de la capacidad, 
incluso a 100 C. En el segundo gráfico (ii) de la Figura
\ref{fig:comparacion}a para 1 $\mu$m, el LTO y el NG tienen
una caída en el SOC por debajo del 50\% para 100 C, mientras que LCO y LNMO
están por encima del 80\%. Por último, el tamaño de partícula más grande que se 
considera, 10 $\mu$m (Figura \ref{fig:comparacion}a, gráfico (iii)), 
todos los materiales presentan una retención de la capacidad por debajo del 80\%
a la C-rate más alta. En la Figure \ref{fig:comparacion}b estos datos 
comportamientos están presentados en el diagrama construido con las simulaciones
galvanostáticas para dar una idea de las regiones en las que cada sistema se
encuentra. 

Cabe destacar que la secuencia de materiales dada en la Figura 
\ref{fig:comparacion} fue obtenida utilizando los valores de $k^0$ y $D$ 
ajustados con el modelo a los datos experimentales. Ajustar otros experimentos
podría alterar esta secuencia. Idealmente, las mediciones deberían estar 
realizadas sobre electrodos de una sola partícula.



\section{Conclusiones del capítulo}



% \subsection{Predicción del tamaño óptimo de partícula}

Como ya ha sido mencionado a lo largo de esta tesis, el criterio de carga rápida
está definido por la obtención del 80\% de la capacidad del electrodo en 15 
minutos, lo cual se traduce en un SOC$_{\max}$ de 0.8 y una C-rate de 4 C. La
Figura \ref{fig:prediccion} muestra donde se encuentra cada sistema analizado en
el diagrama $\log(\Xi)$--$\log(\ell)$ para dicha C-rate. También se presenta una
curva de nivel con una línea roja correspondiente a SOC$_{\max} = 0.8$. Puede
observarse que tres de los materiales ya se encuentran en la región de 
SOC$_{\max}$ mayor a 0.8 (LCO, LMNO y LNMO), mientras que los otros se encuentran
por debajo de este valor (LTO, Grafito amorfo y LFP).
\begin{figure}[h!]
    \centering
    \includegraphics[width=0.7\textwidth]{FastCharging/un/resultados/prediccion/prediccion.png}
    \caption{Diagrama de SOC$_{\max}$ mostrando la ubicación de los materiales 
    usuales de LIBs a 4 C para las referencias consideradas \cite{mancini2022,
    he2012, lei2015, wang2019high, bak2011, nishikawa2017}. En los casos en los 
    que la curva de cargado a 4 C no estaba disponible, el valor del punto fue 
    predicho con el modelo. La línea roja muestra la curva de nivel 
    correspondiente al valor 0.8 de SOC$_{\max}$. Las flechas muestran el cambio
    en el tamaño de la partícula que debería efectuarse para obtener dicho valor
    a la C-rate dada. Las curces sobre la línea muestran la posición de estos
    tamaños de partícula nuevos.}
    \label{fig:prediccion}
\end{figure}
Haciendo uso del diagrama se puede predecir una forma simple y rápida el tamaño 
de partícula requerido para satisfacer el criterio de carga rápida. Dado que los
valores de $D$ y $k^0$ ya fueron ajustados, el valor de $d$ seleccionado por el
experimento y el de C-rate por el criterio, el valor de $\Xi$ es constante. Luego,
para alcanzar el valor de 0.8 de SOC$_{\max}$ hay que variar $\ell$ y esto se
logra disminuyendo o aumentando el tamaño de la partícula, según sea necesario. 
Este desplazamiento necesario está representado por las flechas en la Figura 
\ref{fig:prediccion} para cada caso. Ya se ha apreciado que tres sistemas se 
encuentran en la región ya optimizada (LCO, LMO y LNMO), por lo que en estos casos
los tamaños predichos para alcanzar SOC$_{\max} = 0.8$ a 4 C serán mayores que 
los experimentales. Por el contrario, el resto de los materiales (LTO, Grafito
amorfo y LFP) tienen que ser mejorados con una reducción del tamaño de partícula
para cumplir la condición. En la tabla \ref{t:prediccion} se muestran los tamaños
de partícula predichos para todos los materiales en la tercera columna para este
criterio. Las incertidumbres se determinaron por propagación de errores con 
derivadas parciales. Ya que el tamaño de la partícula sólo aparece en el parámetro
$\ell$, al definir $\ell_{\text{opt}}$ como el valor al cual el SOC$_{\max}$ 
alcanza el valor deseado de 0.8 y usar que $V/A = d/z$ se puede despejar de la 
ecuación \ref{eq:ele} que
\begin{equation}
    d = \sqrt{\frac{t_h z D 10^{\ell_{\text{opt}}}}{C_r}}.
\end{equation}
Si además se supone que toda la incertidumbre está asociada al coeficiente de 
difusión $D$, al cual ya se le calculó su incerteza, se puede obtener que
\begin{equation}
    \Delta d = \frac{1}{2} \sqrt{\frac{t_h z 10^{\ell_{\text{opt}}}}{C_r D}} \Delta D.
\end{equation}

\begin{table}[h!]
    \centering
    \caption{Tamaño experimental y valores predichos para cargar el 80\% del
    electrodo en 15 y 5 minutos.} 
    \setlength\extrarowheight{2pt}\stackon{%
    \begin{tabular}{l c c c}
        \toprule
        \textbf{Material del} &
        \textbf{Tamaño} &  
        \textbf{Tamaño predicho} & 
        \textbf{Tamaño predicho} \\
        \textbf{electrodo} & 
        \textbf{experimental [$\mu$m]} &  
        \textbf{para 15 minutos [$\mu$m]} & 
        \textbf{para 5 minutos [$\mu$m]} \\
        \midrule
        Grafito amorfo & 7.5 & 4.027 $\pm$ 0.002 & 2.167 $\pm$ 0.001 \\
        LTO & 1.75 & 0.962 $\pm$ 0.004 & 0.530 $\pm$ 0.002 \\
        LFP & 0.35 & 0.084 $\pm$ 0.002 & 0.0309 $\pm$ 0.0006 \\
        LCO & 20 & 28.8 $\pm$ 0.6 & 16.4 $\pm$ 0.4 \\
        LMO & 0.025 & 0.0734 $\pm$ 0.0003 & 0.0418 $\pm$ 0.0002 \\
        LNMO & 7.999 & 13 $\pm$ 2 & 7.3 $\pm$ 0.8 \\
        \bottomrule
    \end{tabular}
    }{}
    \label{t:prediccion}
\end{table}

Al observarse un buen desempeño para la carga de 15 minutos, se puede exigir un 
poco más que este criterio y predecir el tamaño de partícula requerido para una
C-rate más alta, digamos 80\% de la carga en 5 minutos (12 C). Si bien esta figura
puede parecer sobredemandante a primera vista, reportes recientes consideran 
protocolos de carga de 10 minutos \cite{mattis2021, attia2020}. Los resultados
se muestran en la última columna de la Tabla \ref{t:prediccion}. Como puede 
observarse, el comportamiento depende del sistema y del experimento en particular
considerado. El único caso donde se cumple este último criterio de carga rápida 
es el LMO, ya que el tamaño experimental sobrecumple el criterio. Aunque el LCO 
y el LNMO no cumplen con este último criterio, los cambios en sus tamaños serían
menores, por lo que estos materiales requieren mejoras menores. En el resto de 
los casos, para el LFP se necesitaría una disminución de un orden de magnitud 
en su tamaño, mientras que para el grafito amorfo o el LTO se requeriría una
disminución de su tamaño en un factor de 3.



% \part{Comentarios finales}

% % Copyright (c) 2024, Francisco Fernandez
% License: CC BY-SA 4.0
%   https://github.com/fernandezfran/thesis/blob/main/LICENSE
\chapter{Comentarios finales y perspectivas futuras}\label{ch:comentarios}

\section{Comentarios finales}

Las baterías de ion-litio han permitido que se desarrollaran una gran variedad de
dispositivos electrónicos, por ejemplo los teléfonos inteligentes o las 
computadoras portátiles. A su vez son utilizadas en vehículos eléctricos y en 
sistemas de almacenamiento estacionarios de energía para fuentes
renovables, lo cual las convierte en un actor crucial para la sustitución de 
los combustibles fósiles en el consumo energético. En rasgos generales, se puede
afirmar que los dos grandes avances logrados aquí se refieren al establecimiento
de una métrica universal sin precedentes para predecir y evaluar el cargado rápido
de materiales al nivel de una partícula, y a la predicción de las propiedades de 
aleaciones amorfas de Li-Si, uno los materiales más promisorios para ser empleado 
como ánodos de baterías de ion-Li de próxima generación. De este modo, en esta 
tesis doctoral se 
aplicaron distintos modelados computacionales para estudiar electrodos de 
baterías de ion-litio de próxima generación, las técnicas utilizadas fueron 
introducidas en el capítulo \ref{ch:metodos}. Los resultados obtenidos fueron 
divididos en dos partes: \textbf{Carga rápida de baterías de ion-litio} (Parte 
\ref{p:fast-charging}) y \textbf{Silicio como ánodo de baterías de ion-litio de 
próxima generación: Estudio de sus aleaciones} (Parte \ref{p:silicio}). Parte de los resultados obtenidos fueron publicados en revistas científicas de referencia en el área:
\begin{enumerate}
    \item \underline{Fernandez, F.}, Otero, M., Oviedo, M. B., Barraco, D. E., Paz, S. A., \& Leiva, E. P. M. (2023). NMR, x-ray, and Mössbauer results for amorphous Li-Si alloys using density functional tight-binding method. \textit{Physical Review B, 108}(14), 144201.
    \item \underline{Fernandez, F.}, Gavilán-Arriazu, E. M., Barraco, D. E., Visintin, A., Ein-Eli, Y., \& Leiva, E. P. M. (2023). Towards a fast-charging of LIBs electrode materials: a heuristic model based on galvanostatic simulations. \textit{Electrochimica Acta, 464}, 142951.
    \item Oviedo, M. B., \underline{Fernandez, F.}, Otero, M., Leiva, E. P., \& Paz, S. A. (2023). Density Functional Tight-Binding Model for Lithium–Silicon Alloys. \textit{The Journal of Physical Chemistry A, 127}(11), 2637-2645.
    \item \underline{Fernandez, F.}, Paz, S. A., Otero, M., Barraco, D., \& Leiva, E. P. (2021). Characterization of amorphous Li x Si structures from ReaxFF via accelerated exploration of local minima. \textit{Physical Chemistry Chemical Physics, 23}(31), 16776-16784.
\end{enumerate}
Se encuentran en redacción otros manuscrítos a ser publicados.

En el capítulo \ref{ch:un} se desarrolló un modelo que permite predecir el tamaño
óptimo que deberían tener las partículas de material activo en un electrodo para 
alcanzar el 80\% del Estado de la Carga (SOC) en 15 minutos, que es el criterio
establecido para considerar que la batería sea de carga rápida. Además, se 
desarrolló un software en Python que cumple con los requisitos establecidos por 
la comunidad, es de libre acceso y fácil de usar para realizar el preprocesamiento
de datos experimentales y las estimaciones de parámetros que emplea el modelo. En esta oportunidad, se 
utilizaron datos experimentales de literatura de distintos materiales de 
relevancia en el área de estudio (NG, LTO, LFP, LCO, LMO, LNMO). Para todos ellos,
en un preprocesamiento de los datos se obtuvo el SOC máximo en función de la 
velocidad de carga galvanostática (C-rate) y se ajustó el modelo, de donde 
se obtuvieron coeficientes de difusión y constantes cinéticas físicamente 
correctas. Dicho ajuste se realiza de manera heurística sobre una superficie 
obtenida mediante un método de simulación universal de una sola partícula en 
condiciones de carga a corriente constante, que es capaz de reproducir la física 
básica del proceso de intercalación de litio, regulado por la difusión de los 
iones dentro de la partícula y la transferencia de carga en su interfase.
Una vez ajustado el modelo a cada sistema, se lo utilizó para predecir el tamaño
óptimo de partícula para cargas de 15 y 5 minutos y se discutió cada caso en 
particular. Por último, se compararon los méritos de los materiales entre sí en 
términos de sus propiedades intrínsecas y a igual tamaño de partícula.

Dentro de la misma área de estudio, en el capítulo \ref{ch:umbem} se propuso, por 
primera vez en la literatura, una métrica universal para comparar el desempeño de 
carga rápida de materiales de electrodos (UMBEM, de sus siglas en inglés, 
\textit{Universal Metric for Benchmarking fast-charging Electrode Materials}). 
Esta se definió como el SOC alcanzado cuando el material se carga 
en condiciones de corriente constante durante 15 minutos. La UMBEM puede ser 
analizada con distintas técnicas experimentales o computacionales, en este caso 
se la analizó utilizando el método empleado en el capítulo anterior. El mismo 
presenta una mejora con respecto a una figura de mérito (FOM, de sus siglas en 
inglés \textit{Figure of Merit}) publicada en un trabajo anterior de literatura,
ya que, además de considerar el tamaño de la partícula y la difusión de los iones, 
considera la transferencia de carga interfacial y la C-rate. Utilizando el mismo 
conjunto de datos de caracterizaciones experimentales que el trabajo de la FOM, se
obtuvo el valor de la UMBEM para cada sistema y se estableció una jerarquía de 
materiales. También se compararon estos valores con los de la FOM. Además,
basándose en un análisis guiado por la superficie del método computacional 
empleado, se predijeron las mejoras necesarias para clasificarlos 
como materiales de carga rápida.

Como primera etapa en el estudio de las aleaciones de LiSi que se forman durante
la litiación de los ánodos de silicio, en el capítulo \ref{ch:caracterizacion} se 
obtuvieron estructuras amorfas de Li$_x$Si para distintos valores de $x$ que 
cubren el intervalo experimental. Se realizaron simulaciones de 
dinámica molecular con un campo de fuerzas reactivo y se encontraron estructuras
cercanas al equilibrio con un método de exploración acelerada de mínimos locales.
Para dichas estructuras se calculó el cambio volumétrico fraccional, que resultó 
en concordancia con experimentos de microscopía de fuerza atómica. También se 
obtuvo una buena representación del comportamiento electroquímico al reproducir 
la curva de potencial en función de la concentración de Li a partir de las 
energías de las estructuras obtenidas. Se analizaron las funciones distribución 
radial (RDF) y los números de coordinación de los primeros y segundos vecinos. 
Mediante análisis de formación e interconexión de clusters se caracterizaron las 
estructuras a los distintos valores de $x$ y se dilucidó la estructura compleja 
observada en el segundo pico de la RDF de Si-Li, respectivamente. Por último, se 
definió un parámetro que permitió determinar el orden de corto alcance de 
estructuras amorfas y los tipos de interacciones.

En el capítulo \ref{ch:modelo} se parametrizó un modelo DFTB (\textit{Density 
Functional Tight-Binding}) para LiSi, que tiene una complejidad intermedia entre 
DFT (\textit{Density Functional Theory}) y los campos de fuerza clásicos. Para 
obtener el conjunto de parámetros se introdujo un algoritmo de ajuste que pondera
las distintas estequiometrías que se consideran en el conjunto de entrenamiento 
para mejorar la predicción de algún observable. En este caso se consideró como 
objetivo las energías de formación relativas de las estructuras cristalinas de 
LiSi, cuyas configuraciones atómicas fueron extraídas de la base de datos del Materials Project.
A este conjunto de estructuras se le agregaron las producidas por compresiones y 
expansiones isotrópicas, a las cuales se les analizaron sus perfiles de energía 
comparándolos con los calculados con DFT. Con el modelo obtenido se realizaron
predicciones óptimas de las energías de formación en el conjunto de entrenamiento 
cristalino y en el conjunto de evaluación amorfo para todo el intervalo de
concentraciones de Li presentes en la litiación de los ánodos de Si. Se compararon 
los residuos de dichas predicciones con las que se obtendrían si se obviara el 
paso del algoritmo de ajuste de pesos para demostrar los beneficios del mismo.
Como las mayores discrepancias con DFT se observaron para Si amorfo puro, se 
amorfizó una estructura de Si mediante un templado simulado y se analizó la RDF 
que resultó en una reproducción excelente del experimento. Como conclusión 
general de este capítulo, se encontró que el modelo DFTB mostró robustez en sus predicciones al
obtenerse una gran concordancia con DFT y superarse el desempeño del potencial 
reactivo del estado-del-arte para este sistema.

Utilizando el modelo DFTB desarrollado en en capítulo \ref{ch:modelo}, se 
obtuvieron configuraciones atómicas de estructuras amorfas Li$_x$Si siguiendo un 
protocolo de litiación de literatura ligeramente modificado. Las estructuras 
obtenidas fueron analizadas en base a modelos que consideran los vecinos más 
cercanos para predecir los resultados de mediciones de rayos x, RMN y Mössbauer.
Para el caso de rayos x se computaron las RDFs parciales, de las cuales se obtuvo
la distribución radial de a pares, $G(r)$, y se la comparó con la PDF 
(\textit{Pair Distribution Function}) de un experimento de Si amorfo y otro de Si
completamente litiado. Para este último caso se hizo un ajuste de coeficientes de 
una combinación lineal de las distintas estructuras que pueden aparecer, como 
sugieren los experimentos. Por otro lado, en los espectros de RMN de corrimiento 
químico de $^7$Li suelen asignarse picos según el criterio de si los vecinos de Si de los átomos de Li están
enlazados a otros átomos de Si o aislados. Siguiendo el argumento experimental, se realizó una 
formulación matemática de esta hipótesis y se propuso un modelo para simular e 
interpretar dichas mediciones. Por último, se propuso una dependencia lineal 
entre la separación de picos en espectroscopia de Mössbauer y el mínimo de 
concentración entre Li y Si. Cuando se usan las concentraciones locales para 
realizar los cálculos se nota una mejora con respecto a los valores globales y
una mejor representación del experimento. Para todos los casos las predicciones
presentaron una buena concordancia con los experimentos.

\section{Perspectivas futuras}

A lo largo de esta tesis se han utilizado distintas técnicas de simulación que
operan en diversas escalas temporales y espaciales. De ambas partes de la misma
surgen distintas alternativas para trabajos futuros.

En el caso del modelo y la métrica propuestos en la parte de la tesis referida a 
la carga rápida, estos podrían ser utilizados en baterías de 
próxima generación además de las de ion-litio, que sigan los mismos fenómenos
físicos, como pueden ser las baterías de sodio, que se presentan entre las 
alternativas más prometedoras \cite{morais2021titanium, leite2020electrochemistry}.
Por otro lado, podrían estudiarse los efectos que tienen las distintas 
suposiciones del modelo para condiciones galvanostáticas. Por ejemplo, se podría utilizar 
la base de datos SQL \path{LiionDB} \cite{wang2022review} o algún sistema de 
extracción automático/semi-automático de datos de publicaciones, como \path{LIBAC}
\cite{el2023libac}, para recopilar una gran cantidad de determinaciones 
experimentales y a partir de ellas estudiar la influencia de la interacción entre los iones 
intercalados a la hora de predecir distintos parámetros, como los coeficientes de 
difusión. Esto sería posible si se modifica ligeramente el modelo para que, además
de recibir los cuatro descriptores del sistema presentados, considere también
la isoterma de inserción.

En lo que refiere a las aleaciones de Li-Si, las estructuras encontradas
que predijeron resultados de mediciones de rayos x, RMN y Mössbauer podrían 
usarse dentro de otros modelos, que podrían elaborarse para predecir otros 
experimentos o utilizarlas en otras técnicas de simulación ya establecidas para 
determinar, por ejemplo, el coeficiente de difusión de litio en silicio amorfo en 
función de la concentración o el potencial aplicado.

Por último, se podría plantear el diseño de un modelo multiescala al estilo del 
de Liu et al \cite{liu2021towards} para electrodos de compositos grafito/silicio,
donde estructuras obtenidas con dinámicas moleculares de grano grueso sean 
utilizadas como parámetros de entrada en un modelo 3D del continuo que acopla 
simulaciones electroquímicas y mecánicas. En el caso del tema abordado en esta tesis podría 
calcularse el coeficiente de difusión de Li en Si en función de la concentración 
de Li, con las estructuras ya obtenidas como configuraciones iniciales, junto con
su isoterma y el cambio volumétrico, e introducir estos parámetros en el modelo 
de una sola partícula, modificando las ecuaciones para que considere 
estos efectos relevantes en este sistema. Por otro lado, también podrían 
aplicarse distintos modelos de aprendizaje automático utilizando los resultados 
de esta tesis como base de datos para obtener un campo de fuerzas. Por ejemplo, se podrían considerar las estructuras amorfas 
de LiSi para predecir resultados experimentales de manera similar al uso que hicimos de los modelos de
vecinos más cercanos propuestos. Se podría también desarrollar un modelo de orden reducido que
permita predecir el estado de carga máxima alcanzado entrenando dicho valor sobre 
los descriptores que permiten obtener los perfiles galvanostáticos simulados.



% \part{Apéndices}

% \appendix
% \renewcommand\chaptername{Apéndice}
% \section{Software}

En la actualidad existe una gran cantidad de códigos que permiten realizar 
simulaciones de dinámica molecular de manera eficiente aprovechando la estructura
paralela de los procesadores. En esta tesis se utilizó principalmente \path{LAMMPS} 
(\textit{Large-scale Atomic/Molecular Massively Parallel Simulator}, sus siglas
en inglés) ~\cite{lammps1, lammps2}, un código centrado en el modelado de 
materiales que permite simular con distintos campos de fuerzas, ensambles y 
condiciones de contorno. También se realizaron simulaciones con versiones 
modificadas de \path{GEMS} ~\cite{gems}, código del Dr. Sergio Alexis Paz 
(FCQ-UNC), que permiten utilizar a \path{DFTB+} ~\cite{dftb+} como una librería
y realizar distintos métodos de aceleración de simulaciones que no se encuentran
en los programas usuales como \path{LAMMPS}.

Para la visualización de las trayectorias y la obtención de imágenes de 
estructuras representativas se utilizó \path{VMD} (\textit{Visual Molecular 
Dynamics}) ~\cite{vmd}. Para el post-procesamiento de datos de las simulaciones
se escribieron distintos programas en Python ~\cite{exma, sierras}.



\bibliographystyle{apalike}
\bibliography{citas}

\end{document}
