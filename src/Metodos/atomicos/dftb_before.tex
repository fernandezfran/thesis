\subsubsection{DFTB}

Un método alternativo para obtener las fuerzas en dinámica molecular es a través
de la utilización de un modelo híbrido entre los métodos \textit{ab-initio},
basados en DFT, y el uso de potenciales completamente empíricos, DFTB (de sus 
siglas en inglés, \textit{Density Functional based Tight Binding}), que tiene
la ventaja de ser más transferibles que estos últimos y requiere menos costo
computacional que los primeros.

El método de DFTB se basa en una expansión de segundo orden de la energía total 
de Kohn-Sham ~\cite{dft1, dft2} en la DFT con respecto a las fluctuaciones de la 
densidad de carga. El enfoque de orden cero es equivalente a un esquema estándar 
no auto-consistente (TB), mientras que en el segundo orden se puede derivar una 
expresión transparente, libre de parámetros y fácilmente calculable para los 
elementos matriciales hamiltonianos generalizados. Estos se modifican mediante 
una redistribución auto-consistente de las cargas de Mulliken (SCC).

La energía total de un sistema de $M$ electrones en el campo de $N$ núcleos en
las posiciones $\mathbf{R}$ puede escribirse a través de DFT como
\begin{equation*}
E = \sum_i^{occ} \langle \psi_i | - \frac{\Delta}{2} + V_{ext} + \frac{1}{2} \int' \frac{n(\mathbf{r}')}{|\mathbf{r} - \mathbf{r}'|} | \psi_i \rangle + E_{XC}(n(\mathbf{r})) + \frac{1}{2} \sum_{\alpha, \beta}^N \frac{Z_{\alpha}Z_{\beta}}{|\mathbf{R}_{\alpha} - \mathbf{R}_{\beta}|},
\end{equation*}
donde la primera suma es sobre los autoestados $\psi_i$ ocupados de Kohn-Sham,
$n(\mathbf{r})$ es la densidad electrónica, el segundo término es la contribución
de la correlación de intercambio (XC), y el último término considera la repulsión
de ion-ion. Si se utiliza una densidad de referencia $n_0$ más un término 
pequeño de fluctuación $\delta n$ y se expande $E_{XC}$ a la densidad de 
referencia:
\begin{equation}\label{eq:dft-fluc}
    \begin{aligned}
        E =& \sum_i^{occ} \langle \psi_i | \hat{H}_0 | \psi_i \rangle - \frac{1}{2} \int \int' \frac{n_0' n_0}{|\mathbf{r} - \mathbf{r}'|} + E_{XC}(n_0) - \int V_{XC}(n_0)n_0 + E_{ii} \\
        &+ \frac{1}{2} \int \int' \left(\frac{1}{|\mathbf{r} - \mathbf{r}'|} + \frac{\delta^2 E_{XC}}{\delta n \delta n'}\bigg\rvert_{n_0} \right)
    \end{aligned}
\end{equation}

\begin{enumerate}
    \item \textbf{Enforque de orden cero}

        El método de DFTB de orden cero calcula los elementos de la matriz 
        Hamiltoniana y de solapamiento a partir de una base orbital local con la
        ayuda de DFT-LDA (DFT-\textit{Local density approximation}) y algunas
        aproximaciones en las integrales. Puede verse como una aproximación de 
        una combinación lineal de los orbitales atómicos (LCAO, de sus siglas en 
        ingles, \textit{linear-combination-of-atomic-orbitals}). De esta forma se
        busca evitar las dificultades que surgen a la hora de parametrizar un 
        potencial empírico ~\cite{dftb1, dftb2}.

        En esta aproximación, las ecuaciones de Kohn-Sham son resultas de una
        forma no consistente, ignorando el último término de la ecuación 
        \ref{eq:dft-fluc} y expandiendo los orbitales de Kohn-Sham $\psi_i$ del 
        sistema en términos de las funciones de la base localizadas centradas en 
        el átomo,
        \begin{equation*}
        \psi_i = \sum_{\nu} C_{\nu i} \phi_{\nu}(\mathbf{r}-\mathbf{R}_k),
        \end{equation*}
        resolviendo las ecuaciones de Kohn-Sham para un potencial efectivo de una
        partícula $V_{eff}(\mathbf{r})$,
        \begin{equation}\label{eq:kohn-sham-mod}
            \hat{H}_0 \psi_i(\mathbf{r}) = \varepsilon_i \psi_i(\mathbf{r}), \quad \hat{H}_0 = \hat{T} + V_{eff}(\mathbf{r}),
        \end{equation}
        se tiene como resultado un conjunto de ecuaciones algebraicas,
        \begin{equation}\label{eq:alg-eq}
        \sum_{\nu} C_{\nu i} (H_{\mu \nu} - \varepsilon S_{\mu \nu}) = 0, \quad \forall \mu, i,
        \end{equation}
        donde
        \begin{equation*}
        H_{\mu \nu} = \langle \phi_{\mu}|\hat{H}_0|\phi_{\nu} \rangle, \quad S_{\mu \nu} = \langle\phi_{\mu}|\phi_{\nu}\rangle.
        \end{equation*}

        La energía total del sistema puede ser aproximada como una suma sobre la
        energía de la estructura de bandas y un potencial repulsivo de dos cuerpos
        de corto alcance,
        \begin{equation*}
            \begin{aligned}
                   E_{tot}(\{\mathbf{R}_k\}) &= E_{BS}(\{\mathbf{R}_k\}) + E_{rep}(\{|\mathbf{R}_k - \mathbf{R}_l|\}) \\
                    &= \sum_i n_i \varepsilon_i(\{\mathbf{R}_k\}) + \sum_k \sum_{<l} V_{rep}(|\mathbf{R}_l - \mathbf{R}_k|),
            \end{aligned}
        \end{equation*}
        donde $n_i$ es el número de ocupación del orbital $i$.
        
        Las funciones de onda pseudoatómicas pueden escribirse en términos de los
        orbitales tipo Slater y armónicos esféricos,
        \begin{equation*}
        \phi_{\nu}(\mathbf{r}) = \sum_{n,\alpha,l_{\nu},m_{\nu}} a_{n\alpha} r^{l_{\nu}+n} e^{-\alpha r} Y_{l_{\nu}m_{\nu}}\left(\frac{\mathbf{r}}{r}\right),
        \end{equation*}
        a la hora de realizar una solución auto-consistente a las ecuaciones
        modificadas de Kohn-Sham \ref{eq:kohn-sham-mod}. Estas soluciones son
        utilizadas como funciones de la base LCAO a la hora de tratar el sistema
        sólo considerando los orbitales de valencia.
        
        Como una aproximación, se escribe el potencial de un electrón de una 
        estructura con muchos átomos como una suma de contribuciones atómicas
        esféricas,
        \begin{equation*}
        V_{eff}(\mathbf{r}) = \sum_k V_0^k(|\mathbf{r} - \mathbf{R}_k|),
        \end{equation*}
        donde $V_0$ es el potencial de Kohn-Sham de un pseudo-átomo neutral.

        La matriz de solapamiento consiste solamente de dos elementos centrales
        y puede ser calculada de una forma sencilla
        \begin{equation*}
            H_{\mu\nu}^0 = 
            \begin{cases*}
                \varepsilon_{\mu}^{atomo\ libre} & si $\mu = \nu$ \\
                \langle \phi_{\mu}^{\alpha} | \hat{T} + V_0^{\alpha} + V_0^{\beta} | \phi_{\nu}^{\beta} \rangle & si $\alpha \neq \beta$ \\
                0 & para el resto de los casos,
            \end{cases*}
        \end{equation*}
        donde los índices $\alpha$ y $\beta$ indican el átomo sobre el cual la
        función de onda y el potencial están centrados, sólo dos elementos de la 
        matriz Hamiltoniana son tratados. Debido a que todos los elementos de la 
        matriz dependen sólo de las distancias interatómicas, sólo es necesario
        calcularlos una vez para cada par de tipo de átomo y guardar los valores
        definiendo un ancho de paso. Luego, los elementos para distancias 
        intermedias pueden ser interpolados entre los valores guardados.

        La repulsión a corto alcance $V_{rep}(R)$ puede ser determinada a partir
        de la diferencia en la energía total resultante de un cálculo 
        auto-consistente, $E_{LDA}^{sc}$, y $E_{BS}$ para distintos valores de 
        distancia $R$,
        \begin{equation*}
        V_{rep}(R) = E_{LDA}^{sc}(R) - E_{BS}(R).
        \end{equation*}

        Por último, las fuerzas interatómicas pueden ser derivadas de manera
        explícita para utilizarlas en dinámica molecular de la siguiente forma
        \begin{equation*}
        \mathbf{F}_{\alpha} = - \sum_i n_i \sum_{\mu} \sum_{\nu} C_{\mu i} C_{\nu i} \left(\frac{\partial H_{\mu \nu}^0}{\partial \mathbf{r}_{\alpha}} - \varepsilon \frac{\partial S_{\mu \nu}}{\partial \mathbf{r}_{\alpha}}\right) - \sum_{\beta \neq \alpha} \frac{\partial E_{rep}(|\mathbf{r_{\alpha} - \mathbf{r}_{\beta}|)}}{\partial \mathbf{r}_{\alpha}}.
        \end{equation*}

    \item \textbf{Enfoque de segundo orden}

        En una aproximación de segundo orden se agrega un término, además de
        el usual correspondiente a la \say{estructura de bandas} y el 
        potencial repulsivo de corto alcance, que considera la energía de
        interacción a largo alcance de Coulomb entre las fluctuaciones de la 
        carga a través una redistribución auto-consistente de las cargas de 
        Mulliken (SCC) ~\cite{dftb3}.

        Ahora sí se considera el último término de la ecuación \ref{eq:dft-fluc}
        al descomponer $\delta n(\mathbf{r})$ en contribuciones centradas en el
        átomo, entonces el término de segundo orden queda
        \begin{equation}\label{eq:q1}
        E_{2nd} = \frac{1}{2} \sum_{\alpha, \beta}^N \int \int' \Gamma(\mathbf{r}, \mathbf{r}', n_0) \delta n_{\alpha}(\mathbf{r}) \delta n_{\beta}(\mathbf{r}'),
        \end{equation}
        donde $\Gamma$ denota los coeficiente Hartree y XC. $\delta n_{\alpha}$
        puede ser extendida en una serie de funciones radiales y angulares,
        \begin{equation*}
            \begin{aligned}
                \delta n_{\alpha}(\mathbf{r}) &= \sum_{l,m} K_{ml} F_{ml}^{\alpha}(|\mathbf{r} - \mathbf{R}_{\alpha}|) Y_{lm} \left(\frac{\mathbf{r}-\mathbf{R}_{\alpha}}{|\mathbf{r}-\mathbf{R}_{\alpha}|}\right) \\
                &\approx \Delta q_{\alpha} F_{00}^{\alpha}(|\mathbf{r} - \mathbf{R}_{\alpha}|) Y_{00},
            \end{aligned}
        \end{equation*}
        donde $F_{ml}^{\alpha}$ denota la dependencia radial normalizada de la
        fluctuación de la densidad en el átomo $\alpha$ para el momento angular
        correspondiente. Esta expresión preserva la carga total del sistema.
        Si a la misma se la sustituye en la ecuación \ref{eq:q1},
        \begin{equation*}
        E_{2nd} = \frac{1}{2} \sum_{\alpha,\beta}^N \Delta q_{\alpha} \Delta q_{\beta} \gamma_{\alpha\beta},
        \end{equation*}
        donde
        \begin{equation*}
        \gamma_{\alpha\beta} = \int \int' \Gamma(\mathbf{r},\mathbf{r}',n_0)\frac{F_{00}^{\alpha}(|\mathbf{r} - \mathbf{R}_{\alpha}|)F_{00}^{\beta}(|\mathbf{r} - \mathbf{R}_{\beta}|)}{4 \pi}
        \end{equation*}
        se introduce como abreviatura. En el límite de distancias interatómicas
        grandes, $E_{2nd}$ puede ser vista como una interacción de Coulomb pura
        entre dos cargas puntuales $\Delta q_{\alpha}$ y $\Delta q_{\beta}$. Una
        aproximación simple y muy utilizada en métodos de química cuántica 
        semi-empíricos es aproximar el valor de $\gamma_{\alpha\alpha}$ por la 
        diferencia potencial de ionización atómica y la afinidad electrónica, que
        está relacionada al parámetro de Hubbard $U_{\alpha}$, 
        $\gamma_{\alpha\alpha} \approx U_{\alpha};$ luego, $\gamma_{\alpha\beta}$ 
        sólo depende de la distancia entre los átomos $\alpha$ y $\beta$ y los 
        parámetros $U_{\alpha}$ y $U_{\beta}$, que pueden ser calculados a partir 
        de la segunda derivada de la energía total de un solo átomo con respecto 
        al número de ocupación del último orbital atómico ocupado en LDA-DFT.

        Por último, en este enfoque de segundo orden, la ecuación 
        \ref{eq:dft-fluc} puede escribirse como
        \begin{equation*}
        E_2^{TB} = \sum_i^{occ} \langle \psi_i | \hat{H}_0 | \psi_i \rangle + \frac{1}{2} \sum_{\alpha, \beta}^N \gamma_{\alpha\beta} \Delta q_{\alpha} q_{\beta} + E_{rep}.
        \end{equation*}

        Para estimar las fluctuaciones de la carga se implementa el análisis de
        Mulliken
        \begin{equation*}
        q_{\alpha} = \frac{1}{2} \sum_i^{occ} n_i \sum_{\mu \in \alpha} \sum_{\nu}^N (C_{\mu i}^{*} C_{\nu i} S_{\mu nu} + C_{\nu i}^{*} C_{\mu i} S_{\nu mu})
        \end{equation*}
        y se obtiene un sistema de ecuaciones algebraicas como en la ecuación
        \ref{eq:alg-eq} pero donde el Hamiltoniano ahora considera una 
        corrección debido a la fluctuación de las cargas $H_{\mu \nu}^1$,
        \begin{equation*}
            \begin{aligned}
                H_{\mu \nu} &= \langle \phi_{\mu} | \hat{H}_0 | \phi_{\nu} \rangle + \frac{1}{2}\sum_{\xi}^N (\gamma_{\alpha\xi} + \gamma{\beta\xi}) \Delta q_{\xi} \\
                &= H_{\mu\nu}^0 + H_{\mu\nu}^1, \quad S_{\mu\nu} = \langle \phi_{\mu} | \phi_{\nu} \rangle, \quad \forall \mu \in \alpha, \quad \nu \in \beta.
            \end{aligned}
        \end{equation*}

        En este nuevo enfoque las fuerzas interatómicas para utilizar en 
        simulaciones de dinámica molecular vienen dadas por
        \begin{equation*}
        \mathbf{F}_{\alpha} = - \sum_i^{occ} n_i \sum_{\mu\nu} C_{\mu i} C_{\nu i} \left[\frac{\partial H_{\mu\nu}^0}{\partial \mathbf{r}_{\alpha}} - \left(\varepsilon_i - \frac{H_{\mu\nu}^1}{S_{\mu\nu}}\right) \frac{\partial S_{\mu\nu}}{\partial \mathbf{r}_{\alpha}} \right] - \Delta q_{\alpha} \sum_{\xi}^N \frac{\partial \gamma_{\alpha\xi}}{\partial \mathbf{r}_{\alpha}} \Delta q_{\xi} - \frac{\partial E_{rep}}{\partial \mathbf{r}_{\alpha}}.
        \end{equation*}
 
\end{enumerate}
