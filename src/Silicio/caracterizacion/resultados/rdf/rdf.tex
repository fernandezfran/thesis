% Copyright (c) 2024, Francisco Fernandez
% License: CC BY-SA 4.0
%   https://github.com/fernandezfran/thesis/blob/main/LICENSE
\subsection{Distribución radial de a pares}

La distribución radial de a pares, introducida en la sección \ref{ss:rdf},
puede ser utilizada para describir la estructura de materiales amorfos. Para 
considerar las distintas $g(r)$ (Li-Li, Si-Si y Si-Li) se utiliza la 
ecuación \ref{eq:prdf}. En la Figura \ref{fig:rdf} se muestran los resultados 
obtenidos para las RDF de Li-Li, Si-Si y Si-Li. En cada una de ellas se analizan
los cambios en la estructura que se dan para los distintos valores de $x$ en 
Li$_x$Si estudiados, las curvas se calculan sobre las estructuras minimizadas 
de la HD.
\begin{figure}[h!]
    \centering
    \includegraphics[width=\textwidth]{Silicio/caracterizacion/resultados/rdf/rdf.png}
    \caption{Distribución radial de a pares parciales de Li-Li, Si-Li y Si-Si 
    para las estructuras minimizadas. Cada curva se corresponde con un valor de 
    concentración distinto.}
    \label{fig:rdf}
\end{figure}

Lo más relevante a destacar de la RDF$_{\text{Li}-\text{Li}}$ es que su primer 
pico comienza 
centrado en 2.45 \AA\ para las concentraciones de iones de Li más bajas y que 
luego dicho pico aumenta su posición a distancias más grandes a medida que aumenta
$x$ hasta permanecer en 2.95 \AA\ para $x$ mayores a 1.71. La altura de este pico
aumenta en un 50\% luego de la litiación completa, relativa a la menor 
concentración.

Este mismo efecto se ve en el primer pico de la RDF$_{\text{Si}-\text{Si}}$, el 
centro del mismo
se encuentra en 2.4 \AA\ para $x = 0.21$ y luego se desplaza a distancias
mayores, después de $x = 1.25$ el centro se encuentra entre 2.52 y 2.56 \AA.
Mientras que la altura del pico aumenta, se ve un decrecimiento en el ancho 
del pico, el valor del FWHM va de 0.14 \AA\ a 0.05 \AA\ para $x = 0.21$ y 
$x = 3.75$, respectivamente. Por otro lado, el segundo pico de la RDF$_{\text{Si}-\text{Si}}$
también se desplaza hacia distancias mayores, se divide en dos picos para valores 
de $x$ entre 0.62 y 1.71 y vuelve a comportarse como un sólo pico para 
concentraciones mayores. Entre el primer y el segundo pico se observa un hombro,
como señalaron previamente Fan \textit{et al.} \cite{fan2013}.% Los resultados 
%obtenidos para la RDF$_{Si-Si}$ están en concordancia con las mediciones 
%experimentales reportadas por Key \textit{et al.} \cite{key2011}.

Para el primer pico de la RDF$_{\text{Si}-\text{Li}}$ se ve el comportamiento contrario, el 
centro del mismo se desplaza a distancias menores a medida que la concentración
de litio aumenta. Esto es acompañado con un aumento de la altura del pico y una
disminución de su ancho. Para el segundo pico también se observa un desplazamiento
del mismo hacia distancias menores, pero por encima de $x = 1.71$ el pico se
divide en dos picos con distintas alturas dependiendo de la concentración. Esto
es analizado con mayor detalle en la sección \ref{s:interconexion}.
