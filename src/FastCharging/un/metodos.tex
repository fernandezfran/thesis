% Copyright (c) 2024, Francisco Fernandez
% License: CC BY-SA 4.0
%   https://github.com/fernandezfran/thesis/blob/main/LICENSE
\section{Métodos computacionales}

\subsection{Construcción de los diagramas con el modelo de una sola partícula}

El modelo de una sola partícula introducido en la sección \ref{s:metodologia}
proporciona la respuesta de los perfiles galvanostáticos en función de dos parámetros
adimensionales $\Xi$ y $\ell$. Esto permite construir diagramas universales. Para 
este fin se definió una grilla logarítmica de $N$ puntos en el espacio 
$\log(\Xi)$--$\log(\ell)$ y se fijó un potencial de corte. Para cada punto en la 
grilla se realizó la simulación y se guardó el valor de intercalación de litio
alcanzado, denotado por $\text{SOC}_{\max}$. Para la construcción de los diagramas 
se simularon algunos miles de puntos que luego fueron suavizados mediante un spline
que también se usa para interpolar.


\subsection{Un modelo heurístico para ajustar sistemas complejos}

El modelo introducido en la sección \ref{s:metodologia} 
se presenta como una alternativa para tener una descripción universal del 
comportamiento de distintos materiales, ya que permite obtener la capacidad en 
condiciones de cargado galvanostático en electrodos de una sola partícula en 
función de 4 parámetros: la C-rate, el tamaño de partícula, el coeficiente de 
difusión y la constante cinética.

Algunas de las suposiciones del modelo pueden no cumplirse en muchos experimentos; sin embargo, el 
mismo puede ser utilizado como una guía heurística para entender el 
comportamiento global de los distintos materiales y para diseñar los materiales
activos, en lo que respecta a su tamaño de partícula medio. La obtención de dicho
comportamiento universal sólo puede obtenerse al resignar a las particularidades
de cada sistema. 

El modelo heurístico basado en simulaciones galvanostáticas, que se presenta a 
continuación, tiene asociado un software que fue escrito en Python, 
\path{galpynostatic}, es open-source y fácil de usar, el mismo se describe en el
Apéndice \ref{software:galpynostatic}. Este modelo pretende obtener información
cinética de una manera simple y rápida, además, también tiene por objeto predecir el 
tamaño óptimo de la partícula para lograr una carga del 80\% en 15 minutos.
El mismo consiste en la utilización de los diagramas universales, como el que 
se presenta en la Figura \ref{fig:diagnostico}, para determinar los valores de
$\Xi$ y $\ell$ que mejor se ajustan a datos experimentales. Esto se logra al 
realizar una búsqueda de grilla con distintas combinaciones posibles del 
coeficiente de difusión $D$ y de la constante cinética $k^0$, manteniendo 
constantes los otros parámetros involucrados en $\Xi$ y $\ell$, como el tamaño 
de la partícula $d$ y el factor geométrico $z$. Luego, para cada valor de C-rate
considerado se tiene el valor experimental del SOC$_{\max}$ y se lo compara con 
el predicho.

Para cada elección de parámetros ($D$, $k^0$) se calcula en valor de 
$\text{SOC}_{\max}$ con el diagrama y se obtiene el error cuadrático medio (MSE,
\textit{mean square error}) referido a los valores experimentales de la siguiente
manera
\begin{equation}
    \text{MSE} = \frac{1}{n} \sum_{i=1}^n \left(\text{SOC}_{\max,i}^{\text{exp}} - \text{SOC}_{\max,i}^{\text{pred}}\right)^2,
\end{equation}
donde $n$ es el número de mediciones, SOC$_{\max, i}^{\exp}$ y 
SOC$_{\max, i}^{\text{pred}}$ son los valores experimentales y predichos por el 
modelo, respectivamente, de la capacidad máxima alcanzada. Luego de una 
exploración exhaustiva, se obtiene el conjunto de parámetros que minimizan el MSE. 
Esta exploración provee los valores de $D$ y $k^0$ que caracterizan a un material de 
intercalación en particular. Con esta información se puede predecir el tamaño 
óptimo de la partícula para que el sistema alcance una capacidad deseada a una dada 
C-rate.
