\section{Campo de fuerzas}

El campo de fuerzas de Fan \textit{et al.} ha sido ampliamente utilizado en 
simulaciones de MD para estudiar el proceso de litiación de distintas estructuras
de silicio, desde estructuras periódicas a nanoestructuras. Previo a la 
realización del trabajo de este capítulo, se consultó la bibliografía para 
verificar esto y asegurar de la transferibilidad del potencial. 

Además de los resultados reportados por Fan \textit{et al.} ~\cite{fan2013}, 
la estructura, el estrés y la difusividad fue estudiada durante la litiación de 
Si amorfo (a-Si) y Si cristalino (c-Si) en diferentes orientaciones 
cristalográficas ~\cite{chen2020, kim2015}. Ding \textit{et al.} ~\cite{ding2017} 
reportaron la variación de la velocidad de migración en la frontera de fases y la 
difusividad de Li en función del estrés externo aplicado, demostrando que la 
tensión acelera la velocidad de litiación, mientras que la compresión la retarda. 
Kim \textit{et al.} ~\cite{kim2014} realizaron simulaciones de MD para caracterizar 
la evolución estructural de la frontera de fases entre c-Si, con diferentes planos 
de orientación, con una fase amorfa de litiación máxima. Posteriormente, Fan 
\textit{et al.} ~\cite{fan2018} estudiaron nanoestructuras, computando la respuesta
mecánica de nanopilares de c-Si en la orientación (111) durante la litiación.
Un trabajo similar, pero para la orientación (100), fue realizado por Cao 
\textit{et al.} ~\cite{cao2019}. Tang \textit{et al.} ~\cite{tang2019} investigaron
la evolución y la permanencia de la porosidad de nanocapas de Si durante los 
procesos de litiación y delitiación. Ostadhossein \textit{et al.} 
~\cite{ostadhossein2015} caracterizaron la litiación de nanohilos de c-Si y mostró
que este potencial ReaxFF reproduce en forma precisa las barreras de energía de 
la migración de Li obtenidas por DFT, tanto en c-Si como en a-Si.

La aplicación de este potencial no estuvo sólo limitada a simulaciones de MD, 
sino que fue empleado en otros métodos de simulación, por ejemplo, simulaciones 
de Monte Carlo en el ensamble gran canónico, que fueron realizadas para estudiar 
un ciclo de litiación y delitiación de un electrodo de a-Si ~\cite{basu2019}. 
Trochet y Mousseau ~\cite{trochet2017} caracterizaron el paisaje energético a 
concentraciones relativamente bajas de impurezas de Li en c-Si, usando una 
técnica de activación-relajación cinética. Kim \textit{et al.} ~\cite{kim2017} 
desarrollaron un algoritmo para investigar la respuesta a la delitiación de una 
capa delgada de silicio recubierta de óxido de aluminio. El ReaxFF también fue 
combinado con otros campos de fuerza, como los potenciales de Tersoff y 
Lennard-Jones, para simular la litiación de nanopartículas de Si recubiertas con 
carbono, que permitieron observar una correlación entre el crecimiento del 
estrés y la densidad de carga ~\cite{zheng2019,zheng2020}. Propiedades mecánicas 
de interfase Si/SiO$_2$ litiada fueron reportadas por Verners y 
Simone ~\cite{verners2019}. 

No es posible llevar a cabo un estudio sobre las propiedades electrónicas con el 
uso del ReaxFF, ya que esta es una de sus limitaciones. Sin embargo, de la 
discusión previa, puede observarse que ha sido capaz de predecir un número 
importante de propiedades del sistema Li-Si.
