\section{Objetivos y estructura de la tesis}

Esta tesis tiene como objetivo estudiar materiales que se utilicen para el 
desarrollo de electrodos de baterías de ion-litio de próxima generación mediante 
distintos modelados computacionales. 
La misma se encuentra dividida en tres partes, la primera de ellas sobre la 
Motivación y fundamentos consistente de dos capítulos, el capítulo 
\ref{ch:introduccion} con esta introducción y el capítulo \ref{ch:metodos} con la
descripción de los distintos métodos computacionales utilizados. 
La Parte \ref{p:fast-charging} se divide en dos capítulos, ambos relacionados con 
la carga rápida de baterías de ion-litio. En el capítulo \ref{ch:un} se 
desarrolla un modelo para ajustar datos experimentales en condiciones 
galvanostáticas y predecir el tamaño óptimo de partículas que permite retener un 
80 \% de su capacidad frente a una carga realizada en 15 minutos 
\cite{fernandez2023towards}. El capítulo \ref{ch:umbem} busca una métrica 
universal que permita estandarizar las comparaciones del desempeño entre 
distintos materiales considerados en aplicaciones de carga rápida.
La Parte \ref{p:silicio} se centra en el estudio de las aleaciones presentes en 
los ánodos de silicio y se divide en tres capítulos. El capítulo 
\ref{ch:caracterizacion} caracteriza las estructuras de Li-Si encontradas con 
un potencial reactivo y con un método de exploración acelerada de mínimos locales
propuesto \cite{fernandez2021characterization}. En el capítulo \ref{ch:modelo} se
parametriza un modelo DFTB (\textit{denstity functional tight-binding}) para la 
interacción Li-Si mediante un algoritmo que asigna pesos a las distintas 
estructuras consideradas para el ajuste \cite{oviedo2023}. En el capítulo 
\ref{ch:prediccion} se proponen modelos de vecinos más cercanos para predecir 
mediciones de rayos x, RMN y Mössbauer a partir de las configuraciones atómicas
\cite{fernandez2023nmr}.
Cada uno de los capítulos mencionados en estas dos últimas partes se componen
de una introducción y detalles de los métodos computacionales utilizados, los 
resultados junto a las discusiones de los mismos y conclusiones parciales.
Por último, se cierra la tesis con el capítulo \ref{ch:comentarios} con los 
comentarios finales de la misma.
