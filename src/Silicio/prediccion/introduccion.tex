% Copyright (c) 2024, Francisco Fernandez
% License: CC BY-SA 4.0
%   https://github.com/fernandezfran/thesis/blob/main/LICENSE
\section{Introducción}

Conocer la estructura de los materiales activos de las baterías a un nivel atómico permite diseñar
estrategias que mitigan sus
limitaciones y mejoran considerablemente su desempeño \cite{liu2019review}.
Esto ha inspirado a la comunidad científica a aplicar numerosas técnicas 
de caracterización microscópicas y espectroscópicas. El comportamiento 
intrínseco de los ánodos de Si lleva a la formación de aleaciones amorfas de 
Li-Si durante la carga/descarga del mismo. Esto convierte a la estructura
de corto alcance especialmente relevante. Se ha afirmado que la transición de 
fase cristal a amorfo que ocurre en este sistema representa el principal 
obstáculo para mejorar su desempeño electroquímico, principalmente porque 
dificulta los intentos de relacionar las estructuras atómicas con las 
observaciones experimentales \cite{key2011}. Aunque hay experimentos 
relacionados con la estructura local como la resonancia magnética nuclear (RMN),
la espectroscopia Mössbauer (MB), la función distribución radial de a pares 
(PDF) de rayos x, entre otras, su interpretación es evasiva sin un modelo 
teórico preciso capaz de predecir las estructuras microscópicas del sistema
y correlacionarlas con los observables experimentales. Como ejemplo, siempre 
se observan dos estados de iones de litio, y una transición intermedia que aún 
no ha sido dilucidada, en voltagramas \cite{pan2019}, coeficientes de 
difusión \cite{ding2009} y experimentos de RMN \cite{key2009}.

En el capítulo \ref{ch:modelo} se parametrizó un potencial DFTB que exhibió 
una precisión notable en la predicción de energías de formación de 
estructuras cristalinas y amorfas de Li$_x$Si en un rango amplio de 
composiciones. Aún más, utilizando este potencial, un templado simulado simple
de una estructura de Si cristalino (c-Si) resultó en una estructura amorfa 
(a-Si) que presentó una concordancia excelente con resultados experimentales
\cite{laaziri1999}. Teniendo en cuenta estos resultados, en este capítulo se realiza
la litiación de dicha estructura y se analizan las configuraciones atómicas obtenidas 
con modelos de vecinos más cercanos propuestos aquí para 
interpretar mediciones de RMN \cite{ogata2014, key2011, koster2011, key2009}, 
MB \cite{li2009} y PDF de rayos x \cite{key2011, laaziri1999}.
