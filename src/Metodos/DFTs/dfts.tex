\section{Mecánica cuántica}

El desarrollo de la mecánica cuántica, y las observaciones experimentales que la
validan, fue uno de los avances científicos más significativos del siglo veinte
en describir el universo en el que vivimos \cite{shankar2012}.

La aproxmiación de Born-Oppenheimer permite separar los núcleos de los electrones 
en problemas de mecánica cuántica. Esta aproximación se basa en que los núcleos 
atómicos son mucho más pesados que los electrones individuales. Eso implica que 
puede considerarse que los electrones responden mucho más rápido a los cambios en 
su entorno que los núcleos, lo que permite resolver las ecuaciones que describen 
el movimiento de los electrones para posiciones fijas de los núcleos atómicos. Así 
se encuentra el estado fundamental de los electrones, es decir, la configuración 
de menor energía de los mismos. Esto permite estudiar como varía la energía de los 
materiales con el movimiento de sus átomos.

La ecuación de Schrödinger no-relativista e independiente del tiempo,
\begin{equation}\label{eq:schrodinger}
    H \psi = E \psi,
\end{equation}
caracteriza un sistema físico desde un punto de vista cuántico, donde $\psi$ es un
conjunto de soluciones, o autoestados, del Hamiltoniano $H$ que tienen asociados los
autovalores $E$ que satisfacen dicha ecuación. Para el caso en el que se desee estudiar
la interacción entre varios electrones y núcleos una descripción más completa de la 
ecuación de Schrödinger es la siguiente,
\begin{equation}\label{eq:schrodinger}
    \left[ - \frac{\hbar^2}{2 m} \sum_{i=1}^N \nabla_i^2 + \sum_{i=1}^N V(\mathbf{r}_i) + \sum_{i=1}^N \sum_{i<j} U(\mathbf{r}_i, \mathbf{r}_j) \right] \psi = E \psi,
\end{equation}
donde $\hbar$ es la constante de Plank reducida, $m$ la masa del electrón, el primer 
término describe la energía cinética de cada electrón, el segundo la energía de 
interacción de cada electrón con todos los núcleos y el tercero la energía de 
interacción entre distintos electrones. Este último término es el más crítico desde 
el punto de vista de la resolución de la ecuación.


\subsection{Teoría del funcional de la densidad (DFT)}

La teoría del funcional de la densidad (DFT por sus siglas en inglés
\textit{Density functional theory}) es un método altamente efectivo para 
encontrar soluciones a la ecuación fundamental que describe el comportamiento 
cuántico de átomos y moleculas en sistemas de materia condensada, la ecuación 
\ref{eq:schrodinger} de Schrödinger, en situaciones de utilidad práctica 
\cite{sholl2022}.

\subsection{Funcional de la densidad de enlace estrecho (DFTB)}\label{s:dftb}

\chapter{Software desarrollado}

En la última década, Python se ha convertido en un lenguaje de programación 
importante dentro de la comunidad científica debido a su facilidad de uso y 
versatilidad en la manipulación y visualización de datos ~\cite{millman2011}. 
Por lo tanto, los software diseñados en esta tesis han sido escritos en este
lenguaje y construidos sobre las librerías usuales del cómputo científico como
NumPy \cite{numpy}, SciPy \cite{scipy}, pandas \cite{pandas}, 
matplotlib \cite{matplotlib} y scikit-learn \cite{sklearn1, sklearn2}. 


\section{Control de calidad de software}

El control de calidad del software hace referencia al conjunto de reglas y 
procedimientos que deben utilizarse para verificar que el software cumple 
determinados estándares de calidad subjetivos. Un procedimiento habitual son las 
pruebas unitarias (\textit{unit testing} en inglés), que consisten en aislar una 
función del código y comprobar que funciona como se espera \cite{jazayeri2007}. 
Otro procedimiento habitual se define a partir de este y es el 
\textit{code-coverage}, que determina que proporción del software se ha testeado
\cite{miller1963}. El estilo y la legibilidad del código también es importante
y aquí se ha seguido la guía de estilo PEP8 de Python, la misma se asegura con 
la herramienta flake8. Además, los mismos fueron desarrollados utilizando control 
de versiones git y distribuidos bajo la Licencia MIT, fomentando su uso tanto en 
entornos académicos como comerciales. Todo esto se realizó buscando que el 
software sea fácil de mantener y que respete los estándares de la comunidad Python.


\section{galpynostatic}

Este paquete denominado \path{galpynostatic} fue escrito para la utilización
del modelo heurístico presentado en el capítulo TODO. El mismo distribuye los 
datos de los diagramas galvanostáticos, un módulo de preprocesamiento de datos
para obtener capacidades de descarga a un potencial de corte dado a partir de 
medidas de perfiles galvanostáticos y una clase que realiza la regresión sobre la 
superficie y permite diferentes tipos de gráficos y estimaciones de parámetros.

A continuación se muestra un ejemplo de uso:

\lstinputlisting[language=Python]{apendices/ejemplo_galpynostatic.py}

A \path{galpynostatic} se le realizan múltiples pruebas unitarias sobre datos de
de electrodos actuales y de materiales de investigación de próxima generación en 
baterías de litio, el \textit{coverage} del mismo alcanza el 100\% del software
en su versión inicial.

Por último, el código fuente está disponible en un repositorio público 
(\url{https://github.com/fernandezfran/galpynostatic}) y todos los nuevos cambios 
confirmados en este repositorio se prueban automáticamente con el servicio de 
integración continua de GitHub Actions. También se genera una documentación a 
partir de los docstrings del código, junto con una guía de instalación,
tutoriales y ejemplos con aplicaciones reales, que se hacen públicos en el 
servicio read-the-docs (\url{https://galpynostatic.readthedocs.io/en/latest/}). 
Además, \path{galpynostatic} está disponible para su instalación en el Python 
Package-Index (\url{https://pypi.org/project/galpynostatic/}).

%
%\section{galpynostatic.metric}
%
%TODO
%
%
%\section{macchiato}
%
%TODO
%
%Ejemplo de cálculo del corrimiento químico de la estructura cristalina 
%Li$_{13}$Si$_{4}$ mediante el uso del modelo a primeros vecinos introducido en 
%el capítulo TODO:
%\lstinputlisting[language=Python]{apendices/ejemplo_macchiato.py}
%
%
%\section{Otros códigos}
%
%\subsection{sierras}
%Con \path{sierras} se automatiza el proceso de ajustar la ecuación de Arrhenius 
%(\ref{eq:arrhenius}) en procesos difusivos y la obtención de información a partir 
%de la misma. Este código también se encuentra público 
%(\url{https://github.com/fernandezfran/sierras}) y cada vez que se realiza un 
%cambio se llevan a cabos tests unitarios sobre distintos datos extraídos de 
%literatura. Un ejemplo simple de como se utiliza se presenta a continuación:
%
%\lstinputlisting[language=Python]{apendices/ejemplo_sierras.py}
%
%La versión inicial de este código fue presentada como trabajo final en la materia
%\href{https://github.com/leliel12/diseno_sci_sfw}{\tt Diseño de software para 
%cómputo científico.}
%
%\subsection{exma}
%Para analizar trayectorias de dinámica molecular, dentro de la comunidad de Python
%se encuentra la librería \path{MDAnalysis} \cite{mdanalysis1, mdanalysis2}. Sin
%embargo, no todos los observables descriptos en la sección \ref{s:observables}
%han sido implementados, por ejemplo, no se puede calcular directamente el número
%de coordinación. Para ello se escribió \path{exma}
%(\url{https://github.com/fernandezfran/exma}), que además de contar con esta
%implementación presenta distintas facilidades para computar observables 
%electroquímicos presentados en distintos capítulos de esta tesis. Algunas partes
%de este software han sido escritas en \path{C}, para tener mayor velocidad de 
%cálculo, y tienen una interfaz en Python para ser utilizadas.
%
%\subsection{aelm}
%Para el método de exploración acelerada de mínimos locales, introducido en 
%la sección \ref{s:aelm} del capítulo \ref{ch:caracterizacion}, se escribió un 
%módulo de Python, \path{aelm} (\url{https://github.com/fernandezfran/aelm}), que 
%toma una trayectoria de una dinámica molecular sesgada, minimiza cada uno de los 
%frames con algún programa de dinámica molecular a elección (\path{LAMMPS} o 
%\path{GEMS}, por ejemplo) y guarda las configuraciones atómicas y las energías 
%para su posterior análisis, como se realizó en el capítulo 
%\ref{ch:caracterizacion} ¿o TODO?
%
%\subsection{cluster-connections}
%Se escribió un código en \path{C++}, 
%\url{https://github.com/fernandezfran/cluster-connections}, con un algoritmo de
%\textit{clustering} para deconvolucionar numéricamente el segundo pico de la RDF
%según la cantidad de vecinos que interconectan a segundos vecinos, como se analizó
%en la sección \ref{s:interconexion} del capítulo \ref{ch:caracterizacion}, 
%en la figura \ref{fig:interconexiones}.

