\subsubsection{MD a temperatura constante: Termostato de Nosé-Hoover}

Debido a que la temperatura es proporcional al promedio de las velocidades al 
cuadrado, como puede verse en la ecuación \ref{eq:tempvel}, es posible variar
la temperatura al ajustar la razón a la cual el tiempo progresa 
~\cite{nose1984a}. Por lo cual puede introducirse una nueva variable dinámica
$s$ en el Lagrangiano que reescalee la unidad de tiempo, y se añaden términos 
adicionales para obtener el comportamiento deseado ~\cite{rapaport2004}. Lo que 
que provoca que haya dos variables temporales distintas: el tiempo real, o físico, 
$t'$, y un tiempo escalado, o virtual, $t$; que están relacionados a través de 
sus diferenciales,
\begin{equation*}
dt = s(t') dt'.
\end{equation*}
El Lagrangiano de este sistema extendido se escribe como
\begin{equation*}
\mathcal{L} = \frac{1}{2} m s^2 \sum_i \dot{\mathbf{r}}_i^2 - \sum_{i<j} u(\mathbf{r}_{ij}) + \frac{1}{2} M_s \dot{s}^2 - n_f T \log s,
\end{equation*}
donde $T$ es la temperatura deseada, $n_f = 3N_m + 1$ el número de grados de 
libertad, $M_s$ es una masa que se necesita para construir la ecuación de 
movimiento de la nueva \say{coordenada} $s$ y el punto representa la derivada
con respecto al tiempo virtual. Las ecuaciones de movimiento de Lagrange pueden 
obtenerse de la forma usual y son
\begin{equation*}
\ddot{\mathbf{r}}_i = \frac{1}{m s^2} \mathbf{f}_i - \frac{2 \dot{s}}{s} \dot{\mathbf{r}}_i
\end{equation*}
\begin{equation*}
M_s \ddot{s} = m s \sum_i \dot{\mathbf{r}}_i^2 - \frac{n_f T}{s}
\end{equation*}

Ya que la relación entre t y t' depende de toda la historia del sistema, es decir,
\begin{equation*}
t = \int s(t') dt'
\end{equation*}
es más conveniente si las ecuaciones se transforman a las unidades del tiempo
físico ~\cite{nose1984b, hoover1985}, ahora el punto pasa a representar la 
derivada con respecto al tiempo real, y las ecuaciones de movimiento pueden 
reescribirse como
\begin{equation*}
\ddot{\mathbf{r}}_i = \frac{1}{m} \mathbf{f}_i - \frac{\dot{s}}{s} \dot{\mathbf{r}}_i
\end{equation*}
\begin{equation*}
\ddot{s} = \frac{\dot{s}^2}{s} + \frac{G_1 s}{M_s} 
\end{equation*}
donde $G_1 = m \sum_i \dot{\mathbf{r}}_i^2 - n_f T$. La primera de estas 
ecuaciones de movimiento se asemeja a la ecuación newtoniana convencional con un 
término adicional similar al de la fricción, aunque no se trata de una fricción 
verdadera ya que el coeficiente puede ser de cualquier signo. La segunda ecuación 
define el mecanismo de retroalimentación por el que $s$ varía para regular la 
temperatura.

Si se integra la función partición microcanónica del sistema extendido sobre la
variable $s$ se obtiene la función de partición canónica, lo cual demuestra que
los promedios de equilibrio del sistema físico son los del ensamble canónico a
temperatura $T$ ~\cite{nose1984a}. Sin embargo, la temperatura en sí no es 
constante, pero la retroalimentación negativa que actúa a través de $s$ garantiza
que las fluctuaciones sean limitadas y el valor medio sea igual a $T$.

El Hamiltoniano del sistema extendido
\begin{equation*}
\mathcal{H} = \frac{1}{2} m \sum_i \dot{\mathbf{r}}_i^2 + \sum_{i<j} u(\mathbf{r}_{ij}) + \frac{1}{2} M_s \left(\frac{\dot{s}}{s}\right)^2 + n_f T \log s
\end{equation*}
se conserva debido a que no hay fuerzas externas que dependan del tiempo, lo cual 
proporciona una comprobación útil de la precisión de la solución numérica. El 
Hamiltoniano no tiene significado físico, sus dos primeros términos representan 
la energía del sistema físico, pero su suma es libre de fluctuar.

La $M_s$ no tiene ningún significado físico en particular, simplemente es una 
parte de la técnica computacional de cálculo y su valor debe determinarse 
empíricamente, que, en principio, no afecta a los resultados de equilibrio final, 
pero sí influye en su precisión y confiabilidad. Para variaciones pequeñas de $s$, 
el período característico de las fluctuaciones es ~\cite{nose1984a}
\begin{equation*}
\tau_s = 2 \pi \sqrt{\frac{M_s \langle s \rangle^2}{2 n_f T}}
\end{equation*}
y la simulación debe extenderse a lo largo de muchos de estos períodos para evitar 
que estas fluctuaciones influyan negativamente en los resultados.

De una manera similar a la realizada en el algoritmo de Nosé-Hoover para el 
control de la temperatura, el algoritmo de \textbf{Parrinello-Rahman} permite
obtener una representación correcta del ensamble isotérmico-isobárico al 
realizar un \textbf{control de la presión}, donde se permite una evolución 
temporal del volumen de la caja ~\cite{parrinello-rahman}.
