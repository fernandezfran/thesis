% Copyright (c) 2024, Francisco Fernandez
% License: CC BY-SA 4.0
%   https://github.com/fernandezfran/thesis/blob/main/LICENSE
Distintos termostatos y barostatos fueron utilizados durante esta tesis, los 
mismos se introducen a continuación:
\begin{itemize}
    \item \textbf{Termostato y barostato de Berendsen}:
        Si se considera la ecuación de Langevin \cite{schneider1978}
        \begin{equation}\label{eq:langevin}
            m_i \dot{v_i} = F_i - m_i \gamma v_i + R_i(t),
        \end{equation}
        donde $F_i$ es la fuerza con la que interactúan los átomos entre sí, 
        $\gamma$ la constante de fricción y $R_i(t)$ una variable estocástica con 
        media cero que cumple 
        \begin{equation}
            \left\langle R_i(t) R_j(t+t') \right\rangle = 2m_i \gamma_i k_B T \delta(t') \delta_{ij},
        \end{equation}
        entonces el termostato de Berendsen \cite{berendsen1984} puede derivarse 
        con una elección en particular de la constante constante de fricción 
        $\gamma$ como $\lambda$, e ignorando la variable estocástica, lo cual 
        representa un escaleo de las velocidades,
        \begin{equation}
            v(t) = \lambda v(t),
        \end{equation}
        por paso temporal $t$, donde
        \begin{equation}
            \lambda = \sqrt{1 + \frac{dt}{\tau_T} \left( \frac{T_0}{T} - 1 \right)},
        \end{equation}
        produce un cambio de temperatura igual a
        \begin{equation}
            \frac{dT}{dt} = \frac{1}{\tau_T} (T_0 - T)
        \end{equation}
        donde $T_0$ es la temperatura de referencia y $\tau_T$ es la constante de
        acoplamiento con el baño térmico y tiene las mismas unidades que el paso 
        temporal. Luego, para considerar un acoplamiento con un baño de presión 
        contante, Berendsen \cite{berendsen1984} agrega un término extra a las 
        ecuaciones de movimiento que consideran el cambio de presión
        \begin{equation}
            \frac{dP}{dt} = \frac{1}{\tau_P} (P_0 - P),
        \end{equation}
        donde $P_0$ es la presión de referencia, $P$ la instantánea y $\tau_P$ la 
        constante de acoplamiento. Este comportamiento puede producirse si se 
        realiza un cambio en el virial, escaleando las distancias entre las 
        partículas. Si el volumen ahora cambia como 
        \begin{equation}
            \frac{dV}{dt} = 3 \alpha V,
        \end{equation}
        y se usa que
        \begin{equation}
            \frac{dP}{dt} = - \frac{1}{\beta V} \frac{dV}{dt} = -\frac{3\alpha}{\beta},
        \end{equation}
        donde $\beta$ es la compresibilidad isotérmica y 
        $\alpha = - \beta (P_0 - P) / (3 \tau_P)$, las posiciones de los átomos 
        dentro de la celda de simulación en cada dirección pueden escalearse como
        \begin{equation}
            x = \mu x,
        \end{equation}
        donde
        \begin{equation}
        \mu = \sqrt[3]{1 - \frac{dt}{\tau_P} (P_0 - P)}.
        \end{equation}
    
    \item \textbf{Termostato de Langevin}:
        En este termostato se considera la ecuación \ref{eq:langevin} junto a 
        tres parámetros \cite{kroger2005} para actualizar las velocidades y 
        las posiciones como
        \begin{align}
            a &= \frac{2 - \xi dt}{2 + \xi dt}, \\
            b &= \sqrt{k_B T_0 \xi \frac{dt}{2}}, \\
            c &= \frac{2 dt}{2 + \xi dt}
        \end{align}
        donde $\xi$ es el factor de fricción, es decir, con qué intensidad se 
        acopla el sistema con el baño térmico. Se tiene entonces una 
        actualización de la siguiente manera
        \begin{align}
            v(t+dt) &= a v(t) + b \eta + \frac{f(t+dt)+f(t)}{2m} dt, \\
            r(t+dt) &= r(t) + c v(t) dt
        \end{align}
        donde $\eta$ es una distribución gaussiana de números aleatorios con 
        media 0 y varianza 1. 
    
    \item \textbf{Termostato de Nosé-Hoover}:
        Debido a que la temperatura es proporcional al promedio de las 
        velocidades al cuadrado, como puede verse en la ecuación \ref{eq:tempvel}, 
        es posible variar la temperatura al ajustar la razón a la cual el tiempo 
        progresa \cite{nose1984a}. Por lo cual puede introducirse una nueva 
        variable dinámica $s$ en el Lagrangiano que reescalee la unidad de 
        tiempo, y pueden añadirse términos adicionales para obtener el 
        comportamiento deseado \cite{rapaport2004}. Lo que que provoca que haya
        dos variables temporales distintas: el tiempo real, o físico, $t$, y un 
        tiempo escalado, o virtual, $t'$; que están relacionados a través de sus 
        diferenciales,
        \begin{equation}
            dt' = s(t) dt.
        \end{equation}
        Con el Lagrangiano del sistema extendido pueden obtenerse las ecuaciones 
        de movimiento para esta nueva \say{coordenada} $s$ con masa $M_s$ y 
        evolucionarla junto al sistema de átomos. Ya que la relación entre $t'$ y
        $t$ depende de toda la historia del sistema, es decir,
        \begin{equation}
            t' = \int s(t) dt
        \end{equation}
        es conveniente resolver ecuaciones en las unidades del tiempo físico 
        \cite{nose1984b, hoover1985}.
    
    \item \textbf{Barostato de Parrinello-Rahman}:
        De una manera similar a la realizada en el algoritmo de Nosé-Hoover para 
        el control de la temperatura, el algoritmo Parrinello-Rahman permite
        obtener una representación correcta del ensamble isotérmico-isobárico al 
        realizar un control de la presión, donde se permite una evolución 
        temporal del volumen de la caja \cite{parrinello-rahman}.
\end{itemize}
