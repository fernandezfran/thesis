\section{Contextualización}

El calentamiento global aparece como el mayor problema ambiental de este siglo.
El mismo se refiere al efecto que producen las actividades humanas en el clima, 
como por ejemplo la quema de combustibles fósiles o la deforestación que emiten 
a la atmósfera grandes cantidades de CO$_2$, entre otros gases de efecto 
invernadero. Estos gases absorben la radiación infrarroja emitida por la tierra 
provocando un incremento de la temperatura de la misma que lleva asociado un 
aumento en la frecuencia y la intensidad de eventos climáticos extremos 
\cite{houghton2005}. De acuerdo a el Panel Intergubernamental del Cambio Climático 
\cite{IPCC}, desde la época preindustrial, las actividades humanas han provocado 
aproximadamente 1.0$^{\circ}$C de calentamiento global y al ritmo actual se van 
a sobrepasar los 1.5$^{\circ}$C antes del 2050, un cambio en la temperatura
media que las emisiones previas por sí solas no habían alcanzado. Limitar el 
calentamiento a esta temperatura requiere que se realicen rápidamente cambios 
sin precedentes en la tecnología y en el comportamiento humano. Uno de los 
cambios más importante es el de la matriz energética, en la cual las energías 
renovables deberán suministrar alrededor del 80 \% de la energía para 2050, donde 
los vectores energéticos, como las baterías de litio, juegan un rol fundamental 
debido a la intermitencia de estas formas de generación de energía.

El litio es el metal más liviano de la tabla periódica y uno de los elementos más
importantes dentro de los minerales necesarios en la producción de baterías de
litio. En particular, para la Argentina tiene un interés económico, social, 
industrial y tecnológico ya que es uno de los países que integran, junto a 
Bolivia y Chile, el Triangulo de Litio, el cual acumula el 70 \% de las reservas 
mundiales de este mineral. Aún más importante que esta cantidad de reservas es 
que las mismas se encuentran en salares que, a grandes rasgos, es más barato
extraer litio de ellos en comparación a las rocas de las cuales se puede extraer 
litio en una míneria usual, como las pegmatitas. A pesar de esto se tienen que
llevar a cabo distintas consideraciones ambientales, sociales y legales del 
proceso de extracción e incentivar el desarrollo de valor agregado a dicha 
extracción \cite{gutierrez2022, petavratzi2022, obaya2021, romero2021, 
heredia2020, fornillo2019}.

En esta tesis se presentan estudios computacionales sobre materiales para el 
desarrollo de electrodos de baterías de ion-litio de próxima generación. Se 
abordan dos perspectivas, una con el objetivo de tener baterías que frente a una 
carga rápida retengan un porcentaje considerable de la capacidad y otra 
utilizando electrodos que permitan almacenar mayor cantidad de energía que los 
actuales.
