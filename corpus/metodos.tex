\section{Teoría}

\subsection{Termodinámica}

Algo de ensambles

\subsection{RMN}

Si agrego algo de RMN.


\section{Dinámica molecular}

La dinámica molecular (MD, de sus siglas en inglés, \textit{molecular dynamics})
es una técnica de simulación computacional que considera un sistema de $N$
partículas atómicas, que interactúan a través de un campo de fuerzas newtoniano,
de las cuales se obtiene su evolución temporal. La misma permite obtener
propiedades termodinámicas macroscópicas (temperatura, presión) de un sistema en 
equilibro a partir de cantidades microscopicas (posiciones, velocidades, fuerzas)
~\cite{frenkel2001, allen2017}.

Para entender mejor como trabaja esta técnica de simulación es conveniente ver
como funciona su código fuente, el mismo sigue, en la mayoria de los casos, la
siguiente forma:
\begin{enumerate}
    \item \textbf{Inicialización del sistema}: se especifican las posiciones y
        velocidades iniciales de los átomos. También se elije un paso temporal, 
        un radio de corte para las interacciones y las condiciones de contorno que
        se van a respetar a lo largo de la simulación. 
    \item \textbf{Cálculo de fuerzas}: con las posiciones específicadas se
        calcula la fuerza sobre cada uno de los átomos a través del campo de 
        fuerzas elegido.
    \item \textbf{Integración de las ecuaciones de movimiento}: se integran las
        ecuaciones de Newton mediante algún integrador que obtiene las posiciones
        y velocidades del paso temporal siguiente a partir del actual.
    \item \textbf{Computo de propiedades termodinámicas}: se realizan los
        cálculos de distintas cantidades de interés, como las energías potencial
        y cinéctica, la presión y la temperatura.
    \item De ser necesario, se aplica algún \textbf{termostato o barostato}
        para realizar simulaciones en el ensamble termodinámico deseado.
    \item \textbf{Evolución temporal}: se incrementa el tiempo adhiriendo un
        paso temporal y se vuelve al cálculo de las fuerzas con las nuevas 
        configuraciones.
\end{enumerate}

Estos pasos pueden verse en la figura \textcolor{red}{HACER UN DIAGRAMA DEL PSEUDO 
CÓDIGO PROPIO}. Veamos a continuación cada una de las partes en detalle.

\subsection{Configuraciones iniciales}

Si bien las posiciones inicialesde algunos sistemas pueden reproducirse a partir
de los vectores de red de estructuras cristalinas (\textit{simple cubic}, 
\textit{body-centered cubic}, \textit{face-centered cubic}), otros casos de
interés, en los cuales hay más de un elemento involucrado, involucran celdas 
unidad más complicadas que han sido calculadas y optimizadas mediente la Teoría
del funcional de la densidad (DFT, de sus siglas en inglés, density functional
theory). Realizar estos calculos suele ser una tarea computacionalmente costosa 
y que requiere interevención de científicos especializados, para evitar esto
existe una base de datos ampliamente utilizada en el ámbito académico y en
la industria, Materials Project \cite{materials_project}, que recopila los datos
que existen sobre estas estructuras cristalinas, realiza nuevos cálculos y está 
abierta a la comunidad para su uso y colaboración. Antes de que los datos se
cargen en la página, los mismos son comparados con resultados experimentales 
para determinar si están dentro de un rango de validez definido. En esta tesis
en particular, fueron utilizadas distintas estructuras cristalinas de esta base de
datos como condiciones iniciales para las posiciones y el tamaño de la celda de 
simulación.

Las velocidades de los átomos suelen ser generadas de manera aleatoria, a través
de un generador de números pseudo-aleatorio, tomando como argumento una semilla 
para la reproducibilidad de la simulación y una temperatura deseada para el
sistema. Estos números suelen ser generados de una distribución gaussiana, donde
el centro se lo fija a cero para que no haya una velocidad en el centro de masa
y el ancho está relacionado a la temperatura.

\subsection{Condiciones de contorno}

Además de dar la configuración inicial de los átomos, es necesario especificar si
los mismos se encuentran dentro de una celda de simulación con un largo tamaño en
particular para cada una de las direcciones del sistema o si no interactúan fuera
del borde de la estructura que conforman los mismos. En el primero de los casos
se tienen condiciones periódicas de contorno (PBC, \textit{periodic boundary 
conditions}), lo que se busca con ellas es reproducir un sistema infinito, para
que no existan efectos de borde, y consiste en considerar que los átomos se 
encuentran dentro de una celda unidad de una red infinita de celdas idénticas; en
donde si un átomo sale por un extremo de la celda, ingresa por el opuesto. Una
condición que debe cumplir esta celda es que su tamaño en cada una de las 
direcciones debe ser mayor al radio de corte de las interacciones entre los átomos. 
\textcolor{red}{agregar figura para explicar un poco mejor y hablar de la imagen
minima.} Por otro lado, el segundo de los casos es útil considerarlo cuando se 
tienen nanoestructuras en las cuales los átomos están ordenados de cierta forma 
que globalmente representan una forma definida y no pueden ser consideradas como
una red inifinita.

\subsection{Potenciales interatómicos}

Existen una gran variedad de potenciales interatómicos para utilizar en dinámicas
moleculares. La elección de cada uno de ellos depende del sistema de estudio ya
que algunos potenciales representan de mejor manera gases y otros metales. A pesar
de esto, un potencial interatómico suele tener contribuciones atractivas y
repulsivas, según la distancia, con un mínimo que se encuentra a la distancia del
enlace entre dos átomos. Como ejemplo se toma un potencial interatómico de 
Lennard-Jones (12-6), dado por la siguiente expresión
$$
V_{LJ} = 4\varepsilon \left[ \left( \frac{\sigma}{r} \right)^{12} - 
                             \left( \frac{\sigma}{r} \right)^{6} \right],
$$
donde $r$ es la distancia entre dos átomos, $\varepsilon$ indica la profundidad 
del pozo del potencial que se encuentra en $r_m = 2^{1/6} \sigma$, $\sigma$ es el
radio del átomo. En la figura \textcolor{red}{X (agregar figura de un potencial 
de LJ)} se muestra el comportamiento de este potencial, si la distancia entre dos
átomos es menor a $r_m$ entonces se repelen, si es mayor a dicha distancia, se 
atraen. Cuando la distancia entre dos átomos es infinita, los mismos no 
interactúan, en el caso práctico se define una distancia de corte, conocida como
el \textit{radio de corte}, $r_{cut}$, a partir de la cual se considera que el 
potencial es nulo. Para evitar discontinuidades en este punto se suelen utilizar 
distintas técnicas como el truncado y desplazado o se multiplica al potencial 
al rededor de dicho punto por una función \textit{smooth}, que hace que el 
potencial se iguale suavemente a cero.

Una vez que el potencial interatómico está bien definido, para calcular la fuerza
que actúa sobre el átomo $i$ es necesario computar la fuerza de a pares con todos
los átomos $j$ del sistema. Para esto es necesario calcular las distancias,
considerando la imagen mínima si las condiciones de contorno son PBC, y ver si
las mismas son mayores o menores a $r_{cut}$, si la distancia es mayor entonces
la contribución de esa interacción es igual a cero y si es menor se computa la 
fuerza a través del potencial de la siguiente manera
$$
f_x(r) = - \frac{\partial V(r)}{\partial x}
       = - \left( \frac{x}{r_{ij}} \right) \cdot \left( \frac{
                                       \partial V(r)}{\partial r} \right)
$$
donde $r$ es la distancia entre los átomos, $x$ la componente en alguna de
las direcciones definidas para el sistema.

A continuación se presentan dos potenciales interatómicos del estado del arte que
fueron utilizados a lo largo de esta tesis.

\subsubsection{ReaxFF}

\subsubsection{DFTB}

\subsection{Integradores}

\subsection{Computo de propiedades termodinámicas}

Una vez que ya se conocen las posiciones, velocidades y fuerzas de todos los 
átomos se tiene toda la información necesaria para para computar distintas
cantidades de interés, el computo de cada una de ellas dependera del ensamble en
el cual se están realizando las simulaciones.

Las energías que contribuyen a la energía total son dos, la potencial y la
cinética. La primera de ellas viene dada por por distintas contribuciones de
pares, ángulos, enlaces, etc, dependiendo de que tan complejo sea el campo de 
fuerzas utilizado. En el caso de que la interacción sea solo de a pares, la
energía potencial puede calcularse de la siguiente forma
$$
E_{pot} = \sum_{i < j} u(r_{ij}),
$$
donde $u(r_{ij})$ es la contribución proveniente de la interacción entre los 
átomos $i$ y $j$.

Por otro lado, la energía cinética traslacional puede calcularse a partir de las
velocidades de cada uno de los átomos como
$$
E_{cin} = \frac{1}{2} \sum_{i=1}^{N} m_i v_i^2, 
$$
donde $m_i$ es la masa del átomo $i$ y $v_i^2$ el módulo de su velocidad. 

También suele ser de interés obtener el valor de la temperatura y de la presión
del sistema. La temperatura en un paso de la simulación puede calcularse 
utilizando que
$$
k_B T = \sum_{i=1}^N \frac{m_i v_i^2}{N_f},
$$
donde $k_B$ es la constante del Boltzmann y $N_f$ los grados de libertad,
aproximados usualmente por $3N$ para sistemas lo suficientemente grandes. Por 
último, la presión puede calcularse como 
$$
P = \rho k_B T + \frac{1}{d V} \left\langle \sum_{i<j} \mathbf{f}(\mathbf{r}_{ij}) 
                                              \cdot \mathbf{r}_{ij} \right\rangle,
$$
donde $\rho$ es la densidad, $d$ la dimensión y $V$ el volumen del sistema. El
segundo término es conocido como el virial, donde $r_{ij}$ y $f(r_{ij})$ son las 
distancias y las fuerzas entre los átomos $i$ y $j$.

\subsection{Termostatos y barostatos}


\section{Métodos de exploración de la superficie energía-potencial}

\subsection{Métodos locales}

\subsubsection{Minimizaciones locales}
\subsubsection{NEB}

\subsection{Métodos globales}

\subsubsection{Templado simulado}
\subsubsection{Dinámica acelerada}


\section{Experimentos computacionales}

\subsection{Distribución radial de a pares}

\subsection{Número de coordinación}

\subsection{Difusión}
