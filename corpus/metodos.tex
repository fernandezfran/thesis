\section{Dinámica molecular}

La dinámica molecular (MD, de sus siglas en inglés, \textit{molecular dynamics})
es una técnica de simulación computacional que considera un sistema de $N$
partículas atómicas, que interactúan a través de un campo de fuerzas newtoniano,
de las cuales se obtiene su evolución temporal. La misma permite obtener
propiedades termodinámicas macroscópicas (temperatura, presión) de un sistema en 
equilibro a partir de cantidades microscopicas (posiciones, velocidades, fuerzas)
~\cite{frenkel2001, allen2017}.

Para entender mejor como trabaja esta técnica de simulación es conveniente ver
como funciona su código fuente, el mismo sigue, en la mayoria de los casos, la
siguiente forma:
\begin{enumerate}
    \item \underline{Inicialización del sistema}: se especifican las posiciones y
        velocidades iniciales de los átomos. También se elije un paso temporal, 
        un radio de corte para las interacciones y las condiciones de contorno que
        se van a respetar a lo largo de la simulación. 
    \item \underline{Cálculo de fuerzas}: con las posiciones específicadas se
        calcula la fuerza sobre cada uno de los átomos a través del campo de 
        fuerzas elegido.
    \item \underline{Integración de las ecuaciones de movimiento}: se integran las
        ecuaciones de Newton mediante algún integrador que obtiene las posiciones
        y velocidades del paso temporal siguiente a partir del actual.
    \item \underline{Computo de propiedades termodinámicas}: se realizan los
        cálculos de distintas cantidades de interés, como las energías potencial
        y cinéctica, la presión y la temperatura.
    \item De ser necesario, se aplica algún \underline{termostato o barostato}
        para realizar simulaciones en el ensamble termodinámico deseado.
    \item \underline{Evolución temporal}: se incrementa el tiempo adhiriendo un
        paso temporal y se vuelve al cálculo de las fuerzas con las nuevas 
        configuraciones.
\end{enumerate}

Estos pasos pueden verse en la figura \textcolor{red}{HACER UN DIAGRAMA DEL PSEUDO 
CÓDIGO PROPIO}. Veamos a continuación cada una de las partes en detalle.

\subsection{Configuraciones iniciales}

Mencionar algo sobre materials project.
Velocidades aleatorias.

\subsection{Condiciones de contorno}

Periódicas - Fijas

\subsection{Potenciales interatómicos}

\subsubsection{ReaxFF}

\subsubsection{DFTB}

\subsection{Integradores}

\subsection{Termostatos y barostatos}

\subsection{Métodos de exploración de la superficie energía-potencial}

\subsubsection{Minimizaciones locales}

\subsubsection{Templado simulado}

\subsubsection{Dinámica acelerada}

\section{Experimentos computacionales}

\subsection{Distribución radial de a pares}

\subsection{Número de coordinación}

\subsection{Difusión}
