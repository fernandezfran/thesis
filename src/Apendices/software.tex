% Copyright (c) 2024, Francisco Fernandez
% License: CC BY-SA 4.0
%   https://github.com/fernandezfran/thesis/blob/main/LICENSE
\chapter{Software desarrollado}\label{a:software}

En la última década, Python se ha convertido en uno de los lenguajes de programación 
más importantes dentro de la comunidad científica debido a su facilidad de uso y 
versatilidad en la manipulación y visualización de datos \cite{millman2011}. 
Por lo tanto, los software diseñados en esta tesis han sido escritos en este
lenguaje y construidos sobre las librerías usuales del cómputo científico como
\path{NumPy} \cite{numpy}, \path{SciPy} \cite{scipy}, \path{pandas} \cite{pandas}, 
\path{matplotlib} \cite{matplotlib} y \path{scikit-learn} \cite{sklearn1, sklearn2}. 


\section{Control de calidad de software}\label{software:control}

El control de calidad del software hace referencia al conjunto de reglas y 
procedimientos que deben utilizarse para verificar que el software cumple 
determinados estándares de calidad subjetivos. Un procedimiento habitual son las 
pruebas unitarias (\textit{unit testing} en inglés), que consisten en aislar una 
función del código y comprobar que funciona como se espera \cite{jazayeri2007}. 
Otro procedimiento habitual se define a partir de este y es el 
\textit{code-coverage}, que determina que proporción del software se ha testeado
\cite{miller1963}. El estilo y la legibilidad del código también es importante
y aquí se ha seguido la guía de estilo PEP8 de Python, la misma se asegura con 
la herramienta \path{flake8}. Además, los mismos fueron desarrollados utilizando control 
de versiones \path{git} y distribuidos bajo la Licencia MIT, fomentando su uso tanto en 
entornos académicos como comerciales. Todo esto se realizó buscando que el 
software sea fácil de mantener y que respete los estándares de la comunidad Python.


\section{galpynostatic}\label{software:galpynostatic}

Este paquete denominado \path{galpynostatic} fue escrito para la utilización
del modelo heurístico presentado en el capítulo \ref{ch:un}. El mismo distribuye 
los datos de los diagramas galvanostáticos, un módulo de preprocesamiento de datos
para obtener capacidades de descarga a un potencial de corte dado a partir de 
medidas de perfiles galvanostáticos y una clase que realiza la regresión sobre la 
superficie y permite diferentes tipos de gráficos y estimaciones de parámetros.

A continuación se muestra un ejemplo de uso:

\lstinputlisting[language=Python]{Apendices/galpynostatic.py}

A \path{galpynostatic} se le realizan múltiples pruebas unitarias sobre datos de
de electrodos actuales y de materiales de investigación de próxima generación en 
baterías de litio, el \textit{coverage} del mismo alcanza el 100\% del software
en su versión inicial.

Por último, el código fuente está disponible en un repositorio público 
(\url{https://github.com/fernandezfran/galpynostatic}) y todos los nuevos cambios 
confirmados en este repositorio se prueban automáticamente con el servicio de 
integración continua de \path{GitHub} \path{Actions}. También se genera una documentación a 
partir de los docstrings del código, junto con una guía de instalación,
tutoriales y ejemplos con aplicaciones reales, que se hacen públicos en el 
servicio \path{read-the-docs} (\url{https://galpynostatic.readthedocs.io/en/latest/}). 
Además, \path{galpynostatic} está disponible para su instalación en el \path{Python}
\path{Package-Index} (\url{https://pypi.org/project/galpynostatic/}).


\section{galpynostatic.metric}\label{software:metric}

En el capítulo \ref{ch:umbem} se definió la métrica UMBEM, para la utilización de 
la misma se extendió el software previo (\path{galpynostatic}) con un nuevo 
modulo (\path{metric}) que tiene dos modos de uso, el primero especificando 
los descriptores del material en un diccionario:
\lstinputlisting[language=Python]{Apendices/metric1.py}
y el segundo pasando un objeto \textit{galpynostatic.model.GalvanostaticRegressor}
ya ajustado
\lstinputlisting[language=Python]{Apendices/metric2.py}
En ambos casos, según el valor de \path{full_output} se tiene como resultado
sólo el valor de UMBEM o a su vez un booleano si se cumple el criterio de USABC 
o no y un objeto que permite graficar el punto en el mapa simulado o hacer una
predicción sobre el tamaño óptimo de partícula.


\section{Código del protocolo de litiación}\label{software:lithiationprotocol}

Para el protocolo de litiación propuesto en el capítulo \ref{ch:prediccion}
se escribió un código utilizando la librería \path{MDAnalysis} 
\cite{mdanalysis1, mdanalysis2} para leer las posiciones de los átomos y realizar 
cálculos de distancias considerando condiciones periódicas de contorno. Dichas 
distancias además de involucrar las posiciones de los átomos de Li y de Si, 
también considera los vértices de un diagrama de Voronoi (calculado 
utilizando \url{http://www.qhull.org/} en \path{SciPy} \cite{scipy}) para 
encontrar la esfera vacía más grande e insertar el nuevo átomo de Li en el centro
de la misma. Adicionalmente, mediante un script de \path{bash} se manejan las 
ubicaciones en directorios de los archivos necesarios para correr con 
\path{DFTB+} \cite{dftb+} las minimizaciones y las simulaciones de dinámica 
molecular en el ensamble $NPT$ desde Python. Dicho código se encuentra disponible
en el siguiente link: \url{https://github.com/fernandezfran/lithiation\_protocol}

\section{macchiato}\label{software:macchiato}

El paquete \path{macchiato} permite predecir resultados de mediciones de rayos x, 
RMN y Mössbauer a partir de las configuraciones atómicas de estructuras de Li-Si
mediante los modelos de vecinos más cercanos introducidos en el capítulo 
\ref{ch:prediccion}. El modo de uso es similar para los distintos modulos
desarrollados, a continuación se muestra un ejemplo de cálculo del corrimiento 
químico de la estructura cristalina Li$_{13}$Si$_{4}$:

\lstinputlisting[language=Python]{Apendices/macchiato.py}

Al igual que \path{galpynostatic}, \path{macchiato} cumple con la calidad de 
software descripta en el Apéndice \ref{software:control}. El código fuente está 
disponible en un repositorio de GitHub público
(\url{https://github.com/fernandezfran/macchiato}), su documentación se encuentra
en el servicio \path{read-the-docs}
(\url{https://macchiato.readthedocs.io/en/latest/}) y para su instalación se 
distribuye en el \path{Python} \path{Package-Index} 
(\url{https://pypi.org/project/macchiato/}).
