\subsection{Predicción del tamaño óptimo de partícula}

Como ya ha sido mencionado a lo largo de esta tesis, el criterio de carga rápida
está definido por la obtención del 80\% de la capacidad del electrodo en 15 
minutos, lo cual se traduce en un SOC$_{\max}$ de 0.8 y una C-rate de 4 C. La
Figura \ref{fig:prediccion} muestra donde se encuentra cada sistema analizado en
el diagrama $\log(\Xi)$--$\log(\ell)$ para dicha C-rate. También se presenta una
curva de nivel con una línea roja correspondiente a SOC$_{\max} = 0.8$. Puede
observarse que tres de los materiales ya se encuentran en la región de 
SOC$_{\max}$ mayor a 0.8 (LCO, LMNO y LNMO), mientras que los otros se encuentran
por debajo de este valor (LTO, Grafito amorfo y LFP).
\begin{figure}[h!]
    \centering
    \includegraphics[width=0.7\textwidth]{FastCharging/un/resultados/prediccion/prediccion.png}
    \caption{Diagrama de SOC$_{\max}$ mostrando la ubicación de los materiales 
    usuales de LIBs a 4 C para las referencias consideradas \cite{mancini2022,
    he2012, lei2015, wang2019high, bak2011, nishikawa2017}. En los casos en los 
    que la curva de cargado a 4 C no estaba disponible, el valor del punto fue 
    predicho con el modelo. La línea roja muestra la curva de nivel 
    correspondiente al valor 0.8 de SOC$_{\max}$. Las flechas muestran el cambio
    en el tamaño de la partícula que debería efectuarse para obtener dicho valor
    a la C-rate dada. Las curces sobre la línea muestran la posición de estos
    tamaños de partícula nuevos.}
    \label{fig:prediccion}
\end{figure}
Haciendo uso del diagrama se puede predecir una forma simple y rápida el tamaño 
de partícula requerido para satisfacer el criterio de carga rápida. Dado que los
valores de $D$ y $k^0$ ya fueron ajustados, el valor de $d$ seleccionado por el
experimento y el de C-rate por el criterio, el valor de $\Xi$ es constante. Luego,
para alcanzar el valor de 0.8 de SOC$_{\max}$ hay que variar $\ell$ y esto se
logra disminuyendo o aumentando el tamaño de la partícula, según sea necesario. 
Este desplazamiento necesario está representado por las flechas en la Figura 
\ref{fig:prediccion} para cada caso. Ya se ha apreciado que tres sistemas se 
encuentran en la región ya optimizada (LCO, LMO y LNMO), por lo que en estos casos
los tamaños predichos para alcanzar SOC$_{\max} = 0.8$ a 4 C serán mayores que 
los experimentales. Por el contrario, el resto de los materiales (LTO, Grafito
amorfo y LFP) tienen que ser mejorados con una reducción del tamaño de partícula
para cumplir la condición. En la tabla \ref{t:prediccion} se muestran los tamaños
de partícula predichos para todos los materiales en la tercera columna para este
criterio. Las incertidumbres se determinaron por propagación de errores con 
derivadas parciales. Ya que el tamaño de la partícula sólo aparece en el parámetro
$\ell$, al definir $\ell_{\text{opt}}$ como el valor al cual el SOC$_{\max}$ 
alcanza el valor deseado de 0.8 y usar que $V/A = d/z$ se puede despejar de la 
ecuación \ref{eq:ele} que
\begin{equation}
    d = \sqrt{\frac{t_h z D 10^{\ell_{\text{opt}}}}{C_r}}.
\end{equation}
Si además se supone que toda la incertidumbre está asociada al coeficiente de 
difusión $D$, al cual ya se le calculó su incerteza, se puede obtener que
\begin{equation}
    \Delta d = \frac{1}{2} \sqrt{\frac{t_h z 10^{\ell_{\text{opt}}}}{C_r D}} \Delta D.
\end{equation}

\begin{table}[h!]
    \centering
    \caption{Tamaño experimental y valores predichos para cargar el 80\% del
    electrodo en 15 y 5 minutos.} 
    \setlength\extrarowheight{2pt}\stackon{%
    \begin{tabular}{l c c c}
        \toprule
        \textbf{Material del} &
        \textbf{Tamaño} &  
        \textbf{Tamaño predicho} & 
        \textbf{Tamaño predicho} \\
        \textbf{electrodo} & 
        \textbf{experimental [$\mu$m]} &  
        \textbf{para 15 minutos [$\mu$m]} & 
        \textbf{para 5 minutos [$\mu$m]} \\
        \midrule
        Grafito amorfo & 7.5 & 4.027 $\pm$ 0.002 & 2.167 $\pm$ 0.001 \\
        LTO & 1.75 & 0.962 $\pm$ 0.004 & 0.530 $\pm$ 0.002 \\
        LFP & 0.35 & 0.084 $\pm$ 0.002 & 0.0309 $\pm$ 0.0006 \\
        LCO & 20 & 28.8 $\pm$ 0.6 & 16.4 $\pm$ 0.4 \\
        LMO & 0.025 & 0.0734 $\pm$ 0.0003 & 0.0418 $\pm$ 0.0002 \\
        LNMO & 7.999 & 13 $\pm$ 2 & 7.3 $\pm$ 0.8 \\
        \bottomrule
    \end{tabular}
    }{}
    \label{t:prediccion}
\end{table}

Al observarse un buen desempeño para la carga de 15 minutos, se puede exigir un 
poco más que este criterio y predecir el tamaño de partícula requerido para una
C-rate más alta, digamos 80\% de la carga en 5 minutos (12 C). Si bien esta figura
puede parecer sobredemandante a primera vista, reportes recientes consideran 
protocolos de carga de 10 minutos \cite{mattis2021, attia2020}. Los resultados
se muestran en la última columna de la Tabla \ref{t:prediccion}. Como puede 
observarse, el comportamiento depende del sistema y del experimento en particular
considerado. El único caso donde se cumple este último criterio de carga rápida 
es el LMO, ya que el tamaño experimental sobrecumple el criterio. Aunque el LCO 
y el LNMO no cumplen con este último criterio, los cambios en sus tamaños serían
menores, por lo que estos materiales requieren mejoras menores. En el resto de 
los casos, para el LFP se necesitaría una disminución de un orden de magnitud 
en su tamaño, mientras que para el grafito amorfo o el LTO se requeriría una
disminución de su tamaño en un factor de 3.
