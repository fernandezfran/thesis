\section{Introducción}

La industria de vehículos eléctricos se encuentra en rápido crecimiento debido a la
necesidad de utilizar energías limpias y renovables para reducir los efectos 
causados por el cambio climático y la contaminación ambiental. Dentro de esta 
industria, la carga rápida es una de las principales características a ser 
mejorada. Para dar una idea del desafío con el cual nos encontramos, el tiempo 
de recarga de la batería de los autos eléctricos de Nivel 2 se encuentra entre 
las 4 y las 10 horas \cite{evcs}. Este número es dos órdenes de magnitud 
mayor que el tiempo que toma recargar un auto a combustión interna.

Existen distintos criterios para definir que una carga sea rápida. Uno de ellos, 
pensado para la aplicación de vehículos eléctricos, busca una autonomía de 30 km 
por minuto de carga \cite{dufek2022}. Otro criterio, definido por el USABC (de sus
siglas en inglés, \textit{United States Advanced Battery Consortium}), tiene como
objetivo obtener el 80\% del Estado de la Carga (SOC, \textit{State-of-Charge})
en 15 minutos \cite{USABC}. Este último es el criterio elegido para esta tesis.
Para cumplir con el mismo, es necesario entender la física de los materiales de
intercalación y cómo estos se comportan durante el cargado, ya que esto podría 
permitir una optimización en el diseño de los electrodos.

La carga rápida es un problema multi-escala \cite{franco2013, franco2019} y, por 
lo tanto, las mejoras en la velocidad de carga de las celdas electroquímicas 
requieren de una comprensión desde el nivel atómico hasta el ingenieril. Dentro 
de lo que es la escala micrométrica, los procesos que determinan la velocidad de 
carga al nivel de una sola partícula de los materiales de electrodos son la 
difusión de iones de litio dentro de electrodos y la transferencia de carga en 
las interfases electrodo/electrolito, por lo cual es necesario favorecer ambos
procesos \cite{liu2019, tomaszewska2019, weiss2021}. También se ha estudiado la 
importancia del rol del tamaño y la geometría de las partículas en el desempeño 
electroquímico \cite{gavilan2020, gavilan2022}. Así, incluso descartando
otros factores que podrían influir en la carga de la batería (como los cambios de 
volumen durante la intercalación de iones de litio, las características termodinámicas particulares
de los sistemas, la formación de interfase electrolito/sólido, etc), es complicado
realizar un análisis detallado del desafío general que representa este problema.

La electroquímica de electrodos de una sola partícula \cite{ventosa2021, 
heubner2020, takahashi2020, wahab2020, xu2020, tao2019, fukui2011} permite 
estudiar la respuesta \say{pura} de los materiales activos, esto es, sin el efecto 
de otros aditivos presentes en los electrodos compuestos. Esta información 
detallada es valiosa, ya que da a conocer los factores limitantes para su 
carga rápida y su respuesta frente a la misma. Enmarcado en este contexto, se utilizó un 
modelo de una sola partícula recientemente propuesto \cite{gavilan2023} 
para construir diagramas galvanostáticos de la capacidad máxima 
alcanzada por una sola partícula en función de dos parámetros adimensionales: uno
cinético,
\begin{equation}\label{eq:xi}
    \Xi = k^0 \sqrt{\frac{t_h}{C_r D}},
\end{equation}
y el otro de difusión finita,
\begin{equation}\label{eq:ele}
    \ell = d \frac{V}{A} \frac{C_r}{D t_h},
\end{equation}
donde $k^0$ es la constante cinética, $D$ el coeficiente de difusión, $V/A$ es la 
proporción volumen/superficie, $d$ es el tamaño característico de la partícula, 
$C_r$ denota la C-rate y $t_h$ el tiempo de una hora (en las unidades que
corresponda). En la Figura \ref{fig:diagnostico}a se muestra en un gráfico de colores en dos dimensiones el
estado de carga máximo alcanzada en función de los dos parámetros adimensionales. Estos valores 
de capacidad se corresponden con los cortes con el eje de las abscisas de los perfiles galvanostáticos presentes en las
Figuras \ref{fig:diagnostico}b y c (ejemplos para algunos valores en particular 
de $\Xi$ y $\ell$) 150 mV debajo del potencial de equilibrio.

\begin{figure}[h!]
    \centering
    \includegraphics[width=\textwidth]{FastCharging/un/introduccion/diagnosis-merged.png}
    \caption{(a) Diagrama de nivel para una geometría esférica con un voltaje de 
    corte de 150 mV y condiciones de contorno detalladas en la sección 
    \ref{s:metodologia}. Los datos de los puntos A y B se encuentran en la 
    Tabla \ref{t:ab}. (b) y (c) muestran perfiles voltaje/capacidad para 
    diferentes valores particulares de $\Xi$ y $\ell$ (ver líneas grises 
    discontinuas en la Figura \ref{fig:diagnostico}a), los triángulos negros 
    sobre el eje SOC indican los valores de SOC$_{\max}$ alcanzados.}
    \label{fig:diagnostico}
\end{figure}


Los parámetros $\Xi$ y $\ell$ introducidos en las ecuaciones \ref{eq:xi} y 
\ref{eq:ele}, respectivamente, son relaciones de escaleo útiles para realizar una
primera predicción cualitativa de la capacidad que un material alcanzaría a una
dada C-rate. Esto es relevante, ya que permite realizar la tarea no-trivial de
clasificar distintos materiales, cada uno caracterizado por los descriptores $D$,
$k^0$ y $d$.

Las relaciones de escaleo suelen dar una respuesta simple, aunque aproximada, a 
problemas complejos, esta cualidad ha hecho que tengan una gran popularidad 
en la física y en la química. Quizás uno de los casos más conocidos sea el de los
gases ideales
\begin{equation}
    \frac{p V}{n T} = R,
\end{equation}
que no es más que escalear el producto de la presión y el volumen, $p V$, por la
temperatura absoluta, $T$, y el número de moles, $n$, para obtener una constante
universal. Aunque de forma aproximada, esta ecuación proporciona una guía 
importante para resolver muchos problemas prácticos y es utilizada como referencia
para entender el comportamiento de fluidos más complejos que los gases ideales.

De manera similar a la de los gases ideales, el valor máximo del SOC alcanzado a 
un dado potencial de corte y a una dada C-rate, SOC$_{\max}$, puede 
pensarse como una función universal de los parámetros de escaleo $\Xi$ y $\ell$,
\begin{equation}\label{eq:socmax}
    \text{SOC}_{\max} = f(\Xi, \ell) = f(d, D, k^0, C_r).
\end{equation}
Como no tenemos una expresión análitica para esta función $f$, la misma puede ser 
obtenida en un mapeo ($\Xi$, $\ell$) de simulaciones galvanostáticas, como se la 
presenta en la Figura \ref{fig:diagnostico}a.

\begin{figure}[h!]
    \centering
    \includegraphics[width=.7\textwidth]{FastCharging/un/introduccion/xiafom.png}
    \caption{Figura de mérito definida como el tiempo característico de difusión, 
    $\tau = d^2/D$, propuesta por Xia \textit{et al.} \cite{xia2022}. La líneas
    grises discontinuas indican valores de $\tau$ constantes que dividen a los 
    materiales en clases con diferentes capacidades de carga rápida.}
    \label{fig:xiafom}
\end{figure}
En un trabajo reciente se ha propuesto una figura de mérito que permite establecer 
una jerarquía entre distintos materiales de carga rápida \cite{xia2022}. 
Esta consiste en combinar los efectos de los coeficientes de difusión y los 
tamaños geométricos para definir el tiempo característico de difusión, 
$\tau = d^2 / D$, y es presentada en la Figura \ref{fig:xiafom}. Esta figura de 
mérito supone un sólido semi-infinito y una reacción interfacial ultrarrápida, 
es decir que supone que el proceso limitante es la difusión de los iones dentro 
del material.
Desde
este punto de vista, la ecuación \ref{eq:socmax} evaluada en el punto ($\Xi$, 
$\ell$) también puede introducirse como una figura de mérito. Supongamos que 
tenemos dos materiales, A y B, con las propiedades dadas en la Tabla \ref{t:ab} y 
se quiere determinar si alguno de ellos es un buen candidato para una carga en 15 
minutos. Entonces, con los datos de las columnas 2--4 de la Tabla \ref{t:ab} 
se pueden calcular los valores de $\Xi$ y $\ell$, utilizando las ecuaciones 
\ref{eq:xi} y \ref{eq:ele}, presentados en la quinta y la sexta columna de la 
Tabla \ref{t:ab}. Los puntos correspondientes a estos materiales se presentan en 
el mapa de la Figura \ref{fig:diagnostico}a. A partir de este, podemos ver que el 
material A es un buen candidato para una carga en 15 minutos mientras que el 
material B no lo es.

\begin{table}[h!]
    \centering
    \caption{Ejemplo de dos materiales, A y B, caracterizados por sus coeficientes 
    de difusión, sus constantes cinéticas y sus tamaños.}
    \setlength\extrarowheight{2pt}\stackon{%
    \begin{tabular}{l c c c c c}
        \toprule
        \textbf{Material} & 
        \textbf{$D$ [cm$^2$/s]} &  
        \textbf{$k^0$ [cm/s]} &
        \textbf{$d$ [cm]} &  
        \textbf{$\log(\Xi)$} & 
        \textbf{$\log(\ell)$} \\
        \midrule
        A & 3.7$\times 10^{-9}$ & 2.03$\times 10^{-7}$ & 0.001 & -1.0 & -1.0 \\
        B & 1.17$\times 10^{-11}$ & 1.14$\times 10^{-6}$ & 0.001 & 1.5 & 1.0 \\
        \bottomrule
    \end{tabular}
    }{}
    \label{t:ab}
\end{table}

El principal objetivo de este capítulo es proponer que estos diagramas generales 
sean utilizados con un enfoque heurístico, aprovechando datos experimentales de la
literatura, para mostrar que pueden proporcionar una guía simple y rápida para 
estudiar la cinética y el comportamiento de materiales específicos bajo 
condiciones de carga rápida.
