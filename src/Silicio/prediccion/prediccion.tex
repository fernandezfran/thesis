\chapter{Predicción de resultados experimentales de mediciones de rayos x, RMN y Mössbauer}\label{ch:prediccion}
\thispagestyle{empty}

\vspace{50pt}

\begin{adjustwidth}{50pt}{50pt}
    En este capítulo se analizan las configuraciones atómicas de las estructuras
    amorfas de Li-Si utilizando el modelo DFTB desarrollado en el capítulo 
    anterior. Se ataca el principal obstáculo para aprovechar la alta capacidad 
    del silicio al relacionar estas estructuras con distintas observaciones 
    experimentales. Para lograr esto se proponen modelos de vecinos más 
    cercanos para predecir resultados experimentales de la función 
    distribución radial de a pares de rayos x, de espectros de corrimiento 
    químico de RMN de $^7$Li y de la división de picos en espectros de 
    Mössbauer. Las estructuras que predicen estos observables se publican
    en un repositorio de libre acceso.
\end{adjustwidth}

\clearpage
\newpage
\thispagestyle{empty}
\mbox{}
\newpage

\chapter{Introducción}\label{ch:introduccion}
\thispagestyle{empty}

\vspace{50pt}

\begin{adjustwidth}{50pt}{50pt}
    TODO
\end{adjustwidth}

\clearpage
\newpage
\thispagestyle{empty}
\mbox{}
\newpage

% Copyright (c) 2024, Francisco Fernandez
% License: CC BY-SA 4.0
%   https://github.com/fernandezfran/thesis/blob/main/LICENSE
\section{Contextualización}

El calentamiento global aparece como el mayor problema ambiental de este siglo.
El mismo se refiere al aumento de la temperatura media de la atmósfera y por 
ende a sus consecuencias en  el clima. Esto es debido al efecto que producen 
las actividades humanas, 
como por ejemplo la quema de combustibles fósiles, que emite 
a la atmósfera grandes cantidades de CO$_2$, entre otros gases de efecto 
invernadero, o la deforestación. Estos gases absorben la radiación infrarroja emitida por la tierra y la reemiten, 
provocando un incremento de la temperatura de la misma que lleva asociado un 
aumento en la frecuencia y la intensidad de eventos climáticos extremos. %\cite{houghton2005}. 
De acuerdo a el Panel Intergubernamental del Cambio Climático 
\cite{IPCC}, desde la época preindustrial, las actividades humanas han provocado 
aproximadamente 1.0$^{\circ}$C de calentamiento global y al ritmo actual se van 
a sobrepasar los 1.5$^{\circ}$C antes del 2050, un cambio en la temperatura
media que las medidas previas al aumento de las actividades mencionadas no habían llegado a alcanzar. Limitar el 
calentamiento a esta temperatura requiere que se realicen rápidamente cambios 
sin precedentes en la tecnología y en el comportamiento humano. Uno de los 
cambios más importante es el de la matriz energética, en la cual las energías 
renovables deberán suministrar alrededor del 80\% de la energía para 2050, donde 
los vectores energéticos, como las baterías de litio, juegan un rol fundamental 
debido a la intermitencia de estas formas de generación de energía.

El litio es el metal más liviano de la tabla periódica y uno de los elementos más
importantes dentro de los minerales necesarios en la producción de baterías de
litio. En particular, para la Argentina tiene un interés económico, social, 
industrial y tecnológico ya que es uno de los países que integran, junto a 
Bolivia y Chile, el Triangulo de Litio, el cual acumula el 70\% de las reservas 
mundiales de fácil extracción de este mineral. Esto último debido a que esta cantidad de reservas
se encuentran en salares de los que, a grandes rasgos, es más barato
extraer litio de ellos en comparación a las rocas de las cuales se puede extraer 
litio en una míneria usual, como las pegmatitas. A pesar de esto se tienen que
llevar a cabo distintas consideraciones ambientales, sociales y legales del 
proceso de extracción e incentivar el desarrollo de valor agregado a dicha 
extracción \cite{gutierrez2022, petavratzi2022, obaya2021, romero2021, 
heredia2020, fornillo2019}.

En esta tesis se presentan estudios computacionales sobre materiales para el 
desarrollo de electrodos de baterías de ion-litio de próxima generación. Se 
abordan dos perspectivas, una con el objetivo de tener baterías que frente a una 
carga rápida retengan un porcentaje considerable de la capacidad y otra 
utilizando electrodos que permitan almacenar mayor cantidad de energía que los 
actuales.


\section{Energía, transporte y litio}

En la actualidad se utilizan distintas formas para generar energía y pueden 
dividirse en dos grandes tipos, las renovables y las no-renovables. Estas últimas
dominan la producción de energía mundial y están compuestas principalmente por 
combustibles fósiles y centrales nucleares, mientras que las energías renovables
abarcan más variantes como la biomasa, la hidráulica, la eólica y la solar, pero 
aún no son lo suficientemente utilizadas. Una de las particularidades de estas 
fuentes de energías renovables es su producción intermitente mientras que el 
consumo de la misma, independientemente de cómo se genere, es a demanda. Esto 
hace que sea necesario el involucramiento de vectores energéticos que permitan 
almacenar y transportar el excedente de energía que se genera en sus períodos de 
mayor producción para luego ser utilizada en los momentos de mayor demanda.

El sector del transporte terrestre, marítimo y aéreo es responsable de más de un 
tercio de las emisiones de CO$_2$ debido a su dependencia en los combustibles 
fósiles \todo{\cite{IEA}}. Dicho esto, está claro que se debe fomentar opciones de desplazamiento menos intensivas
en carbono y con tecnologías más eficientes, como los vehículos eléctricos (EVs),
\todo{cuyos motores poseen una eficiencia para convertir la energía eléctrica en energía para las ruedas que ronda el 80\%, compárese este valor con las
eficiencias entre el 12\% y el 30\% de los motores a combustión interna para la misma tarea \cite{DOE}}.

En los últimos años se ha producido un crecimiento exponencial en las ventas 
anuales de los EVs, como puede observarse en la Figura \ref{fig:evs}a \cite{EVV}. En la
última década, dichas ventas aumentaron aproximadamente un 500\% y se estima que
para la próxima década las ventas se multipliquen por 10. Estas ventas están 
concentradas en China y en algunos países y estados de Europa y Estados Unidos, 
respectivamente, debido a que en los países en desarrollo y emergentes influye 
negativamente su costo alto de adquisición y una falta de infraestructura para la 
recarga de sus baterías. En particular, durante el 2022 en Noruega el 79.3\% de 
los automóviles patentados fueron eléctricos. En el país que le sigue en la lista,
Suecia, se patentó un 32.1\% de EVs en dicho año \cite{PWC}.
\begin{figure}[h!]
    \centering
    \includegraphics[width=\textwidth]{Introduccion/energia/evs.png}
    \caption{(a) Ventas anuales de vehículos eléctricos en la última década. Se 
    proyecta que para el 2030 las ventas asciendan a las 40 millones de unidades 
    frente a las 3 millones del año 2020 \cite{EVV}. (b) Proyección del costo en 
    dólares de vehículos eléctricos y de combustión interna en países 
    desarrollados \cite{BLOOMBERG}.}
    \label{fig:evs}
\end{figure}

En la Figura \ref{fig:evs}b se muestra la proyección en el costo de los vehículos 
eléctricos y de combustión interna realizada por la empresa financiera Bloomberg 
para los países desarrollados \cite{BLOOMBERG}. Se espera que para el año 2026 
los costos se igualen y que para el 2030 los EVs sean aproximadamente un 15\% más
baratos que los vehículos de combustión interna. Este cambio se debe a la 
disminución en el precio de la producción de baterías, que actualmente representa
aproximadamente el 40\% del costo del EV.

El sector energético en Argentina depende altamente de la utilización de 
combustibles fósiles, donde la generación de energía está dominada por el gas 
natural (65\%) y le siguen las centrales hidroeléctricas (18\%), plantas nucleares
(8\%), parques eólicos (7\%) y solares (1\%) \cite{IEA}. En cuanto al potencial de 
producción de fuentes renovables, Argentina tiene una gran capacidad en sus 
fuentes eólicas y solares por desarrollar. Además, es el cuarto productor mundial más 
grande de litio, que es un mineral crítico para la manufactura de sistemas de 
almacenamiento y transporte de energía, claves para la transición energética. 
El mismo representa el 7\% de la demanda para vehículos eléctricos mientras que 
para almacenamiento en la red el porcentaje es del 10\%. Otros metales y 
minerales críticos se encuentran en la región de América Latina; por ejemplo, 
Paraguay posee la reserva más grande del mundo de titanio, Chile es el mayor 
productor de cobre, Brasil tiene las segundas reservas más grande de níquel y
hierro, las terceras de grafito y manganeso, la cuarta de aluminio y la quinta de 
fósforo, por último, Cuba se encuentra en el tercer puesto de reservas de cobalto.

En la Figura \ref{fig:iea-Li} se muestra la proyección en la demanda total de 
litio por año y por aplicación, donde la mayor contribución se encuentra para la 
utilización del mismo en vehículos eléctricos mientras que una menor contribución 
se espera en aplicaciones de sistemas de almacenamiento estacionarios y otras 
aplicaciones que incluyen dispositivos electrónicos, medicamentos, lubricantes, 
entre otras \cite{IEA}. Cabe destacar que para el almacenamiento estacionario 
las baterías de ion-litio es muy probable que compitan con baterías de sodio o magnesio, entre 
otras. En el histograma de la Figura \ref{fig:iea-Li} pueden diferenciarse dos 
regiones, la primera de ellas entre el año 2022 y el 2035, donde los aumentos
porcentuales de la demanda de litio con respecto a 5 años atrás son del 74\%, 
99\% y 76\%. Luego, del año 2035 al 2040, el cambio se encuentra en el 32\% y
dicho aumento porcentual continúa disminuyendo al 10\% y al 3\% en los períodos 
subsiguientes.
\begin{figure}[h!]
    \centering
    \includegraphics[width=.8\textwidth]{Introduccion/energia/iea-Li.png}
    \caption{Proyección de la demanda total de litio en kilotoneladas para el 
    período 2025-2050 para sus distintas aplicaciones: vehículos eléctricos (en 
    azul), sistemas de almacenamiento de energía estacionarios (en naranja) y
    otras aplicaciones (en verde). Fuente: \cite{IEA}.}
    \label{fig:iea-Li}
\end{figure}


\section{Baterías de ion-litio}

A finales del año 2019, año en el que se comenzó esta tesis, la Real Academia 
de Ciencias de Suecia le otorgó el Premio Nobel en Química a J. B. Goodenough, 
M. S. Whittingham y A. Yoshino por sus contribuciones al desarrollo de la batería 
de ion-litio. Esta batería recargable permitió los avances que se vieron en los 
teléfonos móviles y en las computadoras portátiles, entre otras aplicaciones.
Además, permite un mundo libre de combustibles fósiles ya que se utiliza en 
vehículos eléctricos y en almacenamientos estacionarios de energía para fuentes
renovables. Este galardón restaltó la importancia de muchos aspectos de la ciencia
moderna, como la investigación básica, la investigación la aplicada, la 
interdisciplina (JBG fue físico, MSW es un químico y AY un ingeniero) los 
desarrollos tecnológicos y los problemas concretos de las sociedades.
En la década del 1970, MSW desarrolló la primera batería utilizando un ánodo de
litio metálico y un cátodo de disulfuro de titanio. En 1980, JBG duplicó el 
voltaje original de dicha batería al introducir un cátodo de óxido de cobalto.
La desventaja de ambas se encontraba en el ánodo de litio metálico, que en los 
ciclos de carga y descarga se deposita preferentemente en sitios donde ya se 
ha depositado, dando lugar a estructuras ramificadas, llamadas dendritas, que 
pueden cortocircuitar la celda y llevar a la explosión de la misma. En 1985,
AY remplazó este material por uno carbonoso que incorpora los iones de litio
durante la carga y la descarga, disminuyendo los riesgos mencionados. Basandose
en este desarrolló, Sony comenzó a comercializar baterías de ion-litio en 1991.
La densidad de energía de estas baterías rondaba los 80 Wh/kg, en la actualidad
WeLion comercializa para los EVs de Nio una batería de ion-litio con una 
densidad de energía de 360 Wh/kg. En la Figura \ref{fig:whkg} se muestra la 
evolución de la densidad de energía en baterías de ion-litio comercializadas 
en los últimos 30 años. La importancia de esta característica para los EVs 
radica en la relación autonomía/peso.
\begin{figure}[h!]
    \centering
    \includegraphics[width=.8\textwidth]{Introduccion/baterias/whkg.png}
    \caption{Aumento en la densidad de energía en baterías de ion-litio comercializadas
    en los últimos 30 años. Figura adaptada de \cite{li2023700}.}
    \label{fig:whkg}
\end{figure}

Las baterías de ion-litio admiten una gran cantidad de recargas y las mismas están 
compuestas por celdas electroquímicas conectadas entre sí, las mismas son unidades 
fundamentales que permiten transformar la energía química almacenada en energía
eléctrica mediante una reacción redox (reducción-oxidación), en la cual uno de los 
componentes pierde electrones (se oxida) y el otro gana electrones (se reduce).
En la Figura \ref{fig:esquema_bateria} se muestra un esquema general con el 
funcionamiento que presenta una celda electroquímica de ion-litio y se destacan 
las componentes más relevantes: los electrodos positivo (cátodo) y negativo (ánodo) 
donde ocurren las reacciones redox en la carga/descarga de la celda, el electrolito 
por el cual difunden los iones de litio y el separador que suele ser un material 
poroso permeable al electrolito que se encarga de que los electrones circulen por 
el circuito externo. Durante la descarga de la reacción redox es espontánea y 
provoca la difusión de iones de litio por el electrolito desde el ánodo hacia el 
cátodo, junto con una corriente eléctrica en un circuito externo (flechas rojas). 
Durante la carga se debe aplicar una corriente eléctrica externa para tener la 
reacción inversa (flechas verdes).
\begin{figure}[h!]
    \centering
    \includegraphics[width=.8\textwidth]{Introduccion/baterias/esquema_bateria.png}
    \caption{Esquema de las componentes y el funcionamiento de una batería de 
    ion-litio.}
    \label{fig:esquema-bateria}
\end{figure}

En la Figura \ref{fig:scopus} se muestra el incremento en las últimas dos décadas
de los artículos científicos publicados en el área de las baterías de litio y, en 
particular, de las dos ramas estudiadas en esta tesis: la Carga rápida y los 
Ánodos de Si. En dicha figura se presentan datos extraídos de la base de datos 
Scopus \cite{SCOPUS} del número de publicaciones anuales normalizado con respecto 
al número de publicaciones en el año 2003, año en el que hubo 710 publicaciones 
en baterías de litio, 32 sobre ánodos de Si y 0 sobre carga rápida, por lo que 
se normalizó en este caso a la única publicación del 2004 en el tema.
\begin{figure}[h!]
    \centering
    \includegraphics[width=.8\textwidth]{Introduccion/baterias/scopus.png}
    \caption{Número de publicaciones anuales normalizado con respecto al año 2003. 
    Las consultas realizadas en Scopus \cite{SCOPUS} incluyen: 
    \texttt{lithium AND battery} (LIBs, en azul), \texttt{lithium AND battery AND 
    fast-charging} (Carga rápida, en naranja) y \texttt{lithium AND battery AND 
    silicon anodes} (Ánodos de Si, en verde).}
    \label{fig:scopus}
\end{figure}
La normalización y la escala logarítmica en la Figura \ref{fig:scopus} permiten
observar cualitativamente que la pendiente de crecimiento de publicaciones 
realcionadas a la carga rápida de baterías de litio es considerablemente mayor a 
de las otras dos. Además, los ánodos de Si se encuentran dentro de lo que sería
el creciemiento promedio del área de las baterías de litio. Un análisis de datos
cuantitativo permite determinar que en la última década el aumento de porcentaje
anual de publicaciones promedio fue del 15 \% y 16 \% para las baterías de litio 
y para los ánodos de silicio, respectivamente, mientras que para la carga rápida 
este porcentaje promedio asciende al 52 \%. Este análisis demuestra la relevancia
que la comunidad científica le da a los temas estudiados en esta tesis.


\section{Objetivos y estructura de la tesis}

Esta tesis tiene como objetivo estudiar materiales que se utilicen para el 
desarrollo de electrodos de baterías de ion-litio de próxima generación mediante 
distintos modelados computacionales. 
La misma se encuentra dividida en tres partes, la primera de ellas sobre la 
Motivación y fundamentos consistente de dos capítulos, el capítulo 
\ref{ch:introduccion} con esta introducción y el capítulo \ref{ch:metodos} con la
descripción de los distintos métodos computacionales utilizados. 
La Parte \ref{p:fast-charging} se divide en dos capítulos, ambos relacionados con 
la carga rápida de baterías de ion-litio. En el capítulo \ref{ch:un} se 
desarrolla un modelo para ajustar datos experimentales en condiciones 
galvanostáticas y predecir el tamaño óptimo de partículas que permite retener un 
80 \% de su capacidad frente a una carga realizada en 15 minutos 
\cite{fernandez2023towards}. El capítulo \ref{ch:umbem} busca una métrica 
universal que permita estandarizar las comparaciones del desempeño entre 
distintos materiales considerados en aplicaciones de carga rápida.
La Parte \ref{p:silicio} se centra en el estudio de las aleaciones presentes en 
los ánodos de silicio y se divide en tres capítulos. El capítulo 
\ref{ch:caracterizacion} caracteriza las estructuras de Li-Si encontradas con 
un potencial reactivo y con un método de exploración acelerada de mínimos locales
propuesto \cite{fernandez2021characterization}. En el capítulo \ref{ch:modelo} se
parametriza un modelo DFTB (\textit{denstity functional tight-binding}) para la 
interacción Li-Si mediante un algoritmo que asigna pesos a las distintas 
estructuras consideradas para el ajuste \cite{oviedo2023}. En el capítulo 
\ref{ch:prediccion} se proponen modelos de vecinos más cercanos para predecir 
mediciones de rayos x, RMN y Mössbauer a partir de las configuraciones atómicas
\cite{fernandez2023nmr}.
Cada uno de los capítulos mencionados en estas dos últimas partes se componen
de una introducción y detalles de los métodos computacionales utilizados, los 
resultados junto a las discusiones de los mismos y conclusiones parciales.
Por último, se cierra la tesis con el capítulo \ref{ch:comentarios} con los 
comentarios finales de la misma.



\section{Métodos computacionales}

\subsection{Protocolo de litiación}

Se propone un protocolo de litiación, similar al que presentan Chevrier y Dahn
\cite{chevrier2009} y utilizado en la sección \ref{s:dftcalc}, que consiste 
en los siguientes pasos:
\begin{enumerate}
    \item Agregar un átomo de Li en el centro de la esfera vacía más grande.
        Para encontrar dicho punto se calculan los centros de la triangulación 
        de Delaunay, que se corresponden a los vértices de un diagrama de 
        Voronoi \cite{aurenhammer1991}. Desde estos puntos se computa la 
        distancia al átomo más cercano y se selecciona como centro aquel que 
        tenga la mayor distancia. 
    \item Aumentar  el volumen y escalar las coordenadas por un factor para
        seguir la expansión experimental del sistema.
    \item Minimizar localmente, con el algoritmo LBFGS en este caso.
    \item Hacer una simulación de dinámica molecular en el ensamble $NPT$, en 
        este caso utilizando el termostato y el barostato de Berendsen \cite{berendsen1984} 
        disponibles en el código \path{DFTB+} \cite{dftb+} por 10 ps. 
    \item Seleccionar la estructura con menor presión absoluta.
    \item Si $x < 3.75$ se vuelve al paso 1 y si no se termina la litiación.
\end{enumerate}
El paso (4) representa una modificación ligera que mejora la optimización a 
cambio de volumen fijo como se realiza en la referencia \cite{chevrier2009}.
Partiendo de la estructura de silicio amorfo obtenida en la sección 
\ref{s:rdfb} y siguiendo este protocolo de litiación, se obtienen estructuras
amorfas para un rango amplio de concentraciones de Li en Li$_x$Si. Se comienza
con los 64 átomos de Si iniciales ($x=0$) y se llega a un total de 304 átomos
para la estructura completamente litiada ($x=3.75$). Todos los observables que
se presentan a continuación para cada valor particular de $x$ se calcularon 
utilizando una trayectoria más larga de 0.5 ns. Los valores optimizados de $x$
son $0.20, 0.56, 0.89, 1.50, 2.00, 2.50, 3.28, 3.75$. Estas estructuras se 
publicaron en un repositorio de libre acceso \cite{dftb_lisi_amorphous}.


% Copyright (c) 2024, Francisco Fernandez
% License: CC BY-SA 4.0
%   https://github.com/fernandezfran/thesis/blob/main/LICENSE
\section{Resultados y discusiones}

Los resultados se presentan a continuación en el orden en el que se emplean 
los pasos en la librería \path{galpynostatic} de Python: una primera sección
para el preprocesamiento de los datos experimentales, luego otra para el ajuste
de estos datos con el modelo heurístico, y, por último, la utilización de éste
para predecir las condiciones del tamaño de partícula para lograr una carga
rápida de 15 y 5 minutos. Sumado a esto, también se compara el comportamiento que
tendrían los distintos materiales, dados sus parámetros fundamentales, a 
distintos tamaños.

\subsection{Preprocesamiento de los datos experimentales}

Un procedimiento experimental usual para evaluar los materiales de las baterías
consiste en medir los perfiles galvanostáticos a distintos valores de C-rate.
En la Figura \ref{fig:preproc} se muestran como ejemplo las mediciones realizadas
por Wang \textit{et al.} \cite{wang2019high} para LiCoO$_2$ (LCO) recubierto con 
TiO$_2$. Además, se agrega una línea punteada horizontal que se corresponde
con el potencial de equilibrio reportado en el trabajo citado, 3.9 V, y otra
0.15 V por debajo, que es el valor que corresponde al potencial de corte
elegido en este capítulo. Esta es la región de interés en el gráfico, ya que 
los valores en los que el SOC se intersecta con esta última curva (SOC$_{\max}$)
son los que se utilizan para ajustar el modelo en función de C-rate.
\begin{figure}[h!]
    \centering
    \includegraphics[width=0.7\textwidth]{FastCharging/un/resultados/preprocesamiento/preprocesamiento.png}
    \caption{Perfiles galvanostáticos para distintos valores de C-rate para el
    sistema LCO recubierto de TiO$_2$. Las líneas horizontales indican el 
    potencial de equilibrio y el de corte utilizado para determinar la 
    capacidad máxima normalizada alcanzada (SOC$_{\max}$) a cada C-rate. 
    Reproducido del trabajo de Wang \textit{et al.} \cite{wang2019high}.}
    \label{fig:preproc}
\end{figure}

Es importante destacar que, en el trabajo citado, los perfiles galvanostáticos
se presentan en función del SOC normalizado, que no siempre es el caso. La 
forma usual en la que estos resultados son reportados es en función de la 
capacidad de descarga. En estos casos, es necesario normalizarla con respecto
a la capacidad máxima ($Q_{\max}$) alcanzada por el material, para así obtener
el SOC normalizado. El criterio utilizado en este capítulo para encontrar 
$Q_{\max}$ fue considerar el valor máximo de la capacidad alcanzado por la 
medición a la C-rate más baja. Gráficos similares al presentado en la Figura 
\ref{fig:preproc} son obtenidos en el resto de los trabajos experimentales que
se utilizan en los ajustes que siguen.


\subsection{Ajuste del modelo}

\begin{figure}[h!]
    \centering
    \includegraphics[width=0.7\textwidth]{FastCharging/un/resultados/ajuste/ajustes.png}
    \caption{Ajuste del modelo a los datos SOC$_{\max}$ \textit{versus} C-rate
    para los distintos materiales de electrodos considerados: (a) Grafito 
    amorfo \cite{mancini2022}, (b) LTO \cite{he2012}, (c) LFP \cite{lei2015}, 
    (d) LCO \cite{wang2019high}, (e) LMO \cite{bak2011}, (f) LNMO
    \cite{nishikawa2017}.}
    \label{fig:ajustes}
\end{figure}

En la Figura \ref{fig:ajustes} se muestran los datos experimentales y los 
ajustes del modelo para el SOC$_{\max}$ alcanzado al potencial de cortes 
\textit{versus} la C-rate para los resultados de la Figura \ref{fig:preproc}
y otros materiales de uso común en lso electrodos de las baterías de ion-litio.
Puede observarse una buena concordancia en general entre el modelo y los 
experimentos. Esto también puede observarse en la Figura \ref{fig:pred_vs_exp},
donde se muestran los valores predichos para SOC$_{\max}$ en función de los
experimentales, junto con el coeficiente de determinación de cada ajuste.

\begin{figure}[h!]
    \centering
    \includegraphics[width=0.7\textwidth]{FastCharging/un/resultados/ajuste/pred_vs_exp.png}
    \caption{Predicciones del SOC$_{\max}$ \textit{versus} valores 
    experimentales, junto al coeficiente de determinación de cada sistema.}
    \label{fig:pred_vs_exp}
\end{figure}

El trabajo de Mancini \textit{et al} \cite{mancini2022} aportó nuevos 
conocimientos sobre el efecto de esferoidización en las características de las
partículas de grafito y su impacto en el comportamiento electroquímico. A
continuación se hace referencia a estos datos como grafito amorfo, como puede
verse en la Figure \ref{fig:ajustes}a. En este caso, el rango de C-rates
reportadas cubre una amplia región del SOC$_{\max}$, desde un material 
totalmente cargado hasta uno casi totalmente descargado. Esto no es lo habitual,
ya que en la mayoría de los experimentos sólo se reportan curvas con un buen 
rendimiento (alta capacidad), lo que limita los ajustes realizados a una región
concreta del diagrama. El coeficiente de difusión y la constante cinética 
obtenidos en el ajuste para este sistema son $1.23\times10^{-10}$ cm$^2$/s y 
$2.31\times10^{-7}$ cm/s, respectivamente. Para esto se consideró un tamaño
de partícula de $7.5 \mu$m y una geometría esférica. Este valor se corresponde
con la media de la distribución de tamaños, que es reportada junto a los 
cuartiles en el trabajo citado. Para los casos que siguen, en los que no se
tiene información precisa de la distribución de tamaños, se considera el punto
medio del rango reportado para el tamaño de las partículas y utiliza para 
definir el parámetro $d$ en el modelo.

Los ánodos de Li$_4$Ti$_5$O$_{12}$ (LTO) presentan características excelentes
de seguridad y una capacidad teórica de 175 mAhg$^{-1}$. He \textit{et al} 
\cite{he2012} sintetizaron nanopartículas cristalinas y esféricas de LTO a 
diferentes proporciones atómicas de Li/Ti, con un tamaño entre los 0.5 $\mu$m 
y los 3 $\mu$m. Se asume entonces un valor de $d=1.75 \mu$m y se utilizan los
datos de la proporción usual del LTO para ajustar el modelo, teniendose como 
resultado un valor de $D$ de $6.58\times10^{-12}$ cm$^2$/s. El valor experimental 
de $D$ reportado por He \textit{et al} para esta proporción atómica fue de
$5.12\times10^{-12}$ cm$^2$/s. Con respecto al valor de $k^0$, se obtuvo 
$8.11\times10^{-8}$ cm/s. Comparar este valor con el experimental no es tan 
directo, ya que lo que reportan es la densidad de corriente de intercambio, 
$i^0 = 2.7\times10^{-4}$ mA/cm$^2$. Utilizando la ecuación \ref{eq:bv} de 
Butler-Volmer se tiene una relación entre $k^0$ e $i^0$ dada por
\begin{equation}\label{eq:i0k0}
    i^0 = F \frac{\rho}{M_r} k^0 \left(x_s\right)^{\alpha} \left(1 - x_s\right)^{1-\alpha},
\end{equation}
donde las definiciones de los parámetros están dadas en la Tabla \ref{t:params}.
Asumiendo un valor de 0.5 para el coeficiente de transferencia $\alpha$, un 
SOC de 0.5 y utilizando los valores del precusor LTO para $M_r = 459.1$ g/mol y
$\rho = 3.48$ g/cm$^3$ \cite{osti_1284125} se obtiene un valor para $k^0$ de
$7.38\times10^{-10}$ cm/s, que presenta una discrepancia de dos ordenes de 
magnitud con respecto al ajustado en el modelo. Sin embargo, en la literatura
se encuentran valores de $i^0$ con una gran dispersión entre 
$i^0 = 1.1\times10^{-3}$ mA/cm$^2$ \cite{medina2015} y $i^0 = 0.5$ mA/cm$^2$ 
\cite{umirov2019} que darían valores de $k^0$ entre $3.00\times10^{-9}$ cm/s y 
$1.37\times10^{-6}$ cm/s, respectivamente. Por lo cual puede afirmarse que el 
valor estimado por el modelo es razonable, dada la simplicidad del mismo.

Otro sistema en el cual las C-rates a las que se midieron los perfiles 
galvanostáticos cubren un rango amplio de valores de SOC$_{\max}$, de 
completamente cargado a completamente descargado, es el de LiFePO$_4$ (LFP) de 
Lei \textit{et al} \cite{lei2015}, como puede verse en la Figura 
\ref{fig:ajustes}. En este trabajo consideraron sistemas LFP/nanotubos de 
carbono/grafeno (LFP-CNT-G) como materiales catódicos con una capacidad 
superior a velocidades de carga alta y un desmpeño favorable en sucesivos 
ciclados a densidades de corriente relativamente altas, comparados con 
sistemas LFP-CNT y LFP-G. Para este caso seleccionado, obtuvieron coeficientes 
de difusión, a partir de la pendiente de un ajuste lineal a mediciones de 
espectroscopia de impedancia electroquímica (EIS), de $1.04\times10^{-12}$ 
cm$^2$/s, $1.738\times10^{-13}$ cm$^2$/s y $8.225\times10^{-13}$ cm$^2$/s, 
respectivamente para cada uno de los sistemas mencionados. Mientras que al
ajustar los datos experimentales del primer sistema mencionado, con un tamaño
de partícula de $0.35 \mu$m, se obtuvo un valor de $2.85\times10^{-13}$ cm$^2$/s
para este parámetro. Como puede observarse, se aprecia una discrepancia de un 
orden de magnitud pero dentro de los valores obtenidos en las otras sintesís.
El valor obtenido para $k^0$ utilizando la ecuación \ref{eq:i0k0} y el dato 
$i^0=5.127\times10^{-4}$ mA/cm$^2$ es $1.23\times10^{-9}$ cm/s, con un valor
de $M_r$ de $157.75$ g/mol y $\rho$ de $1.36$ g/cm$^3$ \cite{jin2018}.
En este caso se encuentra una correspondencia excelente con el valor ajustado
de $1.00\times10^{-9}$ cm/s.




\subsection{Predicción del tamaño óptimo de partícula}

Como ya ha sido mencionado a lo largo de esta tesis, el criterio de carga rápida
está definido por la obtención del 80\% de la capacidad del electrodo en 15 
minutos, lo cual se traduce en un SOC$_{\max}$ de 0.8 y una C-rate de 4 C. La
Figura \ref{fig:prediccion} muestra donde se encuentra cada sistema analizado en
el diagrama $\log(\Xi)$--$\log(\ell)$ para dicha C-rate. También se presenta una
curva de nivel con una línea roja correspondiente a SOC$_{\max} = 0.8$. Puede
observarse que tres de los materiales ya se encuentran en la región de 
SOC$_{\max}$ mayor a 0.8 (LCO, LMNO y LNMO), mientras que los otros se encuentran
por debajo de este valor (LTO, Grafito amorfo y LFP).
\begin{figure}[h!]
    \centering
    \includegraphics[width=0.7\textwidth]{FastCharging/un/resultados/prediccion/prediccion.png}
    \caption{Diagrama de SOC$_{\max}$ mostrando la ubicación de los materiales 
    usuales de LIBs a 4 C para las referencias consideradas \cite{mancini2022,
    he2012, lei2015, wang2019high, bak2011, nishikawa2017}. En los casos en los 
    que la curva de cargado a 4 C no estaba disponible, el valor del punto fue 
    predicho con el modelo. La línea roja muestra la curva de nivel 
    correspondiente al valor 0.8 de SOC$_{\max}$. Las flechas muestran el cambio
    en el tamaño de la partícula que debería efectuarse para obtener dicho valor
    a la C-rate dada. Las curces sobre la línea muestran la posición de estos
    tamaños de partícula nuevos.}
    \label{fig:prediccion}
\end{figure}
Haciendo uso del diagrama se puede predecir una forma simple y rápida el tamaño 
de partícula requerido para satisfacer el criterio de carga rápida. Dado que los
valores de $D$ y $k^0$ ya fueron ajustados, el valor de $d$ seleccionado por el
experimento y el de C-rate por el criterio, el valor de $\Xi$ es constante. Luego,
para alcanzar el valor de 0.8 de SOC$_{\max}$ hay que variar $\ell$ y esto se
logra disminuyendo o aumentando el tamaño de la partícula, según sea necesario. 
Este desplazamiento necesario está representado por las flechas en la Figura 
\ref{fig:prediccion} para cada caso. Ya se ha apreciado que tres sistemas se 
encuentran en la región ya optimizada (LCO, LMO y LNMO), por lo que en estos casos
los tamaños predichos para alcanzar SOC$_{\max} = 0.8$ a 4 C serán mayores que 
los experimentales. Por el contrario, el resto de los materiales (LTO, Grafito
amorfo y LFP) tienen que ser mejorados con una reducción del tamaño de partícula
para cumplir la condición. En la tabla \ref{t:prediccion} se muestran los tamaños
de partícula predichos para todos los materiales en la tercera columna para este
criterio. Las incertidumbres se determinaron por propagación de errores con 
derivadas parciales. Ya que el tamaño de la partícula sólo aparece en el parámetro
$\ell$, al definir $\ell_{\text{opt}}$ como el valor al cual el SOC$_{\max}$ 
alcanza el valor deseado de 0.8 y usar que $V/A = d/z$ se puede despejar de la 
ecuación \ref{eq:ele} que
\begin{equation}
    d = \sqrt{\frac{t_h z D 10^{\ell_{\text{opt}}}}{C_r}}.
\end{equation}
Si además se supone que toda la incertidumbre está asociada al coeficiente de 
difusión $D$, al cual ya se le calculó su incerteza, se puede obtener que
\begin{equation}
    \Delta d = \frac{1}{2} \sqrt{\frac{t_h z 10^{\ell_{\text{opt}}}}{C_r D}} \Delta D.
\end{equation}

\begin{table}[h!]
    \centering
    \caption{Tamaño experimental y valores predichos para cargar el 80\% del
    electrodo en 15 y 5 minutos.} 
    \setlength\extrarowheight{2pt}\stackon{%
    \begin{tabular}{l c c c}
        \toprule
        \textbf{Material del} &
        \textbf{Tamaño} &  
        \textbf{Tamaño predicho} & 
        \textbf{Tamaño predicho} \\
        \textbf{electrodo} & 
        \textbf{experimental [$\mu$m]} &  
        \textbf{para 15 minutos [$\mu$m]} & 
        \textbf{para 5 minutos [$\mu$m]} \\
        \midrule
        Grafito amorfo & 7.5 & 4.027 $\pm$ 0.002 & 2.167 $\pm$ 0.001 \\
        LTO & 1.75 & 0.962 $\pm$ 0.004 & 0.530 $\pm$ 0.002 \\
        LFP & 0.35 & 0.084 $\pm$ 0.002 & 0.0309 $\pm$ 0.0006 \\
        LCO & 20 & 28.8 $\pm$ 0.6 & 16.4 $\pm$ 0.4 \\
        LMO & 0.025 & 0.0734 $\pm$ 0.0003 & 0.0418 $\pm$ 0.0002 \\
        LNMO & 7.999 & 13 $\pm$ 2 & 7.3 $\pm$ 0.8 \\
        \bottomrule
    \end{tabular}
    }{}
    \label{t:prediccion}
\end{table}

Al observarse un buen desempeño para la carga de 15 minutos, se puede exigir un 
poco más que este criterio y predecir el tamaño de partícula requerido para una
C-rate más alta, digamos 80\% de la carga en 5 minutos (12 C). Si bien esta figura
puede parecer sobredemandante a primera vista, reportes recientes consideran 
protocolos de carga de 10 minutos \cite{mattis2021, attia2020}. Los resultados
se muestran en la última columna de la Tabla \ref{t:prediccion}. Como puede 
observarse, el comportamiento depende del sistema y del experimento en particular
considerado. El único caso donde se cumple este último criterio de carga rápida 
es el LMO, ya que el tamaño experimental sobrecumple el criterio. Aunque el LCO 
y el LNMO no cumplen con este último criterio, los cambios en sus tamaños serían
menores, por lo que estos materiales requieren mejoras menores. En el resto de 
los casos, para el LFP se necesitaría una disminución de un orden de magnitud 
en su tamaño, mientras que para el grafito amorfo o el LTO se requeriría una
disminución de su tamaño en un factor de 3.


\subsection{Comparación de sistemas con un mismo tamaño}

La curva de los distintos experimentos ajustados sigue el mismo comportamiento 
para todos los sistemas, como se mostró en la Figura \ref{fig:ajustes-mapa},
esto es un efecto esperado debido a las definiciones de $\Xi$ y $\ell$, todas las
curvas exhiben una pendiente de $-1/2$ dada por la siguiente ecuación
\begin{equation}
    \log(\Xi) = \log(B) - \frac{1}{2}\log(\ell),
\end{equation}
donde el valor de $B$ puede obtenerse al eliminar C-rate de las ecuaciones 
\ref{eq:xi} y \ref{eq:ele} para obtener
\begin{equation}
    \log(\Xi) = \log\left(\frac{k^0 d}{D \sqrt{z}}\right) - \frac{1}{2}\log(\ell),
\end{equation}
donde se ve que la ordenada al origen contiene una composición de los parámetros
fundamentales considerados. 

\begin{figure}[t]
    \centering
    \includegraphics[width=\textwidth]{FastCharging/un/resultados/comparacion/comparacion.png}
    \caption{Comparación de los sistemas considerados con distintos tamaños de 
    partícula entre 0.1 $\mu$m y 10.0 $\mu$m en el rango experimental usual para 
    valores de C-rates: (a) SOC$_{\max}$ \textit{versus} C-rate. (b) Diagramas.}
    \label{fig:comparacion}
\end{figure}

Si se quieren comparar los méritos de los distintos materiales, en términos de 
sus propiedades intrínsecas de transferencia de carga en la interfase 
electrodo/electrolito ($k^0$) y difusión de iones dentro de ellos ($D$), se 
debería comparar el comportamiento de las partículas para un mismo tamaño a 
distintas C-rate, lo cual se presenta en la Figura \ref{fig:comparacion}. En 
particular, en la Figura \ref{fig:comparacion}a se muestra el SOC$_{\max}$ 
en función de la C-rate, considerando un conjunto de tamaños de partículas y 
C-rates en un rango físicamente razonable. Para obtener estos resultados se 
utilizaron los valores de $D$ y $k^0$ ajustados en la sección \ref{s:ajustes}.

En el gráfico (i) de la Figura \ref{fig:comparacion}a para 0.1 $\mu$m está
claro que los únicos materiales que muestran una pérdida de la capacidad para
C-rates altas son LFP y LMO, el resto retiene más del 80\% de la capacidad, 
incluso a 100 C. En el segundo gráfico (ii) de la Figura
\ref{fig:comparacion}a para 1 $\mu$m, el LTO y el NG tienen
una caída en el SOC por debajo del 50\% para 100 C, mientras que LCO y LNMO
están por encima del 80\%. Por último, el tamaño de partícula más grande que se 
considera, 10 $\mu$m (Figura \ref{fig:comparacion}a, gráfico (iii)), 
todos los materiales presentan una retención de la capacidad por debajo del 80\%
a la C-rate más alta. En la Figure \ref{fig:comparacion}b estos datos 
comportamientos están presentados en el diagrama construido con las simulaciones
galvanostáticas para dar una idea de las regiones en las que cada sistema se
encuentra. 

Cabe destacar que la secuencia de materiales dada en la Figura 
\ref{fig:comparacion} fue obtenida utilizando los valores de $k^0$ y $D$ 
ajustados con el modelo a los datos experimentales. Ajustar otros experimentos
podría alterar esta secuencia. Idealmente, las mediciones deberían estar 
realizadas sobre electrodos de una sola partícula.



\section{Conclusiones del capítulo}

