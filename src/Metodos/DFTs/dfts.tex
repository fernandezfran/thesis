\section{Mecánica cuántica}

El desarrollo de la mecánica cuántica, y las observaciones experimentales que la
validan, fue uno de los avances científicos más significativos del siglo veinte
en describir el universo en el que vivimos \cite{shankar2012}.

La aproxmiación de Born-Oppenheimer permite separar los núcleos de los electrones 
en problemas de mecánica cuántica. Esta aproximación se basa en que los núcleos 
atómicos son mucho más pesados que los electrones individuales. Eso implica que 
puede considerarse que los electrones responden mucho más rápido a los cambios en 
su entorno que los núcleos, lo que permite resolver las ecuaciones que describen 
el movimiento de los electrones para posiciones fijas de los núcleos atómicos. Así 
se encuentra el estado fundamental de los electrones, es decir, la configuración 
de menor energía de los mismos. Esto permite estudiar como varía la energía de los 
materiales con el movimiento de sus átomos.

La ecuación de Schrödinger no-relativista e independiente del tiempo,
\begin{equation}\label{eq:schrodinger}
    H \psi = E \psi,
\end{equation}
caracteriza un sistema físico desde un punto de vista cuántico, donde $\psi$ es un
conjunto de soluciones, o autoestados, del Hamiltoniano $H$ que tienen asociados los
autovalores $E$ que satisfacen dicha ecuación. Para el caso en el que se desee estudiar
la interacción entre varios electrones y núcleos una descripción más completa de la 
ecuación de Schrödinger es la siguiente,
\begin{equation}\label{eq:schrodinger}
    \left[ - \frac{\hbar^2}{2 m} \sum_{i=1}^N \nabla_i^2 + \sum_{i=1}^N V(\mathbf{r}_i) + \sum_{i=1}^N \sum_{i<j} U(\mathbf{r}_i, \mathbf{r}_j) \right] \psi = E \psi,
\end{equation}
donde $\hbar$ es la constante de Plank reducida, $m$ la masa del electrón, el primer 
término describe la energía cinética de cada electrón, el segundo la energía de 
interacción de cada electrón con todos los núcleos y el tercero la energía de 
interacción entre distintos electrones. Este último término es el más crítico desde 
el punto de vista de la resolución de la ecuación.

Ahora, la cantidad que puede medirse es la probabilidad de que los $N$ electrones 
estén en un conjunto de posiciones $\lbrace \mathbf{r}_i \rbrace$ en cualquier orden. 
Esta probabilidad se calcula multiplicando $\psi$ por su complejo conjugado $\psi^*$.
Una cantidad relacionada a esta probabilidad es la densidad de electrones,
$n(\mathbf{r})$. Esto se puede escribir en términos de las funciones de onda 
individuales de los electrones como
\begin{equation}
    n(\mathbf{r}) = 2 \sum_i \psi_i^*(\mathbf{r}) \psi_i(\mathbf{r}),
\end{equation}
donde la suma se realiza sobre todos los electrones $i$. El punto de esta ecuación
es que reduce la solución completa de la función de onda de la ecuación de Schrödinger
de 3$N$ coordenadas a tan sólo 3 al plantearla en términos de la densidad electrónica, 
$n(\mathbf{r})$, que además contiene una gran cantidad de información que es 
observable.

\subsection{Teoría del funcional de la densidad (DFT)}

La teoría del funcional de la densidad (DFT por sus siglas en inglés
\textit{Density functional theory}) es un método altamente efectivo para 
encontrar soluciones a la ecuación fundamental que describe el comportamiento 
cuántico de átomos y moleculas en sistemas de materia condensada, la ecuación 
\ref{eq:schrodinger} de Schrödinger, en situaciones de utilidad práctica 
\cite{sholl2022}.

\subsection{Funcional de la densidad de enlace estrecho (DFTB)}\label{s:dftb}

\section{Software}

En la actualidad existe una gran cantidad de códigos que permiten realizar 
simulaciones de dinámica molecular de manera eficiente aprovechando la estructura
paralela de los procesadores. En esta tesis se utilizó principalmente \path{LAMMPS} 
(\textit{Large-scale Atomic/Molecular Massively Parallel Simulator}, sus siglas
en inglés) ~\cite{lammps1, lammps2}, un código centrado en el modelado de 
materiales que permite simular con distintos campos de fuerzas, ensambles y 
condiciones de contorno. También se realizaron simulaciones con versiones 
modificadas de \path{GEMS} ~\cite{gems}, código del Dr. Sergio Alexis Paz 
(FCQ-UNC), que permiten utilizar a \path{DFTB+} ~\cite{dftb+} como una librería
y realizar distintos métodos de aceleración de simulaciones que no se encuentran
en los programas usuales como \path{LAMMPS}.

Para la visualización de las trayectorias y la obtención de imágenes de 
estructuras representativas se utilizó \path{VMD} (\textit{Visual Molecular 
Dynamics}) ~\cite{vmd}. Para el post-procesamiento de datos de las simulaciones
se escribieron distintos programas en Python ~\cite{exma, sierras}.

