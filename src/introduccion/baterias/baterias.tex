\subsection{Baterías de ion litio}

Las baterías de litio son dispositivos electroquímicos ampliamente utilizados 
como fuentes de energía. Entre las propiedades físicas del litio destacan su 
peso molecular bajo (7 g mol$^{-1}$), su densidad baja (0.534 cm$^3$), su 
capacidad específica alta (3860 mAh g$^{-1}$) y su potencial de reducción bajo 
(-3.04 V vs. SHE), estas propiedades lo convirtieron en un material que puede 
ser utilizado como ánodo para baterías. 

A finales de la década del 1950 se observó que el litio metálico formaba una capa 
de pasivación, llamada interfaz de electrolito sólido (SEI, de sus siglás en 
inglés, \textit{solid electrolyte interface}), con distintos electrolitos 
no-acuosos permitiendo prevenir una reacción química directa entre estas dos 
componentes, pero dejando que los iones de litio la atraviesen ~\cite{peled1979}. 
Esto generó un interés que llevó a la fabricación de baterías primarias de litio 
en la década del 1960 utilizando distintos tipos de cátodos incluyendo dióxido de 
sulfuro de litio (LiSO$_2$), óxido de manganeso de litio (LiMnO$_2$) y óxido de 
cobre de litio (LiCuO), entre otros. Desafortunadamente, la formación de la SEI 
no es estable durante ciclos prolongados y se agrieta, lo que lleva a un consumo 
continuo de electrolito y litio para la reformación continua de la SEI 
~\cite{besenhard1976}. Y lo que es aún peor, la deposición desigual de litio en 
la SEI agrietada conduce al crecimiento de dendritas de litio, que terminan 
rompiéndose y formando islas de tamaño nanométrico de litio altamente reactivo, 
lo que reduce la estabilidad térmica de la celda ~\cite{yamaki1998}. Las 
dendritas también pueden crecer lo suficiente como para generar un contacto 
entre el ánodo y el cátodo, provocando un cortocircuito. 

Avances posteriores en la comprensión de la intercalación de litio en diferentes
materiales dio origen a las baterías recargables de ion-litio. El primero de 
estos materiales fue utilizado por EXXON para una primera versión de las mismas.
Whittingham demostró en 1976 que el litio se intercala entre las capas que se 
forman en los metales de transición en una proporción de un mol que provoca una
expansión del parámetro de red ~\cite{whittingham1976}. El material más 
interesante de estos era el TiS$_2$, que ofrece el menor peso molecular, un costo 
potencialmente bajo y es un conductor eléctrico. Sin embargo, esta batería seguía
utilizando litio metálico como ánodo, presentando los problemas ya mencionados. 

La necesidad de un material catódico de intercalación de litio de alto voltaje 
que se presentaba en la época fue resuelta por el compuesto laminar LiCoO$_2$ 
(LCO) que fue desarrollado por el grupo de investigación de Goodenough en 1980
~\cite{mizushima1980}. La gran diferencia de tamaño entre el cobalto y el litio 
da lugar a un material de capas perfecto con poca o ninguna mezcla de cationes. 
Este aspecto hace que la síntesis estequiométrica de LCO y de fase pura sea muy 
fácil en diversas condiciones. 

Sin embargo, se necesitaba un material anódico estable que permitiera intercalar
los iones de litio. No fue hasta el 1990 que esto se cumplió, año en el que el
grupo de investigación de Dahn descubrió dicha estabilidad en los ánodos de 
grafito en soluciones electrolíticas no acuosas al estudiar la intercalación de 
forma reversible en los primeros 19 ciclos utilizando como electrolito una mezcla 
de disolvente de carbonato de propileno y carbonato de etileno ~\cite{fong1990}. 
Esto se da gracias a la formación de una SEI estable durante el primer ciclo de 
intercalación. 

En la tabla \ref{t:historia} se presentan las primeras baterías secundarias de 
litio comercializadas hasta 1991, año en el que se dejó de utilizar litio 
metálico como ánodo y se comenzó a utilizar uno de grafito. En la misma puede 
notarse el avance con los años de la densidad de energía, sin embargo este no es 
el único parámetro a tener en cuenta en el progreso ya que también tienen que 
considerarse la cantidad de ciclos que se pueden realizar, el tiempo de carga, la 
razón de descarga y el precio de producción ~\cite{reddy2020}. 
\begin{table}[h]
    \centering
    \caption{Primeras baterías de litio recargables comercializadas 
    ~\cite{reddy2020}.}
    \setlength\extrarowheight{2pt}\stackon{%
    \begin{tabular}{c c c c}
        \toprule
        \thead{\normalsize\bfseries Sistema\\\normalsize\bfseries electroquímico} & 
        \thead{\normalsize\bfseries Voltaje (V)} & 
        \thead{\normalsize\bfseries Densidad de energía\\\normalsize\bfseries específica (Wh/kg)} & 
        \thead{\normalsize\bfseries Compañía comercial} \\
        \midrule
        Li||TiS$_2$ & 2.1 & 130 & Exxon (1978) \\
        Li||LiAlCl$_4$-SO$_2$ & 3.2 & 63 & Duracell (1981) \\
        Li||NbSe$_3$ & 2.0 & 95 & Bell Telephone Lab. Inc. (1983) \\
        LiAl||polianilina & 3.0 & -- & Bridgestone (1987) \\
        Li||MoS$_2$ & 1.8 & 52 & MoLi Energy (1987) \\
        Li||V$_2$O$_5$ & 1.5 & 10 & Toshiba (1989) \\
        LiAl||polipirrol & 3.0 & -- & Kanebo (1989) \\
        Li||Li$_{0.3}$MnO$_2$ & 3.0 & 50 & Tadiran (1989) \\
        LiVO$_x$ & 3.2 & 200 & Hydro-Québec (1990) \\
        C||LiCoO$_2$ & 3.6 & 150--190 & Sony (1991) \\
        \bottomrule
    \end{tabular}
    }{}
    \label{t:historia}
\end{table}

La batería que funciona mediante un cátodo de LCO, un ánodo de grafito y una
solución electrolítica no acuosa basada en LiPF$_6$ fue comercializada
exitosamente por Sony en 1991. Desde entonces las baterías de ion litio fueron
adaptadas para su utilización en dispositivos móviles, esto se debió a la
densidad de energía volumétrica alta, de alrededor de los 200 W h L$^{-1}$, que
era el doble que la de níquel-cadmio, su principal competidor en esa época.

En las últimas dos décadas se ha logrado duplicar la densidad de energía 
volumétrica de entonces, lo cual fue realizado en mayor medida gracias a una 
mejor ingeniería en la celda. El próximo desafío en el área de estudio viene 
dado en el desarrollo de LIBs que sean adecuadas para los vehículos eléctricos. 
Para lograr esto es necesario una mejora de la funcionalidad de cada una de las
partes de la batería.

Aunque el LCO supera a otros cátodos de intercalación de litio términos de 
densidad de energía volumétrica, la baja capacidad de 135 mAh g$^{-1}$ da una baja
densidad de energía gravimétrica. Para solucionar este problema se han 
comercializado, o están siendo considerados, distintos cátodos. Como una 
alternativa más barata al LCO se estudió el LiNiO$_2$, que es isoestructural al 
LCO pero posee una capacidad teórica mayor, de 273 mAh g$^{-1}$. Sin embargo,
es difícil de sintetizar y se ha observado la formación de una fase 
electroquímicamente inactiva cuando la concentración de litio alcanza el 50\% 
~\cite{ohzuku1993}. Otra de las variantes propuestas al LCO fue el compuesto 
laminar LiMnO$_2$, pero su capacidad de carga inicial de unos 220 mAh g$^{-1}$ 
desciende a 130 mAh g$^{-1}$ en los ciclos siguientes. Además se produce un 
reordenamiento bastante significativo en el que el material activo se transforma 
en la espinela LiMn$_2$O$_4$ (LMO) tras sólo 5 ciclos ~\cite{shao1999, shin2004}.
Ambos materiales son llamativos debido al bajo costo y la menor toxicidad del 
manganeso con respecto al cobalto o al níquel. Finalmente, con respecto a estos 
tres metales de transición, se desarrollaron compuestos de intercalación basados
en los mismos con la fórmula general Li(Ni$_{1-x-y}$Co$_x$Mn$_y$)O$_2$ (NCM o NMC)
~\cite{liu1999}. Para conseguir un material catódico viable, deben ser controladas
las proporciones de cada uno de los metales según los aspectos que se deseen de 
alta capacidad (más níquel), mejor estabilidad de ciclado (más cobalto) o 
seguridad/costo (más manganeso). Otra de las opciones es el dopaje con cationes
de cobalto y aluminio de LiNiO$_2$, Li(Ni$_{1-x-y}$Co$_x$Al$_y$)O$_2$ (NCA),
que en la composición Li(Ni$_{0.8}$Co$_{0.15}$Al$_{0.05}$)O$_2$ ~\cite{chen2004}
es comercializada por Panasonic y utilizado en algunos de los modelos de autos 
eléctricos de Tesla. El olivino LiFePO$_4$ (LFP) fue introducido por primera vez
por el grupo de Goodenoughs en 1997 ~\cite{padhi1997}, donde se demostraron 
capacidades menores a los 120 mAh g$^{-1}$ pero el material ganó interés debido
al costo barato y la naturaleza benigna del hierro con respecto al cobalto y el
níquel. Los problemas asociados al LFP son la conductividad eléctrica y la 
difusividad del litio bajas, sin embargo, debido a su larga vida de ciclado, su
estabilidad térmica, seguridad, costo e impacto medioambiental son de gran 
interés en baterías grandes y ya se utiliza en algunos autos eléctricos.

Recientemente se vienen estudiado sistemas con reacciones catódicas más 
energéticas, abandonando los cátodos de intercalación de iones de litio, como son 
los cátodos de azufre o de oxígeno molecular. Ambos sistemas se estudian con 
ánodos de litio metálico. A pesar de la capacidad teórica de 1675 m Ah g$^{-1}$,
el precio bajo y la benignidad frente al medio ambiente, las baterías de 
litio-azufre sufren de una cantidad de problemas que hicieron que este sistema 
no fuera atractivo a principios de los años 1970, cuando se realizaron los 
primeros estudios. Estos problemas relacionados con el azufre son su baja
conductividad eléctrica, su expansión volumétrica del 80\% durante la 
carga/descarga y la disolución de polisulfuros en el electrolito y la pérdida de 
contacto entre con el colector de carga que conduce a un aumento de azufre 
inactivo en las celdas y, finalmente, una disminución de la capacidad. En los
últimos años se han llevado a cabo distintos esfuerzos para afrontar estos 
problemas ~\cite{zhao2020}.

Los electrolitos son sistemas complicados que están formados por una mezcla de 
disolventes, una sal y aditivos. Las soluciones electrolíticas que se utilizan en
las LIB actuales consisten disolventes de carbonato de etileno (EC) y carbonatos 
lineales, como el carbonato de etilo y metilo (EMC) o el carbonato de dimetilo 
(DMC), combinados con LiPF$_6$ como sal de litio. Los aditivos que se utilizan 
en las soluciones comerciales no suelen divulgarse ~\cite{schipper2016}. En 2018,
el departamento de energía de los EEUU fijó un objetivo para 2028 de lograr 
la carga del 80\% de la capacidad de la LIB en menos de 15 minutos, esperando que
esto calme la \say{ansiedad por la autonomía} e impulsé la utilización de 
vehículos eléctricos masivamente. En dicho objetivo los electrolitos juegan un 
rol principal. En la actualidad se llevan a cabo investigaciones con solventes de
baja viscosidad ~\cite{logan2020}, en mejorar su seguridad ~\cite{wang2019} 
y en electrolitos de estado sólido ~\cite{zheng2018}.

Con respecto a los ánodos, el grafito sigue siendo el más utilizado en el ámbito
comercial. Entre las ventajas del mismo se presentan su capacidad del doble que 
la mayoría de los cátodos, por lo que el balance de masa en las baterías es fácil
de lograr, su expansión volumétrica leve del 10\% durante la carga y descarga, su
alta conductividad y buena eficiencia; además, es un material abundante y no es
tóxico. A pesar de esto, para las baterías de litio-azufre o litio-oxigeno, con
capacidades mayores, se necesitarían ánodos que presenten capacidades del mismo
orden. Si bien se continúan realizando investigaciones en materiales anódicos de 
intercalación, como lo es el titanato de litio (Li$_4$Ti$_5$O$_{12}$) que tiene
ciertas ventajas al no presentar cambios estructurales luego de la carga/descarga
pero que su densidad energética es baja comparada a otros ánodos, las 
investigaciones en ánodos están orientadas mayoritariamente a materiales que 
presentan densidades de energía mayores, como los son aquellos que forman 
aleaciones con el litio o el litio metálico. Entre estos materiales se encuentran
el silicio y el estaño, que presentan capacidades de 3579 m Ah g$^{-1}$ y
960 m Ah g$^{-1}$, respectivamente. Estos materiales además son baratos, 
abundantes y ambientalmente amigables, sin embargo, presentan un gran cambio de 
volumen, hasta un 300\%, durante la litiación y la delitiación, lo que lleva a
una estabilidad pobre del ciclado, presentando pérdidas en la capacidad 
irreversibles en el primer ciclo. Algunas de las causas de esta pérdida 
irreversible de capacidad son las siguientes: la pulverización de partículas 
activas de la aleación que provoca una desconexión con el colector de carga; la 
formación de una SEI con un comportamiento distinto a la del grafito, ya que aquí
es un proceso dinámico en el cual se rompe y se vuelve a formar con cada 
expansión volumétrica; y, la retención de iones de litio en la aleación debido a 
la cinética lenta o a la formación de aleaciones altamente estables. Para reducir
estos problemas se han desarrollado varias estrategias ~\cite{zhang2011} y se 
continúa estudiando activamente estos materiales.

\subsubsection{Ánodos de baterías de litio basados en silicio}

Dentro de los ánodos ya comentados que forman aleaciones con el litio, el silicio
se posiciona como uno de los más prometedores. Para afrontar los problemas 
críticos ya mencionados se han realizado enormes esfuerzos en las últimas décadas.
La mayoría de ellos están centrados en desarrollar estrategias efectivas que 
resuelvan el problema de la expansión volumétrica. Entre estos se encuentran la 
utilización de nanoestructuras, matrices para disminuir la tensión y construcción
de compartimientos para acomodar dicha expansión ~\cite{zuo2016}.

Para aplicaciones prácticas se han considerados estructura 3D de Silicio con 
estructura porosa, lo cual ayuda a acomodar los cambios volumétricos durante el 
ciclado sin perder la integridad estructural. Por ejemplo, depositando partículas
de Si en una estructura nanoporosa de SiO$_2$ se obtiene una capacidad por encima
de 2800 mAh g$^{-1}$ sin una degradación significativa luego de 100 ciclos 
~\cite{cho2010}. Sin embargo para obtenerla se requieren pasos experimentales 
complicados que consumen mucho tiempo, son costosos y difíciles de escalar.
Macroporos de Si 3D con un recubrimiento de carbono poseen una capacidad de 
2050 mAh g$^{-1}$ con una eficiencia coulómbica del 94\% y una retención estable
del 87\% de la capacidad luego de 50 ciclos.

Las películas delgadas 2D de Si minimizan la variación de volumen y mantienen su 
integridad estructural debido a su estructura uniforme, lo cual conduce a 
mejorar la estabilidad de los ciclos y la capacidad. Por ejemplo, la 
electrodeposición de Si en una película de Ni permitió que el cambio de volumen 
del Si sea de forma eficaz, presentando una capacidad alta de más de 2800 mAh 
g$^{-1}$ durante 80 ciclos ~\cite{zhao2012}. Por otro lado, Li obtuvo una 
película delgada nanoestructurada de tipo core-shell compuesta por nanoesferas 
huecas de CNS/Si/Al$_2$O$_3$, donde la capa fina de Al$_2$O$_3$ evita la 
formación de la SEI y el contacto entre Si y CNS promueve un transporte rápido 
de electrones, todo esto lleva a una capacidad específica de 1560 mAh g$^{-1}$ 
con una retención del 85\% luego de 100 ciclos ~\cite{li2015}. También se han 
propuesto nanoplanchas de Si con un espesor de 5 nm recubiertas de carbono, que 
presentan un rendimiento electroquímico excelente de una capacidad específica de 
865 mAh g$^{-1}$ y una retención del 92.3\% luego de 500 ciclos ~\cite{ryu2016}.

Los nanohilos o nanotubos 1D de Si poseen un camino electrónico 1D que facilita 
el transporte de carga de manera eficiente, además, tienen la ventaja de permitir
una expansión radial de silicio que minimiza la tendencia del material a 
agrietarse. Por el lado de los nanohilos, si a los mismos se le incorporan poros 
y huecos, para proveer espacio extra y poder acomodar la expansión del volumen, 
se obtiene un mejor rendimiento. Por ejemplo, con un diámetro de poro de 8 nm se 
obtuvo una capacidad estable de 2000 mAh g$^{-1}$ ~\cite{ge2012}. Arreglos con 
este tipo de nanohilos y una superficie tipo coral de Cu como colector de carga 
demostraron una capacidad de 2178 mAh g$^{-1}$ luego de 50 ciclos 
~\cite{jing2014}. Al recubrir los nanohilos con una capa de carbono conductora 
los electrodos exhiben capacidades de alrededor de 2000 mAh g$^{-1}$ durante 100 
ciclos ~\cite{bogart2014}. Resultados del mismo orden se obtienen si este 
recubrimiento se realiza con estaño ~\cite{kohandehghan2014}. Por otro lado, 
para los nanotubos, se tiene una retención después de 90 ciclos de una capacidad 
de 1000 mAh g$^{-1}$ ~\cite{wen2013nt}. Para mejorar este comportamiento 
electroquímico también se los ha recubierto con carbono y se obtuvo una 
capacidad de 2085 mAh g$^{-1}$ y una retención del 95\% de la misma luego de 200
ciclos, relativo al ciclo 10 ~\cite{lu2014}.

Las nanopartículas 0D de silicio con tamaños lo suficientemente pequeños pueden 
evitar la fractura mecánica y ofrecer un área superficial grande. Sin embargo 
suelen verse recubiertas por la SEI, lo cual produce los mismos problemas 
mencionados. La utilización de una matriz de carbono y la construcción de 
estructuras jerárquicas resuelven este problema. Dentro de esta sección un
compuesto de nanopartículas embebidas en un polímero conductor demostró una 
capacidad de 1670 mAh g$^{-1}$ con una eficiencia coulómbica inicial del 78\% y 
una retención de la capacidad del 60\% luego de 400 ciclos ~\cite{chen2014}. 
También se desarrollaron estructuras dendríticas de carbono para incrustar 
nanopartículas de Si y presentó una capacidad específica de 1590 mAh g$^{-1}$ 
~\cite{magasinski2010}. Ánodos de nanopartículas recubiertas por una coraza de 
carbono presentaron una capacidad reversible de 1800 mAh g$^{-1}$ durante 50 
ciclos ~\cite{hwa2012}. Un compuesto único de grafeno con nanopartículas de Si 
unidas de manera covalente presentó un buen rendimiento con una capacidad inicial 
de 2250 mAh g$^{-1}$ y de 1800 mAh g$^{-1}$ luego de 120 ciclos ~\cite{wen2013}.

Por último, en adición a esta nanoestructuras, aleaciones o sistemas 
multicomponentes a base de Si también han sido propuestos y estudiados como 
ánodos. Un caso destacado de estos sistemas son las nanopartículas de Si 
embebidas en Ti$_4$Ni$_4$Si$_7$, con una capacidad de 1325 mAh g$^{-1}$, una 
eficiencia coulómbica inicial del 87\% y una retención del 78\% de la capacidad 
luego de 50 ciclos ~\cite{son2012}. También se propusieron estructuras porosas 
del compuesto NiSi$_2$/Si/C, con mesoporos con un tamaño medio de unos 22 nm que
acomodan bien la expansión de volumen del silicio y resulta en una capacidad 
estable durante 200 ciclos de 1272 mAh g$^{-1}$ ~\cite{jia2015}.

\

TODO: Histograma de papers con palabras claves de baterías de litio a lo largo en
función de los años(scopus).
