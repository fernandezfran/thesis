% Copyright (c) 2024, Francisco Fernandez
% License: CC BY-SA 4.0
%   https://github.com/fernandezfran/thesis/blob/main/LICENSE
\chapter{Comentarios finales y perspectivas futuras}\label{ch:comentarios}

\section{Comentarios finales}

Las baterías de ion-litio han permitido que se desarrollaran una gran variedad de
dispositivos electrónicos, por ejemplo los teléfonos inteligentes o las 
computadoras portátiles. A su vez son utilizadas en vehículos eléctricos y en 
sistemas de almacenamiento estacionarios de energía para fuentes
renovables, lo cual las convierte en un actor crucial para la sustitución de 
los combustibles fósiles en el consumo energético. En rasgos generales, se puede
afirmar que los dos grandes avances logrados aquí se refieren al establecimiento
de una métrica universal sin precedentes para predecir y evaluar el cargado rápido
de materiales al nivel de una partícula, y a la predicción de las propiedades de 
aleaciones amorfas de Li-Si, uno los materiales más promisorios para ser empleado 
como ánodos de baterías de ion-Li de próxima generación. De este modo, en esta 
tesis doctoral se 
aplicaron distintos modelados computacionales para estudiar electrodos de 
baterías de ion-litio de próxima generación, las técnicas utilizadas fueron 
introducidas en el capítulo \ref{ch:metodos}. Los resultados obtenidos fueron 
divididos en dos partes: \textbf{Carga rápida de baterías de ion-litio} (Parte 
\ref{p:fast-charging}) y \textbf{Silicio como ánodo de baterías de ion-litio de 
próxima generación: Estudio de sus aleaciones} (Parte \ref{p:silicio}). Parte de los resultados obtenidos fueron publicados en revistas científicas de referencia en el área:
\begin{enumerate}
    \item \underline{Fernandez, F.}, Otero, M., Oviedo, M. B., Barraco, D. E., Paz, S. A., \& Leiva, E. P. M. (2023). NMR, x-ray, and Mössbauer results for amorphous Li-Si alloys using density functional tight-binding method. \textit{Physical Review B, 108}(14), 144201.
    \item \underline{Fernandez, F.}, Gavilán-Arriazu, E. M., Barraco, D. E., Visintin, A., Ein-Eli, Y., \& Leiva, E. P. M. (2023). Towards a fast-charging of LIBs electrode materials: a heuristic model based on galvanostatic simulations. \textit{Electrochimica Acta, 464}, 142951.
    \item Oviedo, M. B., \underline{Fernandez, F.}, Otero, M., Leiva, E. P., \& Paz, S. A. (2023). Density Functional Tight-Binding Model for Lithium–Silicon Alloys. \textit{The Journal of Physical Chemistry A, 127}(11), 2637-2645.
    \item \underline{Fernandez, F.}, Paz, S. A., Otero, M., Barraco, D., \& Leiva, E. P. (2021). Characterization of amorphous Li x Si structures from ReaxFF via accelerated exploration of local minima. \textit{Physical Chemistry Chemical Physics, 23}(31), 16776-16784.
\end{enumerate}
Se encuentran en redacción otros manuscrítos a ser publicados.

En el capítulo \ref{ch:un} se desarrolló un modelo que permite predecir el tamaño
óptimo que deberían tener las partículas de material activo en un electrodo para 
alcanzar el 80\% del Estado de la Carga (SOC) en 15 minutos, que es el criterio
establecido para considerar que la batería sea de carga rápida. Además, se 
desarrolló un software en Python que cumple con los requisitos establecidos por 
la comunidad, es de libre acceso y fácil de usar para realizar el preprocesamiento
de datos experimentales y las estimaciones de parámetros que emplea el modelo. En esta oportunidad, se 
utilizaron datos experimentales de literatura de distintos materiales de 
relevancia en el área de estudio (NG, LTO, LFP, LCO, LMO, LNMO). Para todos ellos,
en un preprocesamiento de los datos se obtuvo el SOC máximo en función de la 
velocidad de carga galvanostática (C-rate) y se ajustó el modelo, de donde 
se obtuvieron coeficientes de difusión y constantes cinéticas físicamente 
correctas. Dicho ajuste se realiza de manera heurística sobre una superficie 
obtenida mediante un método de simulación universal de una sola partícula en 
condiciones de carga a corriente constante, que es capaz de reproducir la física 
básica del proceso de intercalación de litio, regulado por la difusión de los 
iones dentro de la partícula y la transferencia de carga en su interfase.
Una vez ajustado el modelo a cada sistema, se lo utilizó para predecir el tamaño
óptimo de partícula para cargas de 15 y 5 minutos y se discutió cada caso en 
particular. Por último, se compararon los méritos de los materiales entre sí en 
términos de sus propiedades intrínsecas y a igual tamaño de partícula.

Dentro de la misma área de estudio, en el capítulo \ref{ch:umbem} se propuso, por 
primera vez en la literatura, una métrica universal para comparar el desempeño de 
carga rápida de materiales de electrodos (UMBEM, de sus siglas en inglés, 
\textit{Universal Metric for Benchmarking fast-charging Electrode Materials}). 
Esta se definió como el SOC alcanzado cuando el material se carga 
en condiciones de corriente constante durante 15 minutos. La UMBEM puede ser 
analizada con distintas técnicas experimentales o computacionales, en este caso 
se la analizó utilizando el método empleado en el capítulo anterior. El mismo 
presenta una mejora con respecto a una figura de mérito (FOM, de sus siglas en 
inglés \textit{Figure of Merit}) publicada en un trabajo anterior de literatura,
ya que, además de considerar el tamaño de la partícula y la difusión de los iones, 
considera la transferencia de carga interfacial y la C-rate. Utilizando el mismo 
conjunto de datos de caracterizaciones experimentales que el trabajo de la FOM, se
obtuvo el valor de la UMBEM para cada sistema y se estableció una jerarquía de 
materiales. También se compararon estos valores con los de la FOM. Además,
basándose en un análisis guiado por la superficie del método computacional 
empleado, se predijeron las mejoras necesarias para clasificarlos 
como materiales de carga rápida.

Como primera etapa en el estudio de las aleaciones de LiSi que se forman durante
la litiación de los ánodos de silicio, en el capítulo \ref{ch:caracterizacion} se 
obtuvieron estructuras amorfas de Li$_x$Si para distintos valores de $x$ que 
cubren el intervalo experimental. Se realizaron simulaciones de 
dinámica molecular con un campo de fuerzas reactivo y se encontraron estructuras
cercanas al equilibrio con un método de exploración acelerada de mínimos locales.
Para dichas estructuras se calculó el cambio volumétrico fraccional, que resultó 
en concordancia con experimentos de microscopía de fuerza atómica. También se 
obtuvo una buena representación del comportamiento electroquímico al reproducir 
la curva de potencial en función de la concentración de Li a partir de las 
energías de las estructuras obtenidas. Se analizaron las funciones distribución 
radial (RDF) y los números de coordinación de los primeros y segundos vecinos. 
Mediante análisis de formación e interconexión de clusters se caracterizaron las 
estructuras a los distintos valores de $x$ y se dilucidó la estructura compleja 
observada en el segundo pico de la RDF de Si-Li, respectivamente. Por último, se 
definió un parámetro que permitió determinar el orden de corto alcance de 
estructuras amorfas y los tipos de interacciones.

En el capítulo \ref{ch:modelo} se parametrizó un modelo DFTB (\textit{Density 
Functional Tight-Binding}) para LiSi, que tiene una complejidad intermedia entre 
DFT (\textit{Density Functional Theory}) y los campos de fuerza clásicos. Para 
obtener el conjunto de parámetros se introdujo un algoritmo de ajuste que pondera
las distintas estequiometrías que se consideran en el conjunto de entrenamiento 
para mejorar la predicción de algún observable. En este caso se consideró como 
objetivo las energías de formación relativas de las estructuras cristalinas de 
LiSi, cuyas configuraciones atómicas fueron extraídas de la base de datos del Materials Project.
A este conjunto de estructuras se le agregaron las producidas por compresiones y 
expansiones isotrópicas, a las cuales se les analizaron sus perfiles de energía 
comparándolos con los calculados con DFT. Con el modelo obtenido se realizaron
predicciones óptimas de las energías de formación en el conjunto de entrenamiento 
cristalino y en el conjunto de evaluación amorfo para todo el intervalo de
concentraciones de Li presentes en la litiación de los ánodos de Si. Se compararon 
los residuos de dichas predicciones con las que se obtendrían si se obviara el 
paso del algoritmo de ajuste de pesos para demostrar los beneficios del mismo.
Como las mayores discrepancias con DFT se observaron para Si amorfo puro, se 
amorfizó una estructura de Si mediante un templado simulado y se analizó la RDF 
que resultó en una reproducción excelente del experimento. Como conclusión 
general de este capítulo, se encontró que el modelo DFTB mostró robustez en sus predicciones al
obtenerse una gran concordancia con DFT y superarse el desempeño del potencial 
reactivo del estado-del-arte para este sistema.

Utilizando el modelo DFTB desarrollado en en capítulo \ref{ch:modelo}, se 
obtuvieron configuraciones atómicas de estructuras amorfas Li$_x$Si siguiendo un 
protocolo de litiación de literatura ligeramente modificado. Las estructuras 
obtenidas fueron analizadas en base a modelos que consideran los vecinos más 
cercanos para predecir los resultados de mediciones de rayos x, RMN y Mössbauer.
Para el caso de rayos x se computaron las RDFs parciales, de las cuales se obtuvo
la distribución radial de a pares, $G(r)$, y se la comparó con la PDF 
(\textit{Pair Distribution Function}) de un experimento de Si amorfo y otro de Si
completamente litiado. Para este último caso se hizo un ajuste de coeficientes de 
una combinación lineal de las distintas estructuras que pueden aparecer, como 
sugieren los experimentos. Por otro lado, en los espectros de RMN de corrimiento 
químico de $^7$Li suelen asignarse picos según el criterio de si los vecinos de Si de los átomos de Li están
enlazados a otros átomos de Si o aislados. Siguiendo el argumento experimental, se realizó una 
formulación matemática de esta hipótesis y se propuso un modelo para simular e 
interpretar dichas mediciones. Por último, se propuso una dependencia lineal 
entre la separación de picos en espectroscopia de Mössbauer y el mínimo de 
concentración entre Li y Si. Cuando se usan las concentraciones locales para 
realizar los cálculos se nota una mejora con respecto a los valores globales y
una mejor representación del experimento. Para todos los casos las predicciones
presentaron una buena concordancia con los experimentos.

\section{Perspectivas futuras}

A lo largo de esta tesis se han utilizado distintas técnicas de simulación que
operan en diversas escalas temporales y espaciales. De ambas partes de la misma
surgen distintas alternativas para trabajos futuros.

En el caso del modelo y la métrica propuestos en la parte de la tesis referida a 
la carga rápida, estos podrían ser utilizados en baterías de 
próxima generación además de las de ion-litio, que sigan los mismos fenómenos
físicos, como pueden ser las baterías de sodio, que se presentan entre las 
alternativas más prometedoras \cite{morais2021titanium, leite2020electrochemistry}.
Por otro lado, podrían estudiarse los efectos que tienen las distintas 
suposiciones del modelo para condiciones galvanostáticas. Por ejemplo, se podría utilizar 
la base de datos SQL \path{LiionDB} \cite{wang2022review} o algún sistema de 
extracción automático/semi-automático de datos de publicaciones, como \path{LIBAC}
\cite{el2023libac}, para recopilar una gran cantidad de determinaciones 
experimentales y a partir de ellas estudiar la influencia de la interacción entre los iones 
intercalados a la hora de predecir distintos parámetros, como los coeficientes de 
difusión. Esto sería posible si se modifica ligeramente el modelo para que, además
de recibir los cuatro descriptores del sistema presentados, considere también
la isoterma de inserción.

En lo que refiere a las aleaciones de Li-Si, las estructuras encontradas
que predijeron resultados de mediciones de rayos x, RMN y Mössbauer podrían 
usarse dentro de otros modelos, que podrían elaborarse para predecir otros 
experimentos o utilizarlas en otras técnicas de simulación ya establecidas para 
determinar, por ejemplo, el coeficiente de difusión de litio en silicio amorfo en 
función de la concentración o el potencial aplicado.

Por último, se podría plantear el diseño de un modelo multiescala al estilo del 
de Liu et al \cite{liu2021towards} para electrodos de compositos grafito/silicio,
donde estructuras obtenidas con dinámicas moleculares de grano grueso sean 
utilizadas como parámetros de entrada en un modelo 3D del continuo que acopla 
simulaciones electroquímicas y mecánicas. En el caso del tema abordado en esta tesis podría 
calcularse el coeficiente de difusión de Li en Si en función de la concentración 
de Li, con las estructuras ya obtenidas como configuraciones iniciales, junto con
su isoterma y el cambio volumétrico, e introducir estos parámetros en el modelo 
de una sola partícula, modificando las ecuaciones para que considere 
estos efectos relevantes en este sistema. Por otro lado, también podrían 
aplicarse distintos modelos de aprendizaje automático utilizando los resultados 
de esta tesis como base de datos para obtener un campo de fuerzas. Por ejemplo, se podrían considerar las estructuras amorfas 
de LiSi para predecir resultados experimentales de manera similar al uso que hicimos de los modelos de
vecinos más cercanos propuestos. Se podría también desarrollar un modelo de orden reducido que
permita predecir el estado de carga máxima alcanzado entrenando dicho valor sobre 
los descriptores que permiten obtener los perfiles galvanostáticos simulados.
