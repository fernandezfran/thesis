Algo de la energía global, acuerdos, etc

En Argentina el sector energético depende altamente de la utilización de 
combustibles fósiles, donde la capacidad de generación de energía está 
principalmente atada a las centrales térmicas convencionales y a grandes 
centrales hidroeléctricas, mientras que tan sólo una pequeña cantidad proviene 
de plantas nucleares y de fuentes de energías renovables. En cuanto al potencial
de producción de energía de fuentes renovables, Argentina tiene una gran 
capacidad eólica y solar. 

\subsection{Energías renovables}

Existen muchas formas de generación de energías renovables, entre ellas destacan:
\begin{itemize}
    \item la \textbf{biomasa}, que permite obtener la energía química 
        que se encuentra almacenada en la materia orgánica mediante la quema de 
        la misma,
    \item la \textbf{hidráulica}, que aprovecha la energía cinética y potencial
        de la corriente del agua, la \textbf{marina}, transportada en las olas
        del mar,
    \item la \textbf{eólica}, obtenida a partir de la energía cinética del viento,
    \item la \textbf{solar}, que permite producir energía a partir de la radiación
        electromagnética del sol.
\end{itemize}
La producción de dispositivos eficientes de obtención de energía renovable es un 
requisito esencial para mejorar la eficiencia y, finalmente, reducir el costo de 
las fuentes de energía renovables. Este es uno de los retos a los que se 
enfrenta el establecimiento generalizado de las mismas en comparación con fuentes
de energía tradicionales \cite{olabi2022}. Para dar un ejemplo, la energía solar 
se encuentra disponible en todas partes y ya se aplica comercialmente en varios 
sectores. Uno de los principales retos a los que se enfrenta la misma es a los 
días nublados, que afecta negativamente a la producción de energía. La 
generalización de los sistemas solares fotovoltaicos requiere sistemas eficientes 
de almacenamiento de energía, donde las baterías son las más accesibles. 

\subsection{Sistemas de almacenamiento y transporte de energía}

Como una solución al problema de la alta intermitencia, la baja predictibilidad 
diaria y la variación estacional de energías renovables, se introducen sistemas 
de almacenamiento de energía. Estos permiten almacenar la energía de estas 
fuentes, cuando están produciendo energía por demás, y utilizarla cuando se la 
requiera.

La energía de estas fuentes debe ser almacenada 
cuando están produciendo energía por demás y esta puede ser utilizada cuando se 
la requiera. 

Por último, dentro de los sistemas de almacenamiento de energía electroquímicos 
se encuentran las baterías y los capacitores. En las baterías, tanto a la entrada 
como la salida de energía la misma se encuentra en forma de energía eléctrica 
mientras que la electricidad se almacena en energía química. 
