\section{Mecánica estadística}\label{s:me}

En la mayoría de los experimentos que se realizan en un laboratorio se obtiene 
una serie de mediciones sobre sistemas macroscópicos, usualmente constituidos por 
más de 10$^{20}$ moléculas, durante un período de tiempo, a las cuales luego se 
les realiza un promedio. La mecánica estadística ofrece una interpretación de 
las propiedades del equilibrio de sistemas macroscópicos a partir de una teoría 
molecular aplicada a su configuración microscópica ~\cite{hill1986}.

Esta teoría relaciona el promedio temporal de una variable mecánica con el 
promedio de ensambles, donde un ensamble es una colección de un número muy largo
de sistemas construidos de manera tal que reproducen las propiedades 
termodinámicas del sistema en cuestión a partir de sus configuraciones atómicas
\cite{salinas2001}. Esto es el primer postulado de la Mecánica estadística y se 
lo conoce como la \textit{hipótesis ergódica}: \say{El promedio temporal de una 
variable mecánica $M$ en el sistema termodinámico de interés es igual al promedio 
de ensambles de $M$, en el límite del conjunto de ensambles que tiende a infinito, 
siempre que los sistemas del conjunto de ensambles reproduzcan el estado 
termodinámico y el entorno del sistema real de interés}. Para poder aplicar
este postulado se necesita conocer la probabilidad relativa de cada uno de los 
estados presentes en el ensamble. A esto se refiere el segundo postulado de la 
Mecánica estadística de \textit{igual probabilidad a priori}: \say{En un conjunto 
de ensambles representativo de un sistema termodinámico aislado, los sistemas del 
conjunto de ensambles se distribuyen uniformemente, es decir, con igual 
probabilidad o frecuencia, sobre los posibles estados con los valores
especificados de dicho sistema termodinámico aislado}. En otras palabras, cada 
estado esta representado por la misma cantidad de sistemas en el ensamble.

Cuando las \textit{fluctuaciones} son pequeñas, la función de distribución de 
probabilidad de la variable mecánica $M$ tiene una forma gaussiana en torno a su 
valor medio $\overline{M}$, por lo que su dispersión puede caracterizarse 
completamente por su desviación estándar $\sigma_M$, es decir,
\begin{equation}
    \sigma_M = \sqrt{\overline{(M - \overline{M})^2}}.
\end{equation}
Puede demostrarse que las \textit{fluctuaciones} de la variable mecánica $M$ 
decrecen a medida que aumenta el número de particulas $N$ presentes en el sistema 
de forma proporcional a la inversa de su raíz,
\begin{equation}\label{eq:fluctuaciones}
    \frac{\sigma_M}{\overline{M}} \approx \mathcal{O}(N^{-1/2}).
\end{equation}

\section{Mecánica estadística}

\subsection{Dinámica molecular}

