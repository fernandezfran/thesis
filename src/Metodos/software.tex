\section{Software}

En la actualidad existe una gran cantidad de códigos que permiten realizar 
simulaciones de dinámica molecular de manera eficiente aprovechando la estructura
paralela de los procesadores. En esta tesis se utilizó principalmente \path{LAMMPS} 
(\textit{Large-scale Atomic/Molecular Massively Parallel Simulator}, sus siglas
en inglés) ~\cite{lammps1, lammps2}, un código centrado en el modelado de 
materiales que permite simular con distintos campos de fuerzas, ensambles y 
condiciones de contorno. También se realizaron simulaciones con versiones 
modificadas de \path{GEMS} ~\cite{gems}, código del Dr. Sergio Alexis Paz 
(FCQ-UNC), que permiten utilizar a \path{DFTB+} ~\cite{dftb+} como una librería
y realizar distintos métodos de aceleración de simulaciones que no se encuentran
en los programas usuales como \path{LAMMPS}.

Para la visualización de las trayectorias y la obtención de imágenes de 
estructuras representativas se utilizó \path{VMD} (\textit{Visual Molecular 
Dynamics}) ~\cite{vmd}. Para el post-procesamiento de datos de las simulaciones
se escribieron distintos programas en Python ~\cite{exma, sierras}.
