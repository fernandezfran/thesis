\section{Modelos del continuo}

De los modelos atómisticos, introducidos en las secciones \ref{s:mc} (enfoque 
mecánico cuántico) y \ref{s:me} (enfoque mecánico estadístico), se puede obtener
mucha información sobre como se comportan los materiales que componen las
distintas partes de las baterías de ion-litio en la nanoescala. Esta información 
tiene una importancia crucial, sin embargo suelen estar limitados por el tamaño
(unos cientos o miles de átomos) y el tiempo (del orden de los nanosegundos) que 
pueden simular. Como consecuencia a esto se suelen implementar modelos del 
continuo para estudiar al nivel de la celda. Estos modelos consideran las 
diferentes componentes de las baterías como un medio continuo, en vez de 
considerar partículas discretas o átomos, esto permite manejar escalas espaciales 
y temporales mayores \cite{brosa2022}.

Los modelos de batería de tipo continuo pueden dividirse en dos categoría:
empíricos y basados en física. Los modelos empíricos se basan en ajustar 
incrementalmente ecuaciones y parámetros para encontrar la mejor correspondencia
con datos experimentales, representando así el comportamiento de la batería. 
Como ventaja de estos modelos puede destacarse su velocidad de cómputo y su
conjunto reducido de parámetros, sin embargo no se basan en la física subyacente, 
lo que limita obtener una interpretación de los mecanismos internos de la batería.
En contraste, los modelos basados en física representan los fenómenos físicos 
que gobiernan el comportamiento de la batería y pueden utilizarse para producir 
simulaciones con una precisión mayor, respecto a los empíricos. También existen
modelos híbridos que combinan los mejores aspectos de ambas aproximaciones para 
lograr un compromiso adecuado entre la precisión y la velocidad computacional.

Muchos de los electrodos de intercalación de litio que se investigan en los 
laboratorios o que son incorporados en las baterías comerciales, pueden ser 
simulados con modelos matemáticos complejos que consideran las particularidades 
de cada caso \cite{doyle1995}. Los modelos que poseen un mayor nivel de detalles 
pueden ofrecer una mayor precisión sobre las propiedades 
específicas de un electrodo determinado, pero a su vez pueden dificultar la 
compresión de los aspectos físicos ya que requieren una parametrización detallada, 
lo cual, además de ser una carga a veces innecesaria, dificulta una descripción 
más general del problema. 

% Copyright (c) 2024, Francisco Fernandez
% License: CC BY-SA 4.0
%   https://github.com/fernandezfran/thesis/blob/main/LICENSE
\subsection{Modelo de una sola partícula}\label{s:metodologia}

Los modelos de orden reducido buscan simplificar modelos más complejos pero 
conservar sus capacidades predictivas. Estos pueden mejorar la compresión física 
y proporcionar soluciones precisas a un costo computacional significativamente menor.
Este es el caso de los modelos de una sola partícula (SPM, \textit{single particle 
model}), como el que se utiliza en esta tesis, se supone que todas las partículas
dentro del electrodo se comportan de manera similar, lo que hace que todas puedan ser 
representadas por una sola partícula promedio para reducir la complejidad del 
modelado del continuo. Estos modelos pueden ofrecer predicciones rápidas y precisas 
de materiales de baterías reales, lo que tiene numerosas aplicaciones posibles. Por ejemplo, pueden 
utilizarse como herramientas de diseño para 
%facilitar nuevas arquitecturas de electrodos, celdas y paquetes de baterías, y evaluar su rendimiento, lo cual reduce la necesidad de diseñar prototipos costosos. También pueden ser utilizadas para determinar cuál de los distintos tipos de baterías disponibles en el mercado se adapta mejor a un caso de uso particular.
predecir qué tamaño de partícula debería ser necesario para obtener una dada 
performance del material, como se realiza en la parte \ref{p:fast-charging} de 
esta tesis.

En esta tesis se utiliza un modelo de una sola partícula recientemente propuesto 
\cite{gavilan2023} para construir diagramas galvanostáticos de la capacidad máxima 
alcanzada por una sola partícula en función de dos parámetros adimensionales: uno
cinético,
\begin{equation}\label{eq:xi}
    \Xi = k^0 \sqrt{\frac{t_h}{C_r D}},
\end{equation}
y el otro de difusión finita,
\begin{equation}\label{eq:ele}
    \ell = d \frac{V}{A} \frac{C_r}{D t_h},
\end{equation}
donde $k^0$ es la constante cinética, $D$ el coeficiente de difusión, $V/A$ es la 
proporción volumen/superficie, $d$ es el tamaño característico de la partícula, 
$t_h$ el tiempo de una hora (en las unidades que corresponda) y $C_r$ denota la 
velocidad de carga galvanostática (C-rate), que indica cuántas veces se puede 
cargar o descargar una batería en una hora. Los parámetros de todas las ecuaciones 
que siguen a continuación se presentan en la Tabla \ref{t:params}.
\begin{table}[h!]
    \centering
    \caption{Parámetros involucrados en las ecuaciones del modelo de una sola partícula.}
    \setlength\extrarowheight{2pt}\stackon{%
    \begin{tabular}{l l}
        \toprule
        \textbf{Parámetro} & 
        \textbf{Definición} \\
        \midrule
        $\Xi = k^0 \sqrt{t_h / (C_r D)}$ & Parámetro galvanostático cinético \\
        $\ell = d (V/A) (C_r / (D t_h))$ & Parámetro galvanostático de difusión finita \\
        $D$ & Coeficiente de difusión del Li$^+$ \\
        $t, t_h$ & Tiempo y tiempo para una hora \\
        $C_r$ & velocidad de carga galvanostática (C-rate) \\
        $x, x_s$ & Grado de intercalación en el electrodo y en la interfase \\ 
                 & electrodo/electrolito \\
        $Q, Q_{\max}$ & Capacidad y capacidad máxima \\ 
        SOC & Estado de la Carga, $\text{SOC} = Q / Q_{\max}$ \\ 
        $z$ & Parámetro para establecer la geometría de la partícula: \\
            & plana ($z=1$), cilíndrica ($z=2$) o esférica ($z=3$) \\
        $d$ & Longitud de difusión -- radio de la partícula -- espesor \\ 
            & de la lámina \\
        $r$ & Distancia definida entre 0 y $d$ ($0 \leq r \leq d$) \\
        $V$ & Volumen del material activo \\
        $A$ & Área superficial del electrodo/electrolito \\
        $V / A$ & Proporción volumen/superficie, para la geometría plana toma\\
                & el valor de $d$, cilíndrica ($d/2$) o esférica ($d/3$) \\
        $I_c$, $i_c$ & Corriente y densidad de corriente (constantes) \\
        $k^0$ & Constante cinética \\
        $E$, $E^0$, $E_{\text{off}}$ & Potencial del electrodo de trabajo vs Li/Li$^+$,\\
                                     & potencial de equilibrio y de corte \\
        $\alpha$ & Coeficiente de transferencia de carga \\
        $\rho$ & Densidad del material \\
        $M_r$ & Masa molecular del material \\
        $F$ & Constante de Faraday \\
        $R$ & Constante universal de los gases \\
        $T$ & Temperatura absoluta \\
        \bottomrule
    \end{tabular}
    }{}
    \label{t:params}
\end{table}

En este modelo, el grado de intercalación $x$ en el punto $r$ y a tiempo $t$ se 
obtiene resolviendo numéricamente la ecuación diferencial 1D de Fick \cite{bard-electrochemistry}:
\begin{equation}\label{eq:fick}
    \frac{\partial x}{\partial t} = D \left[ \frac{\partial^2 x}{\partial r^2} + \frac{(z - 1)}{r} \left(\frac{\partial x}{\partial r}\right) \right],
\end{equation}
donde $D$ es el coeficiente de difusión y $z$ depende del tipo de geometría de la 
partícula \cite{vassiliev2016}. La ecuación \ref{eq:fick} puede resolverse 
utilizando el método de Crank-Nicolson mediante diferencias finitas 
\cite{crank-nicolson} con las siguientes condiciones de contorno en la superficie
de la partícula ($r = 0$),
\begin{equation}
    \left(\frac{\partial x}{\partial r}\right)_0 = - \frac{I_c}{F A \frac{\rho}{M_r}D},
\end{equation}
y al centro de la partícula ($r = d$),
\begin{equation}
    \left(\frac{\partial x}{\partial r}\right)_d = 0.
\end{equation}

La condición de corriente constante se fija con la ecuación de Butler-Volmer \cite{bard-electrochemistry}
\begin{equation}\label{eq:bv}
    I_c = F A \frac{\rho}{M_r} k^0 \left\{x_s \exp\left[ \frac{(1-\alpha)F(E-E^0)}{RT} \right] - (1 - x_s) \exp\left[ -\frac{\alpha F (E-E^0)}{RT} \right] \right\}
\end{equation}

El modelo parte de las siguientes suposiciones:
\begin{itemize}
    \item El transporte de carga dentro de la partícula está limitado por el
        movimiento de los iones insertados, es decir que el transporte electrónico
        es rápido.
    \item La difusión de los iones obedece la segunda Ley de Fick 1D para 
        geometría plana, cilíndrica o esférica.
    \item Se ignoran limitaciones en la transferencia de masa en el electrolito, dado que el coeficiente de difusión de Li$^+$ en este medio es del orden de 10$^{-5}$ cm$^2$/s \cite{valoen2005}, es decir, varios órdenes de magnitud mayor que en cualquiera de los materiales analizados.
    \item El coeficiente de difusión, $D$, permanece constante durante todo el 
        proceso de intercalación.
    \item El flujo de iones en el centro de la partícula es cero.
    \item La transferencia de carga en la interfase electrodo/electrolito se 
        describe mediante la ecuación de Butler-Volmer.
    \item La constante cinética, $k^0$, es la misma durante todo el proceso de 
        intercalación.
    \item La resistencia de la celda está dada por un valor constante, 
        $R_{\Omega}$.
    \item Se ignora la interacción entre los iones intercalados.
    \item No se consideran los cambios volumétricos de la partícula durante su 
        carga/descarga.
    \item También se ignoran migraciones de los iones por la presencia de un 
        campo eléctrico y reacciones secundarias.
\end{itemize}
La influencia de la mayoría de estas limitaciones a la hora de obtener la 
capacidad de la partícula se discuten en la referencia \cite{gavilan2023}. 


\subsubsection{Derivación de los parámetros adimensionales}\label{s:derivparam}

En esta sección se derivan los parámetros adimensionales considerando una geometría
plana ($z = 1$ en la ecuación \ref{eq:fick}) y las siguientes condiciones de 
contorno para la ecuación de Fick,
\begin{equation}\label{eq:condcort}
    x(r, 0) = 0 \quad y \quad \left(\frac{\partial x(r, t)}{\partial r}\right)_{r=d} = 0,
\end{equation}
lo que significa que la perturbación a la concentración inicial es nula a tiempo
0 y que no hay flujo de iones en el centro del material, respectivamente. Se 
comienza aplicando una transformada de Laplace,
\begin{equation}\label{eq:laplace}
    \hat{f}(s) = \mathcal{L}[f] = \int_0^{\infty} f(t) e^{-s t} dt,
\end{equation}
al término de la izquierda de la ecuación \ref{eq:fick} de Fick 
\todo{(¿acá faltaría aclarar algo?)}
\begin{equation}
    \mathcal{L}\left[\frac{\partial x(r, t)}{\partial t}\right] = s \hat{x}(r, s)
\end{equation}
y al derecho
\begin{equation}
    \mathcal{L}\left[D \frac{\partial^2 x(r, t)}{\partial r^2}\right] = D \frac{\partial^2 \hat{x}(r, s)}{\partial r^2}.
\end{equation}
Esto lleva a una ecuación diferencial ordinaria (ODE) de segundo orden
\begin{equation}\label{eq:ode}
    \frac{\partial^2 \hat{x}(r, s)}{\partial r^2} - \frac{s}{D} \hat{x}(r, s) = 0,
\end{equation}
donde a las condiciones de contorno de la ecuación \ref{eq:condcort} también es 
necesario aplicarles la transformada de Laplace, pero al ser nula para ambos casos
esto se mantiene. Como la ecuación \ref{eq:ode} es una ODE lineal homogénea con 
coeficientes constantes, su solución general tiene la siguiente forma
\begin{equation}
    \hat{x}(r, s) = a(s) e^{-\sqrt{s/D}r} + b(s) e^{\sqrt{s/D}r}.
\end{equation}
donde $a(s)$ y $b(s)$ son constantes de integración independientes de $r$, para 
la condición de contorno considerada y definiendo la transformada de Laplace del 
flujo ($\hat{J}$) como la derivada de la concentración con respecto a la posición
se puede encontrar que 
\begin{equation}
    b(s) = a(s) e^{-2 \sqrt{s/D} r},
\end{equation}
por lo cual 
\begin{equation}
    \hat{x}(r, s) = \frac{\hat{J}(r, s)}{\sqrt{D s}} \frac{\left\{ e^{-\sqrt{s/D}r} + e^{\sqrt{s/D}(r - 2d)} \right\}}{\left\{ e^{-\sqrt{s/D}r} - e^{\sqrt{s/D}(r - 2d)} \right\}}.
\end{equation}
Si se multiplica y divide por $e^{\sqrt{s/D}d}$ se tiene que
\begin{equation}
    \hat{x}(r, s) = \frac{\hat{J}(r, s)}{\sqrt{Ds}} \coth{[\sqrt{s/D} (d - r)]},
\end{equation}
de donde al aplicar el teorema de convolución 
\footnote{Teorema de convolución:
    \begin{equation}
        \mathcal{L}^{-1} \{\hat{f}(s)\hat{g}(s)\} = \int_0^t f(t - \tau) g(\tau) d\tau
    \end{equation}
    con $f(t) = \mathcal{L}^{-1}\{\hat{f}(s)\}$ y $g(t) = \mathcal{L}^{-1}\{\hat{g(s)}\}$.
    En este caso se consideró $g(\tau) = \mathcal{L}^{-1}\hat{J}(r, s)$
    y $f(\tau) = (1/\sqrt{D}) \mathcal{L}^{-1} \left\{ 1/\sqrt{s} \coth{[\sqrt{s/D}(d-x)]}\right\}$.
} 
se obtiene que
\begin{equation}
    x(r, t) = \int_0^t \frac{1}{d - r} \Theta_3\left(0 \left|\frac{D}{(d-r)^2} (t - \tau)\right.\right) J(r, \tau) d\tau.
\end{equation}
donde $\Theta_3(\nu|x)$ es la función tita \cite{bieniasz2015} dada por la 
siguiente expresión
\begin{equation}
    \Theta_3(\nu|x) = \frac{1}{\sqrt{\pi x}} \sum_{n=-\infty}^{\infty} e^{-\frac{1}{x}(\nu + n)^2}.
\end{equation}
En la frontera $r = 0$, donde la transferencia de carga ocurre se tiene
\begin{equation}\label{eq:rel1}
    x(0, t) = \int_0^t \frac{1}{d} \Theta_3\left(0 \left|\frac{D}{d^2} (t - \tau)\right.\right) J(0, \tau) d\tau,
\end{equation}
por lo que se concluye que la concentración está relacionada con la convolución 
de la corriente en dicha frontera.

Ahora, con el propósito de relacionar la concentración con la corriente ahí, se
utiliza la ecuación de Butler-Volmer (ecuación \ref{eq:bv}) con la definición
$\xi = (nF/RT)(E - E^0)$
\begin{equation}
    \frac{i}{nF} = k^0 \left\{ c_R e^{\alpha_a \xi} + c_O e^{-\alpha_c\xi} \right\},
\end{equation}
donde $c_R$ es la concentración de iones de Li insertados y $c_O$ es la 
concentración de sitios disponibles. Como $c_O = c^0 - c_R$ y suponiendo que 
$\alpha_a + \alpha_c = 1$
\begin{equation}\label{eq:rel2}
    \frac{i}{nF} = k^0 (e^{\alpha_a \xi} + e^{-\alpha_c \xi}) \left\{ c^0 \frac{1}{(1 + e^{-\xi})} - c_O \right\},
\end{equation}
Considerando la desviación de la concentración de la especie oxidada con respecto 
a su valor inicial,
\begin{equation}
    x(r, t) = c(r, t) - c^0,
\end{equation}
se puede reescribir la ecuación \ref{eq:rel1} como
\begin{equation}
    c(0, t) = c^0 + \int_0^t \frac{1}{d} \Theta_3\left(0 \left|\frac{D}{d^2} (t - \tau)\right.\right) J(0, \tau) d\tau,
\end{equation}
que junto a la ecuación \ref{eq:rel2} se la puede utilizar para obtener que 
\cite{aoki1984}
\begin{equation}\label{eq:aoki}
    \frac{i}{nFc^0} = k^0 (e^{\alpha_a \xi} + e^{-\alpha_c \xi}) \left\{ c^0 \frac{1}{(1 + e^{-\xi})} - \frac{d}{D} \int_0^{u=\frac{Dt}{d^2}} \Theta_3\left(0 \left| (u - z)\right.\right) \frac{i}{nFc^0} dz \right\},
\end{equation}
donde se realizó el cambio de variables $z = \frac{D}{d^2}\tau$ y se utilizó la 
relación entre el flujo y la densidad de corriente $J(0, \tau) = \frac{i}{nF}$.

Para demostrar la invariancia de la capacidad con los parámetros 
$\Xi = k^0 \sqrt{t_h / (C_r D)}$ (ecuación \ref{eq:xi}) y 
$\ell = d (V/A) (C_r / (D t_h))$ (ecuación \ref{eq:ele}) se necesita relacionar 
la ocupación con el tiempo y la velocidad de carga galvanostática ($C_r$) con
la corriente. Para esto último se tiene que
\begin{equation}
    I_c = \frac{Q}{t_h} C_r
\end{equation}
y a su vez
\begin{equation}
    x = \frac{t}{t_h} C_r,
\end{equation}
por lo que en condiciones galvanostáticas el tiempo de carga y la ocupación 
están linealmente relacionadas. A su vez, puede obtenerse que
\begin{equation}
    \frac{i_C}{nFc^0} = \frac{V}{A} \frac{C_r}{t_h}.
\end{equation}
Si consideramos esta última ecuación y la corriente constante en la ecuación 
\ref{eq:aoki}
\begin{equation}
    \frac{V}{A} \frac{C_r}{t_h} = k^0 (e^{\alpha_a \xi} + e^{-\alpha_c \xi}) \left\{ \frac{1}{(1 + e^{-\xi})} - \frac{V}{A}\frac{C_r}{t_h}\frac{d}{D} \int_0^{u=\frac{Dt}{d^2}} \Theta_3\left(0 \left| (u - z)\right.\right) dz \right\}.
\end{equation}
Si se define ahora la ecuación \ref{eq:ele}, que para el caso de un plano se tiene
$\ell = \frac{d^2}{D} \left(\frac{C_r}{t_h}\right)$, y se usa que
\begin{equation}
    \int_0^{z=u=\frac{Dt}{d^3}} \Theta_3(0|u-z)dz = \frac{D}{d^2} \frac{t_h}{C_r} \int_0^{x'=x} \Theta_3\left(0\left|\frac{D t_h x}{C_r d^2} - \frac{D t_h x'}{C_r d^2}\right.\right)
\end{equation}
se tiene
\begin{equation}
    d \frac{C_r}{t_h} = k^0 (e^{\alpha_a \xi} + e^{-\alpha_c \xi}) \left\{ \frac{1}{(1 + e^{-\xi})} - \int_0^{x'=x} \Theta_3\left(0 \left| \left(\frac{x}{\ell}- \frac{x}{\ell}\right)\right.\right) dx' \right\}.
\end{equation}
Multiplicando ambos lados de la ecuación por el factor $\sqrt{\frac{t_h}{C_r D}}$
\begin{equation}
    d \frac{C_r}{t_h} \sqrt{\frac{t_h}{C_r D}} = k^0 \sqrt{\frac{t_h}{C_r D}} (e^{\alpha_a \xi} + e^{-\alpha_c \xi}) \left\{ \frac{1}{(1 + e^{-\xi})} - \int_0^{x'=x} \Theta_3\left(0 \left| \left(\frac{x}{\ell}- \frac{x}{\ell}\right)\right.\right) dx' \right\},
\end{equation}
de donde emerge la relación para $\Xi$ (ecuación \ref{eq:xi}) y resulta que las 
soluciones de $x$ en función de $\xi$ sólo dependen de los parámetros $\ell$ y
$\Xi$:
\begin{equation}
    \sqrt{\ell} = \Xi (e^{\alpha_a \xi} + e^{-\alpha_c \xi}) \left\{ \frac{1}{(1 + e^{-\xi})} - \int_0^{x'=x} \Theta_3\left(0 \left| \left(\frac{x}{\ell}- \frac{x}{\ell}\right)\right.\right) dx' \right\}.
\end{equation}

