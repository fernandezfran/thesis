\section{Transición energética}

La demanda internacional de energía sigue en aumento debido al crecimiento 
poblacional rápido y a los avances en la civilización, más del 80\% de la misma
sigue siendo producida por combustibles fósiles, que son limitados en recursos 
y tienen un impacto grave en el medio ambiente. Sin políticas comprometidas con 
la transición energética no se espera que esta proporción disminuya, dejando 
lugar a la producción de energía mediante fuentes renovables, en los próximos 20
años. Si el cambio en la matriz energética se deja en manos del mercado, las 
fuentes de energía limpias no sustituirán por sí solas a los métodos tradicionales
hasta que no sólo alcancen una paridad de precios, si no que se vuelvan 
considerablemente más baratas de manera que justifiquen dicho cambio
~\cite{davidson2019}. Dicho esto, cumplir los objetivos climáticos y realizar la 
transición hacia un futuro con menos carbono va a requerir inversiones 
sustanciales por parte de los gobiernos ~\cite{leonhardt2022}.

La gran mayoría de las naciones desarrolladas han implementado un marco legal de 
apoyo y habilitación para ayudar a promover la integración de las energías 
renovables modernas en sus sistemas energéticos. Estas políticas se han 
desarrollado como resultado, o en apoyo, de acuerdos internacionales como el 
Acuerdo de París, el Protocolo de Kioto o el Green Deal europeo. Los objetivos 
de las energías renovables y los incentivos fiscales dirigidos al sector 
energético son las dos políticas más comunes en países en desarrollo para apoyar 
la transición energética ~\cite{cantarero2020}.

En Argentina el sector energético depende altamente de la utilización de 
combustibles fósiles, donde la capacidad de generación de energía está 
principalmente atada a las centrales térmicas convencionales y a grandes 
centrales hidroeléctricas, mientras que tan sólo una pequeña cantidad proviene 
de plantas nucleares y de fuentes de energías renovables. En cuanto al potencial
de producción de energía de fuentes renovables, Argentina tiene una gran 
capacidad eólica y solar. El gobierno nacional viene incentivando la instalación 
de dichas fuentes de energía desde 2009, mediante el programa GENREN. A fines 
del 2015 se estableció un objetivo de que el 20\% de la energía fuera generada
mediante estas fuentes para 2025 y en 2016 se introdujo un nuevo esquema de 
compra con el programa RenovAr ~\cite{schaube2018}. Este desarrollo viene 
acompañado de un amplio espectro de investigaciones académicas 
interdisciplinarias.

\subsection{Energías renovables}

Existen muchas formas de generación de energías renovables, entre ellas destacan:
\begin{itemize}
    \item la \textbf{biomasa}, que permite obtener la energía química 
        que se encuentra almacenada en la materia orgánica mediante la quema de 
        la misma,
    \item la \textbf{hidráulica}, que aprovecha la energía cinética y potencial
        de la corriente del agua, la \textbf{marina}, transportada en las olas
        del mar,
    \item la \textbf{eólica}, obtenida a partir de la energía cinética del viento,
    \item la \textbf{solar}, que permite producir energía a partir de la radiación
        electromagnética del sol.
\end{itemize}
La producción de dispositivos eficientes de obtención de energía renovable es un 
requisito esencial para mejorar la eficiencia y, finalmente, reducir el costo de 
las fuentes de energía renovables. Este es uno de los retos a los que se 
enfrenta el establecimiento generalizado de las mismas en comparación con fuentes
de energía tradicionales ~\cite{olabi2022}. Para dar un ejemplo, la energía solar 
se encuentra disponible en todas partes y ya se aplica comercialmente en varios 
sectores. Uno de los principales retos a los que se enfrenta la misma es a los 
días nublados, que afecta negativamente a la producción de energía. La 
generalización de los sistemas solares fotovoltaicos requiere sistemas eficientes 
de almacenamiento de energía, donde las baterías son las más accesibles. 

\subsection{Sistemas de almacenamiento y transporte de energía}

Como una solución al problema de la alta intermitencia, la baja predictibilidad 
diaria y la variación estacional de energías renovables, se introducen sistemas 
de almacenamiento de energía. La energía de estas fuentes debe ser almacenada 
cuando están produciendo energía por demás y esta puede ser liberada cuando se 
requiera. Dichos sistemas pueden ser clasificados a grandes rasgos en mecánicos, 
electroquímicos, químicos o térmicos ~\cite{khan2019}.

En el caso de los sistemas de almacenamiento de energía mecánicos, la energía se
almacena realizando algún trabajo mecánico, entre ellos se encuentra, por ejemplo,
el aire comprimido. En el almacenamiento de energía térmica se utiliza la energía 
térmica que se produce al calentar o enfriar un medio.

En el sistema químico, la energía se almacena en forma de energía química 
almacenada en distintos materiales. Se tienen principalmente dos tipos, los
biocombustibles o el hidrógeno. En este último caso, la energía eléctrica se 
utiliza para descomponer el agua en oxígeno e hidrógeno, estos gases se almacenan 
y se transportan para luego volver a combinarse y liberar la energía almacenada.

Por último, dentro de los sistemas de almacenamiento de energía electroquímicos 
se encuentran las baterías y los capacitores. En las baterías, tanto a la entrada 
como la salida de energía la misma se encuentra en forma de energía eléctrica 
mientras que la electricidad se almacena en energía química. 
