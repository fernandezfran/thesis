El calentamiento global es el mayor problema ambiental que enfrenta el mundo, el 
mismo se refiere al efecto que producen las actividades humanas en el clima, como 
la quema de combustibles fósiles o la deforestación, que emiten a la atmósfera
grandes cantidades de dióxido de carbono, CO$_2$, entre otros gases de efecto 
invernadero. Estos gases absorben la radiación infrarroja emitida por la tierra 
provocando un incremento de la temperatura de la misma que lleva asociado un 
aumento en la frecuencia y la intensidad de eventos climáticos extremos 
~\cite{houghton2005}. Según el Panel Intergubernamental del Cambio Climático 
(IPCC), desde la época preindustrial, las actividades humanas han provocado 
aproximadamente 1.0$^{\circ}$C de calentamiento global y al ritmo actual se van 
a sobrepasar los 1.5$^{\circ}$C antes del 2050, un cambio en la temperatura
media que las emisiones previas por sí solas no habían alcanzado
~\cite{harvey2018}. Limitar el calentamiento a esta temperatura requiere que se 
realicen rápidamente cambios sin precedentes en la tecnología y en el 
comportamiento humano. Uno de los cambios más importante es el de la matriz 
energética, en la cual las energías renovables deberán suministrar alrededor del 
80\% de la energía para 2050, donde los vectores energéticos, como las baterías 
de litio, juegan un rol fundamental debido a la intermitencia de estas formas de 
generación de energía.

El litio es el metal más liviano de la tabla periódica y uno de los elementos más
importantes dentro de los minerales necesarios en la producción de baterías de
litio. En particular, para la Argentina tiene un interés económico, social, 
industrial y tecnológico ya que es uno de los países que integran, junto a 
Bolivia y Chile, el Triangulo de Litio, el cual acumula el 70\% de las reservas 
mundiales de este mineral. Aún más importante que esta cantidad de reservas es 
que las mismas se encuentran en salares que, a grandes rasgos, es más barato
extraer litio de ellos en comparación a las pegmatitas, heroctitas o jadaritas, 
que son las rocas de las cuales se puede extraer litio en una minería usual.
A pesar de esto se tienen que llevar a cabo distintas consideraciones ambientales,
sociales y legales del proceso de extracción e incentivar el desarrollo de valor
agregado a dicha extracción ~\cite{heredia2020}.

En esta tesis se presentan estudios realizados mediante el uso de simulaciones 
computacionales en... TODO

\section{Transición energética}

La demanda internacional de energía sigue en aumento debido al crecimiento 
poblacional rápido y a los avances en la civilización, más del 80\% de la misma
sigue siendo producida por combustibles fósiles, que son limitados en recursos 
y tienen un impacto grave en el medio ambiente. Sin políticas comprometidas con 
la transición energética no se espera que esta proporción disminuya, dejando 
lugar a la producción de energía mediante fuentes renovables, en los próximos 20
años. Si el cambio en la matriz energética se deja en manos del mercado, las 
fuentes de energía limpias no sustituirán por sí solas a los métodos tradicionales
hasta que no sólo alcancen una paridad de precios, si no que se vuelvan 
considerablemente más baratas de manera que justifiquen dicho cambio
~\cite{davidson2019}. Dicho esto, cumplir los objetivos climáticos y realizar la 
transición hacia un futuro con menos carbono va a requerir inversiones 
sustanciales por parte de los gobiernos ~\cite{leonhardt2022}.

La gran mayoría de las naciones desarrolladas han implementado un marco legal de 
apoyo y habilitación para ayudar a promover la integración de las energías 
renovables modernas en sus sistemas energéticos. Estas políticas se han 
desarrollado como resultado, o en apoyo, de acuerdos internacionales como el 
Acuerdo de París, el Protocolo de Kioto o el Green Deal europeo. Los objetivos 
de las energías renovables y los incentivos fiscales dirigidos al sector 
energético son las dos políticas más comunes en países en desarrollo para apoyar 
la transición energética ~\cite{cantarero2020}.

En Argentina el sector energético depende altamente de la utilización de 
combustibles fósiles, donde la capacidad de generación de energía está 
principalmente atada a las centrales térmicas convencionales y a grandes 
centrales hidroeléctricas, mientras que tan sólo una pequeña cantidad proviene 
de plantas nucleares y de fuentes de energías renovables. En cuanto al potencial
de producción de energía de fuentes renovables, Argentina tiene una gran 
capacidad eólica y solar. El gobierno nacional viene incentivando la instalación 
de dichas fuentes de energía desde 2009, mediante el programa GENREN. A fines 
del 2015 se estableció un objetivo de que el 20\% de la energía fuera generada
mediante estas fuentes para 2025 y en 2016 se introdujo un nuevo esquema de 
compra con el programa RenovAr ~\cite{schaube2018}. Este desarrollo viene 
acompañado de un amplio espectro de investigaciones académicas 
interdisciplinarias.

\subsection{Energías renovables}

Existen muchas formas de generación de energías renovables, entre ellas destacan:
\begin{itemize}
    \item la \textbf{biomasa}, que permite obtener la energía química 
        que se encuentra almacenada en la materia orgánica mediante la quema de 
        la misma,
    \item la \textbf{hidráulica}, que aprovecha la energía cinética y potencial
        de la corriente del agua, la \textbf{marina}, transportada en las olas
        del mar,
    \item la \textbf{eólica}, obtenida a partir de la energía cinética del viento,
    \item la \textbf{solar}, que permite producir energía a partir de la radiación
        electromagnética del sol.
\end{itemize}
La producción de dispositivos eficientes de obtención de energía renovable es un 
requisito esencial para mejorar la eficiencia y, finalmente, reducir el costo de 
las fuentes de energía renovables. Este es uno de los retos a los que se 
enfrenta el establecimiento generalizado de las mismas en comparación con fuentes
de energía tradicionales ~\cite{olabi2022}. Para dar un ejemplo, la energía solar 
se encuentra disponible en todas partes y ya se aplica comercialmente en varios 
sectores. Uno de los principales retos a los que se enfrenta la misma es a los 
días nublados, que afecta negativamente a la producción de energía. La 
generalización de los sistemas solares fotovoltaicos requiere sistemas eficientes 
de almacenamiento de energía, donde las baterías son las más accesibles. 

\subsection{Sistemas de almacenamiento de energía}

Como una solución al problema de la alta intermitencia, la baja predictibilidad 
diaria y la variación estacional de energías renovables, se introducen sistemas 
de almacenamiento de energía. La energía de estas fuentes debe ser almacenada 
cuando están produciendo energía por demás y esta puede ser liberada cuando se 
requiera. Dichos sistemas pueden ser clasificados a grandes rasgos en mecánicos, 
electroquímicos, químicos o térmicos ~\cite{khan2019}.

En el caso de los sistemas de almacenamiento de energía mecánicos, la energía se
almacena realizando algún trabajo mecánico, entre ellos se encuentra, por ejemplo,
el aire comprimido. En el almacenamiento de energía térmica se utiliza la energía 
térmica que se produce al calentar o enfriar un medio.

En el sistema químico, la energía se almacena en forma de energía química 
almacenada en distintos materiales. Se tienen principalmente dos tipos, los
biocombustibles o el hidrógeno. En este último caso, la energía eléctrica se 
utiliza para descomponer el agua en oxígeno e hidrógeno, estos gases se almacenan 
y se transportan para luego volver a combinarse y liberar la energía almacenada.

Por último, dentro de los sistemas de almacenamiento de energía electroquímicos 
se encuentran las baterías y los capacitores. En las baterías, tanto a la entrada 
como la salida de energía la misma se encuentra en forma de energía eléctrica 
mientras que la electricidad se almacena en energía química. 

\section{Baterías}

TODO: Clasificación de baterías en primarias (no recargables) o secundarias 
(recargables). Funcionamiento de una batería, sus partes. Gráfico de densidad
de energía volumétrica vs densidad de energía gravimétrica.

La primera batería concebida fue inventada por Alessandro Volta en 1800 para 
estudiar los descubrimientos de Luigi Galvani, la misma basa su funcionamiento 
en la combinación de zinc con cobre y se la conoce como pila voltaica, fue 
crucial para los primeros experimentos electroquímicos. En 1866 la celda primaria 
de Leclanche, predecesora de la pila de zinc-carbono actual, se convirtió en una
de las primeras baterías comerciales al ser utilizada en estaciones de telegrafía
y en los primeros teléfonos. A mediados del siglo XIX surgió la necesidad de un 
sistema de almacenamiento recargable debido a la invención del generador 
eléctrico, la misma fue satisfecha por Gaston Panté al presentar la batería de 
plomo-ácido. Durante este último periodo de tiempo y principios del siglo XX, 
los motores de combustión interna y los motores eléctricos compitieron como medios
de propulsión en los automóviles. Con el descubrimiento de reservas de petróleo 
mayores a principios del siglo XX y el desarrollo de automóviles de gasolina más
cómodos, los vehículos eléctricos se vieron rápidamente superados por los de 
motores de combustión interna y se dejaron de fabricar. Por desgracia, el sector 
del transporte se convirtió en uno de los responsables de gran parte de la 
contaminación.

Un siglo después aparecieron nuevas iniciativas en el campo de las baterías con
el desarrollo del consumo de la electrónica portátil, que requiere una batería 
recargable de alta densidad energética que no es una condición que cumpla la 
batería de plomo-ácido. Inicialmente se prestó mucha atención al sistema 
níquel-cadmio hasta ser superado por las baterías de iones de litio (LIB) a 
principios de la década del 1990. Desde esa fecha, las LIB continúan mejorando
constantemente y prometen desempeñar un papel vital en el desarrollo de vehículos
eléctricos y el almacenamiento de energía de fuentes alternativas. % schipper2016

\subsection{Baterías de ion litio}

Las baterías de litio son dispositivos electroquímicos ampliamente utilizados 
como fuentes de energía. Entre las propiedades físicas del litio destacan su 
peso molecular bajo (7 g mol$^{-1}$), su densidad baja (0.534 cm$^3$), su 
capacidad específica alta (3860 mAh g$^{-1}$) y su potencial de reducción bajo 
(-3.04 V vs. SHE), estas propiedades lo convirtieron en un material que puede 
ser utilizado como ánodo para baterías. A finales de la década del 1950 se 
observó que el litio metálico formaba una capa de pasivación, llamada interfaz de
electrolito sólido (SEI, de sus siglás en inglés, \textit{solid electrolyte 
interface}), con distintos electrolitos no-acuosos permitiendo prevenir una 
reacción química directa entre estas dos componentes, pero dejando que los iones 
de litio la atraviesen. Esto generó un interés que llevó a la fabricación de 
baterías primarias de litio en la década del 1960 utilizando distintos tipos de 
cátodos incluyendo dióxido de sulfuro de litio (LiSO$_2$), óxido de manganeso de 
litio (LiMnO$_2$) y óxido de cobre de litio (LiCuO), entre otros. 
Desafortunadamente, la formación de la SEI no es estable durante ciclos 
prolongados y se agrieta, lo que lleva a un consumo continuo de electrolito y 
litio para la reformación continua de la SEI. Y lo que es aún peor, la deposición 
desigual de litio en la SEI agrietada conduce al crecimiento de dendritas de 
litio, que terminan rompiéndose y formando islas de tamaño nanométrico de litio 
altamente reactivo, lo que reduce la estabilidad térmica de la célula. Las 
dendritas también pueden crecer lo suficiente como para generar un contacto entre
el ánodo y el cátodo, provocando un cortocircuito en la celda. Avances 
posteriores en la comprensión de la intercalación de litio en diferentes 
materiales dio origen a las baterías recargables de ion-litio. En la tabla 
\ref{t:historia} se presentan las primeras baterías secundarias de litio 
comercializadas hasta 1991, año en el que se dejó de utilizar litio metálico como
ánodo y se comenzó a utilizar uno de carbono. En la misma puede notarse el avance 
con los años de la densidad de energía, sin embargo este no es el único parámetro 
a tener en cuenta en el progreso ya que también tienen que considerarse la 
cantidad de ciclos que se pueden realizar, el tiempo de carga, la razón de 
descarga y el precio de producción ~\cite{reddy2020}. 
\begin{table}[h]
    \centering
    \caption{Primeras baterías de litio recargables comercializadas.}
    \setlength\extrarowheight{2pt}\stackon{%
    \begin{tabular}{c c c c}
        \toprule
        \thead{\normalsize\bfseries Sistema\\\normalsize\bfseries electroquímico} & 
        \thead{\normalsize\bfseries Voltaje (V)} & 
        \thead{\normalsize\bfseries Densidad de energía\\\normalsize\bfseries específica (Wh/kg)} & 
        \thead{\normalsize\bfseries Compañía comercial} \\
        \midrule
        Li||TiS$_2$ & 2.1 & 130 & Exxon (1978) \\
        Li||LiAlCl$_4$-SO$_2$ & 3.2 & 63 & Duracell (1981) \\
        Li||NbSe$_3$ & 2.0 & 95 & Bell Telephone Lab. Inc. (1983) \\
        LiAl||polianilina & 3.0 & -- & Bridgestone (1987) \\
        Li||MoS$_2$ & 1.8 & 52 & MoLi Energy (1987) \\
        Li||V$_2$O$_5$ & 1.5 & 10 & Toshiba (1989) \\
        LiAl||polipirrol & 3.0 & -- & Kanebo (1989) \\
        Li||Li$_{0.3}$MnO$_2$ & 3.0 & 50 & Tadiran (1989) \\
        LiVO$_x$ & 3.2 & 200 & Hydro-Québec (1990) \\
        C||LiCoO$_2$ & 3.6 & 150--190 & Sony (1991) \\
        \bottomrule
    \end{tabular}
    }{}
    \label{t:historia}
\end{table}

La necesidad de un material catódico de intercalación de litio de alto voltaje 
que se presentaba en la época fue resuelta por el compuesto laminar LiCoO$_2$ 
(LCO) que fue desarrollado por el grupo de investigación de Goodenough en 1980.
Sin embargo, se necesitaba un material anódico estable que permitiera intercalar
los iones de litio. No fue hasta el 1990 que esto se cumplió, año en el que Dahn 
descubrió dicha estabilidad en los ánodos de grafito en soluciones electrolíticas
no acuosas al estudiar la intercalación de forma reversible en los primeros 19 
ciclos utilizando como electrolito una mezcla de disolvente de carbonato de 
propileno y carbonato de etileno. Esto se da gracias a la formación de una 
SEI estable durante  el primer ciclo de intercalación. La batería basada en este 
funcionamiento fue comercializada por Sony en 1991 y ha tenido un gran éxito. 

\section{Objetivos y estructura de tesis}

TODO
