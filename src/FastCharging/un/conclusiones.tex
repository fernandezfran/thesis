\section{Conclusiones del capítulo}

En este capítulo se presentó una guía simple para optimizar el tamaño de las 
partículas de los materiales utilizados en las baterías de ion-litio que permita
el cumplimiento de un dado criterio de carga rápida. El enfoque que se siguió
fue uno heurístico, basado en un método universal para una sola partícula en
condiciones de carga a corriente constante. A pesar de la simplicidad del método, 
este muestra un gran potencial al reproducir la física básica detrás del 
fenómeno de intercalación, regulado por la difusión de iones dentro de los 
electrodos y la transferencia de carga en la interfase. Utilizando datos 
experimentales de literatura, se realizaron predicciones sobre los tamaños de
partícula necesarios para cargar el 80\% de su capacidad en 15 y 5 minutos. 
Por otro lado, la herramienta presentada resulta ser atractiva, simple y rápida 
para estimar valores del coeficiente de difusión y la constante cinética mediante
un ajuste simple del Estado de la Carga en función de la C-rate. Para los 
sistemas estudiados se obtuvieron valores sólidos y físicamente correctos de
dichos parámetros. Además, se desarrolló un software en Python que es fácil 
de usar, cumple con los requisitos establecidos por la comunidad y es de libre
acceso.
