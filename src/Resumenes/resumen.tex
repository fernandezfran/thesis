\chapter{Resumen}

La industria de los vehículos eléctricos está en crecimiento, debido a la 
necesidad de utilizar energías renovables para reducir las emisiones de gases de 
efecto invernadero. En este contexto, los sistemas de almacenamiento y transporte 
de energía se vuelven cruciales. Uno de los mayores desafíos que éstos enfrentan es lograr que 
los automóviles eléctricos alcancen la autonomía y el tiempo de recarga de los 
vehículos de combustión interna. Para ello, se requieren electrodos con una 
gran capacidad y que logren una carga rápida. Esta tesis contempla ambas 
cuestiones.

En términos generales, se puede afirmar que los dos grandes avances logrados en 
esta Tesis Doctoral se refieren al establecimiento de una métrica universal sin 
precedentes para predecir y evaluar el cargado rápido de materiales, y a la 
predicción de las propiedades de aleaciones amorfas de Li-Si, uno los materiales
más promisorios para ser empleado como ánodos de baterías de ion litio de próxima 
generación.

En la primera parte de la tesis se desarrolla un modelo para predecir 
el tamaño óptimo de partículas de material activo en los electrodos para que 
logren una carga rápida del 80\% del Estado de la Carga (SOC) en 15 minutos. El 
mismo se basa en simulaciones de técnicas galvanostáticas teniendo en cuenta la 
difusión de los iones dentro del material y la transferencia de carga interfacial. 
Con este modelo se pueden ajustar datos experimentales del SOC máximo en función 
de la velocidad de carga galvanostática (C-rate) para distintos materiales y obtener coeficientes de difusión y 
constantes cinéticas de una forma rápida y simple. Las estimaciones realizadas 
para los sistemas analizados resultaron estar dentro del intervalo de valores 
experimentales esperado. Luego, se propone una métrica universal para 
estandarizar las comparaciones del desempeño de distintos materiales considerados 
para aplicaciones de carga rápida. Esta métrica presenta una mejora con respecto 
a una propuesta de literatura previa que supone una transferencia de carga 
interfacial ultra-rápida.

En la segunda parte de la tesis se estudian las aleaciones amorfas de Li-Si que 
se forman en la litiación de los ánodos de silicio. En el primer capítulo se 
realizan simulaciones de dinámica molecular utilizando un potencial reactivo y 
proponiendo un método acelerado de exploración de mínimos locales. En el segundo 
capítulo se parametriza un modelo DFTB (\textit{density functional tight-binding})
con un algoritmo de ajuste que pondera las distintas estructuras en el conjunto 
de entrenamiento. El mismo supera en su exactitud al potencial reactivo del 
estado del arte para este sistema a la hora de predecir energías de formación 
tanto en las estructuras cristalinas de entrenamiento como en las estructuras 
amorfas de evaluación. La función distribución radial de una estructura de 
silicio amorfa obtenida mediante un templado simulado con este modelo resulta en 
una concordancia excelente con datos experimentales. En el último capítulo se 
describe un protocolo de litiación para obtener estructuras amorfas a distintas 
concentraciones de litio. Las configuraciones atómicas obtenidas se analizan en base a 
modelos considerando los vecinos más cercanos para predecir los resultados de mediciones de resonancia magnética nuclear (RMN), 
rayos x y Mössbauer. Estas predicciones también presentan una buena concordancia 
con los experimentos.
