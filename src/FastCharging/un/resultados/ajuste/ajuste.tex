% Copyright (c) 2024, Francisco Fernandez
% License: CC BY-SA 4.0
%   https://github.com/fernandezfran/thesis/blob/main/LICENSE
\subsection{Ajuste del modelo}\label{s:ajustes}

En la Figura \ref{fig:ajustes} se muestran los datos experimentales y los 
ajustes del modelo para el SOC$_{\max}$ alcanzado al potencial de corte 
\textit{versus} la C-rate para los resultados de la Figura \ref{fig:preproc}
y otros materiales de uso común en los electrodos de las baterías de ion-litio.
Puede observarse una buena concordancia en general entre el modelo y los 
experimentos. Esto también se observa en la Figura \ref{fig:pred_vs_exp},
donde se muestran los valores predichos para SOC$_{\max}$ en función de los
experimentales, junto con el coeficiente de determinación de cada ajuste.
\begin{figure}[h!]
    \centering
    \includegraphics[width=0.7\textwidth]{FastCharging/un/resultados/ajuste/ajustes.png}
    \caption{Ajuste del modelo a los datos SOC$_{\max}$ \textit{versus} C-rate
    para los distintos materiales de electrodos considerados: (a) Grafito natural (NG)
    \cite{mancini2022}, (b) Li$_4$Ti$_5$O$_{12}$ (LTO) \cite{he2012}, (c) LiFePO$_4$ (LFP) \cite{lei2015}, 
    (d) LiCoO$_2$ (LCO) \cite{wang2019high}, (e) LiMn$_2$O$_4$ (LMO) \cite{bak2011}, (f) LiNi$_{0.5}$Mn$_{1.5}$O$_4$ (LNMO)
    \cite{nishikawa2017}.}
    \label{fig:ajustes}
\end{figure}
\begin{figure}[h!]
    \centering
    \includegraphics[width=0.7\textwidth]{FastCharging/un/resultados/ajuste/pred_vs_exp.png}
    \caption{Predicciones del SOC$_{\max}$ en función de los valores 
    experimentales, junto al coeficiente de determinación de cada sistema $R^2$.}
    \label{fig:pred_vs_exp}
\end{figure}

El trabajo de Mancini \textit{et al.} \cite{mancini2022} aportó nuevos 
conocimientos sobre el efecto de esferoidización en las características de las
partículas de grafito natural (NG) y su impacto en el comportamiento 
electroquímico, como se muestra en la Figura \ref{fig:ajustes}a. En este caso, 
el intervalo de C-rates reportadas cubre una amplia región del SOC$_{\max}$, desde un 
material totalmente cargado hasta uno casi totalmente descargado. Esto no es lo 
habitual, ya que en la mayoría de los experimentos sólo se reportan curvas con un 
buen rendimiento (capacidad alta), lo que limita los ajustes realizados a una 
región concreta del diagrama. El coeficiente de difusión y la constante cinética 
obtenidos en el ajuste para este sistema son $1.23\times10^{-10}$ cm$^2$/s y 
$2.31\times10^{-7}$ cm/s, respectivamente. Para esto se consideró un tamaño
de partícula de $7.5$ $\mu$m y una geometría esférica. Este valor se corresponde
con la media de la distribución de tamaños, que es reportada junto a los 
cuartiles en el trabajo citado. Para los casos que siguen, en los que no se
tiene información precisa de la distribución de tamaños, se considera el punto
medio del rango reportado para el tamaño de las partículas y se utiliza para 
definir el parámetro $d$ en el modelo.

Los ánodos de Li$_4$Ti$_5$O$_{12}$ (LTO) presentan características excelentes
de seguridad y una capacidad teórica de 175 mAhg$^{-1}$. He \textit{et al.} 
\cite{he2012} sintetizaron nanopartículas cristalinas y esféricas de LTO a 
diferentes proporciones atómicas de Li/Ti, con un tamaño entre los 0.5 $\mu$m 
y los 3 $\mu$m. Se supone entonces un valor de $d=1.75$ $\mu$m y se utilizan los
datos de la proporción usual del LTO para ajustar el modelo, teniéndose como 
resultado un valor de $D$ de $6.58\times10^{-12}$ cm$^2$/s. El valor experimental 
de $D$ reportado por He \textit{et al.} para esta proporción atómica fue de
$5.12\times10^{-12}$ cm$^2$/s. Con respecto al valor de $k^0$, se obtuvo 
$8.11\times10^{-8}$ cm/s. Comparar este valor con el experimental no es tan 
directo, ya que lo que reportan es la densidad de corriente de intercambio, 
$i^0 = 2.7\times10^{-4}$ mA/cm$^2$. Utilizando la ecuación \ref{eq:bv} de 
Butler-Volmer se tiene una relación entre $k^0$ e $i^0$ dada por
\begin{equation}\label{eq:i0k0}
    i^0 = F \frac{\rho}{M_r} k^0 \left(x_s\right)^{\alpha} \left(1 - x_s\right)^{1-\alpha},
\end{equation}
donde las definiciones de los parámetros están dadas en la Tabla \ref{t:params}.
Suponiendo un valor de 0.5 para el coeficiente de transferencia $\alpha$, un 
SOC de 0.5 y utilizando los valores del precursor LTO, $M_r = 459.1$ g/mol y
$\rho = 3.48$ g/cm$^3$ \cite{osti_1284125} se obtiene un valor para $k^0$ de
$7.38\times10^{-10}$ cm/s, que presenta una discrepancia de dos ordenes de 
magnitud con respecto al ajustado en el modelo. Sin embargo, en la literatura
se encuentran valores de $i^0$ con una gran dispersión entre 
$i^0 = 1.1\times10^{-3}$ mA/cm$^2$ \cite{medina2015} e $i^0 = 0.5$ mA/cm$^2$ 
\cite{umirov2019} que darían valores de $k^0$ entre $3.00\times10^{-9}$ cm/s y 
$1.37\times10^{-6}$ cm/s, respectivamente. Por lo cual puede afirmarse que el 
valor estimado por el modelo es razonable, dada la simplicidad del mismo.

Otro sistema en el cual las C-rates a las que se midieron los perfiles 
galvanostáticos cubren valores de SOC$_{\max}$, que van desde el estado completamente cargado
a completamente descargado, es el de LiFePO$_4$ (LFP) de Lei \textit{et al.} 
\cite{lei2015}, como puede verse en la Figura \ref{fig:ajustes}c. En este trabajo 
se consideraron sistemas LFP/nanotubos de carbono/grafeno (LFP-CNT-G) como 
materiales catódicos. Comparándolos con sistemas LFP-CNT y LFP-G, estos 
presentaron una alta capacidad y un desempeño considerable con el ciclado a 
densidades de corriente relativamente altas. En este trabajo los autores midieron 
los coeficientes de difusión de cada uno de los sistemas, utilizando la técnica de 
espectroscopia de impedancia electroquímica (EIS), y obtuvieron valores de 
$1.04\times10^{-12}$ cm$^2$/s, $1.738\times10^{-13}$ cm$^2$/s y 
$8.225\times10^{-13}$ cm$^2$/s, respectivamente para cada uno de los sistemas 
mencionados. Al ajustar los datos experimentales del primero de ellos
con el modelo que se presenta aquí, considerando un tamaño de partícula de 
$0.35 \mu$m, se obtuvo un valor de $2.85\times10^{-13}$ cm$^2$/s para dicho 
parámetro. Se aprecia una discrepancia de un orden de magnitud para el mismo, 
aunque se encuentra dentro de los valores medidos para las otras síntesis. 
El valor ajustado para $k^0$ de $1.00\times10^{-9}$ cm/s muestra una
concordancia excelente con el valor experimental de $1.23\times10^{-9}$ cm/s, que
fue obtenido utilizando la ecuación \ref{eq:i0k0}, 
$i^0=5.127\times10^{-4}$ mA/cm$^2$, $M_r = 157.75$ g/mol y 
$\rho = 1.36$ g/cm$^3$ \cite{jin2018}.

Para el caso del LCO se obtuvo un valor de $D$ de $5.34\times10^{-9}$ cm$^2$/s y
un valor de $k^0$ de $1.00\times10^{-5}$ al ajustar los datos experimentales 
de Wang \textit{et al.} \cite{wang2019high}. Estos datos ya se presentaron en la Figura \ref{fig:preproc} y ahora se muestran junto al ajuste en la Figura \ref{fig:ajustes}d.
En dicho trabajo sintetizaron partículas esféricas en el rango
de 5 a 40 $\mu$m y utilizaron una técnica de microelectrodo de una sola partícula,
la cual les permitió seleccionar partículas de 20 $\mu$m, por lo que este fue el
valor utilizado para $d$. En este experimento, el recubrimiento de TiO$_2$ mejora 
la eficiencia coulómbica, el rendimiento de la sobrecarga, la 
\textit{rate capability} y la estabilidad con los ciclos a alto voltaje. El valor 
acá ajustado para el coeficiente de difusión se encuentra en el rango de los 
presentados en este trabajo (10$^{-10}$--10$^{-8}$ cm$^2$/s) en la ventana de potencial
4.0--4.4 V.

Bak \textit{et al.} \cite{bak2011} sintetizaron nanopartículas híbridas de la 
espinela LiMn$_2$O$_4$ (LMO) con óxido de grafeno reducido de unos 0.01--0.04 
$\mu$m. Utilizando los datos de este trabajo, considerando una geometría esférica y un tamaño característico de 
difusión de 0.025 $\mu$m se obtuvo un $D$ de $3.51\times10^{-14}$ cm$^2$/s y una
$k^0$ de $1.87\times10^{-8}$ cm/s mediante el ajuste que se muestra en la Figura 
\ref{fig:ajustes}e.

Por último, en la Figura \ref{fig:ajustes}f, se tiene la fase espinela 
LiNi$_{0.5}$Mn$_{1.5}$O$_4$ (LNMO), que es un material prometedor para ser 
utilizado como cátodo debido a su voltaje alto de operación de 4.7 V vs. Li$^+$/Li. 
En este caso, Nishikawa \textit{et al.} \cite{nishikawa2017} también
utilizaron una técnica de medición de una sola partícula para este sistema con 
diámetros en el intervalo de 10 a 20 $\mu$m. En su trabajo estimaron el área 
superficial de la partícula seleccionada, asumiendo una forma esférica, de donde
se obtuvo un tamaño de partícula de 8 $\mu$m. Los valores ajustados para el
coeficiente de difusión y la constante cinética son $1.23\times10^{-9}$ cm$^2$/s 
y $1.23\times10^{-6}$ cm/s, respectivamente. Como se realizó en los casos del 
LTO y LFP, se obtuvo un valor de la $k^0$ al utilizar la densidad de corriente
de intercambio reportada por Nishikawa \textit{et al.}, 0.2 mA/cm$^2$. Además,
se utilizó $M_r$ y $\rho$ de 182.7 g/mol y 2.3 g/cm$^3$, respectivamente, y se
obtuvo $k^0 = 3.29\times10^{-7}$, que sólo difiere en un orden de magnitud con 
el valor ajustado.

También es ilustrativo presentar la zona del diagrama de SOC$_{\max}$ en el plano $\log(\Xi)$--$\log(\ell)$ donde se realizaron los ajustes de las simulaciones galvanostáticas con el presente modelo 
heurístico, 
como se muestra en la Figure \ref{fig:ajustes-mapa}.
\begin{figure}[h!]
    \centering
    \includegraphics[width=0.7\textwidth]{FastCharging/un/resultados/ajuste/mapa.png}
    \caption{Región del diagrama de nivel de capacidades máximas SOC$_{\max}$ en la que se encuentra cada uno de los experimentos de literatura analizados aquí
    \cite{mancini2022, he2012, lei2015, wang2019high, bak2011, nishikawa2017}
    luego del ajuste.}
    \label{fig:ajustes-mapa}
\end{figure}

Los parámetros obtenidos de los ajustes para los distintos materiales están
agrupados en la Tabla \ref{t:dk0} con su incerteza correspondiente. Las mismas
fueron calculadas usando la matriz de covarianza de los parámetros del modelo. 
Para ello se supone que la incerteza en C-rate es mucho menor que en 
SOC$_{\max}$ y que los parámetros del modelo dependen de estos valores, 
entonces dicha matriz está estrechamente relacionada a sus incertidumbres
\begin{equation}
    \sigma_{a_j, a_k}^2 = \sum_{i=0}^{N-1} \frac{\partial a_k}{\partial \text{SOC}_{\max,i}} \frac{\partial a_j}{\partial \text{SOC}_{\max,i}},
\end{equation}
donde en este caso $a_j$ y $a_k$ pueden ser $D$ y $k^0$. Las incertezas en 
estos parámetros están dadas en la diagonal de la matriz,
\begin{equation}
    \sigma_D^2 = \sum_{i=0}^{N-1} \left(\frac{\partial D}{\partial \text{SOC}_{\max,i}}\right)^2
\end{equation}
y
\begin{equation}
    \sigma_{k^0}^2 = \sum_{i=0}^{N-1} \left(\frac{\partial k^0}{\partial \text{SOC}_{\max,i}}\right)^2.
\end{equation}
Si tuviéramos una expresión analítica, es decir si conociéramos la expresión 
explícita dada en la ecuación \ref{eq:socmax}, se podría calcular fácilmente
estos valores. Dado que este no es el caso, lo que se hace es resolverlo 
numéricamente aproximando la matriz de covarianza por la inversa de la matriz
Hessiana, que a su vez también es aproximada por el producto de la matriz
Jacobiana con su transpuesta \cite{bard1974}.
\begin{table}[h!]
    \centering
    \caption{Valores ajustados para $D$ y $k^0$ con su incerteza 
    correspondiente.} 
    \setlength\extrarowheight{2pt}\stackon{%
    \begin{tabular}{l c c}
        \toprule
        \textbf{Material del electrodo} & 
        \textbf{$D$ [cm$^2$/s]} &  
        \textbf{$k^0$ [cm/s]} \\
        \midrule
        NG \cite{mancini2022} & (1.233 $\pm$ 0.001)$\times10^{-10}$ & (2.31 $\pm$ 0.01)$\times10^{-7}$ \\
        LTO \cite{he2012} & (6.58 $\pm$ 0.06)$\times10^{-12}$ & (8.1 $\pm$ 0.7)$\times10^{-8}$ \\
        LFP \cite{lei2015} & (2.8 $\pm$ 0.1)$\times10^{-13}$ & (1.00 $\pm$ 0.03)$\times10^{-11}$ \\
        LCO \cite{wang2019high} & (5.3 $\pm$ 0.2)$\times10^{-9}$ & (1.0 $\pm$ 0.9)$\times10^{-5}$ \\
        LMO \cite{bak2011} & (3.51 $\pm$ 0.03)$\times10^{-14}$ & (1.9 $\pm$ 0.1)$\times10^{-8}$ \\
        LNMO \cite{nishikawa2017} & (1.2 $\pm$ 0.3)$\times10^{-9}$ & (1.2 $\pm$ 0.9)$\times10^{-6}$ \\
        \bottomrule
    \end{tabular}
    }{}
    \label{t:dk0}
\end{table}
