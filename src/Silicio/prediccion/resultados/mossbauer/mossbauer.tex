\subsection{Espectroscopia Mössbauer: División de picos}

\begin{equation}\label{eq:C}
    C_{\text{Li}} = \frac{x}{1+x} \hspace{0.8cm}y \hspace{0.8cm}  C_{\text{Si}} = 1 - C_{\text{Li}}
\end{equation}

\begin{equation}\label{eq:mossbauer}
    \Delta = a\min\left\lbrace C_{\text{Li}},C_{\text{Si}}\right\rbrace + b
\end{equation}

\begin{figure}[h!]
    \centering
    \includegraphics[width=.7\textwidth]{Silicio/prediccion/resultados/mossbauer/mossbauer.png}
    \caption{Desplazamiento entre los dos picos en los espectros de efecto 
    Mössbauer. Los triángulos que apuntan hacia arriba corresponden a dos 
    medidas de Li \text{et al.} (eje de la izquierda), la línea discontinua es la 
    predicción de la ecuación \ref{eq:mossbauer}, utilizando concentraciones 
    medias de átomos de Li y Si. Los círculos azules son las predicciones dadas 
    por la ecuación \ref{eq:mossbauer}, con $C_{\text{Li}}$ , $C_{\text{Si}}$ 
    calculados a partir de la concentración de los vecinos más cercanos (eje de 
    la derecha). Las barras de error son menores que el tamaño de los puntos.}
    \label{fig:mossbauer}
\end{figure}
