\subsection{Análisis de las trayectorias}\label{s:observables}

Para analizar las trayectorias de una MD existen distintos observables usuales 
que pueden calcularse a partir de las posiciones o de las velocidades, algunos
de ellos dan información de carácter estructural, como puede ser la función 
distribución radial, y otros de carácter dinámico, como el desplazamiento 
cuadrático medio que puede ser utilizado para estimar coeficientes de difusión.

\subsubsection{Funciones distribución radial}

La función de distribución radial (RDF, de sus siglas en inglés \textit{radial 
distribution function}), usualmente referida como $g(r)$, permite caracterizar la
estructura local de un fluido describiendo la probabilidad de encontrar un átomo
en una cáscara a una distancia $r$ de un átomo de referencia,
\begin{equation}\label{eq:rdf}
    g(r) = \frac{V}{N^2} \sum_{i=1}^N \sum_{i \neq j} \delta(r - r_{ij}),
\end{equation}
donde $V$ es el volumen de la celda de simulación, $N$ el número de partículas y 
$r_{ij}$ la distancia entre el átomo $i$ y el $j$. Esta cantidad también puede 
calcularse como la razón entre la densidad media $\rho$ a una distancia $r$ del 
átomo de referencia y la densidad a esa misma distancia de un gas ideal.

De la misma forma que en la ecuación \ref{eq:rdf} puede discriminarse el tipo 
de átomos al considerar la probabilidad de encontrar un átomo, de tipo B, en un 
cascarón a una distancia $r$ de un átomo de referencia, de tipo A, para definir 
las funciones distribución radial parciales
\begin{equation}\label{eq:prdf}
    g_{AB}(r) = \frac{V}{4 \pi r^2 N_B} \sum_{i}^{N_A} \sum_{j\neq i}^{N_B} \delta(r - r_{ij}),
\end{equation}
donde $N_A$ y $N_B$ son los números de átomos de tipo A y B, respectivamente. 

Una característica importante de la RDF es que si sus picos están bien definidos,
a distancias largas, entonces la estructura se corresponde con un sólido, si sus 
picos están ensanchados con respecto a estos ya medida que la distancia aumenta, 
la $g(r)$ empieza a oscilar alrededor de la unidad, entonces se corresponde con 
un liquido.


\subsubsection{Función distribución radial de a pares}

La combinación de las RDFs parciales de la ecuación \ref{eq:prdf} es que permiten
calcular la función distribución radial de a pares, $G(r)$ \cite{billinge2019}
\begin{equation}
    G(r) = 4 \pi r \rho_0 \left[\sum_{\langle A,B \rangle} \frac{b_A b_B}{\langle b\rangle^2} g_{AB}(r) - 1\right], 
\end{equation}
donde $\rho_0$ es la densidad de la celda de simulación, $\langle A, B \rangle$
considera las permutaciones sin repeticiones de A y B, $b_A$ y $b_B$ son los 
factores de dispersión de los átomos A y B, respectivamente, y $\langle b \rangle$
es el factor de dispersión promedio de la celda de simulación. La $G(r)$ es 
directamente comparable con la función distribución radial de a pares (PDF) que 
se puede medir con rayos x.


\subsubsection{Número de coordinación}

El número de coordinación (CN, de sus siglás en inglés, \textit{coordination 
number}), también llamado ligancia, de un átomo dado en un sistema, se define 
como el número de átomos, moléculas o iones unidos a él. Para definirlo pueden 
tomarse dos alternativas, la primera de ellas contando el número de vecinos que 
rodean a un determinado tipo de átomo en promedio hasta un cierto radio de corte 
$r_{\text{cut}}$, definido por el mínimo en la RDF después de su primer pico; 
la segunda de ellas a partir de la integral de la RDF,
\begin{equation}
    \int_0^{r_{\text{cut}}} g(r) dV.
\end{equation}
De manera análoga pueden definirse el número de coordinación para segundos 
vecinos y así sucesivamente.

Así como se definieron las RDFs parciales en la ecuación \ref{eq:prdf}, puede 
definirse los número de coordinación de distintos tipos de átomos.


\subsubsection{Difusión}

Si una partícula permanece en un sitio por un periodo de tiempo lo 
suficientemente largo, comparado al tiempo que demora en dar un salto hacia otro
sitio, entonces pierde memoria de donde vino y el próximo salto se dará hacia 
una dirección aleatoria, se dice entonces que el movimiento de la partícula es 
estocástico y que sigue una caminata aleatoria. Si esto se cumple, entonces, el 
desplazamiento cuadrático medio es proporcional al tiempo, donde la constante de 
proporcionalidad es el coeficiente de difusión de traza
\begin{equation}
    D^{*} = \frac{\langle \Delta r^2 \rangle}{2d\cdot t},
\end{equation}
donde $d$ es la dimensión del problema, $t$ el tiempo y 
$\langle \Delta r^2 \rangle$ el desplazamiento cuadrático medio, que en un 
sistema de $N$ partículas puede calcularse de la siguiente manera
\begin{equation}
    \langle \Delta r^2 \rangle = \frac{1}{N} \sum_{i=1}^{N} \langle |\vec{r_i}(t) - \vec{r_i}(t_0)|^2 \rangle.
\end{equation}

Si se conoce el coeficiente de difusión de traza, entonces el coeficiente de 
difusión químico puede obtenerse de la siguiente relación ~\cite{gomer1990}
\begin{equation}
    D = \left( \frac{\partial (\mu / k_BT)}{\partial ln \theta} \right)_T D^{*},
\end{equation}
donde $\mu$ es el potencial químico y $\theta$ es la concentración.

Usualmente, el tiempo que demora un átomo en dar un salto en una dinámica 
molecular a temperatura ambiente es muy largo, lo que impide tener una buena 
estadística en un tiempo de simulación razonable, por lo que es usual calcular 
el desplazamiento cuadrático medio a distintas temperaturas altas y extrapolar
el valor que se obtendría a temperatura ambiente mediante una ecuación de tipo 
Arrhenius, en la cual el coeficiente de difusión puede separarse en un 
prefactor $D_0$ y un término tipo Boltzmann,
\begin{equation}
    D = D_0 e^{-E / k_BT},
\end{equation}
donde $E$ es la energía de activación del proceso. 
