\section{Conclusiones del capítulo}

Se desarrolló un esquema de parametrización DFTB que permite optimizar otros 
observables además de las energías absolutas. En este caso fue utilizado para 
ajustar las energías de formación relativas de las estructuras cristalinas de 
Li-Si. El modelo obtenido presentó predicciones óptimas de las energías de 
formación tanto en el conjunto de entrenamiento cristalino como en el conjunto 
de evaluación amorfo para todo el rango de composiciones Li$_x$Si presente en 
la litiación de los ánodos de silicio. Para el caso del Si amorfo puro, que 
exhibió las mayores discrepancias con respecto a DFT, la distribución radial de 
a pares simulada resultó en una reproducción excelente del experimento. El gran
desempeño de este modelo lo convierte en una elección adecuada para el utilizarlo
en simulaciones futuras, como se lo hace en el capítulo siguiente.
