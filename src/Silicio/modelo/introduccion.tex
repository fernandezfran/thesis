\section{Introducción}

Las simulaciones computacionales pueden facilitar la aplicación industrial de los
ánodos de silicio. Sin embargo, la disciplina debe afrontar los problemas 
intrínsecos de la complejidad del sistema. Las simulaciones deben ser capaces
de seguir los cambios drásticos que se producen en el material durante los ciclos
de carga y descarga. El ánodo de silicio pasa por varias aleaciones con distintas
propiedades fisicoquímicas que ya han sido mencionadas en capítulos anteriores.
En el capítulo \ref{ch:caracterizacion} se utilizó un potencial ReaxFF 
\cite{fan2013} junto con un método de exploración que permitió predecir distintas
estructuras amorfas que podrían presentarse en electrodos altamente ciclados.

En este capítulo se provee un modelo transferible capaz de representar con 
precisión las regiones no exploradas del espacio de configuraciones en su 
parametrización, permitiendo el descubrimiento de estructuras presentes en los
electrodos y previniendo falsos positivos. DFTB implica un nivel de teoría
más riguroso que los campos de fuerza clásicos, como el ReaxFF, y es más rápido que
los cálculos de la Teoría del Funcional de la Densidad (DFT), lo que permite 
estudiar también la dinámica. También se presenta un algoritmo de parametrización
que permite optimizar el modelo reproduciendo otros observables más allá de las
energías absolutas. Con esta idea se centra la discusión en la reproducción de las
energías de formación de Li$_x$Si, una característica clave para lograr robustez
y transferibilidad. 
