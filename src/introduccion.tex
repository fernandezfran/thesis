El calentamiento global es el mayor problema ambiental que enfrenta el mundo, el 
mismo se refiere al efecto que producen las actividades humanas en el clima, como 
la quema de combustibles fósiles o la deforestación, que emiten a la atmósfera
grandes cantidades de dióxido de carbono, CO$_2$, entre otros gases de efecto 
invernadero. Estos gases absorben la radiación infrarroja emitida por la tierra 
provocando un incremento de la temperatura de la misma que lleva asociado un 
aumento en la frecuencia y la intensidad de eventos climáticos extremos 
~\cite{houghton2005}. Según el Panel Intergubernamental del Cambio Climático 
(IPCC), desde la época preindustrial, las actividades humanas han provocado 
aproximadamente 1.0$^{\circ}$C de calentamiento global y al ritmo actual se van 
a sobrepasar los 1.5$^{\circ}$C antes del 2050, un cambio en la temperatura
media que las emisiones previas por sí solas no habían alcanzado
~\cite{harvey2018}. Limitar el calentamiento a esta temperatura requiere que se 
realicen rápidamente cambios sin precedentes en la tecnología y en el 
comportamiento humano. Uno de los cambios más importante es el de la matriz 
energética, en la cual las energías renovables deberán suministrar alrededor del 
80\% de la energía para 2050, donde los vectores energéticos, como las baterías 
de litio, juegan un rol fundamental debido a la intermitencia de estas formas de 
generación de energía.

El litio es el metal más liviano de la tabla periódica y uno de los elementos más
importantes dentro de los minerales necesarios en la producción de baterías de
litio. En particular, para la Argentina tiene un interés económico, social, 
industrial y tecnológico ya que es uno de los países que integran, junto a 
Bolivia y Chile, el Triangulo de Litio, el cual acumula el 70\% de las reservas 
mundiales de este mineral. Aún más importante que esta cantidad de reservas es 
que las mismas se encuentran en salares que, a grandes rasgos, es más barato
extraer litio de ellos en comparación a las pegmatitas, heroctitas o jadaritas, 
que son las rocas de las cuales se puede extraer litio en una minería usual.
A pesar de esto se tienen que llevar a cabo distintas consideraciones ambientales,
sociales y legales del proceso de extracción e incentivar el desarrollo de valor
agregado a dicha extracción ~\cite{heredia2020}.

En esta tesis se presentan estudios realizados mediante el uso de simulaciones 
computacionales en... TODO

\section{Transición energética}

La demanda internacional de energía sigue en aumento y más del 80\% de la misma
sigue siendo producida por combustibles fósiles. Sin políticas comprometidas con 
la transición energética no se espera que esta proporción disminuya, dejando 
lugar a la producción de energía mediante fuentes renovables, en los próximos 20
años. Si el cambio en la matriz energética se deja en manos del mercado, las 
fuentes de energía limpias no sustituirán por sí solas a los métodos tradicionales
hasta que no sólo alcancen una paridad de precios, si no que se vuelvan 
considerablemente más baratas de manera que justifiquen dicho cambio
~\cite{davidson2019}. Dicho esto, cumplir los objetivos climáticos y realizar la 
transición hacia un futuro con menos carbono va a requerir inversiones 
sustanciales por parte de los gobiernos ~\cite{leonhardt2022}.

La gran mayoría de las naciones desarrolladas han implementado un marco legal de 
apoyo y habilitación para ayudar a promover la integración de las energías 
renovables modernas en sus sistemas energéticos. Estas políticas se han 
desarrollado como resultado, o en apoyo, de acuerdos internacionales como el 
Acuerdo de París, el Protocolo de Kioto o el Green Deal europeo. Los objetivos 
de las energías renovables y los incentivos fiscales dirigidos al sector 
energético son las dos políticas más comunes en países en desarrollo para apoyar 
la transición energética ~\cite{cantarero2020}.

En Argentina el sector energético depende altamente de la utilización de 
combustibles fósiles, donde la capacidad de generación de energía está 
principalmente atada a las centrales térmicas convencionales y a grandes 
centrales hidroeléctricas, mientras que tan sólo una pequeña cantidad proviene 
de plantas nucleares y de fuentes de energías renovables. En cuanto al potencial
de producción de energía de fuentes renovables, Argentina tiene una gran 
capacidad eólica y solar. El gobierno nacional viene incentivando la instalación 
de dichas fuentes de energía desde 2009, mediante el programa GENREN. A fines 
del 2015 se estableció un objetivo de que el 20\% de la energía fuera generada
mediante estas fuentes para 2025 y en 2016 se introdujo un nuevo esquema de 
compra con el programa RenovAr ~\cite{schaube2018}. Este desarrollo viene 
acompañado de un amplio espectro de investigaciones académicas 
interdisciplinarias.

\subsection{Formas de generación de energías renovables}

\subsection{Vectores energéticos}

\section{Baterías}

\subsection{Un poco de historia}
\subsection{Funcionamiento}
\subsection{Baterías de litio}

Las baterías de litio son dispositivos electroquímicos ampliamente utilizados 
como fuentes de energía. Entre las propiedades físicas del litio destacan su 
densidad baja (0.534 cm$^3$), su capacidad específica alta (3860 mAh g$^{-1}$) y
su potencial de reducción bajo (-3.04 V vs. SHE), estas propiedades lo convierten
en un material que puede servir como ánodo para baterías. A finales de la década
del '50 del siglo XX se observó que el litio metálico formaba una capa de 
pasivación con distintos electrolitos no-acuosos permitiendo prevenir una 
reacción química directa entre estas dos componentes, pero dejando que los iones 
de litio la atraviesen. Esto generó un interés que llevó a la fabricación de 
baterías primarias de litio en la década del '60 utilizando distintos tipos de 
cátodos incluyendo dióxido de sulfuro de litio (LiSO$_2$), óxido de manganeso de 
litio (LiMnO$_2$) y óxido de cobre de litio (LiCuO), entre otros. Avances 
posteriores en la comprensión de la intercalación de litio en diferentes 
materiales dio origen a las baterías recargables de ion-litio. En la tabla 
\ref{t:historia} se presentan las primeras baterías secundarias de litio 
comercializadas hasta 1991, año en el que se dejó de utilizar litio metálico como
ánodo y se comenzó a utilizar uno de carbono. En la misma puede notarse el avance 
con los años de la densidad de energía, sin embargo este no es el único parámetro 
a tener en cuenta en el progreso ya que también tienen que considerarse la 
cantidad de ciclos que se pueden realizar, el tiempo de carga, la razón de 
descarga y el precio de producción. En la actualidad podemos encontrarlas en una 
gran variedad de dispositivos que va desde los celulares hasta vehículos 
eléctricos ~\cite{reddy2020}. 
\begin{table}[h]
    \centering
    \caption{Primeras baterías de litio recargables comercializadas.}
    \setlength\extrarowheight{2pt}\stackon{%
    \begin{tabular}{c c c c}
        \toprule
        \thead{\normalsize\bfseries Sistema\\\normalsize\bfseries electroquímico} & 
        \thead{\normalsize\bfseries Voltaje (V)} & 
        \thead{\normalsize\bfseries Densidad de energía\\\normalsize\bfseries específica (Wh/kg)} & 
        \thead{\normalsize\bfseries Compañía comercial} \\
        \midrule
        Li//TiS$_2$ & 2.1 & 130 & Exxon (1978) \\
        Li//LiAlCl$_4$-SO$_2$ & 3.2 & 63 & Duracell (1981) \\
        Li//NbSe$_3$ & 2.0 & 95 & Bell Telephone Lab. Inc. (1983) \\
        LiAl//polianilina & 3.0 & -- & Bridgestone (1987) \\
        Li//MoS$_2$ & 1.8 & 52 & MoLi Energy (1987) \\
        Li//V$_2$O$_5$ & 1.5 & 10 & Toshiba (1989) \\
        LiAl//polipirrol & 3.0 & -- & Kanebo (1989) \\
        Li//Li$_{0.3}$MnO$_2$ & 3.0 & 50 & Tadiran (1989) \\
        LiVO$_x$ & 3.2 & 200 & Hydro-Québec (1990) \\
        C//LiCoO$_2$ & 3.6 & 150--190 & Sony (1991) \\
        \bottomrule
    \end{tabular}
    }{}
    \label{t:historia}
\end{table}

\section{Objetivos y estructura de tesis}
