\subsection{Funcional de la densidad de enlace fuerte (DFTB)}\label{s:dftb}

El formalismo del funcional de densidad de enlace fuerte (DFTB, \textit{density
functional tight-binging}) ha sido ampliamente descripto en la literatura 
\cite{elstner1998,frauenheim2000,seifert2007,gaus2011}. El método DFTB se basa 
en una expansión a segundo orden de la energía de la teoría del funcional de la 
densidad (DFT) con respecto a una fluctuación de la densidad electrónica de 
referencia \cite{foulkes1989}. La energía de DFTB resultante puede escribirse de 
la siguiente manera:
\begin{equation}\label{eq:dftb}
    E_{\text{DFTB}}=\sum_i^{\text{occ}}\langle\psi_i|\hat{H}^0|\psi_i\rangle+\frac{1}{2}\sum_{AB}\gamma_{AB}\Delta q_A\Delta q_B+E_{\text{rep}}^{AB}
\end{equation}
donde $\psi_i$ denota los orbitales Kohn-Sham (KS) de una partícula. Con una 
combinación lineal de orbitales atómicos, $\psi_i$ se expande en un conjunto de 
orbitales de valencia pseudoatómicos de tipo Slater $\phi_\nu$,
\begin{equation}
    \psi_i({\bf r})=\sum_\nu c_{\nu i}\phi_\nu({\bf r}-{\bf r}_A),
\end{equation}
que se determinan resolviendo la ecuación secular KS
\begin{equation}\label{eq:ks}
    \sum_\mu c_{\mu i}\left(H^0_{\nu\mu}-\epsilon_iS_{\nu\mu}\right)=0, \;\;\forall \nu,i
\end{equation}
donde $S_{\nu\mu}=\langle \phi_\nu| \phi_\mu\rangle$ y $\epsilon_\nu$ son la 
matriz de superposición y los autovalores de un átomo aislado, respectivamente.
${H}^0_{\nu\mu}$ es el Hamiltoniano efectivo KS generado con la densidad 
electrónica de referencia, $\rho^0$, y está definido como
\begin{equation}\label{eq:h0}
    H^0_{\nu\mu}=\begin{cases}
        \epsilon_\mu & \text{si}\; \nu=\mu\\
        \langle \phi_{\nu}| -\frac{1}{2}\nabla^2+v_{\text{eff}}\left[\rho_A^0+\rho_B^0\right]|\phi_{\nu}\rangle&\text{si}\;\mu\in A,\; \nu\in B\;\text{y} \;A\ne B\\
        0& \text{si no}
    \end{cases}
\end{equation}
donde $\rho_A^0$ es la densidad de referencia de un átomo neutro $A$ y 
$v_{\text{eff}}$ el potencial KS efectivo, construido a partir de la superposición
de densidades centradas en átomos neutros. En particular, los elementos de la 
matriz del Hamiltoniano dependen solo de los átomos $A$ y $B$, por lo tanto sólo
se calculan explícitamente los elementos de dos centros de las matrices del 
Hamiltoniano y de superposición en función de la distancia y la orientación, usando 
las reglas de transformación de Slater-Koster \cite{slater1954}.

Una de las partes cruciales del uso del método DFTB es calcular las funciones 
base y las densidades atómicas $\phi$ y $\rho^0$, respectivamente. Los orbitales
pseudoatómicos y las densidades se obtienen de resolver las ecuaciones atómicas KS 
modificadas en las que se agrega un potencial de confinamiento, $V_{\text{conf}}$,
\begin{equation}\label{eq:dft}
    \left[\hat{T}+V_{\text{eff}}+V_{\text{conf}}\right]\phi_\mu=\epsilon_\mu\phi_\mu.
\end{equation}
Una práctica común dentro de la comunidad de DFTB consiste en elegir un potencial
de confinamiento parabólico, cuadrático, o una función de ley de potencia.

El segundo término en la ecuación \ref{eq:dftb} es la energía debida a las 
fluctuaciones de cargas y se parametriza analíticamente como una función de las
cargas orbitales y de $\gamma_{AB}$, que a su vez es una función de la separación 
interatómica y del parámetro de Hubbard, $U$, que se obtienen suponiendo que son 
iguales a los de los átomos aislados y se calculan como la diferencia de la 
afinidad electrónica y la energía de ionización para distintos momentos angulares 
orbitales \cite{elstner1998b}. $\Delta q_X = q_X - q_X^0$ es la carga de Mulliken 
inducida autoconsistente en el átomo $X$ \cite{elstner1998}.

La contribución restante a la energía total de DFTB en la ecuación \ref{eq:dftb}
es $E_{\text{rep}}$ y se corresponde con el potencial repulsivo diatómico que 
depende de la distancia y contiene los efectos de los electrones del núcleo, los 
términos de repulsión ion-ion y efectos de intercambio-correlación. 
La energía total repulsiva de un sistema es una suma de contribuciones de 
potenciales repulsivos $V_{\text{rep}}(r)$ de cada par de átomos
\begin{equation}\label{eq:rep}
    E_{\text{rep}}=\sum_{i<j} V_{\text{rep}}(r_{ij})
\end{equation}
donde $i$ y $j$ son los índices de los átomos en el sistema y $r_{ij}$ es la 
distancia entre ellos. Generalmente se considera que $V_{\text{rep}}$ es una
función empírica que se determina al ajustar datos de cálculos de estructura 
electrónica de un nivel superior, como DFT.
