\subsection{Modelo de una sola partícula}\label{s:metodologia}

Los modelos de orden reducido buscan simplificar modelos más complejos pero 
conservar sus capacidades predictivas. Estos pueden mejorar la compresión física 
y proporcionar soluciones precisas a un costo computacional significativamente menor.
Este es el caso de los modelos de una sola partícula (SPM, \textit{single particle 
model}), como el que se utiliza en esta tesis, que suponen que todas las partículas
dentro del electrodo se comportan de manera similar, lo que hace que todas pueden ser 
representadas por una sola partícula promedio para reducir la complejidad del 
modelado del continuo. Estos modelos pueden ofrecer predicciones rápidas y precisas 
de baterías reales, lo que tiene numerosas aplicaciones posibles. Por ejemplo, pueden 
utilizarse como herramientas de diseño para facilitar nuevas arquitecturas de 
electrodos, celdas y paquetes de baterías, y evaluar su rendimiento, lo cual reduce la
necesidad de diseñar prototipos costosos. También pueden ser utilizadas para 
determinar cuál de los distintos tipos de baterías disponibles en el mercado se adapta 
mejor a un caso de uso particular.

En esta tesis se utiliza un modelo de una sola partícula recientemente propuesto 
\cite{gavilan2023} para construir diagramas galvanostáticos de la capacidad máxima 
alcanzada por una sola partícula en función de dos parámetros adimensionales: uno
cinético,
\begin{equation}\label{eq:xi}
    \Xi = k^0 \sqrt{\frac{t_h}{C_r D}},
\end{equation}
y el otro de difusión finita,
\begin{equation}\label{eq:ele}
    \ell = d \frac{V}{A} \frac{C_r}{D t_h},
\end{equation}
donde $k^0$ es la constante cinética, $D$ el coeficiente de difusión, $V/A$ es la 
proporción volumen/superficie, $d$ es el tamaño característico de la partícula, 
$C_r$ denota la C-rate y $t_h$ el tiempo de una hora (en las unidades que
corresponda). Los parámetros de todas las ecuaciones que siguen a continuación se
presentan en la Tabla \ref{t:params}.
\begin{table}[h!]
    \centering
    \caption{Parámetros presentes en las ecuaciones del modelo de una sola pártícula}
    \setlength\extrarowheight{2pt}\stackon{%
    \begin{tabular}{l l}
        \toprule
        \textbf{Parámetro} & 
        \textbf{Definición} \\
        \midrule
        $\Xi = k^0 \sqrt{t_h / (C_r D)}$ & Parámetro galvanostático cinético \\
        $\ell = d (V/A) (C_r / (D t_h))$ & Parámetro galvanostático de difusión finita \\
        $D$ & Coeficiente de difusión del Li$^+$ \\
        $t, t_h$ & Tiempo y tiempo para una hora \\
        $C_r$ & C-rate \\
        $x, x_s$ & Grado de intercalación en el electrodo y en la interfase \\ 
                 & electrodo/electrolito \\
        $Q, Q_{\max}$ & Capacidad y capacidad máxima \\ 
        SOC & Estado de la Carga, $\text{SOC} = Q / Q_{\max}$ \\ 
        $z$ & Parámetro para establecer la geometría de la partícula: \\
            & plana ($z=1$), cilíndrica ($z=2$) o esférica ($z=3$) \\
        $d$ & Longitud de difusión -- radio de la partícula -- espesor \\ 
            & de la lámina \\
        $r$ & Distancia definida entre 0 y $d$ ($0 \leq r \leq d$) \\
        $V$ & Volumen del material activo \\
        $A$ & Área superficial del electrodo/electrolito \\
        $V / A$ & Proporción volumen/superficie, para la geometría plana toma\\
                & el valor de $d$, cilíndrica ($d/2$) o esférica ($d/3$) \\
        $I_c$, $i_c$ & Corriente y densidad de corriente (constantes) \\
        $k^0$ & Constante cinética \\
        $E$, $E^0$, $E_{\text{off}}$ & Potencial del electrodo de trabajo vs Li/Li$^+$,\\
                                     & potencial de equilibrio y de corte \\
        $\alpha$ & Coeficiente de transferencia de carga \\
        $\rho$ & Densidad del material \\
        $M_r$ & Masa molecular del material \\
        $F$ & Constante de Faraday \\
        $R$ & Constante universal de los gases \\
        $T$ & Temperatura absoluta \\
        \bottomrule
    \end{tabular}
    }{}
    \label{t:params}
\end{table}

En este modelo, el grado de intercalación $x$ en el punto $r$ y a tiempo $t$ se 
obtiene resolviendo numéricamente la ecuación diferencial 1D de Fick:
\begin{equation}\label{eq:fick}
    \frac{\partial x}{\partial t} = D \left[ \frac{\partial^2 x}{\partial r^2} + \frac{(z - 1)}{r} \left(\frac{\partial x}{\partial r}\right) \right],
\end{equation}
donde $D$ es el coeficiente de difusión y $z$ depende del tipo de geometría de la 
partícula \cite{vassiliev2016}. La ecuación \ref{eq:fick} puede resolverse 
utilizando el método de Crank-Nicolson mediante diferencias finitas 
\cite{crank-nicolson} con las siguientes condiciones de contorno en la superficie
de la partícula ($r = d$),
\begin{equation}
    \left(\frac{\partial x}{\partial r}\right)_d = - \frac{I_c}{F A \frac{\rho}{M_r}D},
\end{equation}
y al centro de la partícula ($r = 0$),
\begin{equation}
    \left(\frac{\partial x}{\partial r}\right)_0 = 0.
\end{equation}

La condición de corriente constante se fija con la ecuación de Butler-Volmer
\begin{equation}\label{eq:bv}
    I_c = F A \frac{\rho}{M_r} k^0 \left\{x_s \exp\left[ \frac{(1-\alpha)F(E-E^0)}{RT} \right] - (1 - x_s) \exp\left[ -\frac{\alpha F (E-E^0)}{RT} \right] \right\}
\end{equation}

El modelo parte de las siguientes suposiciones:
\begin{itemize}
    \item El transporte de carga dentro de la partícula está limitado por el
        movimiento de los iones insertados, es decir que el transporte electrónico
        es rápido.
    \item La difusión de los iones obedece la segunda Ley de Fick 1D para 
        geometría plana, cilíndrica o esférica.
    \item Se ignoran limitaciones en la transferencia de masa en el electrolito, dado que el coeficiente de difusión de Li$^+$ en este medio es del orden de 10$^{-5}$ cm$^2$/s \cite{valoen2005}, es decir, varios órdenes de magnitud mayor que en cualquiera de los materiales analizados.
    \item El coeficiente de difusión, $D$, permanece constante durante todo el 
        proceso de intercalación.
    \item El flujo de iones en el centro de la partícula es cero.
    \item La transferencia de carga en la interfase electrodo/electrolito se 
        describe mediante la ecuación de Butler-Volmer.
    \item La constante cinética, $k^0$, es la misma durante todo el proceso de 
        intercalación.
    \item La resistencia de la celda está dada por un valor constante, 
        $R_{\Omega}$.
    \item Se ignora la interacción entre los iones intercalados.
    \item No se consideran los cambios volumétricos de la partícula durante su 
        carga/descarga.
    \item También se ignoran migraciones de los iones por la presencia de un 
        campo eléctrico y reacciones secundarias.
\end{itemize}
La influencia de la mayoría de estas limitaciones a la hora de obtener la 
capacidad de la partícula se discuten en la referencia \cite{gavilan2023}. 


\subsubsection{Derivación de los parámetros adimensionales}\label{s:derivparam}

En esta sección se derivan los parámetros adimensionales de la ecuación
\ref{eq:bv} de Butler-Volmer en condiciones de corriente constante. Para ello se 
define $\xi = (nF/RT)(E - E^0)$ y $u = D t / d^2$, la ecuación \ref{eq:bv} se 
puede expresar de la siguiente forma integral \cite{aoki1984}
\begin{equation}\label{eq:aoki}
    \frac{i}{n F c^0} = k^0 \left(e^{\alpha_a \xi} + e^{-\alpha_c \xi}\right) \left\{ \frac{1}{1+e^{-\xi}} - \frac{d}{D} \int_0^u \frac{i}{n F c^0} \Theta_3(0|u - z) dz \right\},
\end{equation}
donde $c^0$ es la concentración máxima del ion en el material y $\Theta_3(\nu|x)$ 
es la función tita \cite{bieniasz2015} dada por la siguiente expresión
\begin{equation}
    \Theta_3(\nu|x) = \frac{1}{\sqrt{\pi x}} \sum_{n=-\infty}^{\infty} e^{-\frac{1}{x}(\nu + n)^2}
\end{equation}
Si se considera una corriente constantes $i_c$, la ecuación \ref{eq:aoki} da
\begin{equation}\label{eq:aoki2}
    \frac{i_c}{n F c^0} = k^0 \left(e^{\alpha_a \xi} + e^{-\alpha_c \xi}\right) \left\{ \frac{1}{1+e^{-\xi}} - \frac{d}{D} \frac{i_c}{n F c^0} \int_0^u \Theta_3(0|u - z) dz \right\}.
\end{equation}
Esta corriente constante se la puede expresar en términos de la C-rate como
\begin{equation}
    I_c = \frac{C_r Q}{t_h}
\end{equation}
y usando que la capacidad del electrodo es
\begin{equation}
    Q = V \left( \frac{\rho}{M} \right) F n
\end{equation}
y dividiendo por el área superficial y $n F c^0$ se obtiene que
\begin{equation}\label{eq:i_c}
    \frac{i_c}{n F c^0} = \frac{V C_r}{A t_h}.
\end{equation}
Reemplazando esta ecuación \ref{eq:i_c} en el lado derecho de la ecuación 
\ref{eq:aoki2}, se tiene
\begin{equation}\label{eq:aoki3}
    \frac{i_c}{n F c^0} = k^0 \left(e^{\alpha_a \xi} + e^{-\alpha_c \xi}\right) \left\{ \frac{1}{1+e^{-\xi}} - \frac{d}{D} \frac{V C_r}{A t_h} \int_0^u \Theta_3(0|u - z) dz \right\},
\end{equation}
de donde el factor $d (V/A) (C_r / (D t_h))$ se utiliza para definir el 
parámetro de difusión finita presentado en la ecuación \ref{eq:ele}. Reemplazando 
ahora la ecuación \ref{eq:ele} en \ref{eq:aoki3}, normalizando por la densidad 
de corriente y multiplicando a ambos miembros de la igualdad por el término 
$\sqrt{t_h / (C_r D)}$ se llega a que
\begin{equation}\label{eq:aoki4}
    \frac{i_c}{n F c^0} \sqrt{\frac{t_h}{C_r D}} = k^0 \sqrt{\frac{t_h}{C_r D}}\left(e^{\alpha_a \xi} + e^{-\alpha_c \xi}\right) \left\{ \frac{1}{1+e^{-\xi}} - \ell \int_0^u \Theta_3(0|u - z) dz \right\},
\end{equation}
de donde emerge el parámetro $\Xi$ ya definido en la ecuación \ref{eq:xi} como 
$\Xi = k^0 \sqrt{t_h / (C_r D)}$.
