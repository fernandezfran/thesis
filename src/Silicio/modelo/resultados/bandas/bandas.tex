\subsection{Ajuste de los conjuntos A y B de parámetros}

Como fue introducido en la sección \ref{s:dftb}, para el método DFTB hay dos 
grupos de parámetros a ser determinados, los electrónicos (los orbitales 
pseudoatómicos y las densidades electrónicas) y los potenciales repulsivos para 
cada par de elementos químicos (Si-Si, Si-Li y Li-Li). La optimización de cada 
uno de ellos está sujeta a reproducir alguna propiedad deseada, como la 
estructura de bandas, las energías de atomización, entre otras. Para el ajuste
de la estructura de bandas se siguió el trabajo de van den Bossche \cite{van2019},
considerándose para el Li los electrones de valencia 2s mientras que para el Si 
los 3s y los 3p. La comparación de la estructura de bandas entre DFTB y DFT para 
los conjuntos A y B de parámetros se analiza en profundidad en la referencia 
\cite{oviedo2023}.

%\begin{figure}[th]
%    \centering
%    \includegraphics[width=\textwidth]{Silicio/modelo/resultados/bandas/bandas.png}
%    \caption{Estructuras de banda calculadas por DFTB utilizando los dos conjuntos
%    A y B de parámetros, en comparación con las estructuras de banda calculadas 
%    por DFT/PBE para Li (bcc), Li$_7$Si$_3$, Li$_1$Si$_{15}$, y Si (diamante). 
%    Todas las bandas electrónicas están desplazadas a los niveles de Fermi (0 eV)
%    respectivos.}
%    \label{fig:bandas}
%\end{figure}

Para la parametrización del potencial de repulsión de cada uno de los conjuntos 
A y B se siguió el algoritmo de ajuste descripto en la sección \ref{s:algfit}
que permite optimizar los pesos de cada una de las estructuras en el conjunto de 
entrenamiento. Los coeficientes $\check{\boldsymbol{\xi}}_s$ minimizados para cada
conjunto, A y B, se reportan en la Tabla \ref{t:xiweights}.
\begin{table}[h!]
    \centering
    \caption{Pesos óptimos, $\check{\boldsymbol{xi}}_s$, de cada conjunto.}
    \setlength\extrarowheight{2pt}\stackon{%
    \begin{tabular}{l c c}
        \toprule
        \textbf{$s$} & 
        \textbf{conjunto A} & 
        \textbf{conjunto B} \\ 
        \midrule
        Li & $0.23\times10^{-2}$ & 0.49 \\
        Li$_{15}$Si$_4$ & 0.15 & 0.28$\times10^{-21}$ \\
        Li$_{13}$Si$_4$ & 0.21 & 0.17$\times10^{-1}$ \\
        Li$_7$Si$_3$ & 0.21 & 0.11$\times10^{-1}$ \\
        Li$_{12}$Si$_7$ & 0.23 & 0.11$\times10^{-2}$ \\
        LiSi & 0.21 & 0.35$\times10^{-3}$ \\
        Si & 0.83$\times10^{-7}$ & 0.49 \\
        \bottomrule
    \end{tabular}
    }{}
    \label{t:xiweights}
\end{table}
Es interesante que para el conjunto A el algoritmo reduce los pesos relativos del 
Li y del Si puro y aumenta los de las aleaciones. Mientras tanto, para el conjuntoB sucede lo contrario, los pesos óptimos son mayores para los elementos puros y
menores para las aleaciones (salvo para la Li$_{15}$Si$_4$). Este comportamiento 
se debe a que el término de la energía de bandas del conjunto A se construye 
utilizando los elementos puros mientras que el mismo término para el conjunto B
utiliza una de las aleaciones. Resulta razonable que el término de repulsión, que
busca compensar el residuo de la energía dada por todas las demás contribuciones 
energéticas, sea menos importante para las estructuras consideradas en el ajuste
de las energías de banda y más importante para el resto. Así, los coeficientes
óptimos $\check{\boldsymbol{\xi}}_s$ parecen ser capaces de percibir esta 
situación y tratar de compensarla al centrar la parametrización en las estructurasque más la necesitan.

