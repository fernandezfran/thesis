\section{Conclusiones del capítulo}

Con el fin de emular las estructuras amorfas encontradas en muchos experimentos 
electroquímicos, en este capítulo se generaron por computadora estructuras desordenadas de 
aleaciones de Li$_x$Si para varios valores de $x$ utilizando un algoritmo de 
dinámica acelerada y un campo de fuerzas reactivo. La exploración acelerada de 
mínimos locales (AELM) permitió la caracterización de una amplia gama de 
estructuras amorfas. El cambio de volumen de las estructuras litiadas en relación 
con el Si está en concordancia con los resultados experimentales de microscopía de fuerza atómica. Las
energías de las estructuras obtenidas representan bien el comportamiento 
electroquímico de la curva de potencial en función de la concentración de Li. Se 
analizó la función de distribución radial de a pares para los diferentes tipos de 
pares atómicos y se dilucidó la estructura compleja del segundo pico del RDF 
Si-Li mediante un análisis de interconexión de clusters. Además, haciendo un 
análisis de la formación de clústeres en función del radio de corte, se demostró 
que las estructuras amorfas no presentan diferentes enlaces de Si ni átomos de Si 
aislados. En su lugar, se encontró que el sistema se comporta como una red amorfa.
Estudiando los números de coordinación de primeros y segundos vecinos para las 
diferentes concentraciones, se mostró que esta red amorfa mantiene las conexiones 
tetraédricas para bajas concentraciones de Li y que tiende a formar cadenas lineales para 
altas concentraciones de Li. Por último, la definición de un nuevo parámetro 
permitió determinar el orden de corto alcance de las estructuras amorfas, definido
por interacciones débiles Li-Li e interacciones fuertes Li-Si y Si-Si. El método propuesto AELM 
resulta ser un método rápido y eficaz para obtener mínimos energéticamente 
relevantes. Se hizo una analogía con el templado simulado múltiple.% Un análisis 
%detallado de la eficiencia de AELM en comparación con otros métodos eficientes 
%como el templado simulado múltiple o los métodos de Monte Carlo es una motivación 
%para trabajos futuros.
