\subsection{Software}

En la actualidad existe una gran cantidad de códigos que permiten realizar 
simulaciones de dinámica molecular de manera eficiente aprovechando la estructura
paralela de los procesadores. En esta tesis se utilizó principalmente \path{LAMMPS} 
(\textit{Large-scale Atomic/Molecular Massively Parallel Simulator}, sus siglas
en inglés) ~\cite{lammps1, lammps2}, un código centrado en el modelado de 
materiales que permite simular con distintos campos de fuerzas, ensambles y 
condiciones de contorno. También se realizaron simulaciones con versiones 
modificadas de \path{GEMS} ~\cite{gems}, que posee métodos de aceleración que 
no se encuentran en \path{LAMMPS}.

Para la visualización de las trayectorias y la obtención de imágenes de 
estructuras representativas se utilizó \path{VMD} (\textit{Visual Molecular 
Dynamics}) ~\cite{vmd}. 

Para el post-procesamiento de las trayectorias y las series temporales de las 
propiedades termodinámicas se escribieron distintos códigos en Python, algunos de 
ellos se presentan en el apéndice \ref{a:software}.
