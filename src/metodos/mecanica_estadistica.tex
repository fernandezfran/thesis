\section{Breve introducción a la mecánica estadística}

En la mayoría de los experimentos que se realizan en un laboratorio se obtiene 
una serie de mediciones sobre sistemas macroscópicos, usualmente constituidos por 
más de 10$^{20}$ moléculas, durante un período de tiempo, a las cuales luego se 
les realiza un promedio. La mecánica estadística ofrece una interpretación de 
las propiedades del equilibrio de sistemas macroscópicos a partir de una teoría 
molecular aplicada a su configuración microscópica ~\cite{hill1986}.

Si se quisiera calcular alguna variable mecánica de un sistema termodinámico a
partir de consideraciones moleculares tendría que realizarse durante un período
de tiempo largo para suavizar las fluctuaciones y para que fuera independiente
del paso inicial a la hora de computar el promedio. Dado el gran número de 
moléculas interactuantes entre sí en estos sistemas, este cálculo está fuera de
alcance tanto en una consideración cuántica como en una clásica. Una alternativa
para solucionar esto es conectar el promedio temporal de la variable mecánica de
interés con el promedio de ensambles, donde un ensamble es simplemente una 
colección de un número muy largo de sistemas construidos de manera tal que 
reproducen las propiedades termodinámicas del sistema en cuestión. Si bien todos
los sistemas en el ensamble son idénticos desde el punto de vista termodinámico,
no lo son en sus configuraciones moleculares. De esta manera ahora se tiene
que el valor promedio de la variable mecánica en estudio se realiza sobre estas 
replicas del sistema en vez de sobre su evolución temporal.

\subsection{Ensambles}

Algunos de los ensambles termodinámicos más relevantes son:
\begin{itemize}
    \item \textit{Ensamble microcanónico (NVE)}, un sistema aislado en el cual el 
        número de partículas, el volumen y la energía permanecen constantes.
    \item \textit{Ensamble canónico (NVT)}, un sistema cerrado, con una cantidad
        fija de partículas y volumen constante, en contacto con un baño de 
        temperatura lo suficientemente grande de manera tal que la misma permanece 
        constante.
    \item \textit{Ensamble isotérmico-isobárico (NPT)}, en este sistema el número
        de partículas está fijo y en contacto con un baño de temperatura y un
        pistón que permite variar el volumen para mantener la presión constante.
\end{itemize}

\subsection{Hipótesis ergódica}

El primer postulado de la Mecánica estadística presentado en esta tesis es 
referido como la \textbf{hipótesis ergódica} y nos dice que \textit{El promedio 
temporal de una variable mecánica $M$ en el sistema termodinámico de interés es 
igual al promedio de ensambles de M, en el límite del conjunto de ensambles que 
tiende a infinito, siempre que los sistemas del conjunto de ensambles reproduzcan 
el estado termodinámico y el entorno del sistema real de interés}. Es decir que
es lo mismo calcular el promedio en la evolución temporal que en una cantidad 
grande estructuras instantáneas representativas del sistema. Para poder aplicar
este postulado se necesita conocer la probabilidad relativa de cada uno de los 
estados presentes en el ensamble.

\subsection{Postulado de igual probabilidad a priori}

El segundo postulado de la Mecánica estadística presentado se refiere a esto
último y establece que \textit{En un conjunto de ensambles representativo de un 
sistema termodinámico aislado, los sistemas del conjunto de ensambles se distribuyen 
uniformemente, es decir, con igual probabilidad o frecuencia, sobre los posibles 
estados con los valores especificados de dicho sistema termodinámico aislado}.
En otras palabras, cada estado esta representado por la misma cantidad de sistemas
en el ensamble.

\subsection{Fluctuaciones}

Para definir el valor medio, o examinar la amplitud de las fluctuaciones en torno 
al valor medio, de una propiedad de un sistema que puede existir en varios estados
$j$ con probabilidades $P_j$, la propiedad misma debe definirse en cada estado 
$j$. Una propiedad que cumpla estos criterios es \say{mecánica} por definición.

Si se considera el ejemplo de la fluctuación de la energía de un sistema cerrado 
en  contacto con un reservorio de temperatura lo suficientemente grande (NVT) 
donde las fluctuaciones de energía están asociadas al intercambio de calor entre 
el sistema y el reservorio, dichas fluctuaciones de energía resultan ser muy 
pequeñas, por lo que la función de distribución de probabilidad para las 
diferentes energías tiene forma gaussiana en torno al valor medio $\overline{E}$. 
La dispersión en esta distribución de probabilidad puede, por lo tanto,
caracterizarse completamente por la desviación estándar $\sigma_E$, es decir, 
\begin{equation*}
\sigma_E = \sqrt{\overline{(E - \overline{E})^2}}.
\end{equation*}

% Si se diferencia
% $$
% \overline{E} \sum_j e^{-E_j(N,V)/kT} = \sum_j E_j(N,V) e^{-E_j(N,V)/kT}
% $$
% con respecto a $T$, y luego se divide por $Q$, la función de partición, se tiene
% que
% $$
% \left(\frac{\partial\overline{E}}{\partial T}\right)_{V,N} + \frac{\overline{E}}{Q kT^2} \sum_j E_j e^{-E_j / kT} = \frac{1}{Q kT^2} \sum_j E_j^2 e^{-E_j/kT},
% $$
% o
% $$
% \overline{E^2} - (\overline{E})^2 = \overline{(E-\overline{E})^2} = \sigma_E^2 = k T^2 C_V,
% $$
% de la termodinámica, en general $C_V \approx \mathcal{O}(Nk)$ y $\overline{E} \approx \mathcal{O}(NkT)$. Por lo tanto
% Así encontramos en un sistema típico cerrado e isotérmico el comportamiento de la
% desviación estándar de la distribución de probabilidad de la energía.

Puede encontrarse que el comportamiento de la desviación estándar de la 
distribución de la probabilidad de la energía decrece a medida que aumenta el 
número de moleculas presentes en el sistema de forma proporcional a la inversa
de la raíz de esta cantidad,
\begin{equation}\label{eq:fluctuaciones}
    \frac{\sigma_E}{\overline{E}} = \frac{\sqrt{kT^2C_V}}{\overline{E}} \approx \mathcal{O}(N^{-1/2}).
\end{equation}

% Las variables que fluctúan son diferentes en cada ensamble, aunque las funciones 
% termodinámicas calculadas en mecánica estadística resultan ser independientes del 
% ensamble utilizado en el cálculo, por lo cual, pueden realizarse análisis 
% similares en las distintas cantidades que pueden variar en cada uno de ellos.

A pesar de que las variables que fluctúan son diferentes en cada ensamble, en el 
límite termodinámico, en el cual la cantidad de moléculas en un sistema es 
del orden de $10^{23}$, la relación que se presenta en la ecuación 
\ref{eq:fluctuaciones} muestra que las fluctuaciones son muy pequeñas y pueden 
ser ignoradas. Esto hace que los ensambles sean termodinámicamente equivalentes 
entre sí.

En termodinámica, las relaciones funcionales entre las variables termodinámicas 
de un sistema son independientes del entorno. Otra forma de decir esto es que la 
elección de las variables termodinámicas independientes es arbitraria y no está 
prescrita por el entorno. En la mecánica estadística se llega a la misma 
conclusión: independientemente del entorno, se puede seleccionar cualquier 
ensamble o función de partición para calcular las propiedades termodinámicas; 
los resultados serán independientes de dicha elección.
