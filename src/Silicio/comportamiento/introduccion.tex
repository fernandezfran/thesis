% Copyright (c) 2024, Francisco Fernandez
% License: CC BY-SA 4.0
%   https://github.com/fernandezfran/thesis/blob/main/LICENSE
\section{Introducción}

Desde mediados del siglo pasado, el estudio del coeficiente de difusión del litio
en el silicio ha sido un tema de investigación relevante \cite{fuller1953, 
pell1960}. Por ejemplo, para la aplicación de celdas solares de Si dopadas con Li 
\cite{larue1971} y para baterías de Li \cite{wen1981}. Uno de los valores 
más confiables de la difusión de Li en Si a concentraciones bajas fue medido por 
Yoshimura \textit{et al.} \cite{yoshimura2007}, donde utilizaron una celda bipolar 
con un cristal de silicio en el medio y midieron el tiempo que le toma al litio 
atravesar dicho cristal. Para dos cristales de anchos de 1 $\mu$m y 2.4 $\mu$m el 
valor obtenido para $D$ se encuentra dentro del orden de magnitud de 10$^{-9}$ 
cm$^2$/s, mientras que para un cristal de 180 $\mu$m se tiene una medición dos 
ódenes de magnitud menor. Con respecto a la difusión de litio a concentraciones 
bajas en silicio amorfo, Strauss \textit{et al.} \cite{strauss2018} utilizaron
SIMS (\textit{secondary ion mass spectrometry}) y midieron a altas temperaturas,
de un ajuste tipo Arrhenius determinaron una energía de activación de (1.42 $\pm$
0.03) eV. Por otro lado, Kulova \textit{et al.} \cite{kulova2007} utilizaron 
distintas técnicas electroquímicas para estimar el coeficiente de difusión del 
litio en láminas delgadas de silicio y obtuvieron valores en el órden de 
10$^{-13}$ cm$^2$/s. Zhang \textit{et al.} \cite{zhang2008} también utilizaron 
espectroscopia de impedancia electroquímica (EIS) para determinar $D$ en el 
primero, en el décimo y en el vigésimo ciclo de litiación, para los que obtuvieron 
valores de 6.69$\times 10^{-13}$ cm$^2$/s, 9.83$\times 10^{-12}$ cm$^2$/s y 
7.02$\times 10^{-12}$ cm$^2$/s, respectivamente. El comportamiento de $D$ en 
función de la concentración fue estudiado por primera vez por Ding \textit{et al.}
\cite{ding2009} donde determinaron experimentalmente con EIS y GITT 
(\textit{galvanostatic intermittent titration technique}) que el mismo exhibe un 
comportamiento tipo \say{W} variando entre 10$^{-13}$ cm$^2$/s y 10$^{-10}$ 
cm$^2$/s. Luego, Xie \textit{et al.} \cite{xie2010} utilizaron PITT 
(\textit{potentiostatic intermittent titration technique}) para determinar dicho
comportamiento utilizaron un electrolito líquido y otro polimérico, resultando $D$
en un orden de magnitud menor para este último (10$^{-13}$ cm$^2$/s y 10$^{-14}$ 
cm$^2$/s, respectivamente). Pan \textit{et al.} \cite{pan2019} también midieron
con EIS y GITT el coeficiente de difusión en función de la concentración y
obtuvieron un comportamiento tipo \say{W} en un intervalo desde 10$^{-12}$ 
cm$^2$/s hasta 10$^{-8}$ cm$^2$/s. Los dos mínimos se corresponden con las fases
a-LiSi y a-Li$_{13}$Si$_4$ y, además, se concluye que $D$ es menor en la litiación 
que en la delitiación. De las mediciones revisadas puede resaltarse que el valor 
del coeficiente de difusión de litio en silicio varía en órdenes de magnitud 
según la técnica utilizada.

Además de estas distintas determinaciones experimentales, en la literatura pueden
encontrarse distintos cálculos computacionales, por ejemplo, Moon \textit{et al.}
\cite{moon2014} utilizaron una estructura de 64 átomos de Si y colocaron un átomo
de Li en una posición tetrahédrica para estudiar su difusión a concentraciones
bajas. Con DFT encontraron barreras de activación de 0.6 eV y 0.4 eV para silicio
cristalino y amorfo, respectivamente. Estas energías fueron utilizadas en 
simulaciones de kMC y obtuvieron que el coeficiente de difusión del litio es un
orden de magnitud menor en el c-Si que en el a-Si, 10$^{-12}$ cm$^2$/s y 
10$^{-11}$ cm$^2$/s, respectivamente. En un trabajo posterior \cite{moon2016},
obtuvieron las energías de activación entre 0.3 eV y 0.37 eV para las aleaciones 
cristalinas de Li-Si y las utilizaron en simulaciones de kMC para obtener 
coeficientes de difusión entre 10$^{-11}$ cm$^2$/s y 10$^{-10}$ cm$^2$/s.
De manera similar, Chang \textit{et al.} \cite{chang2015} obtuvieron una variación
de dos órdenes de magnitud entre el coeficiente de difusión de un átomo de Li en 
Si y el de las aleaciones. Johari \textit{et al.} \cite{johari2011} realizaron 
simulaciones de MD \textit{ab initio} (AIMD) de 15 ps a temperaturas altas, entre 
900 K y 1500 K, con las cuales computaron los desplazamientos cuadráticos medios 
en función del tiempo y los ajustaron con una función lineal para obtener el 
coeficiente de difusión a cada temperatura. Con estos datos realizaron un ajuste 
tipo Arrhenius y extrapolaron $D$ a temperatura ambiente, obteniendo valores de
$D$ entre 10$^{-10}$ cm$^2$/s y 10$^{-8}$ cm$^2$/s. Pan \textit{et al.} 
\cite{pan2015} también realizaron simulaciones AIMD para estudiar $D$ en función
del estrés. Para estructuras amorfas de Li-Si obtenidas con un potencial MEAM, Cui 
\textit{et al.} \cite{cui2012} encontraron un comportamiento asintótico del 
coeficiente de difusión en función de la concentración, que comienza en 10$^{-11}$
cm$^2$/s y tiende a 10$^{-7}$ cm$^2$/s. Utilizando este mismo potencial, Chang 
\textit{et al.} \cite{chang2018} estudian con mayor detalle las variaciones de $D$ 
a concentraciones bajas, observando que dentro de este orden de magnitud hay una 
disminución inicial de $D$. Yan \textit{et al.} \cite{yan2015} utilizaron la 
técnica ABC (\textit{Autonomous Basin Clambin}) para explorar la PES del MEAM en 
una estructura de 1 átomo de Li y 64 de c-Si y a-Si y determinar que el 
coeficiente de difusión de traza del Li es dos órdenes de magnitud menor en la
estructura amorfa que en la cristalina. Al ReaxFF \cite{fan2013}, 
caracterizado en el capítulo \ref{ch:caracterización}, también se lo utilizó en 
la literatura para estimar el coeficiente de difusión: a distintas concentraciones 
\cite{kim2015}, a concentraciones bajas \cite{trochet2017} y en función del 
estrés \cite{ding2017}. También se aplicaron potenciales de redes neuronales para
este tipo de estudios, Onat \textit{et al.} \cite{onat2018} encontraron una
dependencia de $D$ con la concentración tipo asintótica, al igual que Cui 
\textit{et al.} \cite{cui2012} con el potencial MEAM, y Li \textit{et al.} 
\cite{li2020effect} obtuvieron valores del coeficiente de difusión de traza en 
concordancia con los medidos por Strauss \textit{et al.} \cite{strauss2018}.

En este capítulo se utiliza la parametrización DFTB desarrollada en el capítulo
\ref{ch:modelo} para analizar el coeficiente de difusión de litio en silicio en 
las estructuras obtenidas en el capítulo \ref{ch:prediccion}.
