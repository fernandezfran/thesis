% Copyright (c) 2024, Francisco Fernandez
% License: CC BY-SA 4.0
%   https://github.com/fernandezfran/thesis/blob/main/LICENSE
\subsection{Teoría del funcional de la densidad (DFT)}

El desarrollo de la mecánica cuántica, y las observaciones experimentales que la
validan, fue uno de los avances científicos más significativos del siglo XX. 
Esta teoría permite estudiar cómo varía la energía de los materiales con la
ubicación de los átomos. Esto gracias a la aproximación de Born-Oppenheimer, que 
separa el movimiento de los núcleos del de los electrones al considerar que los primeros son mucho más
pesados que los segundos. Esto implica que se puede considerar que los electrones 
responden mucho más rápido a los cambios en su entorno que los núcleos, lo que 
permite resolver las ecuaciones que describen su movimiento fijando las 
posiciones de los núcleos atómicos. Así se encuentra el estado fundamental de los
electrones, es decir, su configuración de menor energía \cite{shankar2012}.

La ecuación de Schrödinger no-relativista e independiente del tiempo,
\begin{equation}\label{eq:schrodinger}
    H \psi = E \psi,
\end{equation}
caracteriza un sistema físico desde un enfoque cuántico, donde $\psi$ es un
conjunto de soluciones, o autoestados, del Hamiltoniano $H$ que tienen asociados los
autovalores $E$ que satisfacen dicha ecuación. 

La cantidad que puede medirse es la probabilidad de que los $N$ electrones 
estén en un conjunto de posiciones $\lbrace \mathbf{r}_i \rbrace$ en cualquier orden. 
Una cantidad relacionada a dicha probabilidad es la densidad de electrones,
$n(\mathbf{r})$, que en el caso de un sistema de electrones independientes se puede
escribir como 
\begin{equation}
    n(\mathbf{r}) = 2 \sum_i \psi_i^*(\mathbf{r}) \psi_i(\mathbf{r}),
\end{equation}
donde la suma se realiza sobre todos los electrones $i$ y el producto de $\psi_i$ con su 
complejo conjugado $\psi_i^*$ es el cálculo de la probabilidad asociada al electrón $i$. 
Lo destacable de esta ecuación es que reduce la solución completa de la función de onda
de la ecuación de Schrödinger de $3 N$ coordenadas a tan sólo 3 y, además, contiene una 
gran cantidad de información que es observable, $n(\mathbf{r})$.

La teoría del funcional de la densidad electrónica (DFT por sus siglas en inglés
\textit{Density functional theory}) es un método altamente efectivo para 
encontrar soluciones a la ecuación fundamental que describe el comportamiento 
cuántico de átomos y moléculas en sistemas de materia condensada, la ecuación 
\ref{eq:schrodinger} de Schrödinger, en situaciones de utilidad práctica 
\cite{sholl2022}. La misma se basa en dos teoremas matemáticos fundamentales
demostrados por Hohenberg y Kohn \cite{hohenberg1964} y la derivación de un 
conjunto de ecuaciones realizada por Kohn y Sham \cite{kohn1965}.

El primero de los teoremas establece: \say{La energía del estado fundamental 
de la ecuación de Schrödinger es un funcional unívoco de la densidad
electrónica}, es decir que la energía $E$ puede expresarse como 
$E[n(\mathbf{r})]$.

El segundo teorema define la siguiente propiedad: \say{La densidad electrónica
que minimiza la energía del funcional global es la densidad verdadera de los 
electrones correspondiente a la solución completa de la ecuación de Schrödinger}.
Si se conociera la forma del funcional \say{verdadera} podría variarse la 
densidad electrónica hasta que se minimice su energía, este es el principio 
variacional y en la práctica se lo utiliza con formas aproximadas del funcional.

Al funcional de la energía se lo puede plantear en dos términos,
\begin{equation}
    E[\lbrace \psi_i \rbrace] = E_{\text{conocido}}[\lbrace \psi_i \rbrace] + E_{\text{XC}}[\lbrace \psi_i \rbrace].
\end{equation}
El primero de ellos involucra cuatro contribuciones que pueden expresarse 
analíticamente: la energía cinética del electrón y las interacciones de tipo 
Coulomb (electrón-núcleo, electrón-electrón y núcleo-núcleo), mientras que
el segundo de los términos es el funcional de la correlación de intercambio
que se define para incluir todos los efectos mecánico cuánticos que no estén
incluidos en los términos \say{conocidos}. Kohn y Sham demostraron que la 
solución de la ecuación puede plantearse en términos de la función de onda 
de un sólo electrón que depende de tres variables espaciales.

De la \say{derivada} del funcional de la correlación de intercambio puede 
definirse el potencial de correlación de intercambio como sigue
\begin{equation}
    V_{\text{XC}}(\mathbf{r}) = \frac{\delta E_{\text{XC}}(\mathbf{r})}{\delta n(\mathbf{r})}.
\end{equation}
La solución exacta de la densidad electrónica ha sido demostrada para un gas 
uniforme de electrones. Luego, una aproximación local de la densidad (LDA,
\textit{local density aproximation}) utiliza la densidad local para definir 
un potencial de correlación de intercambio aproximado,
\begin{equation}
    V_{\text{XC}}(\mathbf{r}) = V_{\text{XC}}^{\text{gas de electrones}}[n(\mathbf{r})].
\end{equation}
Otras aproximaciones también utilizan información del gradiente local en 
la densidad electrónica, lo cual define la aproximación generalizada del 
gradiente (GGA, \textit{generalized gradient approximation}). Uno de los
funcionales más utilizados para sólidos es el de Perdew-Burke-Ernzerhof (PBE).
