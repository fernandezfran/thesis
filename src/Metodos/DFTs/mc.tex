\section{Mecánica cuántica}

El desarrollo de la mecánica cuántica, y las observaciones experimentales que la
validan, fue uno de los avances científicos más significativos del siglo veinte. 
Esta teoría permite estudiar cómo varía la energía de los materiales con el 
movimiento de los átomos. Esto gracias a la aproximación de Born-Oppenheimer, que 
separa los núcleos de los electrones al considerar que los primeros son mucho más
pesados que los segundos. Esto implica que se puede considerar que los electrones 
responden mucho más rápido a los cambios en su entorno que los núcleos, lo que 
permite resolver las ecuaciones que describen su movimiento fijando las 
posiciones de los núcleos atómicos. Así se encuentra el estado fundamental de los
electrones, es decir, su configuración de menor energía \cite{shankar2012}.

La ecuación de Schrödinger no-relativista e independiente del tiempo,
\begin{equation}\label{eq:schrodinger}
    H \psi = E \psi,
\end{equation}
caracteriza un sistema físico desde un punto de vista cuántico, donde $\psi$ es un
conjunto de soluciones, o autoestados, del Hamiltoniano $H$ que tienen asociados los
autovalores $E$ que satisfacen dicha ecuación. 
%Para el caso en el que se desee estudiar
%la interacción entre varios electrones y núcleos una descripción más completa de la 
%ecuación de Schrödinger es la siguiente,
%\begin{equation}\label{eq:schrodinger}
%    \left[ - \frac{\hbar^2}{2 m} \sum_{i=1}^N \nabla_i^2 + \sum_{i=1}^N V(\mathbf{r}_i) + \sum_{i=1}^N \sum_{i<j} U(\mathbf{r}_i, \mathbf{r}_j) \right] \psi = E \psi,
%\end{equation}
%donde $\hbar$ es la constante de Plank reducida, $m$ la masa del electrón, el primer 
%término describe la energía cinética de cada electrón, el segundo la energía de 
%interacción de cada electrón con todos los núcleos y el tercero la energía de 
%interacción entre distintos electrones. Este último término es el más crítico desde 
%el punto de vista de la resolución de la ecuación.

%Ahora, la cantidad que puede medirse es la probabilidad de que los $N$ electrones 
La cantidad que puede medirse es la probabilidad de que los $N$ electrones 
estén en un conjunto de posiciones $\lbrace \mathbf{r}_i \rbrace$ en cualquier orden. 
Una cantidad relacionada a dicha probabilidad es la densidad de electrones,
$n(\mathbf{r})$. Esta se puede escribir como 
\begin{equation}
    n(\mathbf{r}) = 2 \sum_i \psi_i^*(\mathbf{r}) \psi_i(\mathbf{r}),
\end{equation}
donde la suma se realiza sobre todos los electrones $i$ y el producto de $\psi_i$ con su 
complejo conjugado $\psi_i^*$ es el cálculo de la probabilidad asociada al electrón $i$. 
Lo destacable de esta ecuación es que reduce la solución completa de la función de onda
de la ecuación de Schrödinger de $3 N$ coordenadas a tan sólo 3 y, además, contiene una 
gran cantidad de información que es observable, $n(\mathbf{r})$.

\subsection{Teoría del funcional de la densidad (DFT)}

El desarrollo de la mecánica cuántica, y las observaciones experimentales que la
validan, fue uno de los avances científicos más significativos del siglo veinte. 
Esta teoría permite estudiar cómo varía la energía de los materiales con el 
movimiento de los átomos. Esto gracias a la aproximación de Born-Oppenheimer, que 
separa el movimiento de los núcleos del de los electrones al considerar que los primeros son mucho más
pesados que los segundos. Esto implica que se puede considerar que los electrones 
responden mucho más rápido a los cambios en su entorno que los núcleos, lo que 
permite resolver las ecuaciones que describen su movimiento fijando las 
posiciones de los núcleos atómicos. Así se encuentra el estado fundamental de los
electrones, es decir, su configuración de menor energía \cite{shankar2012}.

La ecuación de Schrödinger no-relativista e independiente del tiempo,
\begin{equation}\label{eq:schrodinger}
    H \psi = E \psi,
\end{equation}
caracteriza un sistema físico desde un enfoque cuántico, donde $\psi$ es un
conjunto de soluciones, o autoestados, del Hamiltoniano $H$ que tienen asociados los
autovalores $E$ que satisfacen dicha ecuación. 
%Para el caso en el que se desee estudiar
%la interacción entre varios electrones y núcleos una descripción más completa de la 
%ecuación de Schrödinger es la siguiente,
%\begin{equation}\label{eq:schrodinger}
%    \left[ - \frac{\hbar^2}{2 m} \sum_{i=1}^N \nabla_i^2 + \sum_{i=1}^N V(\mathbf{r}_i) + \sum_{i=1}^N \sum_{i<j} U(\mathbf{r}_i, \mathbf{r}_j) \right] \psi = E \psi,
%\end{equation}
%donde $\hbar$ es la constante de Planck reducida, $m$ la masa del electrón, el primer 
%término describe la energía cinética de cada electrón, el segundo la energía de 
%interacción de cada electrón con todos los núcleos y el tercero la energía de 
%interacción entre distintos electrones. Este último término es el más crítico desde 
%el punto de vista de la resolución de la ecuación.

%Ahora, la cantidad que puede medirse es la probabilidad de que los $N$ electrones 
La cantidad que puede medirse es la probabilidad de que los $N$ electrones 
estén en un conjunto de posiciones $\lbrace \mathbf{r}_i \rbrace$ en cualquier orden. 
Una cantidad relacionada a dicha probabilidad es la densidad de electrones,
$n(\mathbf{r})$. Esta se puede escribir como 
\begin{equation}
    n(\mathbf{r}) = 2 \sum_i \psi_i^*(\mathbf{r}) \psi_i(\mathbf{r}),
\end{equation}
donde la suma se realiza sobre todos los electrones $i$ y el producto de $\psi_i$ con su 
complejo conjugado $\psi_i^*$ es el cálculo de la probabilidad asociada al electrón $i$. 
Lo destacable de esta ecuación es que reduce la solución completa de la función de onda
de la ecuación de Schrödinger de $3 N$ coordenadas a tan sólo 3 y, además, contiene una 
gran cantidad de información que es observable, $n(\mathbf{r})$.

La teoría del funcional de la densidad electrónica (DFT por sus siglas en inglés
\textit{Density functional theory}) es un método altamente efectivo para 
encontrar soluciones a la ecuación fundamental que describe el comportamiento 
cuántico de átomos y moléculas en sistemas de materia condensada, la ecuación 
\ref{eq:schrodinger} de Schrödinger, en situaciones de utilidad práctica 
\cite{sholl2022}. La misma se basa en dos teoremas matemáticos fundamentales
demostrados por Hohenberg y Kohn \cite{hohenberg1964} y la derivación de un 
conjunto de ecuaciones realizada por Kohn y Sham \cite{kohn1965}.

El primero de los teoremas establece: \say{La energía del estado fundamental 
de la ecuación de Schrödinger es un funcional unívoco de la densidad
electrónica}, es decir que la energía $E$ puede expresarse como 
$E[n(\mathbf{r})]$.

El segundo teorema define la siguiente propiedad: \say{La densidad electrónica
que minimiza la energía del funcional global es la densidad verdadera de los 
electrones correspondiente a la solución completa de la ecuación de Schrödinger}.
Si se conociera la forma del funcional \say{verdadera} podría variarse la 
densidad electrónica hasta que se minimice su energía, este es el principio 
variacional y en la práctica se lo utiliza con formas aproximadas del funcional.

Al funcional de la energía se lo puede plantear en dos términos,
\begin{equation}
    E[\lbrace \psi_i \rbrace] = E_{\text{conocido}}[\lbrace \psi_i \rbrace] + E_{\text{XC}}[\lbrace \psi_i \rbrace].
\end{equation}
El primero de ellos involucra cuatro contribuciones que pueden expresarse 
analíticamente: la energía cinética del electrón y las interacciones de tipo 
Coulomb (electrón-núcleo, electrón-electrón y núcleo-núcleo), mientras que
el segundo de los términos es el funcional de la correlación de intercambio
que se define para incluir todos los efectos mecánico cuánticos que no estén
incluidos en los términos \say{conocidos}. Kohn y Sham demostraron que la 
solución de la ecuación puede plantearse en términos de la función de onda 
de un sólo electrón que depende de tres variables espaciales.

De la \say{derivada} del funcional de la correlación de intercambio puede 
definirse el potencial de correlación de intercambio como sigue
\begin{equation}
    V_{\text{XC}}(\mathbf{r}) = \frac{\delta E_{\text{XC}}(\mathbf{r})}{\delta n(\mathbf{r})}.
\end{equation}
La solución exacta de la densidad electrónica ha sido demostrada para un gas 
uniforme de electrones. Luego, una aproximación local de la densidad (LDA,
\textit{local density aproximation}) utiliza la densidad local para definir 
un potencial de correlación de intercambio aproximado,
\begin{equation}
    V_{\text{XC}}(\mathbf{r}) = V_{\text{XC}}^{\text{gas de electrones}}[n(\mathbf{r})].
\end{equation}
Otras aproximaciones también utilizan información del gradiente local en 
la densidad electrónica, lo cual define la aproximación generalizada del 
gradiente (GGA, \textit{generalized gradient approximation}). Uno de los
funcionales más utilizados para sólidos es el de Perdew-Burke-Ernzerhof (PBE).


\subsection{Funcional de la densidad de enlace fuerte (DFTB)}\label{s:dftb}

El formalismo del funcional de densidad de enlace fuerte (DFTB, \textit{density
functional tight-binging}) ha sido ampliamente descripto en la literatura 
\cite{elstner1998,frauenheim2000,seifert2007,gaus2011}. El método DFTB se basa 
en una expansión a segundo orden de la energía de la teoría del funcional de la 
densidad (DFT) con respecto a una fluctuación de la densidad electrónica de 
referencia \cite{foulkes1989}. La energía de DFTB resultante puede escribirse de 
la siguiente manera:
\begin{equation}\label{eq:dftb}
    E_{\text{DFTB}}=\sum_i^{\text{occ}}\langle\psi_i|\hat{H}^0|\psi_i\rangle+\frac{1}{2}\sum_{AB}\gamma_{AB}\Delta q_A\Delta q_B+E_{\text{rep}}^{AB}
\end{equation}
donde $\psi_i$ denota los orbitales Kohn-Sham (KS) de una partícula. Con una 
combinación lineal de orbitales atómicos, $\psi_i$ se expande en un conjunto de 
orbitales de valencia pseudoatómicos de tipo Slater $\phi_\nu$,
\begin{equation}
    \psi_i({\bf r})=\sum_\nu c_{\nu i}\phi_\nu({\bf r}-{\bf r}_A),
\end{equation}
que se determinan resolviendo la ecuación secular KS
\begin{equation}\label{eq:ks}
    \sum_\mu c_{\mu i}\left(H^0_{\nu\mu}-\epsilon_iS_{\nu\mu}\right)=0, \;\;\forall \nu,i
\end{equation}
donde $S_{\nu\mu}=\langle \phi_\nu| \phi_\mu\rangle$ y $\epsilon_\nu$ son la 
matriz de superposición y los autovalores de un átomo aislado, respectivamente.
${H}^0_{\nu\mu}$ es el Hamiltoniano efectivo KS generado con la densidad 
electrónica de referencia, $\rho^0$, y está definido como
\begin{equation}\label{eq:h0}
    H^0_{\nu\mu}=\begin{cases}
        \epsilon_\mu & \text{si}\; \nu=\mu\\
        \langle \phi_{\nu}| -\frac{1}{2}\nabla^2+v_{\text{eff}}\left[\rho_A^0+\rho_B^0\right]|\phi_{\nu}\rangle&\text{si}\;\mu\in A,\; \nu\in B\;\text{y} \;A\ne B\\
        0& \text{si no}
    \end{cases}
\end{equation}
donde $\rho_A^0$ es la densidad de referencia de un átomo neutro $A$ y 
$v_{\text{eff}}$ el potencial KS efectivo, construido a partir de la superposición
de densidades centradas en átomos neutros. En particular, los elementos de la 
matriz del Hamiltoniano dependen solo de los átomos $A$ y $B$, por lo tanto sólo
se calculan explícitamente los elementos de dos centros de las matrices del 
Hamiltoniano y de superposición en función de la distancia y la orientación, usando 
las reglas de transformación de Slater-Koster \cite{slater1954}.

Una de las partes cruciales del uso del método DFTB es calcular las funciones 
base y las densidades atómicas $\phi$ y $\rho^0$, respectivamente. Los orbitales
pseudoatómicos y las densidades se obtienen de resolver las ecuaciones atómicas KS 
modificadas en las que se agrega un potencial de confinamiento, $V_{\text{conf}}$,
\begin{equation}\label{eq:dft}
    \left[\hat{T}+V_{\text{eff}}+V_{\text{conf}}\right]\phi_\mu=\epsilon_\mu\phi_\mu.
\end{equation}
Una práctica común dentro de la comunidad de DFTB consiste en elegir un potencial
de confinamiento parabólico, cuadrático, o una función de ley de potencia.

El segundo término en la ecuación \ref{eq:dftb} es la energía debida a las 
fluctuaciones de cargas y se parametriza analíticamente como una función de las
cargas orbitales y de $\gamma_{AB}$, que a su vez es una función de la separación 
interatómica y del parámetro de Hubbard, $U$, que se obtienen suponiendo que son 
iguales a los de los átomos aislados y se calculan como la diferencia de la 
afinidad electrónica y la energía de ionización para distintos momentos angulares 
orbitales \cite{elstner1998b}. $\Delta q_X = q_X - q_X^0$ es la carga de Mulliken 
inducida autoconsistente en el átomo $X$ \cite{elstner1998}.

La contribución restante a la energía total de DFTB en la ecuación \ref{eq:dftb}
es $E_{\text{rep}}$ y se corresponde con el potencial repulsivo diatómico que 
depende de la distancia y contiene los efectos de los electrones del núcleo, los 
términos de repulsión ion-ion y efectos de intercambio-correlación. 
La energía total repulsiva de un sistema es una suma de contribuciones de 
potenciales repulsivos $V_{\text{rep}}(r)$ de cada par de átomos
\begin{equation}\label{eq:rep}
    E_{\text{rep}}=\sum_{i<j} V_{\text{rep}}(r_{ij})
\end{equation}
donde $i$ y $j$ son los índices de los átomos en el sistema y $r_{ij}$ es la 
distancia entre ellos. Generalmente se considera que $V_{\text{rep}}$ es una
función empírica que se determina al ajustar datos de cálculos de estructura 
electrónica de un nivel superior, como DFT.

