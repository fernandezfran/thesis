% Copyright (c) 2024, Francisco Fernandez
% License: CC BY-SA 4.0
%   https://github.com/fernandezfran/thesis/blob/main/LICENSE
\subsection{Exploración acelerada de mínimos locales}\label{s:aelm}

Las simulaciones de MD tienen un gran poder predictivo para el estudio de 
procesos presentes en las baterías de litio, sin embargo, las escalas de tiempo
están limitadas a unos pocos ns o $\mu$s. El número de operaciones que se 
necesita para alcanzar las escalas de tiempo de la operación de una batería 
experimental son prohibitivos, incluso considerando el uso de potenciales 
semi-empíricos como el ReaxFF en supercomputadoras. Como consecuencia de esto,
la MD usual no es suficiente para una exploración del espacio de las fases y las
estructuras de Li-Si observadas van a estar cercanas a las configuraciones 
iniciales mientras que en el sistema real probablemente pueden aparecer otras
configuraciones. Un método simple y poderoso para acelerar la exploración de 
mínimos locales en sistemas moleculares es el templado simulado 
\cite{kirkpatrick1983}, en el cual básicamente se busca mejorar la exploración
del espacio de las fases en simulaciones de MD utilizando temperaturas altas y
luego reduciéndola progresivamente hasta encontrar un mínimo de energía a 
temperatura ambiente. El templado simulado múltiple (MSA, de sus siglas en inglés 
\textit{Multiple simluated annealing}) utiliza esta idea en distintas copias del 
sistema y fue utilizado para explorar y encontrar distintas estructuras mínimas 
relevantes cercanas al equilibrio \cite{hao2015}.

La presente técnica de simulación, exploración acelerada de mínimos locales (AELM,
de sus siglas en inglés \textit{accelerated exploration of local minima}), es 
similar a la MSA pero en vez de calentar y enfriar lentamente el sistema, se 
utiliza un sesgo en la función de energía potencial para sobrepasar las barreras
de energía y luego se realiza una minimización local, con algún minimizador local 
como gradientes conjugados o LBFGS, para encontrar el mínimo. Este método permite 
obtener muchas estructuras con energías mínimas relevantes, que son de interés a 
la hora de estudiar electrodos de Li-Si muy ciclados.

Las aleaciones de Li-Si presentan interacciones fuertes entre los átomos que las
conforman, especialmente en el enlace Si-Si donde la energía de enlace es del
orden de $\approx$2 eV \cite{wypych2018handbook}. Se espera que las barreras de 
energía potencial sean de ese orden de magnitud, por lo cual un 
muestreo de una MD a temperatura ambiente parece no ser viable. Para explorar 
ampliamente las distintas configuraciones del sistema, $\mathbf{r}$, se 
transforma la superficie de energía potencial (PES, de sus siglas en inglés 
\textit{potencial energy surface}), $V(\mathbf{r})$, usando un potencial sesgado
\begin{equation}\label{eq:bias}
    V_b(\mathbf{r}) = V(\mathbf{r}) + (\alpha - 1) V(\mathbf{r}) = \alpha V(\mathbf{r}),
\end{equation}
donde $\alpha$ es el factor de compresión. El principal efecto de la ecuación 
\ref{eq:bias} es reducir las barreras de la PES, por lo cual el tiempo de
residencia en estados metaestables es menor que en el sistema sin sesgar y la 
exploración de configuraciones de sistemas diferentes es más eficiente y 
alcanzada en un tiempo de simulación razonable. El término 
$(\alpha - 1) V(\mathbf{r})$ es usualmente referido como la 
\say{función de sesgo}.

La adición de esta función de sesgo a la PES constituye la base del método de 
hiper-dinámica (HD), desarrollado por Voter \cite{voter1997HD,voter1997method} 
para acelerar la exploración de un sistema sin perder su comportamiento dinámico a tiempos largos. En una 
simulación típica de HD, para recuperar el promedio de alguna propiedad, las 
configuraciones muestreadas son repesadas por un factor $w$ que involucra una función
exponencial y depende del sesgo aplicado. Debido a que este sistema involucra 
cambios grandes en las energías de interacción, comparado con la energía térmica
$k_BT$, lo que implica que la función exponencial en $w$ toma valores muy grandes,
esto provoca que el procedimiento numérico sea inestable y la recuperación de 
la propiedad de interés, como la energía potencial, no sea posible. Ya que este
capítulo se centra en un estudio estructural de los sistemas, podemos ignorar
el cálculo del tiempo real evolucionado en la simulación. Además, como el 
funcionamiento de las baterías luego se da a temperatura ambiente, es de esperar
que una vez que se alcance un mínimo local, el sistema explore configuraciones
cercanas a este. Por este motivo se aplica el método de gradientes conjugados (CG)
a cada una de las configuraciones de la HD y de esa forma se muestrea la 
multiplicidad de estructuras.

Este método de exploración introducido en esta tesis se asemeja al templado
simulado, aunque el objetivo es explorar muchas estructuras diferentes en vez de 
encontrar el mínimo global. El templado simulado fue utilizado anteriormente
con este mismo objetivo, Hao \textit{et al.} utilizó la técnica MSA para obtener
distintas estructuras de mínima energía de péptidos \cite{hao2015}. En este 
método AELM se usa HD en vez de temperaturas altas para favorecer la exploración,
y se realizan múltiples minimizaciones por CG en vez de simular un enfriado. 
