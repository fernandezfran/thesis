% Copyright (c) 2024, Francisco Fernandez
% License: CC BY-SA 4.0
%   https://github.com/fernandezfran/thesis/blob/main/LICENSE
\thispagestyle{empty}

\onehalfspacing

\begin{center}

    \includegraphics[height=2.5cm]{Caratula/logo.png}

    \vspace{1cm}

    {\LARGE Modelado computacional para el desarrollo de electrodos de baterías de ion-litio de próxima generación}

    \vspace{0.5cm}
    por
    \vspace{0.5cm}

    {\Large Francisco Fernandez}

    \vspace{0.5cm}

    Presentado ante la Facultad de Matemática, Astronomía, Física y Computación 
    como parte de los requerimientos para la obtención del grado de

    \vspace{0.5cm}

    {\Large Doctor en Física}

    \vspace{0.5cm}
    de la

    UNIVERSIDAD NACIONAL DE CÓRDOBA

    \vspace{0.5cm}

    Abril 2024

\end{center}

\vspace{0.5cm}

\textbf{Director}: 

\hspace{1.5cm} Dr. Daniel BARRACO DÍAZ

\textbf{Codirector}: 

\hspace{1.5cm} Dr. Ezequiel LEIVA (FCQ, UNC)

\vspace{0.25cm}

\textbf{Tribunal especial}:

\hspace{1.5cm} Dr. Roberto Manuel TORRESI (Instituto de Química, Universidade de São Paulo)

\hspace{1.5cm} Dr. Alejandro FRANCO (LRCS, Université de Picardie Jules Verne)

\hspace{1.5cm} Dr. Fabián VACA CHÁVEZ FORNASERO (FAMAF, UNC)

\textbf{Tribunal suplente}:

\hspace{1.5cm} Dra. Mariela ORTIZ (INIFTA, Universidad Nacional de La Plata)

\hspace{1.5cm} Dr. Andrés RUDERMAN (FAMAF, UNC)

\vspace{1.0cm}

\begin{center}
    
    \vfill
    \href{https://creativecommons.org/licenses/by-sa/4.0/deed.es}{
        \includegraphics[height=0.75cm]{Caratula/cc-by-sa.png}
    }

    {\footnotesize 
    Este trabajo se distribuye bajo una licencia
    \href{https://creativecommons.org/licenses/by-sa/4.0/deed.es}{Licencia 
    Creative Commons Atribución-CompartirIgual 4.0 Internacional}
    }

\end{center}

\singlespacing
