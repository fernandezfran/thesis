\subsection{Teoría del funcional de la densidad (DFT)}

La teoría del funcional de la densidad (DFT por sus siglas en inglés
\textit{Density functional theory}) es un método altamente efectivo para 
encontrar soluciones a la ecuación fundamental que describe el comportamiento 
cuántico de átomos y moleculas en sistemas de materia condensada, la ecuación 
\ref{eq:schrodinger} de Schrödinger, en situaciones de utilidad práctica 
\cite{sholl2022}. La misma se basa en dos teoremas matemáticos fundamentales
demostrados por Hohenberg y Kohn \cite{hohenberg1964} y la derivación de un 
conjunto de ecuaciones realizada por Kohn y Sham \cite{kohn1965}.

El primero de los teoremas establece: \say{La energía del estado fundamental 
de la ecuación de Schrödinger es un funcional unívoco de la densidad
electrónica}, es decir que la energía $E$ puede expresarse como 
$E[n(\mathbf{r})]$.

El segundo teorema define la siguiente propiedad: \say{La densidad electrónica
que minimiza la energía del funcional global es la densidad verdadera de los 
electrones correspondiente a la solución completa de la ecuación de Schrödinger}.
Si se conociera la forma del funcional \say{verdadera} podría variarse la 
densidad electrónica hasta que se minimice su energía, este es el principio 
variacional y en la práctica se lo utiliza con formas aproximadas del funcional.

Al funcional de la energía se lo puede plantear en dos términos,
\begin{equation}
    E[\lbrace \psi_i \rbrace] = E_{\text{conocido}}[\lbrace \psi_i \rbrace] + E_{\text{XC}}[\lbrace \psi_i \rbrace].
\end{equation}
El primero de ellos involucra cuatro contriubciones que pueden expresarse 
analíticamente: la energía cinética del electrón y las interacciónes de tipo 
Coulomb (electrón-núcleo, electrón-electrón y núcleo-núcleo). Mientras que
el segundo de los términos es el funcional de la correlación de intercambio
que se define para incluir todos los efectos mecánico cuánticos que no estén
incluidos en los términos \say{conocidos}. Kohn y Sham demostraron que la 
solución de la ecuación puede plantearse en términos de la función de onda 
sólo un electrón que depende de tres variables espaciales.

El potencial de correlación de intercambio puede definirse como una 
\say{derivada del funcional}
\begin{equation}
    V_{\text{XC}}(\mathbf{r}) = \frac{\delta E_{\text{XC}}(\mathbf{r})}{\delta n(\mathbf{r})}.
\end{equation}
La solución exacta de la densidad electrónica ha sido demostrada para un gas 
uniforme de electrones. Luego, una aproximación local de la densidad (LCA,
\textit{local density aproximation}) utiliza la densidad local para definir 
un potencial de correlación de intercambio aproximado,
\begin{equation}
    V_{\text{XC}}(\mathbf{r}) = V_{\text{XC}}^{\text{gas de electrones}}[n(\mathbf{r})].
\end{equation}
Otras aproximaciones también utilizan información del gradiente local en 
la densidad electrónica, lo cual define la aproximación generalizada del 
gradiente (GGA, \textit{generalized gradient approximation}). Uno de los
funcionales más utilizados para sólidos es el de Perdew-Burke-Ernzerhof (PBE).
