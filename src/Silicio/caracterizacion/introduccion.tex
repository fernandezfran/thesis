% Copyright (c) 2024, Francisco Fernandez
% License: CC BY-SA 4.0
%   https://github.com/fernandezfran/thesis/blob/main/LICENSE
\section{Introducción}

La información experimental que puede obtenerse de la estructuras de las 
distintas fases amorfas que se forman durante el ciclado de los electrodos de 
silicio es bastante limitada. Estas estructuras son 
inestables y amorfas, lo cual dificulta su caracterización mediante técnicas
experimentales tradicionales. Por ejemplo, la difracción de rayos x permitió
caracterizar la fase cristalina Li$_{15}$Si$_4$ que está presente en el electrodo
cuando este se encuentra completamente cargado \cite{obrovac2004}, pero esta 
técnica tiene ciertas limitaciones a la hora de estudiar estructuras amorfas 
que se encuentran en los procesos de carga y descarga. Por otro lado, el análisis
de la función distribución de a pares de Si \textit{ex-situ} de datos de rayos x
hizo posible investigar el orden a corto alcance de las estructuras amorfas de
Li$_x$Si \cite{key2011} y proponer una explicación al mecanismo de litiación.

Dentro de este contexto, las simulaciones computacionales se posicionan como una
herramienta poderosa para acceder al comportamiento atomístico de las 
estructuras de Li$_x$Si y los cambios que sufren durante la litiación. Actualmente
no existe un único modelo computacional robusto que permita estudiar todos los
diferentes procesos presentes en los electrodos de silicio, por lo que se han 
llevado a cabo distintos esfuerzos en los últimos años para estudiar este sistema,
donde el mayor obstáculo está relacionado con la naturaleza intrínseca de 
multi-escala del silicio. A pesar de su gran precisión, los estudios de DFT se
encuentran drásticamente limitados en el número de átomos que se pueden utilizar
para modelar las estructuras complejas de las fases litiadas. Una solución a este
problema es utilizar potenciales interatómicos semi-empíricos, para los cuales 
se necesita una parametrización que sea robusta y transferible. En este 
capítulo se utiliza un potencial reactivo
para sistemas de Li-Si parametrizado por Fan 
\textit{et al.} \cite{fan2013}, en la cual optimizaron el campo de fuerzas usando
cálculos de DFT, considerando datos de las energías, distintas geometrías y cargas
de las fases cristalinas de Li, Si y aleaciones de Li-Si. En su trabajo lo 
utilizaron en simulaciones de MD para caracterizar las propiedades mecánicas de
las estructuras amorfas de Li$_x$Si, incluyendo litiación de capa fina, compresión
biaxial, tensión y compresión uniaxial y la tensión que puede soportar el sistema 
antes de deformarse.
